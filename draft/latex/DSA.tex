%% Generated by Sphinx.
\def\sphinxdocclass{report}
\documentclass[letterpaper,10pt,lithuanian]{sphinxmanual}
\ifdefined\pdfpxdimen
   \let\sphinxpxdimen\pdfpxdimen\else\newdimen\sphinxpxdimen
\fi \sphinxpxdimen=.75bp\relax
\ifdefined\pdfimageresolution
    \pdfimageresolution= \numexpr \dimexpr1in\relax/\sphinxpxdimen\relax
\fi
%% let collapsible pdf bookmarks panel have high depth per default
\PassOptionsToPackage{bookmarksdepth=5}{hyperref}

\PassOptionsToPackage{booktabs}{sphinx}
\PassOptionsToPackage{colorrows}{sphinx}

\PassOptionsToPackage{warn}{textcomp}

\catcode`^^^^00a0\active\protected\def^^^^00a0{\leavevmode\nobreak\ }
\usepackage{cmap}
\usepackage{fontspec}
\defaultfontfeatures[\rmfamily,\sffamily,\ttfamily]{}
\usepackage{amsmath,amssymb,amstext}
\usepackage{polyglossia}
\setmainlanguage{lithuanian}



\setmainfont{DejaVu Serif}
\setsansfont{DejaVu Sans}
\setmonofont{DejaVu Sans Mono}



\usepackage[Sonny]{fncychap}
\ChNameVar{\Large\normalfont\sffamily}
\ChTitleVar{\Large\normalfont\sffamily}
\usepackage{sphinx}

\fvset{fontsize=auto}
\usepackage{geometry}


% Include hyperref last.
\usepackage{hyperref}
% Fix anchor placement for figures with captions.
\usepackage{hypcap}% it must be loaded after hyperref.
% Set up styles of URL: it should be placed after hyperref.
\urlstyle{same}


\usepackage{sphinxmessages}
\setcounter{tocdepth}{1}

\usepackage{enumitem} \setlistdepth{9}

\title{DSA Specifikacija}
\date{2024-12-17}
\release{}
\author{VSSA}
\newcommand{\sphinxlogo}{\vbox{}}
\renewcommand{\releasename}{}
\makeindex
\begin{document}

\pagestyle{empty}
\sphinxmaketitle
\pagestyle{plain}
\sphinxtableofcontents
\pagestyle{normal}
\phantomsection\label{\detokenize{index::doc}}

\begin{description}
\sphinxlineitem{Ši versija:}
\sphinxAtStartPar
\sphinxhref{https://ivpk.github.io/dsa/draft/}{Naujausias redaktoriaus juodraštis}

\sphinxlineitem{Klaidų sekimas:}
\sphinxAtStartPar
\sphinxhref{https://github.com/ivpk/dsa/issues}{Github}

\sphinxlineitem{Redaktorius:}
\sphinxAtStartPar
\sphinxhref{https://vssa.lrv.lt/lt/}{VSSA}

\sphinxlineitem{Vertimas (nenorminis):}
\sphinxAtStartPar
Nėra

\sphinxlineitem{Naujausia paskelbta versija:}
\sphinxAtStartPar
\sphinxhref{https://ivpk.github.io/dsa/}{Naujausia paskelbta versija}

\sphinxlineitem{Naujausias redaktoriaus juodraštis:}
\sphinxAtStartPar
\sphinxhref{https://ivpk.github.io/dsa/draft/}{Naujausias redaktoriaus juodraštis}

\end{description}

\sphinxAtStartPar
Čia rasite \sphinxstyleemphasis{Duomenų struktūros aprašo} ({\hyperref[\detokenize{savokos:term-DSA}]{\sphinxtermref{\DUrole{xref}{\DUrole{std}{\DUrole{std-term}{DSA}}}}}}) lentelės specifikaciją.

\sphinxAtStartPar
\sphinxstyleemphasis{Duomenų struktūros aprašas} yra lentelė skirta fizinio, loginio ir
semantinio duomenų modelių susiejimui, prieigos lygio nustatymui ir duomenų
brandos lygio vertinimui.

\noindent\sphinxincludegraphics{{dsa_overview}.png}

\sphinxAtStartPar
\sphinxstyleemphasis{Koncepcinis modelis} yra UML klasių diagrama, sudaryta laikantis
\sphinxhref{https://semiceu.github.io/style-guide/1.0.0/gc-conceptual-model-conventions.html}{Conceptual model conventions (UML)} reikalavimų. Koncepcinis modelis laikomas
kaip vienintelis tiesios šaltinis ir yra sudaromas remiantis teisės aktais,
informacinės sistemos nuostatais, semantiniais žodynais ir duomenų modeliu iš
duomenų šaltinio.

\sphinxAtStartPar
\sphinxstyleemphasis{Fizinis modelis} šio dokumento kontekste yra duomenų schema apibūdinanti
kur ir kaip duomenys yra saugomi ir kaip juos pasiekti. Schema apibrėžianti
duomenų modelį priklauso nuo duomenų saugojimo formato. Jei duomenys saugomi
SQL duomenų bazėse, tada DSA lentelėje nurodomi lentelių ir stulpelių
pavadinimai, XML atveju nurodomos \sphinxhref{https://www.w3.org/TR/2010/REC-xpath20-20101214/}{XPath} išraiškos, JSON atveju nurodomos
\sphinxhref{https://www.ietf.org/archive/id/draft-goessner-dispatch-jsonpath-00.html}{JSONPath} išraiškos. DSA lentelėje fizinis modelis nurodomas \DUrole{xref}{\DUrole{std}{\DUrole{std-ref}{source}}}
stulpelyje.

\sphinxAtStartPar
\sphinxstyleemphasis{Loginis modelis} yra duomenų schema, kuri naudojama duomenų apsikeitimui
\sphinxhref{https://ivpk.github.io/uapi}{UDTS} protokolu, loginis modelis rengiamas pagal koncepcinį modelį ir yra
artimas semantiniam modeliui, tačiau skirtas duomenų publikavimui per API.
Loginis modelis siejamas su fiziniu ir semantiniu modeliais.

\sphinxAtStartPar
\sphinxstyleemphasis{Semantinis modelis} yra nepriklausomas nuo to, kaip duomenys saugomi ar
perduodami fiziškai, siejamas su tarptautiniais standartais ir plačiai
naudojamais sąvokų žodynais.


\chapter{Specifikacijos}
\label{\detokenize{index:specifikacijos}}\begin{itemize}
\item {} 
\sphinxAtStartPar
\sphinxhref{https://ivpk.github.io/uapi}{Universali duomenų teikimo sąsaja} (UDTS)

\item {} 
\sphinxAtStartPar
\sphinxhref{https://ivpk.github.io/DCAT-AP-LT}{Duomenų katalogo Lietuvos taikymo profilis} (DCAT\sphinxhyphen{}AP\sphinxhyphen{}LT)

\end{itemize}


\chapter{Turinys}
\label{\detokenize{index:turinys}}
\sphinxstepscope


\section{Koncepcinis modelis}
\label{\detokenize{modelis:koncepcinis-modelis}}\label{\detokenize{modelis:uml-index}}\label{\detokenize{modelis::doc}}
\sphinxAtStartPar
Prieš pradedant darbą su strūktūros aprašais, būtina pasirengti koncepcinio
modelio UML diagramą, vadovaujantis \sphinxhref{https://semiceu.github.io/style-guide/1.0.0/gc-conceptual-model-conventions.html}{Conceptual model conventions (UML)}
reikalavimais.

\sphinxAtStartPar
Koncepcinis modelis vienareikšmiškai apibrėžia duomenų modelį grafine forma ir
naudojamas, kaip vienintelis tiesos šaltinis, kadangi vizualinę duomenų modelio
reprezentaciją nesunkiai gali suprasti skirtingose srityse dirbandys žmonės ir
pasitivrtinti duomenų modelį, kuris vėliau bus taikomas rengiant duomenų
struktūros aprašus.

\sphinxAtStartPar
Reikia atkreipti dėmesį, kad koncepcinis modelis yra vienas, o jį atitinkančių
duomenų šaltinių gali būti daug.

\sphinxAtStartPar
Kaip pavyzdį, naudosime žemiau pateiktą koncepcinį duomenų modelį:

\sphinxAtStartPar
Pagal šį koncepcinį modelį, DSA lentelė atrodytu taip:


\begin{savenotes}\sphinxattablestart
\sphinxthistablewithglobalstyle
\centering
\begin{tabulary}{\linewidth}[t]{TTTTTTT}
\sphinxtoprule
\sphinxstyletheadfamily 
\sphinxAtStartPar
d
&\sphinxstyletheadfamily 
\sphinxAtStartPar
r
&\sphinxstyletheadfamily 
\sphinxAtStartPar
m
&\sphinxstyletheadfamily 
\sphinxAtStartPar
property
&\sphinxstyletheadfamily 
\sphinxAtStartPar
type
&\sphinxstyletheadfamily 
\sphinxAtStartPar
ref
&\sphinxstyletheadfamily 
\sphinxAtStartPar
prepare
\\
\sphinxmidrule
\sphinxtableatstartofbodyhook\sphinxstartmulticolumn{4}%
\begin{varwidth}[t]{\sphinxcolwidth{4}{7}}
\sphinxAtStartPar
datasets/gov/example
\par
\vskip-\baselineskip\vbox{\hbox{\strut}}\end{varwidth}%
\sphinxstopmulticolumn
&&&\\
\sphinxhline
\sphinxAtStartPar

&&\sphinxstartmulticolumn{2}%
\begin{varwidth}[t]{\sphinxcolwidth{2}{7}}
\sphinxAtStartPar
\sphinxstylestrong{Administracija}
\par
\vskip-\baselineskip\vbox{\hbox{\strut}}\end{varwidth}%
\sphinxstopmulticolumn
&&
\sphinxAtStartPar
kodas
&\\
\sphinxhline
\sphinxAtStartPar

&&&
\sphinxAtStartPar
kodas
&
\sphinxAtStartPar
string
&&\\
\sphinxhline
\sphinxAtStartPar

&&&
\sphinxAtStartPar
pavadinimas
&
\sphinxAtStartPar
string
&&\\
\sphinxhline
\sphinxAtStartPar

&&&
\sphinxAtStartPar
tipas
&
\sphinxAtStartPar
string
&&\\
\sphinxhline
\sphinxAtStartPar

&&&&
\sphinxAtStartPar
enum
&&
\sphinxAtStartPar
"APSKRITIS"
\\
\sphinxhline
\sphinxAtStartPar

&&&&&&
\sphinxAtStartPar
"SAVIVALDYBE"
\\
\sphinxhline
\sphinxAtStartPar

&&\sphinxstartmulticolumn{2}%
\begin{varwidth}[t]{\sphinxcolwidth{2}{7}}
\sphinxAtStartPar
\sphinxstylestrong{Gyvenviete}
\par
\vskip-\baselineskip\vbox{\hbox{\strut}}\end{varwidth}%
\sphinxstopmulticolumn
&&
\sphinxAtStartPar
id
&\\
\sphinxhline
\sphinxAtStartPar

&&&
\sphinxAtStartPar
id
&
\sphinxAtStartPar
integer
&&\\
\sphinxhline
\sphinxAtStartPar

&&&
\sphinxAtStartPar
pavadinimas
&
\sphinxAtStartPar
string
&&\\
\sphinxhline
\sphinxAtStartPar

&&&
\sphinxAtStartPar
savivaldybe
&
\sphinxAtStartPar
ref
&
\sphinxAtStartPar
\sphinxstylestrong{Savivaldybe}
&\\
\sphinxhline
\sphinxAtStartPar

&&\sphinxstartmulticolumn{2}%
\begin{varwidth}[t]{\sphinxcolwidth{2}{7}}
\sphinxAtStartPar
\sphinxstylestrong{Apskritis}
\par
\vskip-\baselineskip\vbox{\hbox{\strut}}\end{varwidth}%
\sphinxstopmulticolumn
&
\sphinxAtStartPar
\sphinxstylestrong{Administracija}
&
\sphinxAtStartPar
kodas, tipas
&\\
\sphinxhline
\sphinxAtStartPar

&&&
\sphinxAtStartPar
kodas
&
\sphinxAtStartPar
string
&&\\
\sphinxhline
\sphinxAtStartPar

&&&
\sphinxAtStartPar
pavadinimas
&
\sphinxAtStartPar
string
&&\\
\sphinxhline
\sphinxAtStartPar

&&&
\sphinxAtStartPar
tipas
&
\sphinxAtStartPar
string
&&
\sphinxAtStartPar
"APSKRITIS"
\\
\sphinxhline
\sphinxAtStartPar

&&\sphinxstartmulticolumn{2}%
\begin{varwidth}[t]{\sphinxcolwidth{2}{7}}
\sphinxAtStartPar
\sphinxstylestrong{Savivaldybe}
\par
\vskip-\baselineskip\vbox{\hbox{\strut}}\end{varwidth}%
\sphinxstopmulticolumn
&
\sphinxAtStartPar
\sphinxstylestrong{Administracija}
&
\sphinxAtStartPar
kodas, tipas
&\\
\sphinxhline
\sphinxAtStartPar

&&&
\sphinxAtStartPar
kodas
&
\sphinxAtStartPar
string
&&\\
\sphinxhline
\sphinxAtStartPar

&&&
\sphinxAtStartPar
pavadinimas
&
\sphinxAtStartPar
string
&&\\
\sphinxhline
\sphinxAtStartPar

&&&
\sphinxAtStartPar
tipas
&
\sphinxAtStartPar
string
&&
\sphinxAtStartPar
"SAVIVALDYBE"
\\
\sphinxhline
\sphinxAtStartPar

&&&
\sphinxAtStartPar
apskritis
&
\sphinxAtStartPar
ref
&
\sphinxAtStartPar
\sphinxstylestrong{Apskritis}
&\\
\sphinxbottomrule
\end{tabulary}
\sphinxtableafterendhook\par
\sphinxattableend\end{savenotes}

\sphinxAtStartPar
Pavadinimai nurodyti koncepciniame modelyje, turi identiškai sutapti su
pavadinimai nurodytais DSA lentelės loginio modelio \sphinxcode{\sphinxupquote{model}}, \sphinxcode{\sphinxupquote{property}},
\sphinxcode{\sphinxupquote{type}}, \sphinxcode{\sphinxupquote{ref}} ir \sphinxcode{\sphinxupquote{prepare}} stulpeluose.

\sphinxAtStartPar
DSA lentelėje fizinio modelio, \sphinxcode{\sphinxupquote{source}} stulpelyje nurodyti pavadinimai
skirtinguose šaltiniuose gali skirtis, tačiau loginio modelio pavadinimai turi
išlikti tokie patys.


\subsection{Objektas}
\label{\detokenize{modelis:objektas}}\label{\detokenize{modelis:id1}}
\sphinxAtStartPar
\sphinxstyleemphasis{Objektas} yra viena duomenų eilutė, arba vienas duomenų įrašas ar
atvejis. Kalbant apie objektus, naudojamas \DUrole{underline}{pavyzdys} žymėjimas.

\sphinxAtStartPar
Pavyzdžiui iš aukščiau pateikto duomenų modelio, klasės \sphinxcode{\sphinxupquote{Gyvenviete}} objektai
gali būti:
\begin{itemize}
\item {} 
\sphinxAtStartPar
\DUrole{underline}{Vilnius}

\item {} 
\sphinxAtStartPar
\DUrole{underline}{Kaunas}

\item {} 
\sphinxAtStartPar
\DUrole{underline}{Klaipeda}

\end{itemize}

\sphinxAtStartPar
Sąvoka {\hyperref[\detokenize{savokos:term-objektas}]{\sphinxtermref{\DUrole{xref}{\DUrole{std}{\DUrole{std-term}{objektas}}}}}} kalba apie konkretų individualų atvejį ar pavyzdį.

\sphinxAtStartPar
Imant duomenų lentelę iš \sphinxcode{\sphinxupquote{Gyvenviete}} modelio, gausime tokius duomenis.


\begin{savenotes}\sphinxattablestart
\sphinxthistablewithglobalstyle
\centering
\begin{tabulary}{\linewidth}[t]{TTT}
\sphinxtoprule
\sphinxstyletheadfamily 
\sphinxAtStartPar
id
&\sphinxstyletheadfamily 
\sphinxAtStartPar
pavadinimas
&\sphinxstyletheadfamily 
\sphinxAtStartPar
savivaldybe
\\
\sphinxmidrule
\sphinxtableatstartofbodyhook
\sphinxAtStartPar
1
&
\sphinxAtStartPar
Vilnius
&
\sphinxAtStartPar
10
\\
\sphinxhline
\sphinxAtStartPar
2
&
\sphinxAtStartPar
Kaunas
&
\sphinxAtStartPar
11
\\
\sphinxhline
\sphinxAtStartPar
3
&
\sphinxAtStartPar
Klaipeda
&
\sphinxAtStartPar
12
\\
\sphinxbottomrule
\end{tabulary}
\sphinxtableafterendhook\par
\sphinxattableend\end{savenotes}

\sphinxAtStartPar
Šioje lentelėje yra trys objektai.

\sphinxAtStartPar
Objekto pavyzdys UML diagramoje:

\sphinxAtStartPar
UML diagramoje turime tris objetus \DUrole{underline}{Vilnius}, \DUrole{underline}{Kaunas} ir
\DUrole{underline}{Klaipeda}, priskirti klasei \sphinxcode{\sphinxupquote{Gyvenviete}}.

\sphinxAtStartPar
Skirtingi objektai gali būti klasifikuojami į klases arba esybes.


\subsection{Klasė}
\label{\detokenize{modelis:klase}}
\sphinxAtStartPar
Klasė arba Esybė yra vienodas savybes ir vienodą apibrėžimą turinčių objektų
aibė, kuriems suteikiamas tam tikras pavadinimas.

\sphinxAtStartPar
Tarkime \DUrole{underline}{Vilniaus}, \DUrole{underline}{Kauno} ir \DUrole{underline}{Klaipėdos}
objektus galime priskirti vienai klasei ir suteikti tai klasei pavadinimą
\sphinxcode{\sphinxupquote{Gyvenviete}}.

\sphinxAtStartPar
Klasės pavyzdys UML diagramoje:

\sphinxAtStartPar
Klasė gali neturėti jokių savybių, arba gali turėti savybes, kurios apibūdina
pačią klasę.

\sphinxAtStartPar
Tarkime modelis \sphinxcode{\sphinxupquote{Gyvenvietė}} turi savybę \sphinxcode{\sphinxupquote{pavadinimas}}, tačiau tai nėra klasės
savybė, todėl, kad \sphinxcode{\sphinxupquote{pavadinimas}} yra duomenų atributas, kuris nėra klasę
apibūdinanti savybė.

\sphinxAtStartPar
Nurodžius savybes prie klasės, iškeliamas griežtas reikalavimas, visiems
modeliams ir subklasėms, atitikti visas klasės savybes.

\sphinxAtStartPar
Tuo tarpu duomenų modelis, gali atitikti tam tikrą klasę, bet gali būti
pateikiamas su skirtingomis savybėmis.

\sphinxAtStartPar
Sudarant ontologijas, pateikiami klasių apibrėžimai, dažniausiai be savybių,
kad neriboti klasės taikymo. Tačiau tam tikrais atvejais, ontologijoje klasės
pateikiamos ir su keliomis savybėmis, kurios apibrėžia pačią klasę.


\subsection{Modelis}
\label{\detokenize{modelis:modelis}}\label{\detokenize{modelis:id2}}
\noindent\sphinxincludegraphics{{modelis}.png}

\sphinxAtStartPar
Klasės savybės apibrežia pačią klasę ir tampa klasės dalimi, tačiau modelio
savybės neturi įtakos klasės semantiniam apibrėžimui, tai yra tiesiog duomenų
laukų sąrašas pateikiams su klase.

\sphinxAtStartPar
Viena klasė gali turėti daug skirtingų modelių, su skirtingomis savybėmis arba
su skirtingais duomenų laukais.

\sphinxAtStartPar
Modelis, schema arba profilis yra konkretus savybių, duomenų tipų sąrašas,
kuriame nurodoma kurios savybės yra privalomos, kurios gali turėti daugiau nei
vieną reikšmę ir kitas detales.

\sphinxAtStartPar
Sudarant taikymo profilius (angl. \sphinxstyleemphasis{Application profile}) UML klasių diagramoje
pateikiami konkretūs duomenų modeliai, su konkrečiomis savybėmis ir jų tipais.

\sphinxAtStartPar
Modelio pavyzdys UML diagramoje:

\sphinxAtStartPar
Modelis atvaizduojas lygiai taip pat, kaip ir klasės. Ar tai yra klasės ar
modelis galima atskirti tik pagal diagramos pavadinimą, jei diagrama vaizduoja
ontologiją, tada joje yra klasės, jei taikymo profilį, tada diagramoje yra
modeliai.

\sphinxAtStartPar
Jei UML diagramose prie klasių yra pateikti pilni sąrašai savybių su tipais,
tada tai greičiausiai yra taikymo profilis.


\subsection{Apibendrinimas}
\label{\detokenize{modelis:apibendrinimas}}\label{\detokenize{modelis:uml-generalization}}
\sphinxAtStartPar
Objektai gali būti skirstomi į klases, tačiau pačios klasės gali būti
skirstomos į bendresnes klases, toks apibendrinimo procesas vadinamas
generalizacija.

\sphinxAtStartPar
UML diagramose gneralizacija žymima užpildyta rodykle, kurios krypts iš labiau
specializuotos siauresnę prasmę turinčios klasės, į labiau apibendrintą,
platesnę prasmę turinčią klasę, pavyzdžiui:

\sphinxAtStartPar
Šiame pavyzdyje nurodome, kad \sphinxcode{\sphinxupquote{Savivaldybe}} yra \sphinxcode{\sphinxupquote{Administracija}} poaibis. Arba
\sphinxcode{\sphinxupquote{Administracija}} yra platesnė klasė, o \sphinxcode{\sphinxupquote{Savivaldybę}} yra siauresnė, labiau
specifinę prasmę nurodanti klasė.


\subsection{Identifikatorius}
\label{\detokenize{modelis:identifikatorius}}
\sphinxAtStartPar
Kad galėtume vienareikšmiškai įvardinti ar nurodyti tam tikrą objektą, visi
objektai privalo turėti unikalius identifikatorius.

\sphinxAtStartPar
Kiekvienam objektui priskiriamas vienas globalus identifikatorius \index{RFC@\spxentry{RFC}!RFC 9562@\spxentry{RFC 9562}}\sphinxhref{https://datatracker.ietf.org/doc/html/rfc9562.html}{\sphinxstylestrong{UUID}} formatu, tačiau objektas gali turėti vieną ar daugiau lokalius
identifikatorius.

\sphinxAtStartPar
Globalūs identifikatoriai priskiriami esybei ir atspindi vieną realaus pasaulio
objektą, lokalus identifikatorius yra siejams su konkrečiu duomenų modeliu ar
duomenų šaltiniu ir skirtinguose modeliuose gali būti naudojami keli skirtingi
lokalūs identifikatoriai rodantys į vieną realaus pasaulio objektą.

\sphinxAtStartPar
UML diagramoje aukščiau turime du skirtingus duomenų objektus, kurie turi
vienodą globalų identifikatorių \sphinxcode{\sphinxupquote{dd79d2a6\sphinxhyphen{}d3d6\sphinxhyphen{}4fc2\sphinxhyphen{}83bb\sphinxhyphen{}da9dd15b2a89}}, tačiau
skirtingus lokalius \sphinxcode{\sphinxupquote{id = 7}} ir \sphinxcode{\sphinxupquote{kodas = 23}}.

\sphinxAtStartPar
Globalus identifikatorius suteikiamas esybei \sphinxcode{\sphinxupquote{Gyvenviete}}, lokalūs
identifikatoriai suteikiami konkrečiam duomenų modeliui ir konkrečiam duomenų
šaltiniui.

\sphinxAtStartPar
Rengiant {\hyperref[\detokenize{savokos:term-DSA}]{\sphinxtermref{\DUrole{xref}{\DUrole{std}{\DUrole{std-term}{DSA}}}}}} lentelę globalūs identifikatoriai žymimi {\hyperref[\detokenize{dimensijos:model.ref}]{\sphinxcrossref{\sphinxcode{\sphinxupquote{model.ref}}}}}
stulpelyje arba rezervuotu savybės pavadinimu \sphinxcode{\sphinxupquote{\_id}} ir yra privalomas.


\subsection{Savybė}
\label{\detokenize{modelis:savybe}}
\sphinxAtStartPar
UML diagramos savybės žymimos sutartine forma:

\begin{sphinxadmonition}{note}{Sintaksė}

\sphinxAtStartPar
\sphinxstylestrong{access} \sphinxstylestrong{property} \sphinxcode{\sphinxupquote{:}} \sphinxstylestrong{type} \sphinxcode{\sphinxupquote{{[}}} \sphinxstylestrong{cardinality} \sphinxcode{\sphinxupquote{..}} \sphinxstylestrong{multiplicity} \sphinxcode{\sphinxupquote{{]}}}
\end{sphinxadmonition}
\begin{description}
\sphinxlineitem{access}
\sphinxAtStartPar
Prieigos lygis. Gali būti naudojami tokie žymėjimai:
\begin{itemize}
\item {} 
\sphinxAtStartPar
\sphinxcode{\sphinxupquote{+}} \sphinxhyphen{} atviri duomenys, žiūrėti {\hyperref[\detokenize{prieiga:open}]{\sphinxcrossref{\sphinxcode{\sphinxupquote{open}}}}}.

\item {} 
\sphinxAtStartPar
\sphinxcode{\sphinxupquote{\#}} \sphinxhyphen{} vieši duomenys, žiūrįti {\hyperref[\detokenize{prieiga:public}]{\sphinxcrossref{\sphinxcode{\sphinxupquote{public}}}}}.

\item {} 
\sphinxAtStartPar
\sphinxcode{\sphinxupquote{\textasciitilde{}}} \sphinxhyphen{} duomenys teikiami pagal sutartį, žiūrėti {\hyperref[\detokenize{prieiga:protected}]{\sphinxcrossref{\sphinxcode{\sphinxupquote{protected}}}}}.

\item {} 
\sphinxAtStartPar
\sphinxcode{\sphinxupquote{\sphinxhyphen{}}} \sphinxhyphen{} nepublikuojami duomenys, žiūrėti {\hyperref[\detokenize{prieiga:private}]{\sphinxcrossref{\sphinxcode{\sphinxupquote{private}}}}}.

\end{itemize}

\sphinxlineitem{property}
\sphinxAtStartPar
Savybė, žiūrėti {\hyperref[\detokenize{formatas:property}]{\sphinxcrossref{\sphinxcode{\sphinxupquote{property}}}}}. Nurodoma savybės URI forma.

\sphinxlineitem{type}
\sphinxAtStartPar
Duomenų tipas, žiūrėti {\hyperref[\detokenize{tipai:duomenu-tipai}]{\sphinxcrossref{\DUrole{std}{\DUrole{std-ref}{Duomenų tipai}}}}}. UML diagramose, jei duomenų
tipas yra {\hyperref[\detokenize{formatas:ref}]{\sphinxcrossref{\sphinxcode{\sphinxupquote{ref}}}}} arba \sphinxcode{\sphinxupquote{backref}}, tada nurodomas modelio
pavadinimas, URI forma, su kuriuo daroma asociacija.

\sphinxlineitem{cardinality}
\sphinxAtStartPar
Nurodo ar laukas yra privalomas:
\begin{itemize}
\item {} 
\sphinxAtStartPar
\sphinxcode{\sphinxupquote{0}} \sphinxhyphen{} laukas yra neprivalomas.

\item {} 
\sphinxAtStartPar
\sphinxcode{\sphinxupquote{1}} \sphinxhyphen{} laukas yra privalomas.

\end{itemize}

\sphinxlineitem{multiplicity}
\sphinxAtStartPar
Nurodo kiek kartų gali būti pateikta lauko reikšmė.
\begin{itemize}
\item {} 
\sphinxAtStartPar
\sphinxcode{\sphinxupquote{1}} \sphinxhyphen{} lauko reikšmė gali būti pateikta tik vieną kartą.

\item {} 
\sphinxAtStartPar
\sphinxcode{\sphinxupquote{*}} \sphinxhyphen{} laukė reikšmė gali būti pateikta daugiau nei veiną kartą.

\end{itemize}

\end{description}

\sphinxAtStartPar
Pavyzdys:

\sphinxAtStartPar
UML diagramoje matote \sphinxcode{\sphinxupquote{Gyvenviete}} duomenų modelį, kuris turi dvi savybes:

\begin{sphinxVerbatim}[commandchars=\\\{\}]
\PYG{o}{+} \PYG{n+nb}{id}\PYG{p}{:} \PYG{n}{integer} \PYG{p}{[}\PYG{l+m+mf}{1.}\PYG{l+m+mf}{.1}\PYG{p}{]}
\PYG{o}{+} \PYG{n}{pavadinimas}\PYG{p}{:} \PYG{n}{text} \PYG{p}{[}\PYG{l+m+mf}{1.}\PYG{l+m+mf}{.1}\PYG{p}{]}
\end{sphinxVerbatim}

\sphinxAtStartPar
Abi savybės turi atvirą prieigos lygmenį, \sphinxcode{\sphinxupquote{id}} ir \sphinxcode{\sphinxupquote{pavadinimas}} kodinius
savybės pavadinimus, \sphinxcode{\sphinxupquote{integer}} ir \sphinxcode{\sphinxupquote{text}} duomenų tipus ir abi savybės yra
privalomos ir gali turėti tik vieną reikšmę.


\subsection{Asociacija}
\label{\detokenize{modelis:asociacija}}

\subsubsection{Per duomenų tipą}
\label{\detokenize{modelis:per-duomenu-tipa}}
\sphinxAtStartPar
UML diagramose nurodant ryšį su kitomis esybėmis, galima naudoti įprastą
savybių žymėjimo formą \sphinxcode{\sphinxupquote{+ savivaldybe: Savivaldybe {[}1..1{]}}}, kur po \sphinxcode{\sphinxupquote{:}}
dvitaškio nurodomas kitas modelis, su kuriuo daroma asociacija.

\sphinxAtStartPar
Tokia asociacija daroma, kai siejame su išoriniais modeliais, arba kai turime
per daug asociacijų ir norime UML diagramoje sumažinti rodyklių skaičių.


\subsubsection{Tiesioginė}
\label{\detokenize{modelis:tiesiogine}}
\sphinxAtStartPar
Tiesioginė asociacija nurodoma rodyklės pagalba, jei yra pateikta rodyklė, tada
savybių sąraše, savybės, kuri yra pateikta prie rodyklės neberodome.

\sphinxAtStartPar
Rodyklės kryptis visada rodo iš modelio, prie kurio savybė yra apibrėžta, į
kitą modelį, su kuriuo savybė yra siejama.

\sphinxAtStartPar
Tiesioginė asociacija {\hyperref[\detokenize{savokos:term-DSA}]{\sphinxtermref{\DUrole{xref}{\DUrole{std}{\DUrole{std-term}{DSA}}}}}} yra nurodoma {\hyperref[\detokenize{tipai:type.ref}]{\sphinxcrossref{\sphinxcode{\sphinxupquote{type.ref}}}}} pagalba.


\subsubsection{Atvirkštinė}
\label{\detokenize{modelis:atvirkstine}}
\sphinxAtStartPar
Asociacijai gali būti naudojami ir atvirkštiniai ryšiai, pavyzdžiui:

\sphinxAtStartPar
Šiuo atveju nurodome {\hyperref[\detokenize{tipai:type.backref}]{\sphinxcrossref{\sphinxcode{\sphinxupquote{type.backref}}}}} tipo atvirkštinę asociaciją, rodyklės
kryptis ir daugiareikšmiškumas keičiasi, turime vieną savyvaldybę, kuri gali
turėti daug gyvenviečių.


\subsection{Klasifikatorius}
\label{\detokenize{modelis:klasifikatorius}}
\sphinxAtStartPar
Klasifikatoriai arba kontroliuojami žodynai, yra galimų reikšmių sąrašas
naudojamas tam tikrai savybei.

\sphinxAtStartPar
UML diagramoje klasifikatoriai pateikiami naudojant \sphinxcode{\sphinxupquote{<<enumeration>>}}
stereotipą ir punktyrinę priklausomybės rodyklę:

\sphinxAtStartPar
\sphinxcode{\sphinxupquote{AdministracijosTipas}} yra klasifikatorius, turintis kontroliuojamą žodyną,
kuriame apibrėžtos dvi galimos reikšmės \sphinxcode{\sphinxupquote{APSKRITIS}} ir \sphinxcode{\sphinxupquote{SAVIVALDYBE}}.

\sphinxAtStartPar
Struktūros apraše klasifikatoriai aprašomi naudojant {\hyperref[\detokenize{dimensijos:module-enum}]{\sphinxcrossref{\sphinxcode{\sphinxupquote{enum}}}}} dimensiją.


\subsection{Žodynas}
\label{\detokenize{modelis:zodynas}}
\sphinxAtStartPar
Visos klasės ir savybės (\sphinxstyleemphasis{sąvokos}) yra skirstomos į žodynus. Dažnai
viename duomenų modelyje yra naudojamos \DUrole{xref}{\DUrole{std}{\DUrole{std-term}{sąvokos}}} iš skirtingų
žodynų.

\sphinxAtStartPar
Kad atskirti, kuri sąvoka yra iš kokio žodyno, naudojami žodyno prefiksai.

\sphinxAtStartPar
Žodyno prefiksai gali būti naudojami tiek klasės pavadinime, tie savybių ir
tipų pavadinimuose.

\sphinxAtStartPar
Jei žodyno prefiksas nėra nurodytas, tai reiškia, kad naudojamas esamas
žodynas, kuris yra apibrėžtas duomenų modelyje.

\sphinxAtStartPar
Žodynai taip pat gali būti nurodomi naudojant UML paketus arba vardų erdves:

\sphinxAtStartPar
Sąvokoms, kurios yra vardų erdvės rėmuose, žodyno prefiksai nenurodomi. Žodyno
prefiksai nurodomi tik tuo atveju, jei sąvoka yra iš kito žodyno.


\subsection{IRI}
\label{\detokenize{modelis:iri}}
\sphinxAtStartPar
Visos sąvokos, tokios kaip klasės, savybės, duomenų tipai, taip pat yra
objektai, turintys savo identifikatorius.

\sphinxAtStartPar
UML diagramose nurodomi būtent sąvokų identifikatoriai sutrumpinta IRI forma.

\sphinxAtStartPar
IRI yra identifikatorius schema sudaryta iš sekančių komponentų:

\sphinxAtStartPar
\sphinxstylestrong{scheme} \sphinxcode{\sphinxupquote{://}} \sphinxstylestrong{host} \sphinxcode{\sphinxupquote{/}} \sphinxstylestrong{path} \sphinxcode{\sphinxupquote{?}} \sphinxstylestrong{query} \sphinxcode{\sphinxupquote{\#}} \sphinxstylestrong{fragment}

\sphinxAtStartPar
Lietuvos viešąjame sektoriuje naudojama sekanti URI schema:

\sphinxAtStartPar
\sphinxcode{\sphinxupquote{https://data.gov.lt/id/}} \sphinxstylestrong{vocab} \sphinxcode{\sphinxupquote{/}} \sphinxstylestrong{term} {[} \sphinxcode{\sphinxupquote{/}} \sphinxstylestrong{id} {]}
\begin{description}
\sphinxlineitem{vocab}
\sphinxAtStartPar
Žodyno kodinis pavadinimas.

\sphinxlineitem{term}
\sphinxAtStartPar
Sąvokos kodinis pavadinimas.

\sphinxlineitem{id}
\sphinxAtStartPar
Objekto identifikatorius.

\end{description}

\sphinxAtStartPar
Jei mūsų kuriamam žodynui būtų suteiktas kodinis pavadinimas \sphinxcode{\sphinxupquote{adresai}}, tada
mūsų sąvokoms būtų suteikti tokie IRI identifikatoriai:

\begin{sphinxVerbatim}[commandchars=\\\{\}]
\PYG{n}{https}\PYG{p}{:}\PYG{o}{/}\PYG{o}{/}\PYG{n}{data}\PYG{o}{.}\PYG{n}{gov}\PYG{o}{.}\PYG{n}{lt}\PYG{o}{/}\PYG{n+nb}{id}\PYG{o}{/}\PYG{n}{adresai}\PYG{o}{/}\PYG{n}{Gyvenviete}
\PYG{n}{https}\PYG{p}{:}\PYG{o}{/}\PYG{o}{/}\PYG{n}{data}\PYG{o}{.}\PYG{n}{gov}\PYG{o}{.}\PYG{n}{lt}\PYG{o}{/}\PYG{n+nb}{id}\PYG{o}{/}\PYG{n}{adresai}\PYG{o}{/}\PYG{n+nb}{id}
\PYG{n}{https}\PYG{p}{:}\PYG{o}{/}\PYG{o}{/}\PYG{n}{data}\PYG{o}{.}\PYG{n}{gov}\PYG{o}{.}\PYG{n}{lt}\PYG{o}{/}\PYG{n+nb}{id}\PYG{o}{/}\PYG{n}{adresai}\PYG{o}{/}\PYG{n}{pavadinimas}
\end{sphinxVerbatim}

\sphinxAtStartPar
Kadangi pilnas IRI yra gan ilgas, UML diagramose naudojame sutrumpintą IRI
formą su prefiksu. Šiuo atveju, galime deklaruoti, kad \sphinxcode{\sphinxupquote{ar}} prefiksas atitinka
\sphinxcode{\sphinxupquote{https://data.gov.lt/id/adresai/}} URI, todėl sutrumpinta forma atrodys taip:

\begin{sphinxVerbatim}[commandchars=\\\{\}]
\PYG{k}{@prefix}\PYG{+w}{ }\PYG{n+nn}{ar:}\PYG{+w}{ }\PYG{n+nv}{\PYGZlt{}https://data.gov.lt/id/adresai/\PYGZgt{}}

\PYG{n+nn}{ar}\PYG{p}{:}\PYG{n+nt}{Gyvenviete}
\PYG{n+nn}{ar}\PYG{p}{:}\PYG{n+nt}{id}
\PYG{n+nn}{ar}\PYG{p}{:}\PYG{n+nt}{pavadinimas}
\end{sphinxVerbatim}

\sphinxAtStartPar
UML diagramoje naudojame sutrumpintus URI pavadinimus, tačiau kartu su diagrama
būtina pateikti ir prefiksų sąrašą, kad būtų ašku, ką reiškia kiekvienas
prefikas:


\begin{savenotes}\sphinxattablestart
\sphinxthistablewithglobalstyle
\centering
\begin{tabulary}{\linewidth}[t]{TT}
\sphinxtoprule
\sphinxstyletheadfamily 
\sphinxAtStartPar
Prefiksas
&\sphinxstyletheadfamily 
\sphinxAtStartPar
Vardų erdvės IRI
\\
\sphinxmidrule
\sphinxtableatstartofbodyhook
\sphinxAtStartPar
ar
&
\sphinxAtStartPar
https://data.gov.lt/id/adresai/
\\
\sphinxhline
\sphinxAtStartPar
locn
&
\sphinxAtStartPar
http://www.w3.org/ns/locn\#
\\
\sphinxhline
\sphinxAtStartPar
dct
&
\sphinxAtStartPar
http://purl.org/dc/terms/
\\
\sphinxhline
\sphinxAtStartPar
rdfs
&
\sphinxAtStartPar
http://www.w3.org/2000/01/rdf\sphinxhyphen{}schema\#
\\
\sphinxbottomrule
\end{tabulary}
\sphinxtableafterendhook\par
\sphinxattableend\end{savenotes}

\sphinxstepscope


\section{Lentelės formatas}
\label{\detokenize{formatas:lenteles-formatas}}\label{\detokenize{formatas::doc}}
\sphinxAtStartPar
{\hyperref[\detokenize{savokos:term-DSA}]{\sphinxtermref{\DUrole{xref}{\DUrole{std}{\DUrole{std-term}{DSA}}}}}} yra sudarytas taip, kad būtų patogu dirbti tiek žmonėms, tiek
programoms. Žmonės su {\hyperref[\detokenize{savokos:term-DSA}]{\sphinxtermref{\DUrole{xref}{\DUrole{std}{\DUrole{std-term}{DSA}}}}}} lentele gali dirbti naudojantis, bet kuria
skaičiuoklės programa (Excel, LibreOffice Calc) ar kitas pasirinktas priemones.
Kadangi {\hyperref[\detokenize{savokos:term-DSA}]{\sphinxtermref{\DUrole{xref}{\DUrole{std}{\DUrole{std-term}{DSA}}}}}} turi aiškią ir griežtą struktūrą, lentelėje pateiktus
duomenis taip pat gali lengvai nuskaityti ir interpretuoti kompiuterinės
programos.

\sphinxAtStartPar
Tais atvejais, kai su {\hyperref[\detokenize{savokos:term-DSA}]{\sphinxtermref{\DUrole{xref}{\DUrole{std}{\DUrole{std-term}{DSA}}}}}} lentele dirba žmonės, lentelė gali būti
saugoma įstaigos pasirinktos skaičiuoklės programos ar kitų priemonių formatu.

\sphinxAtStartPar
Automatizuotoms priemonėms {\hyperref[\detokenize{savokos:term-DSA}]{\sphinxtermref{\DUrole{xref}{\DUrole{std}{\DUrole{std-term}{DSA}}}}}} turi būti teikiamas CSV formatu laikantis
\index{RFC@\spxentry{RFC}!RFC 4180@\spxentry{RFC 4180}}\sphinxhref{https://datatracker.ietf.org/doc/html/rfc4180.html}{\sphinxstylestrong{RFC 4180}} taisyklių, failo koduotė turi būti UTF\sphinxhyphen{}8.

\sphinxAtStartPar
DSA lentelė gali būti importuojama į \sphinxhref{https://data.gov.lt/}{Duomenų katalogą}, kuriame DSA lentelės
turinys gali būti tvarkomas naudojantis grafine naudotojo sąsaja.

\sphinxAtStartPar
Rengiant duomenų struktūros aprašus darbas vyksta su viena lentele. Lentelės
stuleliai sudaryti iš dimensijų ir metaduomenų.

\noindent\sphinxincludegraphics{{struktura}.png}

\sphinxAtStartPar
Lentelė sudaryta hierarchiniu principu. Kiekvienas metaduomenų stulpelis gali
turėti skirtingą prasmę, priklausomai nuo dimensijos. Todėl toliau
dokumentacijoje konkrečios dimensijos stulpelis yra žymimas nurodant tiek
dimensijos, tiek metaduomenų pavadinimus, pavyzdžiui {\hyperref[\detokenize{dimensijos:property.type}]{\sphinxcrossref{\sphinxcode{\sphinxupquote{property.type}}}}},
kuris nurodo {\hyperref[\detokenize{tipai:module-type}]{\sphinxcrossref{\sphinxcode{\sphinxupquote{type}}}}} metaduomenų stulpelį, esantį {\hyperref[\detokenize{formatas:property}]{\sphinxcrossref{\sphinxcode{\sphinxupquote{property}}}}}
dimensijoje.


\subsection{Dimensijos}
\label{\detokenize{formatas:dimensijos}}\label{\detokenize{formatas:dimensijos-stulpeliai}}
\sphinxAtStartPar
Duomenų struktūros aprašo lentelė sudaryta hierarchiniu principu. Kiekvienos
lentelės eilutės prasmę apibrėžia {\hyperref[\detokenize{dimensijos:dimensijos}]{\sphinxcrossref{\DUrole{std}{\DUrole{std-ref}{Dimensijos}}}}} stulpelis.

\sphinxAtStartPar
Kiekvienoje eilutėje gali būti užpildytas tik vienas dimensijos stulpelis.

\sphinxAtStartPar
Be šių penkių dimensijų, yra kelios {\hyperref[\detokenize{dimensijos:papildomos-dimensijos}]{\sphinxcrossref{\DUrole{std}{\DUrole{std-ref}{papildomos dimensijos}}}}}, jos nurodomos {\hyperref[\detokenize{tipai:module-type}]{\sphinxcrossref{\sphinxcode{\sphinxupquote{type}}}}} stulpelyje, neužpildžius
nei vieno dimensijos stulpelio.
\index{dataset (įtaisytasis kintamasis)@\spxentry{dataset}\spxextra{įtaisytasis kintamasis}}

\begin{fulllineitems}
\phantomsection\label{\detokenize{formatas:dataset}}
\pysigstartsignatures
\pysigline
{\sphinxbfcode{\sphinxupquote{dataset}}}
\pysigstopsignatures
\sphinxAtStartPar
\sphinxstylestrong{Duomenų rinkinys}

\sphinxAtStartPar
Kodinis duomenų rinkinio pavadinimas. Naudojant duomenų rinkinio kodinį
pavadinimą formuojamas API.

\sphinxAtStartPar
Duomenų rinkinio kodinis pavadinimas užrašomas pagal tokį šabloną:

\sphinxAtStartPar
\sphinxcode{\sphinxupquote{datasets/}} \sphinxstylestrong{forma} \sphinxcode{\sphinxupquote{/}} \sphinxstylestrong{organizacija} \sphinxcode{\sphinxupquote{/}} \sphinxstylestrong{katalogas} \sphinxcode{\sphinxupquote{/}} \sphinxstylestrong{rinkinys}

\sphinxAtStartPar
Visi duomenų rinkinio pavadinimo komponenta užrašomi mažosiomis raidėmis,
jei reikia žodžiai atskiriami \sphinxcode{\sphinxupquote{\_}} simbolio pagalba.
\begin{description}
\sphinxlineitem{forma}
\sphinxAtStartPar
Nurodo organizacijos teisinę formą, galimi variantai:

\begin{DUlineblock}{0em}
\item[] \sphinxstylestrong{gov} \sphinxhyphen{} Viešasis sektorius.
\item[] \sphinxstylestrong{com} \sphinxhyphen{} Privatusis sektorius.
\end{DUlineblock}

\sphinxlineitem{organizacija}
\sphinxAtStartPar
Organizacijos pavadinimo trumpinys. Viena organizacija gali turėti
vieną trumpinį, kuris yra registruojamas {\hyperref[\detokenize{savokos:term-duomenu-katalogas}]{\sphinxtermref{\DUrole{xref}{\DUrole{std}{\DUrole{std-term}{duomenų kataloge}}}}}}.

\sphinxlineitem{katalogas}
\sphinxAtStartPar
Organizacijos informacinės sistemos trumpinys.

\sphinxlineitem{rinkinys}
\sphinxAtStartPar
Informacinės sistemos teikiamas duomenų rinkinys.

\end{description}

\sphinxAtStartPar
Visi pavadinimai užrašomi mažosiomis lotyniškomis raidėmis, žodžiams
atskirti gali būti naudojamas \sphinxcode{\sphinxupquote{\_}} simbolis.

\sphinxAtStartPar
Pagal semantinę prasmę atitinka \sphinxhref{https://www.w3.org/TR/vocab-dcat-2/\#Class:Resource}{dcat:Resource}.

\begin{sphinxadmonition}{note}{Pavyzdys}

\begin{DUlineblock}{0em}
\item[] \sphinxcode{\sphinxupquote{datasets/gov/rc/jar/ws}}
\item[] \sphinxcode{\sphinxupquote{datasets/gov/ivkp/adp/adk}}
\end{DUlineblock}
\end{sphinxadmonition}


\begin{sphinxseealso}{Taip pat žiūrėkite:}

\begin{DUlineblock}{0em}
\item[] {\hyperref[\detokenize{dimensijos:dataset}]{\sphinxcrossref{\DUrole{std}{\DUrole{std-ref}{dataset}}}}}
\item[] {\hyperref[\detokenize{pavadinimai:kodiniai-pavadinimai}]{\sphinxcrossref{\DUrole{std}{\DUrole{std-ref}{Kodiniai pavadinimai}}}}}
\end{DUlineblock}


\end{sphinxseealso}


\end{fulllineitems}

\index{resource (įtaisytasis kintamasis)@\spxentry{resource}\spxextra{įtaisytasis kintamasis}}

\begin{fulllineitems}
\phantomsection\label{\detokenize{formatas:resource}}
\pysigstartsignatures
\pysigline
{\sphinxbfcode{\sphinxupquote{resource}}}
\pysigstopsignatures
\sphinxAtStartPar
\sphinxstylestrong{Duomenų šaltinis}

\sphinxAtStartPar
Kodinis duomenų šaltinio pavadinimas, užrašomas mažosiomis lotyniškomis
raidėmis, žodžiai skiriami \sphinxcode{\sphinxupquote{\_}} simboliu.

\sphinxAtStartPar
Duomenų šaltinis yra duomenų failas, duomenų bazė ar API, per kurį teikiami
duomenys.

\sphinxAtStartPar
Pagal semantinę prasmę atitinka \sphinxhref{https://www.w3.org/TR/vocab-dcat-2/\#Class:Distribution}{dcat:Distribution} arba \sphinxhref{https://rml.io/specs/rml/\#logical-source}{rml:logicalSource}.

\begin{sphinxadmonition}{note}{Pavyzdys}

\begin{DUlineblock}{0em}
\item[] \sphinxcode{\sphinxupquote{resource1}}
\item[] \sphinxcode{\sphinxupquote{db1}}
\end{DUlineblock}
\end{sphinxadmonition}


\begin{sphinxseealso}{Taip pat žiūrėkite:}

\begin{DUlineblock}{0em}
\item[] {\hyperref[\detokenize{dimensijos:resource}]{\sphinxcrossref{\DUrole{std}{\DUrole{std-ref}{resource}}}}}
\item[] {\hyperref[\detokenize{saltiniai:duomenu-saltiniai}]{\sphinxcrossref{\DUrole{std}{\DUrole{std-ref}{Duomenų šaltiniai}}}}}
\end{DUlineblock}


\end{sphinxseealso}


\end{fulllineitems}

\index{base (įtaisytasis kintamasis)@\spxentry{base}\spxextra{įtaisytasis kintamasis}}

\begin{fulllineitems}
\phantomsection\label{\detokenize{formatas:base}}
\pysigstartsignatures
\pysigline
{\sphinxbfcode{\sphinxupquote{base}}}
\pysigstopsignatures
\sphinxAtStartPar
\sphinxstylestrong{Modelio bazė}

\sphinxAtStartPar
Kodinis bazinio modelio pavadinimas. Atitinka \sphinxhref{https://www.w3.org/TR/rdf-schema/\#ch\_subclassof}{rdfs:subClassOf} prasmę
({\hyperref[\detokenize{formatas:model}]{\sphinxcrossref{\sphinxcode{\sphinxupquote{model}}}}} \sphinxcode{\sphinxupquote{rdfs:subClassOf}} {\hyperref[\detokenize{formatas:base}]{\sphinxcrossref{\sphinxcode{\sphinxupquote{base}}}}}).

\sphinxAtStartPar
Šiame stulpelyje įrašomas kito {\hyperref[\detokenize{formatas:model}]{\sphinxcrossref{\sphinxcode{\sphinxupquote{model}}}}} stulpelyje įrašyto modelio
kodinis pavadinimas.

\sphinxAtStartPar
Galima nurodyti absoliutų modelio pavadinimą, kuris prasideda \sphinxcode{\sphinxupquote{/}} simboliu,
taikoma, kai nurodomas bazinis modelis iš kito duomenų rinkinio,
pavyzdžiui:

\begin{sphinxadmonition}{note}{Pavyzdys}

\sphinxAtStartPar
\sphinxcode{\sphinxupquote{/datasets/gov/example/Country}}
\end{sphinxadmonition}

\sphinxAtStartPar
Arba galima nurodyti reliatyvų pavadinimą, kuris neprasideda \sphinxcode{\sphinxupquote{/}} simboliu,
taikoma, kai bazinis modelis yra tame pačiame duomenų rinkinyje, pavyzdžiui:

\begin{sphinxadmonition}{note}{Pavyzdys}

\sphinxAtStartPar
\sphinxcode{\sphinxupquote{Country}}
\end{sphinxadmonition}

\sphinxAtStartPar
Jei \sphinxcode{\sphinxupquote{base}} stulpelis neužpildytas, tada visi modeliai, neturintys \sphinxcode{\sphinxupquote{base}}
yra laikomi baziniais modeliais.


\begin{sphinxseealso}{Taip pat žiūrėkite:}

\begin{DUlineblock}{0em}
\item[] {\hyperref[\detokenize{dimensijos:base}]{\sphinxcrossref{\DUrole{std}{\DUrole{std-ref}{Baziniai modeliai}}}}}
\item[] {\hyperref[\detokenize{apibendrinimas:generalization}]{\sphinxcrossref{\DUrole{std}{\DUrole{std-ref}{Apibendrinimas}}}}}
\end{DUlineblock}


\end{sphinxseealso}


\end{fulllineitems}

\index{model (įtaisytasis kintamasis)@\spxentry{model}\spxextra{įtaisytasis kintamasis}}

\begin{fulllineitems}
\phantomsection\label{\detokenize{formatas:model}}
\pysigstartsignatures
\pysigline
{\sphinxbfcode{\sphinxupquote{model}}}
\pysigstopsignatures
\sphinxAtStartPar
\sphinxstylestrong{Modelis (lentelė)}

\sphinxAtStartPar
Kodinis modelio pavadinimas, užrašomas lotyniškomis raidėmis, kiekvieno
žodžio pirma raidė didžioji, kitos mažosios, žodžiai atskiriami didžiąja
raide.

\sphinxAtStartPar
Pagal semantinę prasmę atitinka \sphinxhref{https://www.w3.org/TR/rdf-schema/\#ch\_class}{rdfs:Class} arba \sphinxhref{https://www.w3.org/TR/r2rml/\#subject-map}{r2rml:SubjectMap}.

\begin{sphinxadmonition}{note}{Pavyzdys}

\begin{DUlineblock}{0em}
\item[] \sphinxcode{\sphinxupquote{Gyvenviete}}
\item[] \sphinxcode{\sphinxupquote{AdministracijosTipas}}
\end{DUlineblock}
\end{sphinxadmonition}


\begin{sphinxseealso}{Taip pat žiūrėkite:}

\begin{DUlineblock}{0em}
\item[] {\hyperref[\detokenize{dimensijos:model}]{\sphinxcrossref{\DUrole{std}{\DUrole{std-ref}{model}}}}}
\item[] {\hyperref[\detokenize{modelis:modelis}]{\sphinxcrossref{\DUrole{std}{\DUrole{std-ref}{Modelis}}}}}
\end{DUlineblock}


\end{sphinxseealso}


\end{fulllineitems}

\index{property (įtaisytasis kintamasis)@\spxentry{property}\spxextra{įtaisytasis kintamasis}}

\begin{fulllineitems}
\phantomsection\label{\detokenize{formatas:property}}
\pysigstartsignatures
\pysigline
{\sphinxbfcode{\sphinxupquote{property}}}
\pysigstopsignatures
\sphinxAtStartPar
\sphinxstylestrong{Savybė (stulpelis)}

\sphinxAtStartPar
Kodinis savybės pavadinimas, užrašomas mažosiomis lotyniškomis raidėmis,
žodžiai atskiriami \sphinxcode{\sphinxupquote{\_}} simoboliu.

\sphinxAtStartPar
Savybių pavadinimai prasidedantys \sphinxcode{\sphinxupquote{\_}} simboliu yra rezervuoti ir turi
apibrėžtą prasmę.

\sphinxAtStartPar
Savybės pavadinime gali būti naudojami tokie specialūs simboliai:
\begin{description}
\sphinxlineitem{.}
\sphinxAtStartPar
(taško simbolis) nurodo objektų kompoziciją. Naudojamas su
{\hyperref[\detokenize{tipai:type.ref}]{\sphinxcrossref{\sphinxcode{\sphinxupquote{ref}}}}} ir {\hyperref[\detokenize{tipai:type.object}]{\sphinxcrossref{\sphinxcode{\sphinxupquote{object}}}}} duomenų tipais.

\begin{sphinxadmonition}{note}{Pavyzdys}

\begin{DUlineblock}{0em}
\item[] \sphinxcode{\sphinxupquote{adresas.gatve}}
\end{DUlineblock}
\end{sphinxadmonition}

\sphinxlineitem{{[}{]}}
\sphinxAtStartPar
Duomenų masyvas arba sąrašas, gali būti naudojamas su visais tipais.

\begin{sphinxadmonition}{note}{Pavyzdys}

\begin{DUlineblock}{0em}
\item[] \sphinxcode{\sphinxupquote{miestai{[}{]}}}
\end{DUlineblock}
\end{sphinxadmonition}

\end{description}
\begin{description}
\sphinxlineitem{@}
\sphinxAtStartPar
Kalbos žymė, naudojama su {\hyperref[\detokenize{tipai:type.text}]{\sphinxcrossref{\sphinxcode{\sphinxupquote{text}}}}} tipu.

\begin{sphinxadmonition}{note}{Pavyzdys}

\begin{DUlineblock}{0em}
\item[] \sphinxcode{\sphinxupquote{pavadinimas@lt}}
\item[] \sphinxcode{\sphinxupquote{pavadinimas@en}}
\end{DUlineblock}
\end{sphinxadmonition}

\end{description}

\sphinxAtStartPar
Pagal semantinę prasmę atitinka \sphinxhref{https://www.w3.org/TR/rdf-schema/\#ch\_property}{rdfs:Property},
\sphinxhref{https://www.w3.org/TR/r2rml/\#predicate-object-map}{r2rml:PredicateObjectMap}.


\begin{sphinxseealso}{Taip pat žiūrėkite:}

\begin{DUlineblock}{0em}
\item[] {\hyperref[\detokenize{dimensijos:property}]{\sphinxcrossref{\DUrole{std}{\DUrole{std-ref}{property}}}}}
\end{DUlineblock}


\end{sphinxseealso}


\end{fulllineitems}



\subsection{Metaduomenys}
\label{\detokenize{formatas:metaduomenys}}\label{\detokenize{formatas:metaduomenu-stulpeliai}}
\sphinxAtStartPar
Kaip ir minėta aukščiau, kiekvienos metaduomenų eilutės prasmė priklauso nuo
{\hyperref[\detokenize{dimensijos:dimensijos}]{\sphinxcrossref{\DUrole{std}{\DUrole{std-ref}{Dimensijos}}}}}. Todėl, toliau dokumentacijoje, kalbant apie tam tikros
dimensijos stulpelį, stulpelis bus įvardinamas pridedant dimensijos
pavadinimą, pavyzdžiui {\hyperref[\detokenize{dimensijos:model.ref}]{\sphinxcrossref{\sphinxcode{\sphinxupquote{model.ref}}}}}, kas reikštų, kad kalbama apie
{\hyperref[\detokenize{formatas:ref}]{\sphinxcrossref{\sphinxcode{\sphinxupquote{ref}}}}} stulpelį, {\hyperref[\detokenize{formatas:model}]{\sphinxcrossref{\sphinxcode{\sphinxupquote{model}}}}} dimensijoje.
\index{id (įtaisytasis kintamasis)@\spxentry{id}\spxextra{įtaisytasis kintamasis}}

\begin{fulllineitems}
\phantomsection\label{\detokenize{formatas:id}}
\pysigstartsignatures
\pysigline
{\sphinxbfcode{\sphinxupquote{id}}}
\pysigstopsignatures
\sphinxAtStartPar
\sphinxstylestrong{Eilutės identifikatorius}

\sphinxAtStartPar
Unikalus elemento numeris, gali būti sveikas, monotoniškai didėjantis
skaičius arba \index{RFC@\spxentry{RFC}!RFC 9562@\spxentry{RFC 9562}}\sphinxhref{https://datatracker.ietf.org/doc/html/rfc9562.html}{\sphinxstylestrong{UUID}}. Svarbu užtikrinti, kad visi elementai
turėtu unikalų id.

\sphinxAtStartPar
Šis stulpelis pildomas automatinėmis priemonėmis, siekiant identifikuoti
konkrečias metaduomenų eilutes, kad būtų galima atpažinti metaduomenis,
kurie jau buvo pateikti ir po to atnaujinti.

\sphinxAtStartPar
Šio stulpelio pildyti nereikia.

\end{fulllineitems}

\index{type (įtaisytasis kintamasis)@\spxentry{type}\spxextra{įtaisytasis kintamasis}}

\begin{fulllineitems}
\phantomsection\label{\detokenize{formatas:type}}
\pysigstartsignatures
\pysigline
{\sphinxbfcode{\sphinxupquote{type}}}
\pysigstopsignatures
\sphinxAtStartPar
\sphinxstylestrong{Tipas}

\sphinxAtStartPar
Prasmė priklauso nuo dimensijos.

\sphinxAtStartPar
Jei nenurodytas nei vienas {\hyperref[\detokenize{formatas:dimensijos-stulpeliai}]{\sphinxcrossref{\DUrole{std}{\DUrole{std-ref}{dimensijos stulpelis}}}}}, tuomet šiame stulpelyje nurodoma {\hyperref[\detokenize{dimensijos:papildomos-dimensijos}]{\sphinxcrossref{\DUrole{std}{\DUrole{std-ref}{papildoma
dimensija}}}}}.


\begin{sphinxseealso}{Taip pat žiūrėkite:}

\sphinxAtStartPar
{\hyperref[\detokenize{tipai:duomenu-tipai}]{\sphinxcrossref{\DUrole{std}{\DUrole{std-ref}{Duomenų tipai}}}}}


\end{sphinxseealso}


\end{fulllineitems}

\index{ref (įtaisytasis kintamasis)@\spxentry{ref}\spxextra{įtaisytasis kintamasis}}

\begin{fulllineitems}
\phantomsection\label{\detokenize{formatas:ref}}
\pysigstartsignatures
\pysigline
{\sphinxbfcode{\sphinxupquote{ref}}}
\pysigstopsignatures
\sphinxAtStartPar
\sphinxstylestrong{Ryšys}

\sphinxAtStartPar
Prasmė priklauso nuo dimensijos.


\begin{sphinxseealso}{Taip pat žiūrėkite:}

\begin{DUlineblock}{0em}
\item[] {\hyperref[\detokenize{identifikatoriai:rysiai}]{\sphinxcrossref{\DUrole{std}{\DUrole{std-ref}{Asociacija}}}}}
\item[] {\hyperref[\detokenize{vienetai:matavimo-vienetai}]{\sphinxcrossref{\DUrole{std}{\DUrole{std-ref}{Matavimo vienetai}}}}}
\item[] {\hyperref[\detokenize{dimensijos:enum}]{\sphinxcrossref{\DUrole{std}{\DUrole{std-ref}{enum}}}}}
\end{DUlineblock}


\end{sphinxseealso}


\end{fulllineitems}

\index{source (įtaisytasis kintamasis)@\spxentry{source}\spxextra{įtaisytasis kintamasis}}

\begin{fulllineitems}
\phantomsection\label{\detokenize{formatas:source}}
\pysigstartsignatures
\pysigline
{\sphinxbfcode{\sphinxupquote{source}}}
\pysigstopsignatures
\sphinxAtStartPar
\sphinxstylestrong{Šaltinis}

\sphinxAtStartPar
Duomenų šaltinio struktūros elementai.


\begin{sphinxseealso}{Taip pat žiūrėkite:}

\begin{DUlineblock}{0em}
\item[] {\hyperref[\detokenize{saltiniai:duomenu-saltiniai}]{\sphinxcrossref{\DUrole{std}{\DUrole{std-ref}{Duomenų šaltiniai}}}}}
\end{DUlineblock}


\end{sphinxseealso}


\end{fulllineitems}

\index{prepare (įtaisytasis kintamasis)@\spxentry{prepare}\spxextra{įtaisytasis kintamasis}}

\begin{fulllineitems}
\phantomsection\label{\detokenize{formatas:prepare}}
\pysigstartsignatures
\pysigline
{\sphinxbfcode{\sphinxupquote{prepare}}}
\pysigstopsignatures
\sphinxAtStartPar
\sphinxstylestrong{Formulė}

\sphinxAtStartPar
Formulė skirta duomenų atrankai, nuasmeninimui, transformavimui, tikrinimui
ir pan.


\begin{sphinxseealso}{Taip pat žiūrėkite:}

\sphinxAtStartPar
{\hyperref[\detokenize{formules:formules}]{\sphinxcrossref{\DUrole{std}{\DUrole{std-ref}{Formulės}}}}}


\end{sphinxseealso}


\end{fulllineitems}

\index{level (įtaisytasis kintamasis)@\spxentry{level}\spxextra{įtaisytasis kintamasis}}

\begin{fulllineitems}
\phantomsection\label{\detokenize{formatas:level}}
\pysigstartsignatures
\pysigline
{\sphinxbfcode{\sphinxupquote{level}}}
\pysigstopsignatures
\sphinxAtStartPar
\sphinxstylestrong{Brandos lygis}

\sphinxAtStartPar
Duomenų brandos lygis, atitinka \sphinxhref{https://5stardata.info/en/}{5 Star Data}.


\begin{sphinxseealso}{Taip pat žiūrėkite:}

\sphinxAtStartPar
{\hyperref[\detokenize{branda:level}]{\sphinxcrossref{\DUrole{std}{\DUrole{std-ref}{Brandos lygiai}}}}}


\end{sphinxseealso}


\end{fulllineitems}

\index{access (įtaisytasis kintamasis)@\spxentry{access}\spxextra{įtaisytasis kintamasis}}

\begin{fulllineitems}
\phantomsection\label{\detokenize{formatas:access}}
\pysigstartsignatures
\pysigline
{\sphinxbfcode{\sphinxupquote{access}}}
\pysigstopsignatures
\sphinxAtStartPar
\sphinxstylestrong{Prieiga}

\sphinxAtStartPar
Duomenų prieigos lygis.


\begin{sphinxseealso}{Taip pat žiūrėkite:}

\sphinxAtStartPar
{\hyperref[\detokenize{prieiga:access}]{\sphinxcrossref{\DUrole{std}{\DUrole{std-ref}{Prieigos lygiai}}}}}


\end{sphinxseealso}


\end{fulllineitems}

\index{uri (įtaisytasis kintamasis)@\spxentry{uri}\spxextra{įtaisytasis kintamasis}}

\begin{fulllineitems}
\phantomsection\label{\detokenize{formatas:uri}}
\pysigstartsignatures
\pysigline
{\sphinxbfcode{\sphinxupquote{uri}}}
\pysigstopsignatures
\sphinxAtStartPar
\sphinxstylestrong{Žodyno atitikmuo}

\sphinxAtStartPar
Sąsaja su išoriniu žodynu.


\begin{sphinxseealso}{Taip pat žiūrėkite:}

\sphinxAtStartPar
{\hyperref[\detokenize{zodynai:vocab}]{\sphinxcrossref{\DUrole{std}{\DUrole{std-ref}{Išoriniai žodynai}}}}}


\end{sphinxseealso}


\end{fulllineitems}

\index{title (įtaisytasis kintamasis)@\spxentry{title}\spxextra{įtaisytasis kintamasis}}

\begin{fulllineitems}
\phantomsection\label{\detokenize{formatas:title}}
\pysigstartsignatures
\pysigline
{\sphinxbfcode{\sphinxupquote{title}}}
\pysigstopsignatures
\sphinxAtStartPar
\sphinxstylestrong{Pavadinimas}

\sphinxAtStartPar
Elemento pavadinimas.

\end{fulllineitems}

\index{description (įtaisytasis kintamasis)@\spxentry{description}\spxextra{įtaisytasis kintamasis}}

\begin{fulllineitems}
\phantomsection\label{\detokenize{formatas:description}}
\pysigstartsignatures
\pysigline
{\sphinxbfcode{\sphinxupquote{description}}}
\pysigstopsignatures
\sphinxAtStartPar
\sphinxstylestrong{Aprašymas}

\sphinxAtStartPar
Elemento aprašymas. Galima naudoti \sphinxhref{https://en.wikipedia.org/wiki/Markdown}{Markdown} sintaksę.

\end{fulllineitems}


\sphinxAtStartPar
Visi stulpeliai lentelėje yra neprivalomi. Stulpelių tvarka taip pat nėra
svarbi. Pavyzdžiui jei reikia apsirašyti tik globalių modelių struktūrą,
nebūtina įtraukti {\hyperref[\detokenize{formatas:dataset}]{\sphinxcrossref{\sphinxcode{\sphinxupquote{dataset}}}}}, {\hyperref[\detokenize{formatas:resource}]{\sphinxcrossref{\sphinxcode{\sphinxupquote{resource}}}}} ir {\hyperref[\detokenize{formatas:base}]{\sphinxcrossref{\sphinxcode{\sphinxupquote{base}}}}} stulpelių.
Jei norima apsirašyti tik prefiksus naudojamus {\hyperref[\detokenize{formatas:uri}]{\sphinxcrossref{\sphinxcode{\sphinxupquote{uri}}}}} lauke, užtenka
turėti tik prefiksų aprašymui reikalingus stulpelius.

\sphinxAtStartPar
Įrankiai skaitantys {\hyperref[\detokenize{savokos:term-DSA}]{\sphinxtermref{\DUrole{xref}{\DUrole{std}{\DUrole{std-term}{DSA}}}}}}, stulpelius, kurių nėra lentelėje turi
interpretuoti juos kaip tuščius. Taip pat įrankiai neturėtų tikėtis, kad stulpeliai
bus išdėstyti būtent tokia tvarka. Nors įrankių atžvilgiu stulpelių tvarka nėra
svarbi, tačiau rekomenduotina išlaikyti vienodą stulpelių tvarką, tam kad
lenteles būtų lengviau skaityti.

\sphinxstepscope


\section{Dimensijos}
\label{\detokenize{dimensijos:dimensijos}}\label{\detokenize{dimensijos:id1}}\label{\detokenize{dimensijos::doc}}
\sphinxAtStartPar
Demensijos leidžia vienoje lentelėje sutalpinti kelias skirtingas lenteles
turinčias bendrų savybių.

\sphinxAtStartPar
DSA lentelėje turime tokius dimensijų stulpelius:
\begin{itemize}
\item {} 
\sphinxAtStartPar
{\hyperref[\detokenize{dimensijos:dataset}]{\sphinxcrossref{\DUrole{std}{\DUrole{std-ref}{dataset}}}}}

\item {} 
\sphinxAtStartPar
{\hyperref[\detokenize{dimensijos:resource}]{\sphinxcrossref{\DUrole{std}{\DUrole{std-ref}{resource}}}}}

\item {} 
\sphinxAtStartPar
{\hyperref[\detokenize{dimensijos:model}]{\sphinxcrossref{\DUrole{std}{\DUrole{std-ref}{model}}}}}

\item {} 
\sphinxAtStartPar
{\hyperref[\detokenize{dimensijos:property}]{\sphinxcrossref{\DUrole{std}{\DUrole{std-ref}{property}}}}}

\end{itemize}


\begin{savenotes}\sphinxattablestart
\sphinxthistablewithglobalstyle
\centering
\begin{tabulary}{\linewidth}[t]{TTTTT}
\sphinxtoprule
\sphinxstyletheadfamily 
\sphinxAtStartPar
dataset
&\sphinxstyletheadfamily 
\sphinxAtStartPar
resource
&\sphinxstyletheadfamily 
\sphinxAtStartPar
model
&\sphinxstyletheadfamily 
\sphinxAtStartPar
property
&\sphinxstyletheadfamily 
\sphinxAtStartPar
title
\\
\sphinxmidrule
\sphinxtableatstartofbodyhook\sphinxstartmulticolumn{4}%
\begin{varwidth}[t]{\sphinxcolwidth{4}{5}}
\sphinxAtStartPar
datasets/gov/rc/ar/ws
\par
\vskip-\baselineskip\vbox{\hbox{\strut}}\end{varwidth}%
\sphinxstopmulticolumn
&
\sphinxAtStartPar
Duomenų rinkinys
\\
\sphinxhline
\sphinxAtStartPar

&\sphinxstartmulticolumn{3}%
\begin{varwidth}[t]{\sphinxcolwidth{3}{5}}
\sphinxAtStartPar
db
\par
\vskip-\baselineskip\vbox{\hbox{\strut}}\end{varwidth}%
\sphinxstopmulticolumn
&
\sphinxAtStartPar
Duomenų teikimo paslauga
\\
\sphinxhline
\sphinxAtStartPar

&&\sphinxstartmulticolumn{2}%
\begin{varwidth}[t]{\sphinxcolwidth{2}{5}}
\sphinxAtStartPar
\sphinxstylestrong{Gyvenviete}
\par
\vskip-\baselineskip\vbox{\hbox{\strut}}\end{varwidth}%
\sphinxstopmulticolumn
&
\sphinxAtStartPar
Esybė
\\
\sphinxhline
\sphinxAtStartPar

&&&
\sphinxAtStartPar
pavadinimas
&
\sphinxAtStartPar
Savybė
\\
\sphinxbottomrule
\end{tabulary}
\sphinxtableafterendhook\par
\sphinxattableend\end{savenotes}

\sphinxAtStartPar
Pavyzdyje aukščiau turime tris lenteles, turinčias vieną bendrą stulpelį
\sphinxcode{\sphinxupquote{title}}.

\sphinxAtStartPar
Daugiamatė lentelė pateikta viršuje, atitiktų tokią vienamatę lentelę:


\begin{savenotes}\sphinxattablestart
\sphinxthistablewithglobalstyle
\centering
\begin{tabulary}{\linewidth}[t]{TTT}
\sphinxtoprule
\sphinxstyletheadfamily 
\sphinxAtStartPar
type
&\sphinxstyletheadfamily 
\sphinxAtStartPar
name
&\sphinxstyletheadfamily 
\sphinxAtStartPar
title
\\
\sphinxmidrule
\sphinxtableatstartofbodyhook
\sphinxAtStartPar
dataset
&
\sphinxAtStartPar
datasets/gov/rc/ar/ws
&
\sphinxAtStartPar
Duomenų rinkinys
\\
\sphinxhline
\sphinxAtStartPar
resource
&
\sphinxAtStartPar
db
&
\sphinxAtStartPar
Duomenų teikimo paslauga
\\
\sphinxhline
\sphinxAtStartPar
model
&
\sphinxAtStartPar
\sphinxstylestrong{Gyvenviete}
&
\sphinxAtStartPar
Esybė
\\
\sphinxhline
\sphinxAtStartPar
property
&
\sphinxAtStartPar
pavadinimas
&
\sphinxAtStartPar
Savybė
\\
\sphinxbottomrule
\end{tabulary}
\sphinxtableafterendhook\par
\sphinxattableend\end{savenotes}

\sphinxAtStartPar
Kadangi DSA lentelė yra daugiamatė, nurodant stulpelį, jei kalbama apie
konkrečios dimensijos stulpelį, reikia nurodyti ir dimensiją, pavyzdžiui
\sphinxcode{\sphinxupquote{dataset.title}} nurodo būtent apie \sphinxcode{\sphinxupquote{dataset}} diemensijos \sphinxcode{\sphinxupquote{title}} stulpelį.


\subsection{dataset}
\label{\detokenize{dimensijos:dataset}}\label{\detokenize{dimensijos:id2}}\index{module@\spxentry{module}!dataset@\spxentry{dataset}}\index{dataset@\spxentry{dataset}!module@\spxentry{module}}\phantomsection\label{\detokenize{dimensijos:module-dataset}}
\sphinxAtStartPar
Duomenų rinkinys struktūros apraše nurodomas tam, kad būtų galimybė susieti
duomenų struktūros elementus su duomenų rinkiniais registruotais duomenų
kataloge. Toks susiejimas atliekamas naudojant duomenų rinkinio kodinį
pavadinimą.

\begin{sphinxtopic}
\sphinxstyletopictitle{Brandos lygis}
\begin{description}
\sphinxlineitem{{\hyperref[\detokenize{branda:l203}]{\sphinxcrossref{\DUrole{std}{\DUrole{std-ref}{L203: Nestandartiniai kodiniai pavadinimai}}}}}}
\sphinxAtStartPar
Duomenų rinkinio ar duomenų erdvės kodinis pavadinimas neatitinka
reikalavimų keliamų kodiniams pavadinimams.

\end{description}
\end{sphinxtopic}

\begin{sphinxadmonition}{note}{Pastaba:}
\sphinxAtStartPar
Nurodytas duomenų rinkinio ar vardų erdvės kodinis pavadinimas turi būti
unikalus tarp visų duomenų struktūros aprašų.
\end{sphinxadmonition}


\begin{sphinxseealso}{Taip pat žiūrėkite:}

\begin{DUlineblock}{0em}
\item[] {\hyperref[\detokenize{vardu-erdves:ns}]{\sphinxcrossref{\DUrole{std}{\DUrole{std-ref}{Vardų erdvės}}}}}
\item[] {\hyperref[\detokenize{pavadinimai:kodiniai-pavadinimai}]{\sphinxcrossref{\DUrole{std}{\DUrole{std-ref}{Kodiniai pavadinimai}}}}}
\end{DUlineblock}


\end{sphinxseealso}

\index{id (modulje dataset)@\spxentry{id}\spxextra{modulje dataset}}

\begin{fulllineitems}
\phantomsection\label{\detokenize{dimensijos:dataset.id}}
\pysigstartsignatures
\pysigline
{\sphinxcode{\sphinxupquote{dataset.}}\sphinxbfcode{\sphinxupquote{id}}}
\pysigstopsignatures
\sphinxAtStartPar
Duomenų rinkinio arba duomenų erdvės identifikatorius.

\end{fulllineitems}

\index{type (modulje dataset)@\spxentry{type}\spxextra{modulje dataset}}

\begin{fulllineitems}
\phantomsection\label{\detokenize{dimensijos:dataset.type}}
\pysigstartsignatures
\pysigline
{\sphinxcode{\sphinxupquote{dataset.}}\sphinxbfcode{\sphinxupquote{type}}}
\pysigstopsignatures
\sphinxAtStartPar
Jei nenurodyta, pagal nutylėjimą naudojama \sphinxcode{\sphinxupquote{dataset}} reikšmė, kuri nurodo
duomenų rinkinio kodinį pavadinimą nurodyta {\hyperref[\detokenize{savokos:term-duomenu-katalogas}]{\sphinxtermref{\DUrole{xref}{\DUrole{std}{\DUrole{std-term}{duomenų kataloge}}}}}}.

\sphinxAtStartPar
Galimos reikšmės:
\begin{description}
\sphinxlineitem{ns}
\sphinxAtStartPar
Vardų erdvė.

\sphinxlineitem{dataset}
\sphinxAtStartPar
Duomenų rinkinys.

\end{description}

\begin{sphinxadmonition}{note}{Pavyzdys}


\begin{savenotes}\sphinxattablestart
\sphinxthistablewithglobalstyle
\centering
\begin{tabulary}{\linewidth}[t]{TTTTTT}
\sphinxtoprule
\sphinxstyletheadfamily 
\sphinxAtStartPar
dataset
&\sphinxstyletheadfamily 
\sphinxAtStartPar
resource
&\sphinxstyletheadfamily 
\sphinxAtStartPar
model
&\sphinxstyletheadfamily 
\sphinxAtStartPar
property
&\sphinxstyletheadfamily 
\sphinxAtStartPar
type
&\sphinxstyletheadfamily 
\sphinxAtStartPar
title
\\
\sphinxmidrule
\sphinxtableatstartofbodyhook\sphinxstartmulticolumn{4}%
\begin{varwidth}[t]{\sphinxcolwidth{4}{6}}
\sphinxAtStartPar
datasets/gov/rc
\par
\vskip-\baselineskip\vbox{\hbox{\strut}}\end{varwidth}%
\sphinxstopmulticolumn
&
\sphinxAtStartPar
ns
&
\sphinxAtStartPar
Registrų centras
\\
\sphinxhline\sphinxstartmulticolumn{4}%
\begin{varwidth}[t]{\sphinxcolwidth{4}{6}}
\sphinxAtStartPar
datasets/gov/rc/ar
\par
\vskip-\baselineskip\vbox{\hbox{\strut}}\end{varwidth}%
\sphinxstopmulticolumn
&
\sphinxAtStartPar
ns
&
\sphinxAtStartPar
Adresų registras
\\
\sphinxhline\sphinxstartmulticolumn{4}%
\begin{varwidth}[t]{\sphinxcolwidth{4}{6}}
\sphinxAtStartPar
datasets/gov/rc/ar/ws
\par
\vskip-\baselineskip\vbox{\hbox{\strut}}\end{varwidth}%
\sphinxstopmulticolumn
&&
\sphinxAtStartPar
Duomenų teikimo paslauga
\\
\sphinxbottomrule
\end{tabulary}
\sphinxtableafterendhook\par
\sphinxattableend\end{savenotes}
\end{sphinxadmonition}

\end{fulllineitems}

\index{ref (modulje dataset)@\spxentry{ref}\spxextra{modulje dataset}}

\begin{fulllineitems}
\phantomsection\label{\detokenize{dimensijos:dataset.ref}}
\pysigstartsignatures
\pysigline
{\sphinxcode{\sphinxupquote{dataset.}}\sphinxbfcode{\sphinxupquote{ref}}}
\pysigstopsignatures
\sphinxAtStartPar
Duomenų rinkinio identifikatorius duomenų kataloge. Alternatyviai, galima
naudoti {\hyperref[\detokenize{dimensijos:dataset.source}]{\sphinxcrossref{\sphinxcode{\sphinxupquote{dataset.source}}}}}.

\sphinxAtStartPar
Nenaudojamas jei {\hyperref[\detokenize{dimensijos:dataset.type}]{\sphinxcrossref{\sphinxcode{\sphinxupquote{dataset.type}}}}} yra \sphinxcode{\sphinxupquote{ns}}.

\end{fulllineitems}

\index{source (modulje dataset)@\spxentry{source}\spxextra{modulje dataset}}

\begin{fulllineitems}
\phantomsection\label{\detokenize{dimensijos:dataset.source}}
\pysigstartsignatures
\pysigline
{\sphinxcode{\sphinxupquote{dataset.}}\sphinxbfcode{\sphinxupquote{source}}}
\pysigstopsignatures
\sphinxAtStartPar
Nuoroda į duomenų rinkinio puslapį duomenų kataloge.

\sphinxAtStartPar
Nenaudojama, jei \sphinxcode{\sphinxupquote{dataset.type}} yra \sphinxcode{\sphinxupquote{ns}}.

\end{fulllineitems}

\index{prepare (modulje dataset)@\spxentry{prepare}\spxextra{modulje dataset}}

\begin{fulllineitems}
\phantomsection\label{\detokenize{dimensijos:dataset.prepare}}
\pysigstartsignatures
\pysigline
{\sphinxcode{\sphinxupquote{dataset.}}\sphinxbfcode{\sphinxupquote{prepare}}}
\pysigstopsignatures
\sphinxAtStartPar
Nenaudojama.

\end{fulllineitems}

\index{level (modulje dataset)@\spxentry{level}\spxextra{modulje dataset}}

\begin{fulllineitems}
\phantomsection\label{\detokenize{dimensijos:dataset.level}}
\pysigstartsignatures
\pysigline
{\sphinxcode{\sphinxupquote{dataset.}}\sphinxbfcode{\sphinxupquote{level}}}
\pysigstopsignatures
\sphinxAtStartPar
Nenaudojamas.

\sphinxAtStartPar
Duomenų rinkinio brandos lygis yra išskaičiuojamas iš {\hyperref[\detokenize{dimensijos:model.level}]{\sphinxcrossref{\sphinxcode{\sphinxupquote{model.level}}}}}
ir {\hyperref[\detokenize{dimensijos:property.level}]{\sphinxcrossref{\sphinxcode{\sphinxupquote{property.level}}}}}.

\end{fulllineitems}

\index{access (modulje dataset)@\spxentry{access}\spxextra{modulje dataset}}

\begin{fulllineitems}
\phantomsection\label{\detokenize{dimensijos:dataset.access}}
\pysigstartsignatures
\pysigline
{\sphinxcode{\sphinxupquote{dataset.}}\sphinxbfcode{\sphinxupquote{access}}}
\pysigstopsignatures
\sphinxAtStartPar
Prieigos lygis, naudojamas pagal nutylėjimą viesiems šios vardų erdvės
elementams.

\end{fulllineitems}

\index{title (modulje dataset)@\spxentry{title}\spxextra{modulje dataset}}

\begin{fulllineitems}
\phantomsection\label{\detokenize{dimensijos:dataset.title}}
\pysigstartsignatures
\pysigline
{\sphinxcode{\sphinxupquote{dataset.}}\sphinxbfcode{\sphinxupquote{title}}}
\pysigstopsignatures
\sphinxAtStartPar
Duomenų rinkinio ar vardų erdvės pavadinimas.

\end{fulllineitems}

\index{description (modulje dataset)@\spxentry{description}\spxextra{modulje dataset}}

\begin{fulllineitems}
\phantomsection\label{\detokenize{dimensijos:dataset.description}}
\pysigstartsignatures
\pysigline
{\sphinxcode{\sphinxupquote{dataset.}}\sphinxbfcode{\sphinxupquote{description}}}
\pysigstopsignatures
\sphinxAtStartPar
Duomenų rinkinio ar vardų erdvės aprašymas.

\end{fulllineitems}



\subsection{resource}
\label{\detokenize{dimensijos:resource}}\label{\detokenize{dimensijos:id3}}\index{module@\spxentry{module}!resource@\spxentry{resource}}\index{resource@\spxentry{resource}!module@\spxentry{module}}\phantomsection\label{\detokenize{dimensijos:module-resource}}
\sphinxAtStartPar
Fizinis duomenų šaltinis, kuriame saugomi duomenys.

\sphinxAtStartPar
Kiekvienam duomenų šaltiniui suteikiamas \DUrole{xref}{\DUrole{std}{\DUrole{std-ref}{kodinis pavadinimas}}}, kuris nėra naudojamas formuojant API URI, tačiau naudojamas
identifikuojant patį duomenų šaltinį.

\sphinxAtStartPar
Nurodytas duomenų šaltinio kodinis pavadinimas turi būti unikalus duomenų
rinkinio kontekste.


\begin{sphinxseealso}{Taip pat žiūrėkite:}

\sphinxAtStartPar
{\hyperref[\detokenize{saltiniai:duomenu-saltiniai}]{\sphinxcrossref{\DUrole{std}{\DUrole{std-ref}{Duomenų šaltiniai}}}}}


\end{sphinxseealso}

\index{id (modulje resource)@\spxentry{id}\spxextra{modulje resource}}

\begin{fulllineitems}
\phantomsection\label{\detokenize{dimensijos:resource.id}}
\pysigstartsignatures
\pysigline
{\sphinxcode{\sphinxupquote{resource.}}\sphinxbfcode{\sphinxupquote{id}}}
\pysigstopsignatures
\sphinxAtStartPar
Duomenų šaltinio unikalus identifikatorius UUID formatu.

\end{fulllineitems}

\index{type (modulje resource)@\spxentry{type}\spxextra{modulje resource}}

\begin{fulllineitems}
\phantomsection\label{\detokenize{dimensijos:resource.type}}
\pysigstartsignatures
\pysigline
{\sphinxcode{\sphinxupquote{resource.}}\sphinxbfcode{\sphinxupquote{type}}}
\pysigstopsignatures
\sphinxAtStartPar
Duomenų šaltinio tipas. Galimos reikšmės:


\begin{savenotes}\sphinxattablestart
\sphinxthistablewithglobalstyle
\centering
\begin{tabulary}{\linewidth}[t]{TT}
\sphinxtoprule
\sphinxtableatstartofbodyhook
\sphinxAtStartPar
\sphinxcode{\sphinxupquote{sql}}
&
\sphinxAtStartPar
Reliacinės duomenų bazės
\\
\sphinxhline
\sphinxAtStartPar
\sphinxcode{\sphinxupquote{csv}}
&
\sphinxAtStartPar
CSV lentelės
\\
\sphinxhline
\sphinxAtStartPar
\sphinxcode{\sphinxupquote{json}}
&
\sphinxAtStartPar
JSON resursai
\\
\sphinxhline
\sphinxAtStartPar
\sphinxcode{\sphinxupquote{xml}}
&
\sphinxAtStartPar
XML resursai
\\
\sphinxbottomrule
\end{tabulary}
\sphinxtableafterendhook\par
\sphinxattableend\end{savenotes}

\end{fulllineitems}

\index{ref (modulje resource)@\spxentry{ref}\spxextra{modulje resource}}

\begin{fulllineitems}
\phantomsection\label{\detokenize{dimensijos:resource.ref}}
\pysigstartsignatures
\pysigline
{\sphinxcode{\sphinxupquote{resource.}}\sphinxbfcode{\sphinxupquote{ref}}}
\pysigstopsignatures
\sphinxAtStartPar
Identifikatorius, naudojamas konfiguracijoje, kurioje pateikiamas pilnas
resurso adresas ir kiti parametrai, tokie kaip slaptažodžiai ar
prisijungimo vardai.

\sphinxAtStartPar
Alternatyviai resurso pilną adresą galima nurodyti {\hyperref[\detokenize{dimensijos:resource.source}]{\sphinxcrossref{\sphinxcode{\sphinxupquote{resource.source}}}}}
stulpelyje.

\end{fulllineitems}

\index{source (modulje resource)@\spxentry{source}\spxextra{modulje resource}}

\begin{fulllineitems}
\phantomsection\label{\detokenize{dimensijos:resource.source}}
\pysigstartsignatures
\pysigline
{\sphinxcode{\sphinxupquote{resource.}}\sphinxbfcode{\sphinxupquote{source}}}
\pysigstopsignatures
\sphinxAtStartPar
Pilnas resurso adresas URI formatu.

\begin{sphinxadmonition}{warning}{Įspėjimas:}
\sphinxAtStartPar
Jei duomenų šaltinis reikalauja naudotojo vardo ir slaptažodžio,
rekomenduojama nerodyti URI struktūros apraše, vietoj to prisijungimo
duomenis prie šaltinio pateikti atskirame konfigūraciniame faile,
naudojant {\hyperref[\detokenize{dimensijos:resource.ref}]{\sphinxcrossref{\sphinxcode{\sphinxupquote{resource.ref}}}}} stulpelį.
\end{sphinxadmonition}

\sphinxAtStartPar
\sphinxstylestrong{dialect} {[} \sphinxcode{\sphinxupquote{+}} \sphinxstylestrong{driver} {]} \sphinxcode{\sphinxupquote{://}} {[} \sphinxstylestrong{user} \sphinxcode{\sphinxupquote{:}} \sphinxstylestrong{password} \sphinxcode{\sphinxupquote{@}} {]}
\sphinxstylestrong{host} {[} \sphinxcode{\sphinxupquote{:}} \sphinxstylestrong{port} {]} \sphinxcode{\sphinxupquote{/}} \sphinxstylestrong{path} {[} \sphinxcode{\sphinxupquote{?}} \sphinxstylestrong{params} {]}
\begin{description}
\sphinxlineitem{dialect}
\sphinxAtStartPar
Duomenų šaltinio dialektas arba protokolas, kuriuo teikiami duomenys,
galimi variantai:


\begin{savenotes}\sphinxattablestart
\sphinxthistablewithglobalstyle
\centering
\begin{tabulary}{\linewidth}[t]{TT}
\sphinxtoprule
\sphinxtableatstartofbodyhook
\sphinxAtStartPar
\sphinxcode{\sphinxupquote{postgresql}}
&
\sphinxAtStartPar
PostgreSQL duomenų bazė.
\\
\sphinxhline
\sphinxAtStartPar
\sphinxcode{\sphinxupquote{mysql}}
&
\sphinxAtStartPar
MySQL duomenų bazė.
\\
\sphinxhline
\sphinxAtStartPar
\sphinxcode{\sphinxupquote{mariadb}}
&
\sphinxAtStartPar
MariaDB duomenų bazė.
\\
\sphinxhline
\sphinxAtStartPar
\sphinxcode{\sphinxupquote{sqlite}}
&
\sphinxAtStartPar
SQLite duomenų bazė.
\\
\sphinxhline
\sphinxAtStartPar
\sphinxcode{\sphinxupquote{oracle}}
&
\sphinxAtStartPar
Oracle duomenų bazė.
\\
\sphinxhline
\sphinxAtStartPar
\sphinxcode{\sphinxupquote{mssql}}
&
\sphinxAtStartPar
Microsoft SQL Server duomenų bazė.
\\
\sphinxhline
\sphinxAtStartPar
\sphinxcode{\sphinxupquote{http}}, \sphinxcode{\sphinxupquote{https}}
&
\sphinxAtStartPar
Duomenų failas publikuojamas HTTP protokolu.
\\
\sphinxbottomrule
\end{tabulary}
\sphinxtableafterendhook\par
\sphinxattableend\end{savenotes}

\sphinxlineitem{driver}
\sphinxAtStartPar
Priklauso nuo \sphinxstylestrong{dialect} ir nuo naudojamo duomenų agento.

\sphinxlineitem{user}
\sphinxAtStartPar
Duomenų šaltinio naudotojo vardas, jei duomenų šaltinis to reikalauja.

\sphinxlineitem{password}
\sphinxAtStartPar
Duomenų šaltinio slaptažodis, jei duomenų šaltinis to reikalauja.

\sphinxlineitem{host}
\sphinxAtStartPar
Duomenų šaltinio serverio adresas, jei duomenų šaltinis yra
nuotoliniame serveryje.

\sphinxlineitem{port}
\sphinxAtStartPar
Nuotolinio serverio prievado numeris.

\sphinxlineitem{path}
\sphinxAtStartPar
Duomenų bazės pavadinimas arba kelias iki duomenų failo.

\sphinxlineitem{params}
\sphinxAtStartPar
Papildomi parametrai, priklauso nuo naudojamo \sphinxstylestrong{driver}.

\end{description}

\end{fulllineitems}

\index{level (modulje resource)@\spxentry{level}\spxextra{modulje resource}}

\begin{fulllineitems}
\phantomsection\label{\detokenize{dimensijos:resource.level}}
\pysigstartsignatures
\pysigline
{\sphinxcode{\sphinxupquote{resource.}}\sphinxbfcode{\sphinxupquote{level}}}
\pysigstopsignatures
\sphinxAtStartPar
Duomenų šaltinio {\hyperref[\detokenize{branda:level}]{\sphinxcrossref{\DUrole{std}{\DUrole{std-ref}{brandos lygis}}}}}, vertinant tik pagal formatą,
nežiūrint į šaltinyje esančių duomenų turinį.

\end{fulllineitems}

\index{access (modulje resource)@\spxentry{access}\spxextra{modulje resource}}

\begin{fulllineitems}
\phantomsection\label{\detokenize{dimensijos:resource.access}}
\pysigstartsignatures
\pysigline
{\sphinxcode{\sphinxupquote{resource.}}\sphinxbfcode{\sphinxupquote{access}}}
\pysigstopsignatures
\sphinxAtStartPar
Duomenų šaltinio {\hyperref[\detokenize{prieiga:access}]{\sphinxcrossref{\DUrole{std}{\DUrole{std-ref}{prieigos lygis}}}}}.

\sphinxAtStartPar
Pildyti neprivaloma, jei nurodytas, tada visoms žemesnio lygio dimensijoms,
pagal nutylėjimą taikomas nurodytas šaltinio prieigos lygis.

\end{fulllineitems}

\index{title (modulje resource)@\spxentry{title}\spxextra{modulje resource}}

\begin{fulllineitems}
\phantomsection\label{\detokenize{dimensijos:resource.title}}
\pysigstartsignatures
\pysigline
{\sphinxcode{\sphinxupquote{resource.}}\sphinxbfcode{\sphinxupquote{title}}}
\pysigstopsignatures
\sphinxAtStartPar
Duomenų šaltinio pavadinimas.

\end{fulllineitems}

\index{description (modulje resource)@\spxentry{description}\spxextra{modulje resource}}

\begin{fulllineitems}
\phantomsection\label{\detokenize{dimensijos:resource.description}}
\pysigstartsignatures
\pysigline
{\sphinxcode{\sphinxupquote{resource.}}\sphinxbfcode{\sphinxupquote{description}}}
\pysigstopsignatures
\sphinxAtStartPar
Duomenų šaltinio aprašymas.

\end{fulllineitems}



\subsubsection{Funkcijos}
\label{\detokenize{dimensijos:module-0}}\label{\detokenize{dimensijos:funkcijos}}\index{module@\spxentry{module}!resource@\spxentry{resource}}\index{resource@\spxentry{resource}!module@\spxentry{module}}\index{http() (modulyje resource)@\spxentry{http()}\spxextra{modulyje resource}}

\begin{fulllineitems}
\phantomsection\label{\detokenize{dimensijos:resource.http}}
\pysigstartsignatures
\pysiglinewithargsret
{\sphinxcode{\sphinxupquote{resource.}}\sphinxbfcode{\sphinxupquote{http}}}
{\sphinxparam{\DUrole{n}{method}\DUrole{o}{=}\DUrole{default_value}{'GET'}}\sphinxparamcomma \sphinxparam{\DUrole{n}{body}\DUrole{o}{=}\DUrole{default_value}{form}}}
{}
\pysigstopsignatures
\sphinxAtStartPar
Papildomi parametrai, reikaling konstruojant HTTP užklausas.

\sphinxAtStartPar
\sphinxstylestrong{Argumentai}
\begin{description}
\sphinxlineitem{method (vardinis)}
\sphinxAtStartPar
HTTP \sphinxhref{https://developer.mozilla.org/en-US/docs/Web/HTTP/Methods}{methodas}.

\sphinxlineitem{body (vardinis)}
\sphinxAtStartPar
HTTP užklausos perduodamų duomenų formatas.

\sphinxAtStartPar
Galimi variantai:


\begin{savenotes}\sphinxattablestart
\sphinxthistablewithglobalstyle
\centering
\begin{tabulary}{\linewidth}[t]{TT}
\sphinxtoprule
\sphinxtableatstartofbodyhook
\sphinxAtStartPar
\sphinxcode{\sphinxupquote{json}}
&
\sphinxAtStartPar
Duomenys perduodami JSON formatu.
\\
\sphinxhline
\sphinxAtStartPar
\sphinxcode{\sphinxupquote{xml}}
&
\sphinxAtStartPar
Duomenys perduodami XML formatu.
\\
\sphinxhline
\sphinxAtStartPar
\sphinxcode{\sphinxupquote{from}}
&
\sphinxAtStartPar
Duomenys perduodami \sphinxcode{\sphinxupquote{application/x\sphinxhyphen{}www\sphinxhyphen{}form\sphinxhyphen{}urlencoded}} arba
\sphinxcode{\sphinxupquote{multipart/form\sphinxhyphen{}data}} (jei formoje pateikiami failai) formatu.
\\
\sphinxbottomrule
\end{tabulary}
\sphinxtableafterendhook\par
\sphinxattableend\end{savenotes}

\end{description}

\begin{sphinxadmonition}{note}{Pavyzdys}


\begin{savenotes}\sphinxattablestart
\sphinxthistablewithglobalstyle
\centering
\begin{tabulary}{\linewidth}[t]{TTTTT}
\sphinxtoprule
\sphinxstyletheadfamily 
\sphinxAtStartPar
resource
&\sphinxstyletheadfamily 
\sphinxAtStartPar
type
&\sphinxstyletheadfamily 
\sphinxAtStartPar
ref
&\sphinxstyletheadfamily 
\sphinxAtStartPar
source
&\sphinxstyletheadfamily 
\sphinxAtStartPar
prepare
\\
\sphinxmidrule
\sphinxtableatstartofbodyhook
\sphinxAtStartPar
resource1
&
\sphinxAtStartPar
json
&&
\sphinxAtStartPar
https://example.com/
&\\
\sphinxhline
\sphinxAtStartPar

&
\sphinxAtStartPar
param
&
\sphinxAtStartPar
name1
&
\sphinxAtStartPar
NAME1
&
\sphinxAtStartPar
\sphinxcode{\sphinxupquote{query("value1")}}
\\
\sphinxhline
\sphinxAtStartPar

&&
\sphinxAtStartPar
name2
&
\sphinxAtStartPar
NAME2
&
\sphinxAtStartPar
\sphinxcode{\sphinxupquote{query("value2")}}
\\
\sphinxbottomrule
\end{tabulary}
\sphinxtableafterendhook\par
\sphinxattableend\end{savenotes}

\sphinxAtStartPar
Bus konstruojamas toks URI:

\begin{sphinxVerbatim}[commandchars=\\\{\}]
https://example.com/?NAME1=value1\PYGZam{}NAME2=value2
\end{sphinxVerbatim}
\end{sphinxadmonition}

\end{fulllineitems}



\subsection{base}
\label{\detokenize{dimensijos:base}}\label{\detokenize{dimensijos:id4}}\index{module@\spxentry{module}!base@\spxentry{base}}\index{base@\spxentry{base}!module@\spxentry{module}}\phantomsection\label{\detokenize{dimensijos:module-base}}
\sphinxAtStartPar
Modelio bazė naudojama objekto identifikatoriams susieti, kai keli skirtingi
duomenų modeliai aprašo tą pačią realaus pasaulio esybę.
\index{ref (modulje base)@\spxentry{ref}\spxextra{modulje base}}

\begin{fulllineitems}
\phantomsection\label{\detokenize{dimensijos:base.ref}}
\pysigstartsignatures
\pysigline
{\sphinxcode{\sphinxupquote{base.}}\sphinxbfcode{\sphinxupquote{ref}}}
\pysigstopsignatures
\sphinxAtStartPar
{\hyperref[\detokenize{dimensijos:model.property}]{\sphinxcrossref{\sphinxcode{\sphinxupquote{model.property}}}}} reikšmė, kurios pagalba {\hyperref[\detokenize{formatas:model}]{\sphinxcrossref{\sphinxcode{\sphinxupquote{model}}}}} objektai
siejami su {\hyperref[\detokenize{formatas:base}]{\sphinxcrossref{\sphinxcode{\sphinxupquote{base}}}}} objektais. Jei susiejimas pagal vieną {\hyperref[\detokenize{dimensijos:model.property}]{\sphinxcrossref{\sphinxcode{\sphinxupquote{model.property}}}}}
yra neįmanomas, galima nurodyti kelis {\hyperref[\detokenize{dimensijos:model.property}]{\sphinxcrossref{\sphinxcode{\sphinxupquote{model.property}}}}} pavadinimus
atskirtus kableliu.

\sphinxAtStartPar
Galima naudoti tik tuos {\hyperref[\detokenize{dimensijos:model.property}]{\sphinxcrossref{\sphinxcode{\sphinxupquote{model.property}}}}}, kurie neturi nurodyto
{\hyperref[\detokenize{dimensijos:property.type}]{\sphinxcrossref{\sphinxcode{\sphinxupquote{property.type}}}}}, kas reiškia, kad toks pat laukas turi būti tiek
{\hyperref[\detokenize{formatas:base}]{\sphinxcrossref{\sphinxcode{\sphinxupquote{base}}}}}, tiek {\hyperref[\detokenize{formatas:model}]{\sphinxcrossref{\sphinxcode{\sphinxupquote{model}}}}} laukų sąraše.

\sphinxAtStartPar
Tais atvejais, kai {\hyperref[\detokenize{dimensijos:base.ref}]{\sphinxcrossref{\sphinxcode{\sphinxupquote{base.ref}}}}} rodo į modelio lauką, kuris turi tipą,
tada {\hyperref[\detokenize{dimensijos:base.level}]{\sphinxcrossref{\sphinxcode{\sphinxupquote{base.level}}}}} negali būti didesnis nei \sphinxcode{\sphinxupquote{3}}, kadangi jei modelio
laukas turi tipą, tai reiškia, kad jo duomenys nesutampa su bazės
duomenimis ir todėl jungimas negali būti daromas.

\end{fulllineitems}

\index{level (modulje base)@\spxentry{level}\spxextra{modulje base}}

\begin{fulllineitems}
\phantomsection\label{\detokenize{dimensijos:base.level}}
\pysigstartsignatures
\pysigline
{\sphinxcode{\sphinxupquote{base.}}\sphinxbfcode{\sphinxupquote{level}}}
\pysigstopsignatures
\sphinxAtStartPar
{\hyperref[\detokenize{branda:level}]{\sphinxcrossref{\DUrole{std}{\DUrole{std-ref}{Brandos lygis}}}}}, nurodantis modelio susiejamumą su nurodytu
baziniu modeliu. Plačiau žiūrėti {\hyperref[\detokenize{identifikatoriai:ref-level}]{\sphinxcrossref{\DUrole{std}{\DUrole{std-ref}{Ryšiai tarp modelių | Brandos lygis}}}}}.

\sphinxAtStartPar
Jei brandos lygis yra žemesnis nei \sphinxcode{\sphinxupquote{3}}, tada identifikatorių siejimas nėra
atliekamas, tokiu būdu tiesiog nurodomas semantinis susiejimas metaduomenų,
o ne duomenų lygmenyje.

\end{fulllineitems}

\index{access (modulje base)@\spxentry{access}\spxextra{modulje base}}

\begin{fulllineitems}
\phantomsection\label{\detokenize{dimensijos:base.access}}
\pysigstartsignatures
\pysigline
{\sphinxcode{\sphinxupquote{base.}}\sphinxbfcode{\sphinxupquote{access}}}
\pysigstopsignatures
\sphinxAtStartPar
Nenaudojamas.

\end{fulllineitems}



\subsubsection{Išoriniai identifikatoriai}
\label{\detokenize{dimensijos:isoriniai-identifikatoriai}}
\sphinxAtStartPar
Modelis ir jo bazė turi vienodus išorinius identifikatorius, nors vidiniai
šaltinio identifikatoriai gali skirtis.

\sphinxAtStartPar
Siejant {\hyperref[\detokenize{formatas:model}]{\sphinxcrossref{\sphinxcode{\sphinxupquote{model}}}}} ir {\hyperref[\detokenize{formatas:base}]{\sphinxcrossref{\sphinxcode{\sphinxupquote{base}}}}} duomenis tarpusavyje, {\hyperref[\detokenize{formatas:model}]{\sphinxcrossref{\sphinxcode{\sphinxupquote{model}}}}}
lentelė įgauna lygiai tokius pačius unikalius identifikatorius, kurie yra
{\hyperref[\detokenize{formatas:base}]{\sphinxcrossref{\sphinxcode{\sphinxupquote{base}}}}} lentelėje. Tai reiškia, kad {\hyperref[\detokenize{formatas:model}]{\sphinxcrossref{\sphinxcode{\sphinxupquote{model}}}}} lentelėje negali būti
duomenų, kurių nėra {\hyperref[\detokenize{formatas:base}]{\sphinxcrossref{\sphinxcode{\sphinxupquote{base}}}}} lentelėje.

\sphinxAtStartPar
Identifikatorių apjungimas atliekamas pagal {\hyperref[\detokenize{dimensijos:model.ref}]{\sphinxcrossref{\sphinxcode{\sphinxupquote{model.ref}}}}} ir
{\hyperref[\detokenize{dimensijos:base.ref}]{\sphinxcrossref{\sphinxcode{\sphinxupquote{base.ref}}}}} pateiktus pirminius raktus, kurie turi sutapti.

\sphinxAtStartPar
Visi {\hyperref[\detokenize{dimensijos:base.ref}]{\sphinxcrossref{\sphinxcode{\sphinxupquote{base.ref}}}}} laukai turi būti aprašyti tiek {\hyperref[\detokenize{formatas:base}]{\sphinxcrossref{\sphinxcode{\sphinxupquote{base}}}}}, tiek
{\hyperref[\detokenize{formatas:model}]{\sphinxcrossref{\sphinxcode{\sphinxupquote{model}}}}} modeliusoe.


\subsubsection{Paveldimumas}
\label{\detokenize{dimensijos:paveldimumas}}
\sphinxAtStartPar
{\hyperref[\detokenize{formatas:model}]{\sphinxcrossref{\sphinxcode{\sphinxupquote{model}}}}} paveldi visus laukus iš {\hyperref[\detokenize{formatas:base}]{\sphinxcrossref{\sphinxcode{\sphinxupquote{base}}}}}, įskaitant ir tuos, kurie
nėra nurodyti prie {\hyperref[\detokenize{formatas:model}]{\sphinxcrossref{\sphinxcode{\sphinxupquote{model}}}}} laukų sąrašo. Tai reiškia, kad galima skaityti
ir rašyti duomenis į {\hyperref[\detokenize{formatas:base}]{\sphinxcrossref{\sphinxcode{\sphinxupquote{base}}}}}, per {\hyperref[\detokenize{formatas:model}]{\sphinxcrossref{\sphinxcode{\sphinxupquote{model}}}}}. Jei skaitomas ar rašomas
laukas, kurio nėra {\hyperref[\detokenize{formatas:model}]{\sphinxcrossref{\sphinxcode{\sphinxupquote{model}}}}} laukų sąraše, tada to lauko duomenys skaitomi
iš arba rašomi į {\hyperref[\detokenize{formatas:base}]{\sphinxcrossref{\sphinxcode{\sphinxupquote{base}}}}} modelį.

\sphinxAtStartPar
Skaitymas ir rašymas iš base įmanomas tik tuo atveju, jei tai palaiko duomenų
šaltinis.


\subsubsection{Duomenų lokalumas}
\label{\detokenize{dimensijos:duomenu-lokalumas}}
\sphinxAtStartPar
Visi modelio laukai, kurie neturi {\hyperref[\detokenize{dimensijos:property.type}]{\sphinxcrossref{\sphinxcode{\sphinxupquote{property.type}}}}}, fiziškai saugomi
{\hyperref[\detokenize{formatas:base}]{\sphinxcrossref{\sphinxcode{\sphinxupquote{base}}}}} modelio šaltinyje.

\sphinxAtStartPar
Jei {\hyperref[\detokenize{formatas:base}]{\sphinxcrossref{\sphinxcode{\sphinxupquote{base}}}}} stulpelyje nurodoma \sphinxcode{\sphinxupquote{/}} reikšmė, tai reiškia, kad
{\hyperref[\detokenize{formatas:model}]{\sphinxcrossref{\sphinxcode{\sphinxupquote{model}}}}} neturi bazės, arba modelio bazė yra panaikinama. \sphinxcode{\sphinxupquote{/}} naudojamas
tais atvejais, kai norima vieną ar kelis modelius prijungti prie vienos bazės,
tačiau sekantys modeliai nebeturi priklausyti jokiai bazei.


\subsubsection{Persidengimas}
\label{\detokenize{dimensijos:persidengimas}}
\sphinxAtStartPar
Tais atvejais, kai {\hyperref[\detokenize{formatas:property}]{\sphinxcrossref{\sphinxcode{\sphinxupquote{property}}}}} yra saugomas tiek {\hyperref[\detokenize{formatas:base}]{\sphinxcrossref{\sphinxcode{\sphinxupquote{base}}}}}, tiek
{\hyperref[\detokenize{formatas:model}]{\sphinxcrossref{\sphinxcode{\sphinxupquote{model}}}}} lentelėse, norint gauti persidengiančios savybės duomenis iš
{\hyperref[\detokenize{formatas:base}]{\sphinxcrossref{\sphinxcode{\sphinxupquote{base}}}}}, reikia naudoti \sphinxcode{\sphinxupquote{\_base.}} prefiksą.

\sphinxAtStartPar
\sphinxcode{\sphinxupquote{\_base}} rodo į bazinį modelė.


\subsubsection{Pavyzdžiai}
\label{\detokenize{dimensijos:pavyzdziai}}
\begin{sphinxadmonition}{note}{Pavyzdys}


\begin{savenotes}\sphinxattablestart
\sphinxthistablewithglobalstyle
\centering
\begin{tabulary}{\linewidth}[t]{TTTTTTT}
\sphinxtoprule
\sphinxstyletheadfamily 
\sphinxAtStartPar
dataset
&\sphinxstyletheadfamily 
\sphinxAtStartPar
base
&\sphinxstyletheadfamily 
\sphinxAtStartPar
model
&\sphinxstyletheadfamily 
\sphinxAtStartPar
property
&\sphinxstyletheadfamily 
\sphinxAtStartPar
type
&\sphinxstyletheadfamily 
\sphinxAtStartPar
ref
&\sphinxstyletheadfamily 
\sphinxAtStartPar
source
\\
\sphinxmidrule
\sphinxtableatstartofbodyhook\sphinxstartmulticolumn{4}%
\begin{varwidth}[t]{\sphinxcolwidth{4}{7}}
\sphinxAtStartPar
example
\par
\vskip-\baselineskip\vbox{\hbox{\strut}}\end{varwidth}%
\sphinxstopmulticolumn
&&&\\
\sphinxhline
\sphinxAtStartPar

&&\sphinxstartmulticolumn{2}%
\begin{varwidth}[t]{\sphinxcolwidth{2}{7}}
\sphinxAtStartPar
Location
\par
\vskip-\baselineskip\vbox{\hbox{\strut}}\end{varwidth}%
\sphinxstopmulticolumn
&&
\sphinxAtStartPar
id
&\\
\sphinxhline
\sphinxAtStartPar

&&&
\sphinxAtStartPar
id
&
\sphinxAtStartPar
integer
&&\\
\sphinxhline
\sphinxAtStartPar

&&&
\sphinxAtStartPar
name@lt
&
\sphinxAtStartPar
text
&&\\
\sphinxhline
\sphinxAtStartPar

&&&
\sphinxAtStartPar
population
&
\sphinxAtStartPar
integer
&&\\
\sphinxhline
\sphinxAtStartPar

&\sphinxstartmulticolumn{3}%
\begin{varwidth}[t]{\sphinxcolwidth{3}{7}}
\sphinxAtStartPar
Loocation
\par
\vskip-\baselineskip\vbox{\hbox{\strut}}\end{varwidth}%
\sphinxstopmulticolumn
&&
\sphinxAtStartPar
name@lt
&\\
\sphinxhline
\sphinxAtStartPar

&&\sphinxstartmulticolumn{2}%
\begin{varwidth}[t]{\sphinxcolwidth{2}{7}}
\sphinxAtStartPar
City
\par
\vskip-\baselineskip\vbox{\hbox{\strut}}\end{varwidth}%
\sphinxstopmulticolumn
&&
\sphinxAtStartPar
name@lt
&
\sphinxAtStartPar
CITY
\\
\sphinxhline
\sphinxAtStartPar

&&&
\sphinxAtStartPar
name@lt
&&&
\sphinxAtStartPar
NAME
\\
\sphinxhline
\sphinxAtStartPar

&&&
\sphinxAtStartPar
population
&&&
\sphinxAtStartPar
POPULATION
\\
\sphinxhline
\sphinxAtStartPar

&\sphinxstartmulticolumn{3}%
\begin{varwidth}[t]{\sphinxcolwidth{3}{7}}
\sphinxAtStartPar
/
\par
\vskip-\baselineskip\vbox{\hbox{\strut}}\end{varwidth}%
\sphinxstopmulticolumn
&&&\\
\sphinxhline
\sphinxAtStartPar

&&\sphinxstartmulticolumn{2}%
\begin{varwidth}[t]{\sphinxcolwidth{2}{7}}
\sphinxAtStartPar
Village
\par
\vskip-\baselineskip\vbox{\hbox{\strut}}\end{varwidth}%
\sphinxstopmulticolumn
&&
\sphinxAtStartPar
name@lt
&
\sphinxAtStartPar
VILLAGE
\\
\sphinxhline
\sphinxAtStartPar

&&&
\sphinxAtStartPar
name@lt
&&&
\sphinxAtStartPar
VILLAGE
\\
\sphinxhline
\sphinxAtStartPar

&&&
\sphinxAtStartPar
population
&&&
\sphinxAtStartPar
POPULATION
\\
\sphinxhline
\sphinxAtStartPar

&&&
\sphinxAtStartPar
region
&
\sphinxAtStartPar
ref
&
\sphinxAtStartPar
Location
&
\sphinxAtStartPar
REGION
\\
\sphinxhline
\sphinxAtStartPar

&\sphinxstartmulticolumn{3}%
\begin{varwidth}[t]{\sphinxcolwidth{3}{7}}
\sphinxAtStartPar
/
\par
\vskip-\baselineskip\vbox{\hbox{\strut}}\end{varwidth}%
\sphinxstopmulticolumn
&&&\\
\sphinxbottomrule
\end{tabulary}
\sphinxtableafterendhook\par
\sphinxattableend\end{savenotes}

\sphinxAtStartPar
Šiame pavyzdyje esminis skirtumas yra tas, kad nurodyta kaip daromas jungimas.
\sphinxcode{\sphinxupquote{City}} ir \sphinxcode{\sphinxupquote{Village}} su \sphinxcode{\sphinxupquote{Location}} jungiame per \sphinxcode{\sphinxupquote{name@lt}} lauką.
\end{sphinxadmonition}


\subsection{model}
\label{\detokenize{dimensijos:model}}\label{\detokenize{dimensijos:id5}}\index{module@\spxentry{module}!model@\spxentry{model}}\index{model@\spxentry{model}!module@\spxentry{module}}\phantomsection\label{\detokenize{dimensijos:module-model}}
\sphinxAtStartPar
Duomenų modelio {\hyperref[\detokenize{pavadinimai:kodiniai-pavadinimai}]{\sphinxcrossref{\DUrole{std}{\DUrole{std-ref}{kodinis pavadinimas}}}}}. Užrašomas
vienaskaitos forma iš didžiosios raidės, jei pavadinimas iš kelių žodžių,
žodžiai atskiriami didžiąja raide.

\begin{sphinxadmonition}{note}{Pavyzdžiai}

\begin{DUlineblock}{0em}
\item[] \sphinxcode{\sphinxupquote{Gyvenviete}}
\item[] \sphinxcode{\sphinxupquote{AdministracinisTipas}}
\end{DUlineblock}
\end{sphinxadmonition}

\sphinxAtStartPar
Modelis yra siejamas su realaus pasaulio esybėmis. Viena esybė gali turėti
kelis skirtingus duomenų modelius, su skirtingomis savybėmis, tačiau skirtingi
vienos esybės modeliai turi turėti vienodus identifikatorius.

\begin{sphinxtopic}
\sphinxstyletopictitle{Brandos lygis}
\begin{description}
\sphinxlineitem{{\hyperref[\detokenize{branda:l203}]{\sphinxcrossref{\DUrole{std}{\DUrole{std-ref}{L203: Nestandartiniai kodiniai pavadinimai}}}}}}
\sphinxAtStartPar
Modelio kodinis pavadinimas neatitinka reikalavimų keliamų kodiniams
pavadinimams.

\end{description}
\end{sphinxtopic}
\index{type (modulje model)@\spxentry{type}\spxextra{modulje model}}

\begin{fulllineitems}
\phantomsection\label{\detokenize{dimensijos:model.type}}
\pysigstartsignatures
\pysigline
{\sphinxcode{\sphinxupquote{model.}}\sphinxbfcode{\sphinxupquote{type}}}
\pysigstopsignatures
\sphinxAtStartPar
\DUrole{versionmodified}{\DUrole{changed}{Pakeista 0.2 versijoje: }}Nuo 0.2 versijos nurodo modelio bazę.

\sphinxAtStartPar
Nurodo modelio bazę arba esybę, kurios pagalba skirtingiems modeliams
suteikiami vienodi identifikatoriai.

\sphinxAtStartPar
Jei nurodyta modelio bazė, {\hyperref[\detokenize{dimensijos:model.ref}]{\sphinxcrossref{\sphinxcode{\sphinxupquote{model.ref}}}}} nurodytas pirminis raktas turi
sutapti su bazinio modelio pirminiu raktu.

\sphinxAtStartPar
Taip pat turi sutapti ir modelio savybės su baziniu modeliu. Tačiau modelis
gali turėti ir papildomų savybių, kurių nėra baziniame modelyje.
Vienintelis privalomas reikalavimas yra pirminio rakto susiejimas, kad
modelis ir bazinis modelis turėtu vienodus identifikatorius.

\begin{sphinxtopic}
\sphinxstyletopictitle{Brandos lygis}
\begin{description}
\sphinxlineitem{{\hyperref[\detokenize{branda:l103}]{\sphinxcrossref{\DUrole{std}{\DUrole{std-ref}{L103: Neįmanomas jungimas}}}}}}
\sphinxAtStartPar
Modelis yra susietas su bazinio registro esybe metaduomenų
lygmeniu, tačiau nėra tokio identifikatoriaus kuris leistu susieti
ir pačius duomenis.

\sphinxlineitem{{\hyperref[\detokenize{branda:l209}]{\sphinxcrossref{\DUrole{std}{\DUrole{std-ref}{L209: Nenurodyta modelio bazė}}}}}}
\sphinxAtStartPar
Modelis nėra susietas su baziniame registre apibrėžta esybe.

\end{description}
\end{sphinxtopic}

\begin{sphinxadmonition}{note}{Pavyzdys}

\sphinxAtStartPar
\sphinxstylestrong{Duomenų modelis}

\begin{DUlineblock}{0em}
\item[] 
\end{DUlineblock}

\sphinxAtStartPar
\sphinxstylestrong{Struktūros aprašas}


\begin{savenotes}\sphinxattablestart
\sphinxthistablewithglobalstyle
\centering
\begin{tabulary}{\linewidth}[t]{TTTTT}
\sphinxtoprule
\sphinxstyletheadfamily 
\sphinxAtStartPar
dataset
&\sphinxstyletheadfamily 
\sphinxAtStartPar
model
&\sphinxstyletheadfamily 
\sphinxAtStartPar
property
&\sphinxstyletheadfamily 
\sphinxAtStartPar
type
&\sphinxstyletheadfamily 
\sphinxAtStartPar
ref
\\
\sphinxmidrule
\sphinxtableatstartofbodyhook\sphinxstartmulticolumn{3}%
\begin{varwidth}[t]{\sphinxcolwidth{3}{5}}
\sphinxAtStartPar
locn
\par
\vskip-\baselineskip\vbox{\hbox{\strut}}\end{varwidth}%
\sphinxstopmulticolumn
&&\\
\sphinxhline
\sphinxAtStartPar

&\sphinxstartmulticolumn{2}%
\begin{varwidth}[t]{\sphinxcolwidth{2}{5}}
\sphinxAtStartPar
\sphinxstylestrong{Location}
\par
\vskip-\baselineskip\vbox{\hbox{\strut}}\end{varwidth}%
\sphinxstopmulticolumn
&&
\sphinxAtStartPar
code
\\
\sphinxhline
\sphinxAtStartPar

&&
\sphinxAtStartPar
code
&
\sphinxAtStartPar
integer
&\\
\sphinxhline
\sphinxAtStartPar

&&
\sphinxAtStartPar
name@en
&
\sphinxAtStartPar
string
&\\
\sphinxhline\sphinxstartmulticolumn{3}%
\begin{varwidth}[t]{\sphinxcolwidth{3}{5}}
\sphinxAtStartPar
ns1
\par
\vskip-\baselineskip\vbox{\hbox{\strut}}\end{varwidth}%
\sphinxstopmulticolumn
&&\\
\sphinxhline
\sphinxAtStartPar

&\sphinxstartmulticolumn{2}%
\begin{varwidth}[t]{\sphinxcolwidth{2}{5}}
\sphinxAtStartPar
\sphinxstylestrong{Gyvenviete}
\par
\vskip-\baselineskip\vbox{\hbox{\strut}}\end{varwidth}%
\sphinxstopmulticolumn
&
\sphinxAtStartPar
\sphinxstylestrong{/locn/Location}
&
\sphinxAtStartPar
code
\\
\sphinxhline
\sphinxAtStartPar

&&
\sphinxAtStartPar
code
&
\sphinxAtStartPar
integer
&\\
\sphinxhline
\sphinxAtStartPar

&&
\sphinxAtStartPar
name@lt
&
\sphinxAtStartPar
string
&\\
\sphinxhline\sphinxstartmulticolumn{3}%
\begin{varwidth}[t]{\sphinxcolwidth{3}{5}}
\sphinxAtStartPar
ns2
\par
\vskip-\baselineskip\vbox{\hbox{\strut}}\end{varwidth}%
\sphinxstopmulticolumn
&&\\
\sphinxhline
\sphinxAtStartPar

&\sphinxstartmulticolumn{2}%
\begin{varwidth}[t]{\sphinxcolwidth{2}{5}}
\sphinxAtStartPar
\sphinxstylestrong{Gyvenviete}
\par
\vskip-\baselineskip\vbox{\hbox{\strut}}\end{varwidth}%
\sphinxstopmulticolumn
&
\sphinxAtStartPar
\sphinxstylestrong{/locn/Location}
&
\sphinxAtStartPar
code
\\
\sphinxhline
\sphinxAtStartPar

&&
\sphinxAtStartPar
code
&
\sphinxAtStartPar
integer
&\\
\sphinxhline
\sphinxAtStartPar

&&
\sphinxAtStartPar
name@lt
&
\sphinxAtStartPar
string
&\\
\sphinxbottomrule
\end{tabulary}
\sphinxtableafterendhook\par
\sphinxattableend\end{savenotes}

\sphinxAtStartPar
Pavyzdyje turime tris modelius iš skirtingų duomenų rinkinių,
\sphinxcode{\sphinxupquote{ns1:Gyvenviete}} ir \sphinxcode{\sphinxupquote{ns2:Gyvenviete}} nurodo \sphinxcode{\sphinxupquote{locn:Location}} kaip šių
modelių bazę, tai reiškia, kad visi trys modeliai realiame pasaulyje
yra viena esybė, turinti vienodus identifikatorius skirtinguose
modeliuose.
\end{sphinxadmonition}

\end{fulllineitems}

\index{ref (modulje model)@\spxentry{ref}\spxextra{modulje model}}

\begin{fulllineitems}
\phantomsection\label{\detokenize{dimensijos:model.ref}}
\pysigstartsignatures
\pysigline
{\sphinxcode{\sphinxupquote{model.}}\sphinxbfcode{\sphinxupquote{ref}}}
\pysigstopsignatures
\sphinxAtStartPar
Kableliu atskirtas sąrašas {\hyperref[\detokenize{dimensijos:model.property}]{\sphinxcrossref{\sphinxcode{\sphinxupquote{model.property}}}}} duomenų laukų pavadinimų,
kurie kartu unikaliai identifikuoja vieną duomenų eilutę (pirminis lentelės
raktas arba identifikatorius).

\sphinxAtStartPar
Jei nurodytas {\hyperref[\detokenize{dimensijos:model.type}]{\sphinxcrossref{\sphinxcode{\sphinxupquote{model.type}}}}}, pirminis raktas būtinai turi sutapti su
{\hyperref[\detokenize{dimensijos:model.type}]{\sphinxcrossref{\sphinxcode{\sphinxupquote{model.type}}}}} pirminiu raktu.

\sphinxAtStartPar
Jei modelio objektą unikaliai identifikuoja keli duomenų laukai,
{\hyperref[\detokenize{dimensijos:model.ref}]{\sphinxcrossref{\sphinxcode{\sphinxupquote{model.ref}}}}} stulpelyje galima nurodyti kelis duomenų laukus atskirtus
kableliu.

\begin{sphinxtopic}
\sphinxstyletopictitle{Brandos lygis}
\begin{description}
\sphinxlineitem{{\hyperref[\detokenize{branda:l003}]{\sphinxcrossref{\DUrole{std}{\DUrole{std-ref}{L003: Nėra identifikatoriaus}}}}}}
\sphinxAtStartPar
Nenurodytas objekto identifikatorius.

\sphinxlineitem{{\hyperref[\detokenize{branda:l104}]{\sphinxcrossref{\DUrole{std}{\DUrole{std-ref}{L104: Identifikatorius nėra unikalus}}}}}}
\sphinxAtStartPar
Nurodytas objekto identifikatorius nėra unikalus, turi
pasikartojančių reikšmių.

\sphinxlineitem{{\hyperref[\detokenize{branda:l204}]{\sphinxcrossref{\DUrole{std}{\DUrole{std-ref}{L204: Nepatikimi identifikatoriai}}}}}}
\sphinxAtStartPar
Nurodytas objekto identifikatorius yra unikalus, tačiau
nepatikimas, kadangi nurodytas duomenų laukas, kuris gali keistis,
tarkime pavadinimas.

\sphinxlineitem{{\hyperref[\detokenize{branda:l301}]{\sphinxcrossref{\DUrole{std}{\DUrole{std-ref}{L301: Nėra globalaus objekto identifikatoriaus}}}}}}
\sphinxAtStartPar
Nurodytas objekto identifikatorius yra patikimas, tačiau nėra
siejamas su globaliu objekto identifikatoriumi.

\end{description}
\end{sphinxtopic}

\end{fulllineitems}

\index{source (modulje model)@\spxentry{source}\spxextra{modulje model}}

\begin{fulllineitems}
\phantomsection\label{\detokenize{dimensijos:model.source}}
\pysigstartsignatures
\pysigline
{\sphinxcode{\sphinxupquote{model.}}\sphinxbfcode{\sphinxupquote{source}}}
\pysigstopsignatures
\sphinxAtStartPar
Modelio duomenų šaltinis, vieta ar pavadinimas fiziniame duomenų modelyje.

\sphinxAtStartPar
Kas įrašoma į šį stulpelį priklauso nuo duomenų šaltinio {\hyperref[\detokenize{dimensijos:resource.type}]{\sphinxcrossref{\sphinxcode{\sphinxupquote{resource.type}}}}}.

\sphinxAtStartPar
SQL atveju, tai bus lentelės pavadinimas, XML atveju \sphinxhyphen{} XPath išraiška, JSON
atveju \sphinxhyphen{} JSONPath išraiška, skirtingi duomenų šaltiniai gali naudoti
skirtingą sintaksę vietai (kur fiziškai saugomi duomenys) apibūdinti.

\sphinxAtStartPar
Jei duomenys publikuojami \DUrole{xref}{\DUrole{std}{\DUrole{std-ref}{vidinėje saugykloje}}},
{\hyperref[\detokenize{dimensijos:model.source}]{\sphinxcrossref{\sphinxcode{\sphinxupquote{model.source}}}}} pildyti nereikia, kadangi vidinės saugyklos fizinio ir
loginio modelio pavadinimai yra tokie patys.

\begin{sphinxtopic}
\sphinxstyletopictitle{Brandos lygis}
\begin{description}
\sphinxlineitem{{\hyperref[\detokenize{branda:l004}]{\sphinxcrossref{\DUrole{std}{\DUrole{std-ref}{L004: Duomenų nėra}}}}}}
\sphinxAtStartPar
Nenurodytas modelio duomenų šaltinis {\hyperref[\detokenize{dimensijos:model.source}]{\sphinxcrossref{\sphinxcode{\sphinxupquote{model.source}}}}} ir duomenys nėra
publikuojami \DUrole{xref}{\DUrole{std}{\DUrole{std-ref}{vidinėje saugykloje}}}.

\end{description}
\end{sphinxtopic}

\end{fulllineitems}

\index{prepare (modulje model)@\spxentry{prepare}\spxextra{modulje model}}

\begin{fulllineitems}
\phantomsection\label{\detokenize{dimensijos:model.prepare}}
\pysigstartsignatures
\pysigline
{\sphinxcode{\sphinxupquote{model.}}\sphinxbfcode{\sphinxupquote{prepare}}}
\pysigstopsignatures
\sphinxAtStartPar
Formulė skirta duomenų filtravimui ir paruošimui, iš dalies priklauso nuo
{\hyperref[\detokenize{dimensijos:resource.type}]{\sphinxcrossref{\sphinxcode{\sphinxupquote{resource.type}}}}}.


\begin{sphinxseealso}{Taip pat žiūrėkite:}

\begin{DUlineblock}{0em}
\item[] {\hyperref[\detokenize{formules:formules}]{\sphinxcrossref{\DUrole{std}{\DUrole{std-ref}{Formulės}}}}}
\item[] {\hyperref[\detokenize{formules:duomenu-atranka}]{\sphinxcrossref{\DUrole{std}{\DUrole{std-ref}{Duomenų atranka}}}}}
\end{DUlineblock}


\end{sphinxseealso}


\end{fulllineitems}

\index{level (modulje model)@\spxentry{level}\spxextra{modulje model}}

\begin{fulllineitems}
\phantomsection\label{\detokenize{dimensijos:model.level}}
\pysigstartsignatures
\pysigline
{\sphinxcode{\sphinxupquote{model.}}\sphinxbfcode{\sphinxupquote{level}}}
\pysigstopsignatures
\sphinxAtStartPar
Modelio {\hyperref[\detokenize{branda:level}]{\sphinxcrossref{\DUrole{std}{\DUrole{std-ref}{brandos lygis}}}}}, nusakantis pačio modelio brandos
lygį, pavyzdžiui ar nurodytas pirminis raktas, ar modelio pavadinimas
atitinka kodiniams pavadinimams keliamus reikalavimus.


\begin{sphinxseealso}{Taip pat žiūrėkite:}

\sphinxAtStartPar
{\hyperref[\detokenize{identifikatoriai:ref-level}]{\sphinxcrossref{\DUrole{std}{\DUrole{std-ref}{Brandos lygis}}}}}


\end{sphinxseealso}


\end{fulllineitems}

\index{access (modulje model)@\spxentry{access}\spxextra{modulje model}}

\begin{fulllineitems}
\phantomsection\label{\detokenize{dimensijos:model.access}}
\pysigstartsignatures
\pysigline
{\sphinxcode{\sphinxupquote{model.}}\sphinxbfcode{\sphinxupquote{access}}}
\pysigstopsignatures
\sphinxAtStartPar
Modeliui priklausančių laukų {\hyperref[\detokenize{prieiga:access}]{\sphinxcrossref{\DUrole{std}{\DUrole{std-ref}{prieigos lygis}}}}}.

\sphinxAtStartPar
Modelio prieigos lygis yra išskaičiuojamas iš modeliui priskirtų duomenų laukų, imant didžiausią prieigos lygmenį nurodytą prie duomenų lauko. Pavyzdžiui, jei bent vienas duomenų laukas turi aukščiausią \sphinxcode{\sphinxupquote{open}} prieigos lygmenį, tada ir viso modelio prieigos lygis tampa \sphinxcode{\sphinxupquote{open}}.


\begin{sphinxseealso}{Taip pat žiūrėkite:}

\sphinxAtStartPar
{\hyperref[\detokenize{prieiga:access}]{\sphinxcrossref{\DUrole{std}{\DUrole{std-ref}{Prieigos lygiai}}}}}


\end{sphinxseealso}


\end{fulllineitems}

\index{uri (modulje model)@\spxentry{uri}\spxextra{modulje model}}

\begin{fulllineitems}
\phantomsection\label{\detokenize{dimensijos:model.uri}}
\pysigstartsignatures
\pysigline
{\sphinxcode{\sphinxupquote{model.}}\sphinxbfcode{\sphinxupquote{uri}}}
\pysigstopsignatures
\sphinxAtStartPar
Sąsaja su \sphinxhref{https://www.w3.org/TR/owl2-overview/}{OWL}, \sphinxhref{https://www.w3.org/TR/rdf-schema/}{RDFS} ontologijomis ar \sphinxhref{https://www.w3.org/TR/skos-primer/}{SKOS} kontroliuojamais žodynais.

\sphinxAtStartPar
Jei nenurodyta, generuojamas pavadinimas pagal tokį šabloną:

\sphinxAtStartPar
\sphinxcode{\sphinxupquote{https://data.gov.lt/id/}} \sphinxstylestrong{dataset} \sphinxcode{\sphinxupquote{/}} \sphinxstylestrong{model}

\begin{sphinxadmonition}{note}{Pavyzdys}

\begin{sphinxVerbatim}[commandchars=\\\{\}]
\PYG{n}{https}\PYG{p}{:}\PYG{o}{/}\PYG{o}{/}\PYG{n}{data}\PYG{o}{.}\PYG{n}{gov}\PYG{o}{.}\PYG{n}{lt}\PYG{o}{/}\PYG{n+nb}{id}\PYG{o}{/}\PYG{n}{datasets}\PYG{o}{/}\PYG{n}{gov}\PYG{o}{/}\PYG{n}{rc}\PYG{o}{/}\PYG{n}{ar}\PYG{o}{/}\PYG{n}{ws}\PYG{o}{/}\PYG{n}{Location}
\end{sphinxVerbatim}
\end{sphinxadmonition}

\sphinxAtStartPar
Struktūros apraše galima nurodyti automatiškai generuojamus URI.

\begin{sphinxadmonition}{note}{Pavyzdys}


\begin{savenotes}\sphinxattablestart
\sphinxthistablewithglobalstyle
\centering
\begin{tabulary}{\linewidth}[t]{TTTTTT}
\sphinxtoprule
\sphinxstyletheadfamily 
\sphinxAtStartPar
dataset
&\sphinxstyletheadfamily 
\sphinxAtStartPar
model
&\sphinxstyletheadfamily 
\sphinxAtStartPar
property
&\sphinxstyletheadfamily 
\sphinxAtStartPar
type
&\sphinxstyletheadfamily 
\sphinxAtStartPar
ref
&\sphinxstyletheadfamily 
\sphinxAtStartPar
uri
\\
\sphinxmidrule
\sphinxtableatstartofbodyhook\sphinxstartmulticolumn{3}%
\begin{varwidth}[t]{\sphinxcolwidth{3}{6}}
\sphinxAtStartPar
adresai
\par
\vskip-\baselineskip\vbox{\hbox{\strut}}\end{varwidth}%
\sphinxstopmulticolumn
&&&\\
\sphinxhline
\sphinxAtStartPar

&&&
\sphinxAtStartPar
prefix
&
\sphinxAtStartPar
ar
&
\sphinxAtStartPar
https://data.gov.lt/id/adresai/
\\
\sphinxhline
\sphinxAtStartPar

&\sphinxstartmulticolumn{2}%
\begin{varwidth}[t]{\sphinxcolwidth{2}{6}}
\sphinxAtStartPar
\sphinxstylestrong{Location}
\par
\vskip-\baselineskip\vbox{\hbox{\strut}}\end{varwidth}%
\sphinxstopmulticolumn
&&
\sphinxAtStartPar
code
&\\
\sphinxhline
\sphinxAtStartPar

&&
\sphinxAtStartPar
code
&
\sphinxAtStartPar
integer
&&
\sphinxAtStartPar
ar:code
\\
\sphinxhline
\sphinxAtStartPar

&&
\sphinxAtStartPar
name@en
&
\sphinxAtStartPar
string
&&
\sphinxAtStartPar
ar:name
\\
\sphinxhline\sphinxstartmulticolumn{3}%
\begin{varwidth}[t]{\sphinxcolwidth{3}{6}}
\sphinxAtStartPar
datasets/gov/ivpk/dp/api
\par
\vskip-\baselineskip\vbox{\hbox{\strut}}\end{varwidth}%
\sphinxstopmulticolumn
&&&\\
\sphinxhline
\sphinxAtStartPar

&&&
\sphinxAtStartPar
prefix
&
\sphinxAtStartPar
ar
&
\sphinxAtStartPar
https://data.gov.lt/id/adresai/
\\
\sphinxhline
\sphinxAtStartPar

&\sphinxstartmulticolumn{2}%
\begin{varwidth}[t]{\sphinxcolwidth{2}{6}}
\sphinxAtStartPar
\sphinxstylestrong{Gyvenviete}
\par
\vskip-\baselineskip\vbox{\hbox{\strut}}\end{varwidth}%
\sphinxstopmulticolumn
&&
\sphinxAtStartPar
code
&
\sphinxAtStartPar
ar:Location
\\
\sphinxhline
\sphinxAtStartPar

&&
\sphinxAtStartPar
code
&
\sphinxAtStartPar
integer
&&
\sphinxAtStartPar
ar:code
\\
\sphinxhline
\sphinxAtStartPar

&&
\sphinxAtStartPar
name@lt
&
\sphinxAtStartPar
string
&&
\sphinxAtStartPar
ar:name
\\
\sphinxbottomrule
\end{tabulary}
\sphinxtableafterendhook\par
\sphinxattableend\end{savenotes}

\sphinxAtStartPar
Šiame pavyzdyje \sphinxcode{\sphinxupquote{ar:Location}} yra URI, kuris yra automatiškai
generuojamas \sphinxcode{\sphinxupquote{adresai}} duomenų rinkinyje.
\end{sphinxadmonition}


\begin{sphinxseealso}{Taip pat žiūrėkite:}

\sphinxAtStartPar
{\hyperref[\detokenize{zodynai:vocab}]{\sphinxcrossref{\DUrole{std}{\DUrole{std-ref}{Išoriniai žodynai}}}}}


\end{sphinxseealso}


\end{fulllineitems}

\index{title (modulje model)@\spxentry{title}\spxextra{modulje model}}

\begin{fulllineitems}
\phantomsection\label{\detokenize{dimensijos:model.title}}
\pysigstartsignatures
\pysigline
{\sphinxcode{\sphinxupquote{model.}}\sphinxbfcode{\sphinxupquote{title}}}
\pysigstopsignatures
\sphinxAtStartPar
Trumpas modelio pavadinimas pirmas žodis iš didžiosios raidės, pavadinimo
gale taško nereikia.

\sphinxAtStartPar
Pavadinime nereikia kartoti duomenų rinkinio pavadinimo. Modelio
pavadinimas rašomas duomenų rinkinio kontekste.

\end{fulllineitems}

\index{description (modulje model)@\spxentry{description}\spxextra{modulje model}}

\begin{fulllineitems}
\phantomsection\label{\detokenize{dimensijos:model.description}}
\pysigstartsignatures
\pysigline
{\sphinxcode{\sphinxupquote{model.}}\sphinxbfcode{\sphinxupquote{description}}}
\pysigstopsignatures
\sphinxAtStartPar
Modelio aprašymas.

\end{fulllineitems}

\index{property (modulje model)@\spxentry{property}\spxextra{modulje model}}

\begin{fulllineitems}
\phantomsection\label{\detokenize{dimensijos:model.property}}
\pysigstartsignatures
\pysigline
{\sphinxcode{\sphinxupquote{model.}}\sphinxbfcode{\sphinxupquote{property}}}
\pysigstopsignatures
\sphinxAtStartPar
Modeliui priklausantis duomenų laukas.

\end{fulllineitems}



\subsubsection{Funkcijos}
\label{\detokenize{dimensijos:id6}}\index{module@\spxentry{module}!model@\spxentry{model}}\index{model@\spxentry{model}!module@\spxentry{module}}\index{distinct() (modulyje model)@\spxentry{distinct()}\spxextra{modulyje model}}\phantomsection\label{\detokenize{dimensijos:module-1}}

\begin{fulllineitems}
\phantomsection\label{\detokenize{dimensijos:model.distinct}}
\pysigstartsignatures
\pysiglinewithargsret
{\sphinxcode{\sphinxupquote{model.}}\sphinxbfcode{\sphinxupquote{distinct}}}
{}
{}
\pysigstopsignatures
\sphinxAtStartPar
Jei {\hyperref[\detokenize{dimensijos:model.ref}]{\sphinxcrossref{\sphinxcode{\sphinxupquote{model.ref}}}}} pirminis raktas nėra unikalus ir norma panaikinti
besidubliuojančias reikšmes, galima nurodyti \sphinxcode{\sphinxupquote{distinct()}} funkciją, kuri
panaidins objktus su besidubliuojančiais pirminiais raktais.

\begin{sphinxadmonition}{note}{Pavyzdys}

\sphinxAtStartPar
Turint tokius duomenis duomenų šaltinyje:


\begin{savenotes}\sphinxattablestart
\sphinxthistablewithglobalstyle
\centering
\begin{tabulary}{\linewidth}[t]{TT}
\sphinxtoprule
\sphinxstyletheadfamily 
\sphinxAtStartPar
CITY
&\sphinxstyletheadfamily 
\sphinxAtStartPar
COUNTRY
\\
\sphinxmidrule
\sphinxtableatstartofbodyhook
\sphinxAtStartPar
Vilnius
&
\sphinxAtStartPar
Lithuania
\\
\sphinxhline
\sphinxAtStartPar
Kaunas
&
\sphinxAtStartPar
Lithuania
\\
\sphinxbottomrule
\end{tabulary}
\sphinxtableafterendhook\par
\sphinxattableend\end{savenotes}

\sphinxAtStartPar
Ir struktūros aprašą, kuriame \sphinxcode{\sphinxupquote{COUNTY}} aprašytas, kaip atskiras
modelis:


\begin{savenotes}\sphinxattablestart
\sphinxthistablewithglobalstyle
\centering
\begin{tabulary}{\linewidth}[t]{TTTTTTTT}
\sphinxtoprule
\sphinxstyletheadfamily 
\sphinxAtStartPar
model
&\sphinxstyletheadfamily 
\sphinxAtStartPar
property
&\sphinxstyletheadfamily 
\sphinxAtStartPar
type
&\sphinxstyletheadfamily 
\sphinxAtStartPar
ref
&\sphinxstyletheadfamily 
\sphinxAtStartPar
source
&\sphinxstyletheadfamily 
\sphinxAtStartPar
prepare
&\sphinxstyletheadfamily 
\sphinxAtStartPar
level
&\sphinxstyletheadfamily 
\sphinxAtStartPar
access
\\
\sphinxmidrule
\sphinxtableatstartofbodyhook\sphinxstartmulticolumn{2}%
\begin{varwidth}[t]{\sphinxcolwidth{2}{8}}
\sphinxAtStartPar
\sphinxstylestrong{Country}
\par
\vskip-\baselineskip\vbox{\hbox{\strut}}\end{varwidth}%
\sphinxstopmulticolumn
&&
\sphinxAtStartPar
name@en
&
\sphinxAtStartPar
CITIES
&
\sphinxAtStartPar
\sphinxcode{\sphinxupquote{distinct()}}
&
\sphinxAtStartPar
4
&\\
\sphinxhline
\sphinxAtStartPar

&
\sphinxAtStartPar
name@en
&
\sphinxAtStartPar
string
&&
\sphinxAtStartPar
COUNTRY
&&
\sphinxAtStartPar
4
&
\sphinxAtStartPar
open
\\
\sphinxhline\sphinxstartmulticolumn{2}%
\begin{varwidth}[t]{\sphinxcolwidth{2}{8}}
\sphinxAtStartPar
\sphinxstylestrong{City}
\par
\vskip-\baselineskip\vbox{\hbox{\strut}}\end{varwidth}%
\sphinxstopmulticolumn
&&
\sphinxAtStartPar
name@en
&
\sphinxAtStartPar
CITIES
&&
\sphinxAtStartPar
4
&\\
\sphinxhline
\sphinxAtStartPar

&
\sphinxAtStartPar
name@en
&
\sphinxAtStartPar
string
&&
\sphinxAtStartPar
CITY
&&
\sphinxAtStartPar
4
&
\sphinxAtStartPar
open
\\
\sphinxhline
\sphinxAtStartPar

&
\sphinxAtStartPar
country
&
\sphinxAtStartPar
ref
&
\sphinxAtStartPar
\sphinxstylestrong{Country}
&
\sphinxAtStartPar
COUNTRY
&&
\sphinxAtStartPar
3
&
\sphinxAtStartPar
open
\\
\sphinxbottomrule
\end{tabulary}
\sphinxtableafterendhook\par
\sphinxattableend\end{savenotes}

\sphinxAtStartPar
\sphinxcode{\sphinxupquote{distinct()}} funkcija panaikina besidubliuojančius objektus ir grąžina
tik vieną šalį.
\end{sphinxadmonition}

\end{fulllineitems}



\subsection{property}
\label{\detokenize{dimensijos:property}}\label{\detokenize{dimensijos:id7}}\index{module@\spxentry{module}!property@\spxentry{property}}\index{property@\spxentry{property}!module@\spxentry{module}}\phantomsection\label{\detokenize{dimensijos:module-property}}
\sphinxAtStartPar
Savybė yra duomenų laukas, modelio atributas.
\index{source (modulje property)@\spxentry{source}\spxextra{modulje property}}

\begin{fulllineitems}
\phantomsection\label{\detokenize{dimensijos:property.source}}
\pysigstartsignatures
\pysigline
{\sphinxcode{\sphinxupquote{property.}}\sphinxbfcode{\sphinxupquote{source}}}
\pysigstopsignatures
\sphinxAtStartPar
Duomenų lauko pavadinimas šaltinyje. Prasmė priklauso nuo
{\hyperref[\detokenize{dimensijos:resource.type}]{\sphinxcrossref{\sphinxcode{\sphinxupquote{resource.type}}}}}.

\end{fulllineitems}

\index{prepare (modulje property)@\spxentry{prepare}\spxextra{modulje property}}

\begin{fulllineitems}
\phantomsection\label{\detokenize{dimensijos:property.prepare}}
\pysigstartsignatures
\pysigline
{\sphinxcode{\sphinxupquote{property.}}\sphinxbfcode{\sphinxupquote{prepare}}}
\pysigstopsignatures
\sphinxAtStartPar
Formulė skirta duomenų tikrinimui ir transformavimui arba statinės reikšmės
pateikimui.

\end{fulllineitems}

\index{type (modulje property)@\spxentry{type}\spxextra{modulje property}}

\begin{fulllineitems}
\phantomsection\label{\detokenize{dimensijos:property.type}}
\pysigstartsignatures
\pysigline
{\sphinxcode{\sphinxupquote{property.}}\sphinxbfcode{\sphinxupquote{type}}}
\pysigstopsignatures
\sphinxAtStartPar
Nurodomas loginis duomenų tipas. Dėl galimų tipų sąrašo žiūrėti
{\hyperref[\detokenize{tipai:duomenu-tipai}]{\sphinxcrossref{\DUrole{std}{\DUrole{std-ref}{Duomenų tipai}}}}}.

\sphinxAtStartPar
Loginis duomenų tipas yra toks tipas, kurį tikitės gauti publikuojant
duomenis per API. Loginis tipas gali skirtis nuo duomenų šaltinio tipo.

\sphinxAtStartPar
Visi duomenų tipai gali turėti tokius parametrus:
\begin{itemize}
\item {} 
\sphinxAtStartPar
\sphinxcode{\sphinxupquote{required}} \sphinxhyphen{} nurodo, kad šis duomenų laukas yra privalomas, tai reiškia,
kad šio duomenų lauko reikšmė visada turi būti pateikta. Pagal nutylėjimą
visi modelio duomenų laukai yra neprivalomi.

\end{itemize}

\sphinxAtStartPar
Kai kurie duomenų tipai, gali turėti konkrečiam duomenų tipui pateikiamus
papildomus parametrus, tokie parametrai nurodomi skliausteliuose.

\sphinxAtStartPar
Dupmenų tipų pavyzdžiai:
\begin{itemize}
\item {} 
\sphinxAtStartPar
\sphinxcode{\sphinxupquote{integer}}

\item {} 
\sphinxAtStartPar
\sphinxcode{\sphinxupquote{integer required}}

\item {} 
\sphinxAtStartPar
\sphinxcode{\sphinxupquote{geometry}}

\item {} 
\sphinxAtStartPar
\sphinxcode{\sphinxupquote{geometry(linestringm, 3345) required}}

\end{itemize}

\end{fulllineitems}

\index{ref (modulje property)@\spxentry{ref}\spxextra{modulje property}}

\begin{fulllineitems}
\phantomsection\label{\detokenize{dimensijos:property.ref}}
\pysigstartsignatures
\pysigline
{\sphinxcode{\sphinxupquote{property.}}\sphinxbfcode{\sphinxupquote{ref}}}
\pysigstopsignatures
\sphinxAtStartPar
Priklauso nuo \sphinxcode{\sphinxupquote{property.type}}, nurodo matavimo vienetus, laiko ar vietos
tikslumą, {\hyperref[\detokenize{dimensijos:enum}]{\sphinxcrossref{\DUrole{std}{\DUrole{std-ref}{klasifikatorių}}}}} arba {\hyperref[\detokenize{identifikatoriai:rysiai}]{\sphinxcrossref{\DUrole{std}{\DUrole{std-ref}{ryšį su kitais modeliais}}}}}. Ką tiksliai reiškia šis laukas, patikslinta skyrelyje
{\hyperref[\detokenize{tipai:duomenu-tipai}]{\sphinxcrossref{\DUrole{std}{\DUrole{std-ref}{Duomenų tipai}}}}}.

\end{fulllineitems}

\index{level (modulje property)@\spxentry{level}\spxextra{modulje property}}

\begin{fulllineitems}
\phantomsection\label{\detokenize{dimensijos:property.level}}
\pysigstartsignatures
\pysigline
{\sphinxcode{\sphinxupquote{property.}}\sphinxbfcode{\sphinxupquote{level}}}
\pysigstopsignatures
\sphinxAtStartPar
Nurodo duomenų lauko brandos lygį. Žiūrėti {\hyperref[\detokenize{branda:level}]{\sphinxcrossref{\DUrole{std}{\DUrole{std-ref}{Brandos lygiai}}}}}.

\end{fulllineitems}

\index{access (modulje property)@\spxentry{access}\spxextra{modulje property}}

\begin{fulllineitems}
\phantomsection\label{\detokenize{dimensijos:property.access}}
\pysigstartsignatures
\pysigline
{\sphinxcode{\sphinxupquote{property.}}\sphinxbfcode{\sphinxupquote{access}}}
\pysigstopsignatures
\sphinxAtStartPar
Nurodo prieigos prie duomenų lygį. Žiūrėti skyrių {\hyperref[\detokenize{prieiga:access}]{\sphinxcrossref{\DUrole{std}{\DUrole{std-ref}{Prieigos lygiai}}}}}.

\end{fulllineitems}

\index{uri (modulje property)@\spxentry{uri}\spxextra{modulje property}}

\begin{fulllineitems}
\phantomsection\label{\detokenize{dimensijos:property.uri}}
\pysigstartsignatures
\pysigline
{\sphinxcode{\sphinxupquote{property.}}\sphinxbfcode{\sphinxupquote{uri}}}
\pysigstopsignatures
\sphinxAtStartPar
Sąsaja su išoriniu žodynu. Žiūrėti {\hyperref[\detokenize{zodynai:vocab}]{\sphinxcrossref{\DUrole{std}{\DUrole{std-ref}{Išoriniai žodynai}}}}}.

\end{fulllineitems}

\index{title (modulje property)@\spxentry{title}\spxextra{modulje property}}

\begin{fulllineitems}
\phantomsection\label{\detokenize{dimensijos:property.title}}
\pysigstartsignatures
\pysigline
{\sphinxcode{\sphinxupquote{property.}}\sphinxbfcode{\sphinxupquote{title}}}
\pysigstopsignatures
\sphinxAtStartPar
Duomenų lauko pavadinimas. Šis pavadinimas yra skirtas skaityti žmonėms
ir bus rodomas duomenų laukų sąrašuose ir antraštėse. Jei nenurodyta, bus
naudojamas {\hyperref[\detokenize{formatas:property}]{\sphinxcrossref{\sphinxcode{\sphinxupquote{property}}}}} kodinis pavadinimas.

\end{fulllineitems}

\index{description (modulje property)@\spxentry{description}\spxextra{modulje property}}

\begin{fulllineitems}
\phantomsection\label{\detokenize{dimensijos:property.description}}
\pysigstartsignatures
\pysigline
{\sphinxcode{\sphinxupquote{property.}}\sphinxbfcode{\sphinxupquote{description}}}
\pysigstopsignatures
\sphinxAtStartPar
Duomenų lauko aprašymas.

\end{fulllineitems}

\index{enum (modulje property)@\spxentry{enum}\spxextra{modulje property}}

\begin{fulllineitems}
\phantomsection\label{\detokenize{dimensijos:property.enum}}
\pysigstartsignatures
\pysigline
{\sphinxcode{\sphinxupquote{property.}}\sphinxbfcode{\sphinxupquote{enum}}}
\pysigstopsignatures
\sphinxAtStartPar
Žiūrėti {\hyperref[\detokenize{dimensijos:enum}]{\sphinxcrossref{\DUrole{std}{\DUrole{std-ref}{enum}}}}}.

\end{fulllineitems}



\section{Papildomos dimensijos}
\label{\detokenize{dimensijos:papildomos-dimensijos}}\label{\detokenize{dimensijos:id8}}

\subsection{prefix}
\label{\detokenize{dimensijos:prefix}}\label{\detokenize{dimensijos:id9}}\index{module@\spxentry{module}!prefix@\spxentry{prefix}}\index{prefix@\spxentry{prefix}!module@\spxentry{module}}\phantomsection\label{\detokenize{dimensijos:module-prefix}}
\sphinxAtStartPar
Sąsają su išoriniais žodynais galima pateikti {\hyperref[\detokenize{dimensijos:model.uri}]{\sphinxcrossref{\sphinxcode{\sphinxupquote{model.uri}}}}} ir
{\hyperref[\detokenize{dimensijos:property.uri}]{\sphinxcrossref{\sphinxcode{\sphinxupquote{property.uri}}}}} stulpeliuose. Tačiau prieš naudojant žodynus, pirmiausia
reikia apsirašyti žodynų prefiksus. Žodynų prefiksai aprašomi taip:
\index{ref (modulje prefix)@\spxentry{ref}\spxextra{modulje prefix}}

\begin{fulllineitems}
\phantomsection\label{\detokenize{dimensijos:prefix.ref}}
\pysigstartsignatures
\pysigline
{\sphinxcode{\sphinxupquote{prefix.}}\sphinxbfcode{\sphinxupquote{ref}}}
\pysigstopsignatures
\sphinxAtStartPar
Prefikso pavadinimas.

\sphinxAtStartPar
Rekomenduojama naudoti \sphinxhref{https://prefix.cc/}{prefix.cc} paslaugą URI prefiksų pavadinimams.

\end{fulllineitems}

\index{uri (modulje prefix)@\spxentry{uri}\spxextra{modulje prefix}}

\begin{fulllineitems}
\phantomsection\label{\detokenize{dimensijos:prefix.uri}}
\pysigstartsignatures
\pysigline
{\sphinxcode{\sphinxupquote{prefix.}}\sphinxbfcode{\sphinxupquote{uri}}}
\pysigstopsignatures
\sphinxAtStartPar
Išorinio žodyno URI.

\end{fulllineitems}

\index{title (modulje prefix)@\spxentry{title}\spxextra{modulje prefix}}

\begin{fulllineitems}
\phantomsection\label{\detokenize{dimensijos:prefix.title}}
\pysigstartsignatures
\pysigline
{\sphinxcode{\sphinxupquote{prefix.}}\sphinxbfcode{\sphinxupquote{title}}}
\pysigstopsignatures
\sphinxAtStartPar
Prefikso antraštė.

\end{fulllineitems}

\index{description (modulje prefix)@\spxentry{description}\spxextra{modulje prefix}}

\begin{fulllineitems}
\phantomsection\label{\detokenize{dimensijos:prefix.description}}
\pysigstartsignatures
\pysigline
{\sphinxcode{\sphinxupquote{prefix.}}\sphinxbfcode{\sphinxupquote{description}}}
\pysigstopsignatures
\sphinxAtStartPar
Prefikso aprašymas.

\end{fulllineitems}


\sphinxAtStartPar
Rekomenduojama naudoti \sphinxhref{https://lov.linkeddata.es/dataset/lov/}{LOV} prefiksus.

\sphinxAtStartPar
Aprašyti prefiksai gali būti naudojami {\hyperref[\detokenize{dimensijos:model.uri}]{\sphinxcrossref{\sphinxcode{\sphinxupquote{model.uri}}}}} ir {\hyperref[\detokenize{dimensijos:property.uri}]{\sphinxcrossref{\sphinxcode{\sphinxupquote{property.uri}}}}}
stulpeliuose tokiu būdu: \sphinxcode{\sphinxupquote{prefix:name}}.

\sphinxAtStartPar
Pavyzdys:


\begin{savenotes}\sphinxattablestart
\sphinxthistablewithglobalstyle
\centering
\begin{tabulary}{\linewidth}[t]{TTTTTTTT}
\sphinxtoprule
\sphinxstyletheadfamily 
\sphinxAtStartPar
d
&\sphinxstyletheadfamily 
\sphinxAtStartPar
r
&\sphinxstyletheadfamily 
\sphinxAtStartPar
b
&\sphinxstyletheadfamily 
\sphinxAtStartPar
m
&\sphinxstyletheadfamily 
\sphinxAtStartPar
property
&\sphinxstyletheadfamily 
\sphinxAtStartPar
type
&\sphinxstyletheadfamily 
\sphinxAtStartPar
ref
&\sphinxstyletheadfamily 
\sphinxAtStartPar
uri
\\
\sphinxmidrule
\sphinxtableatstartofbodyhook\sphinxstartmulticolumn{5}%
\begin{varwidth}[t]{\sphinxcolwidth{5}{8}}
\sphinxAtStartPar
dataset1
\par
\vskip-\baselineskip\vbox{\hbox{\strut}}\end{varwidth}%
\sphinxstopmulticolumn
&&&\\
\sphinxhline
\sphinxAtStartPar

&&&&&
\sphinxAtStartPar
prefix
&
\sphinxAtStartPar
spinta
&
\sphinxAtStartPar
https://github.com/atviriduomenys/spinta/issues/
\\
\sphinxhline
\sphinxAtStartPar

&&&&&&
\sphinxAtStartPar
dsa
&
\sphinxAtStartPar
https://ivpk.github.io/dsa/
\\
\sphinxhline
\sphinxAtStartPar

&&&&&&
\sphinxAtStartPar
dct
&
\sphinxAtStartPar
http://purl.org/dc/dcmitype/
\\
\sphinxhline\sphinxstartmulticolumn{5}%
\begin{varwidth}[t]{\sphinxcolwidth{5}{8}}
\sphinxAtStartPar
dataset2
\par
\vskip-\baselineskip\vbox{\hbox{\strut}}\end{varwidth}%
\sphinxstopmulticolumn
&&&\\
\sphinxhline
\sphinxAtStartPar

&&&&&
\sphinxAtStartPar
prefix
&
\sphinxAtStartPar
dcat
&
\sphinxAtStartPar
http://www.w3.org/ns/dcat\#
\\
\sphinxhline
\sphinxAtStartPar

&&&&&&
\sphinxAtStartPar
dct
&
\sphinxAtStartPar
http://purl.org/dc/terms/
\\
\sphinxhline
\sphinxAtStartPar

&&&&&&
\sphinxAtStartPar
dctype
&
\sphinxAtStartPar
http://purl.org/dc/dcmitype/
\\
\sphinxhline
\sphinxAtStartPar

&&&&&&
\sphinxAtStartPar
foaf
&
\sphinxAtStartPar
http://xmlns.com/foaf/0.1/
\\
\sphinxhline
\sphinxAtStartPar

&&&&&&
\sphinxAtStartPar
owl
&
\sphinxAtStartPar
http://www.w3.org/2002/07/owl\#
\\
\sphinxhline
\sphinxAtStartPar

&&&&&&
\sphinxAtStartPar
prov
&
\sphinxAtStartPar
http://www.w3.org/ns/prov\#
\\
\sphinxhline
\sphinxAtStartPar

&&&&&&
\sphinxAtStartPar
rdf
&
\sphinxAtStartPar
http://www.w3.org/1999/02/22\sphinxhyphen{}rdf\sphinxhyphen{}syntax\sphinxhyphen{}ns\#
\\
\sphinxhline
\sphinxAtStartPar

&&&&&&
\sphinxAtStartPar
rdfs
&
\sphinxAtStartPar
http://www.w3.org/2000/01/rdf\sphinxhyphen{}schema\#
\\
\sphinxhline
\sphinxAtStartPar

&&&&&&
\sphinxAtStartPar
sdo
&
\sphinxAtStartPar
http://schema.org/
\\
\sphinxhline
\sphinxAtStartPar

&&&&&&
\sphinxAtStartPar
skos
&
\sphinxAtStartPar
http://www.w3.org/2004/02/skos/core\#
\\
\sphinxhline
\sphinxAtStartPar

&&&&&&
\sphinxAtStartPar
vcard
&
\sphinxAtStartPar
http://www.w3.org/2006/vcard/ns\#
\\
\sphinxhline
\sphinxAtStartPar

&&&&&&
\sphinxAtStartPar
xsd
&
\sphinxAtStartPar
http://www.w3.org/2001/XMLSchema\#
\\
\sphinxbottomrule
\end{tabulary}
\sphinxtableafterendhook\par
\sphinxattableend\end{savenotes}

\sphinxAtStartPar
Prefiksai turi būti apibrėžti duomenų rinkinio kontekste, kadangi skirtingi
duomenų rinkiniai gali naudoti skirtingus prefiksus, tiems patiems URI.
Pavyzdžiui abiejuose rinkinių pavyzdžiuose aukščiau, \sphinxcode{\sphinxupquote{dct}} iš \sphinxcode{\sphinxupquote{dataset1}} ir
\sphinxcode{\sphinxupquote{dctype}} iš \sphinxcode{\sphinxupquote{dataset2}} rodo į tą patį URI.


\subsection{enum}
\label{\detokenize{dimensijos:enum}}\label{\detokenize{dimensijos:id10}}\index{module@\spxentry{module}!enum@\spxentry{enum}}\index{enum@\spxentry{enum}!module@\spxentry{module}}\phantomsection\label{\detokenize{dimensijos:module-enum}}
\sphinxAtStartPar
Tam tikri duomenų laukai turi fiksuotą reikšmių variantų aibę. Dažnai duomenų
bazėse fiksuotos reikšmės saugomos skaitine forma ar kitais kodiniais
pavadinimais. Tokias fiksuotas reikšmes duomenų struktūros apraše galima
pateikti neužpildant hierarchinių stulpelių ir nurodant \sphinxcode{\sphinxupquote{type}} reikšmę
\sphinxcode{\sphinxupquote{enum}}, pavyzdžiui:


\begin{savenotes}\sphinxattablestart
\sphinxthistablewithglobalstyle
\centering
\begin{tabulary}{\linewidth}[t]{TTTTTTTTTTTTTTT}
\sphinxtoprule
\sphinxstyletheadfamily 
\sphinxAtStartPar
id
&\sphinxstyletheadfamily 
\sphinxAtStartPar
d
&\sphinxstyletheadfamily 
\sphinxAtStartPar
r
&\sphinxstyletheadfamily 
\sphinxAtStartPar
b
&\sphinxstyletheadfamily 
\sphinxAtStartPar
m
&\sphinxstyletheadfamily 
\sphinxAtStartPar
property
&\sphinxstyletheadfamily 
\sphinxAtStartPar
type
&\sphinxstyletheadfamily 
\sphinxAtStartPar
ref
&\sphinxstyletheadfamily 
\sphinxAtStartPar
source
&\sphinxstyletheadfamily 
\sphinxAtStartPar
prepare
&\sphinxstyletheadfamily 
\sphinxAtStartPar
level
&\sphinxstyletheadfamily 
\sphinxAtStartPar
access
&\sphinxstyletheadfamily 
\sphinxAtStartPar
uri
&\sphinxstyletheadfamily 
\sphinxAtStartPar
title
&\sphinxstyletheadfamily 
\sphinxAtStartPar
description
\\
\sphinxmidrule
\sphinxtableatstartofbodyhook
\sphinxAtStartPar
1
&\sphinxstartmulticolumn{5}%
\begin{varwidth}[t]{\sphinxcolwidth{5}{15}}
\sphinxAtStartPar
datasets/example/places
\par
\vskip-\baselineskip\vbox{\hbox{\strut}}\end{varwidth}%
\sphinxstopmulticolumn
&&&&&&&&&\\
\sphinxhline
\sphinxAtStartPar
2
&&\sphinxstartmulticolumn{4}%
\begin{varwidth}[t]{\sphinxcolwidth{4}{15}}
\sphinxAtStartPar
places
\par
\vskip-\baselineskip\vbox{\hbox{\strut}}\end{varwidth}%
\sphinxstopmulticolumn
&
\sphinxAtStartPar
sql
&&
\sphinxAtStartPar
sqlite://
&&&&&&\\
\sphinxhline
\sphinxAtStartPar
3
&&&&\sphinxstartmulticolumn{2}%
\begin{varwidth}[t]{\sphinxcolwidth{2}{15}}
\sphinxAtStartPar
Place
\par
\vskip-\baselineskip\vbox{\hbox{\strut}}\end{varwidth}%
\sphinxstopmulticolumn
&&
\sphinxAtStartPar
id
&
\sphinxAtStartPar
PLACES
&&&&&&\\
\sphinxhline
\sphinxAtStartPar
4
&&&&&
\sphinxAtStartPar
id
&
\sphinxAtStartPar
integer
&&
\sphinxAtStartPar
ID
&&
\sphinxAtStartPar
3
&
\sphinxAtStartPar
open
&&&\\
\sphinxhline
\sphinxAtStartPar
5
&&&&&
\sphinxAtStartPar
type
&
\sphinxAtStartPar
string
&&
\sphinxAtStartPar
CODE
&&
\sphinxAtStartPar
3
&
\sphinxAtStartPar
open
&&&\\
\sphinxhline
\sphinxAtStartPar
6
&&&&&&
\sphinxAtStartPar
enum
&&
\sphinxAtStartPar
1
&
\sphinxAtStartPar
"city"
&&&&
\sphinxAtStartPar
City
&\\
\sphinxhline
\sphinxAtStartPar
7
&&&&&&&&
\sphinxAtStartPar
2
&
\sphinxAtStartPar
"town"
&&&&
\sphinxAtStartPar
Town
&\\
\sphinxhline
\sphinxAtStartPar
8
&&&&&&&&
\sphinxAtStartPar
3
&
\sphinxAtStartPar
"village"
&&&&
\sphinxAtStartPar
Village
&\\
\sphinxhline
\sphinxAtStartPar
9
&&&&&
\sphinxAtStartPar
name
&
\sphinxAtStartPar
string
&&
\sphinxAtStartPar
NAME
&&
\sphinxAtStartPar
3
&
\sphinxAtStartPar
open
&&&\\
\sphinxbottomrule
\end{tabulary}
\sphinxtableafterendhook\par
\sphinxattableend\end{savenotes}

\sphinxAtStartPar
Šiame pavyzdyje \sphinxcode{\sphinxupquote{Place.type}} laukas yra klasifikatorius, kurio reikšmės yra
kodai 1, 2 ir 3, kurios duomenų struktūros apraše keičiamos į \sphinxcode{\sphinxupquote{city}}, \sphinxcode{\sphinxupquote{town}}
ir \sphinxcode{\sphinxupquote{village}}, papildomai \sphinxcode{\sphinxupquote{title}} stulpelyje nurodant reikšmės pavadinimą.

\sphinxAtStartPar
Jei tas pats klasifikatorius gali būti naudojamas keliose skirtingose vietose,
tada galima iškelti klasifikatorių ir suteikti jam pavadinimą, pavyzdžiui:


\begin{savenotes}\sphinxattablestart
\sphinxthistablewithglobalstyle
\centering
\begin{tabulary}{\linewidth}[t]{TTTTTTTTTTTTTTT}
\sphinxtoprule
\sphinxstyletheadfamily 
\sphinxAtStartPar
id
&\sphinxstyletheadfamily 
\sphinxAtStartPar
d
&\sphinxstyletheadfamily 
\sphinxAtStartPar
r
&\sphinxstyletheadfamily 
\sphinxAtStartPar
b
&\sphinxstyletheadfamily 
\sphinxAtStartPar
m
&\sphinxstyletheadfamily 
\sphinxAtStartPar
property
&\sphinxstyletheadfamily 
\sphinxAtStartPar
type
&\sphinxstyletheadfamily 
\sphinxAtStartPar
ref
&\sphinxstyletheadfamily 
\sphinxAtStartPar
source
&\sphinxstyletheadfamily 
\sphinxAtStartPar
prepare
&\sphinxstyletheadfamily 
\sphinxAtStartPar
level
&\sphinxstyletheadfamily 
\sphinxAtStartPar
access
&\sphinxstyletheadfamily 
\sphinxAtStartPar
uri
&\sphinxstyletheadfamily 
\sphinxAtStartPar
title
&\sphinxstyletheadfamily 
\sphinxAtStartPar
description
\\
\sphinxmidrule
\sphinxtableatstartofbodyhook
\sphinxAtStartPar
1
&\sphinxstartmulticolumn{5}%
\begin{varwidth}[t]{\sphinxcolwidth{5}{15}}
\sphinxAtStartPar
datasets/example/places
\par
\vskip-\baselineskip\vbox{\hbox{\strut}}\end{varwidth}%
\sphinxstopmulticolumn
&&&&&&&&&\\
\sphinxhline
\sphinxAtStartPar
2
&&&&&&
\sphinxAtStartPar
enum
&
\sphinxAtStartPar
place
&
\sphinxAtStartPar
1
&
\sphinxAtStartPar
"city"
&&&&
\sphinxAtStartPar
City
&\\
\sphinxhline
\sphinxAtStartPar
3
&&&&&&&&
\sphinxAtStartPar
2
&
\sphinxAtStartPar
"town"
&&&&
\sphinxAtStartPar
Town
&\\
\sphinxhline
\sphinxAtStartPar
4
&&&&&&&&
\sphinxAtStartPar
3
&
\sphinxAtStartPar
"village"
&&&&
\sphinxAtStartPar
Village
&\\
\sphinxhline
\sphinxAtStartPar
5
&&\sphinxstartmulticolumn{4}%
\begin{varwidth}[t]{\sphinxcolwidth{4}{15}}
\sphinxAtStartPar
places
\par
\vskip-\baselineskip\vbox{\hbox{\strut}}\end{varwidth}%
\sphinxstopmulticolumn
&
\sphinxAtStartPar
sql
&&
\sphinxAtStartPar
sqlite://
&&&&&&\\
\sphinxhline
\sphinxAtStartPar
6
&&&&\sphinxstartmulticolumn{2}%
\begin{varwidth}[t]{\sphinxcolwidth{2}{15}}
\sphinxAtStartPar
Place
\par
\vskip-\baselineskip\vbox{\hbox{\strut}}\end{varwidth}%
\sphinxstopmulticolumn
&&
\sphinxAtStartPar
id
&
\sphinxAtStartPar
PLACES
&&&&&&\\
\sphinxhline
\sphinxAtStartPar
7
&&&&&
\sphinxAtStartPar
id
&
\sphinxAtStartPar
integer
&&
\sphinxAtStartPar
ID
&&
\sphinxAtStartPar
3
&
\sphinxAtStartPar
open
&&&\\
\sphinxhline
\sphinxAtStartPar
8
&&&&&
\sphinxAtStartPar
type
&
\sphinxAtStartPar
string
&
\sphinxAtStartPar
place
&
\sphinxAtStartPar
CODE
&&
\sphinxAtStartPar
3
&
\sphinxAtStartPar
open
&&&\\
\sphinxhline
\sphinxAtStartPar
9
&&&&&
\sphinxAtStartPar
name
&
\sphinxAtStartPar
string
&&
\sphinxAtStartPar
NAME
&&
\sphinxAtStartPar
3
&
\sphinxAtStartPar
open
&&&\\
\sphinxbottomrule
\end{tabulary}
\sphinxtableafterendhook\par
\sphinxattableend\end{savenotes}

\sphinxAtStartPar
Šiuo atveju, klasifikatoriui buvo suteiktas pavadinimas \sphinxcode{\sphinxupquote{place}} įrašytas
\sphinxcode{\sphinxupquote{enum.ref}} stulpelyje, 2\sphinxhyphen{}oje eilutėje. O \sphinxcode{\sphinxupquote{Place.type}} laukui, \sphinxcode{\sphinxupquote{property.ref}}
stulpelyje nurodyta, kad šis laukas naudoja vardinį \sphinxcode{\sphinxupquote{place}} klasifikatorių.
\index{ref (modulje enum)@\spxentry{ref}\spxextra{modulje enum}}

\begin{fulllineitems}
\phantomsection\label{\detokenize{dimensijos:enum.ref}}
\pysigstartsignatures
\pysigline
{\sphinxcode{\sphinxupquote{enum.}}\sphinxbfcode{\sphinxupquote{ref}}}
\pysigstopsignatures
\sphinxAtStartPar
Pasirinkimų sąrašo pavadinimas.

\end{fulllineitems}

\index{source (modulje enum)@\spxentry{source}\spxextra{modulje enum}}

\begin{fulllineitems}
\phantomsection\label{\detokenize{dimensijos:enum.source}}
\pysigstartsignatures
\pysigline
{\sphinxcode{\sphinxupquote{enum.}}\sphinxbfcode{\sphinxupquote{source}}}
\pysigstopsignatures
\sphinxAtStartPar
Pateikiama originali reikšmė, taip kaip ji saugoma duomenų šaltinyje.
Pateiktos reikšmės turi būti unikalios ir negali kartotis.

\sphinxAtStartPar
Jei pageidaujama aprašyti tuščią šaltinio reikšmę, tada
{\hyperref[\detokenize{dimensijos:property.prepare}]{\sphinxcrossref{\sphinxcode{\sphinxupquote{property.prepare}}}}} celėje reikia nurodyti formulę, kuri tuščią
reikšmę pakeičia, į kokią nors kitą. Formulės pavyzdys:

\begin{sphinxVerbatim}[commandchars=\\\{\}]
\PYG{n}{swap}\PYG{p}{(}\PYG{l+s+s1}{\PYGZsq{}}\PYG{l+s+s1}{\PYGZsq{}}\PYG{p}{,} \PYG{l+s+s1}{\PYGZsq{}}\PYG{l+s+s1}{\PYGZhy{}}\PYG{l+s+s1}{\PYGZsq{}}\PYG{p}{)}
\end{sphinxVerbatim}

\end{fulllineitems}

\index{prepare (modulje enum)@\spxentry{prepare}\spxextra{modulje enum}}

\begin{fulllineitems}
\phantomsection\label{\detokenize{dimensijos:enum.prepare}}
\pysigstartsignatures
\pysigline
{\sphinxcode{\sphinxupquote{enum.}}\sphinxbfcode{\sphinxupquote{prepare}}}
\pysigstopsignatures
\sphinxAtStartPar
Pateikiama reikšmė, tokia kuri bus naudojama atveriant duomenis.
{\hyperref[\detokenize{dimensijos:model.prepare}]{\sphinxcrossref{\sphinxcode{\sphinxupquote{model.prepare}}}}} filtruose taip pat bus naudojama būtent ši
reikšmė.

\sphinxAtStartPar
\sphinxcode{\sphinxupquote{enum.prepare}} reikšmės gali kartotis, tokiu būdu, kelios skirtingos
\sphinxcode{\sphinxupquote{enum.source}} reikšmės bus susietos su viena \sphinxcode{\sphinxupquote{enum.prepare}} reikšme.

\end{fulllineitems}

\index{access (modulje enum)@\spxentry{access}\spxextra{modulje enum}}

\begin{fulllineitems}
\phantomsection\label{\detokenize{dimensijos:enum.access}}
\pysigstartsignatures
\pysigline
{\sphinxcode{\sphinxupquote{enum.}}\sphinxbfcode{\sphinxupquote{access}}}
\pysigstopsignatures
\sphinxAtStartPar
Klasifikatoriams galima nurodyti skirtingas prieigos teises, tokiu
atveju, naudotojas turintis \sphinxcode{\sphinxupquote{open}} prieigą matys tik tuos duomenis,
kurių klasifikatorių reikšmės turi \sphinxcode{\sphinxupquote{open}} prieigos teises, visi kiti bus
išfiltruoti.

\end{fulllineitems}

\index{title (modulje enum)@\spxentry{title}\spxextra{modulje enum}}

\begin{fulllineitems}
\phantomsection\label{\detokenize{dimensijos:enum.title}}
\pysigstartsignatures
\pysigline
{\sphinxcode{\sphinxupquote{enum.}}\sphinxbfcode{\sphinxupquote{title}}}
\pysigstopsignatures
\sphinxAtStartPar
Fiksuotos reikšmės pavadinimas.

\end{fulllineitems}

\index{description (modulje enum)@\spxentry{description}\spxextra{modulje enum}}

\begin{fulllineitems}
\phantomsection\label{\detokenize{dimensijos:enum.description}}
\pysigstartsignatures
\pysigline
{\sphinxcode{\sphinxupquote{enum.}}\sphinxbfcode{\sphinxupquote{description}}}
\pysigstopsignatures
\sphinxAtStartPar
Fiksuotos reikšmės aprašymas.

\end{fulllineitems}


\sphinxAtStartPar
Pagal nutylėjimą, jei {\hyperref[\detokenize{dimensijos:property.prepare}]{\sphinxcrossref{\sphinxcode{\sphinxupquote{property.prepare}}}}} yra tuščias ir {\hyperref[\detokenize{formatas:property}]{\sphinxcrossref{\sphinxcode{\sphinxupquote{property}}}}}
turi {\hyperref[\detokenize{dimensijos:enum}]{\sphinxcrossref{\DUrole{std}{\DUrole{std-ref}{enum}}}}} sąrašą, tada jei šaltinis turi neaprašytą reikšmę, turėtų
būti fiksuojama klaida.

\sphinxAtStartPar
Jei yra poreikis fiksuoti tik tam tikras reikšmes, o visas kitas palikti tokias,
kokios yra šaltinyje, tada {\hyperref[\detokenize{dimensijos:property.prepare}]{\sphinxcrossref{\sphinxcode{\sphinxupquote{property.prepare}}}}} stulpelyje reikia įrašyti
\sphinxcode{\sphinxupquote{self.choose(self)}}.


\subsection{param}
\label{\detokenize{dimensijos:param}}\label{\detokenize{dimensijos:id11}}\index{module@\spxentry{module}!param@\spxentry{param}}\index{param@\spxentry{param}!module@\spxentry{module}}\phantomsection\label{\detokenize{dimensijos:module-param}}
\sphinxAtStartPar
Parametrai leidžia iškelti tam tikras duomenų paruošimo operacijas į parametrus
kurie gali būti naudojami {\hyperref[\detokenize{dimensijos:dimensijos}]{\sphinxcrossref{\DUrole{std}{\DUrole{std-ref}{Dimensijos}}}}}, kurioje apibrėžtas parametras
kontekste. Parametrai gali gražinti {\hyperref[\detokenize{savokos:term-iteratorius}]{\sphinxtermref{\DUrole{xref}{\DUrole{std}{\DUrole{std-term}{iteratorius}}}}}}, kurių pagalba galima
dinamiškai kartoti {\hyperref[\detokenize{formatas:resource}]{\sphinxcrossref{\sphinxcode{\sphinxupquote{resource}}}}} duomenų skaitymą, panaudojant aprašytus
parametrus. Taip pat parametrų pagalba galima sudaryti reikšmių sąrašus, kurių
pagalba galima kartoti {\hyperref[\detokenize{formatas:resource}]{\sphinxcrossref{\sphinxcode{\sphinxupquote{resource}}}}} su kiekviena reikšme.

\sphinxAtStartPar
Parametrai dažniausiai naudojami žemesnio brandos lygio duomenų šaltiniams
aprašyti, o taip pat API atvejais, kai duomenys atiduodami dinamiškai.

\sphinxAtStartPar
Parametrai aprašomi pasitelkiant papildomą {\hyperref[\detokenize{dimensijos:param}]{\sphinxcrossref{\DUrole{std}{\DUrole{std-ref}{param}}}}} dimensiją.


\begin{savenotes}\sphinxattablestart
\sphinxthistablewithglobalstyle
\centering
\begin{tabulary}{\linewidth}[t]{TTTTTTTT}
\sphinxtoprule
\sphinxstyletheadfamily 
\sphinxAtStartPar
d
&\sphinxstyletheadfamily 
\sphinxAtStartPar
r
&\sphinxstyletheadfamily 
\sphinxAtStartPar
m
&\sphinxstyletheadfamily 
\sphinxAtStartPar
property
&\sphinxstyletheadfamily 
\sphinxAtStartPar
type
&\sphinxstyletheadfamily 
\sphinxAtStartPar
ref
&\sphinxstyletheadfamily 
\sphinxAtStartPar
source
&\sphinxstyletheadfamily 
\sphinxAtStartPar
prepare
\\
\sphinxmidrule
\sphinxtableatstartofbodyhook\sphinxstartmulticolumn{4}%
\begin{varwidth}[t]{\sphinxcolwidth{4}{8}}
\sphinxAtStartPar
datasets/example/cities
\par
\vskip-\baselineskip\vbox{\hbox{\strut}}\end{varwidth}%
\sphinxstopmulticolumn
&&&&\\
\sphinxhline
\sphinxAtStartPar

&\sphinxstartmulticolumn{3}%
\begin{varwidth}[t]{\sphinxcolwidth{3}{8}}
\sphinxAtStartPar
places
\par
\vskip-\baselineskip\vbox{\hbox{\strut}}\end{varwidth}%
\sphinxstopmulticolumn
&
\sphinxAtStartPar
csv
&&
\sphinxAtStartPar
https://example.com/\{\}.csv
&\\
\sphinxhline
\sphinxAtStartPar

&&\sphinxstartmulticolumn{2}%
\begin{varwidth}[t]{\sphinxcolwidth{2}{8}}
\sphinxAtStartPar
\sphinxstylestrong{Country}
\par
\vskip-\baselineskip\vbox{\hbox{\strut}}\end{varwidth}%
\sphinxstopmulticolumn
&&
\sphinxAtStartPar
id
&
\sphinxAtStartPar
countries
&\\
\sphinxhline
\sphinxAtStartPar

&&&
\sphinxAtStartPar
code
&
\sphinxAtStartPar
string
&&
\sphinxAtStartPar
CODE
&\\
\sphinxhline
\sphinxAtStartPar

&&&
\sphinxAtStartPar
title
&
\sphinxAtStartPar
string
&&
\sphinxAtStartPar
TITLE
&\\
\sphinxhline
\sphinxAtStartPar

&&\sphinxstartmulticolumn{2}%
\begin{varwidth}[t]{\sphinxcolwidth{2}{8}}
\sphinxAtStartPar
\sphinxstylestrong{City}
\par
\vskip-\baselineskip\vbox{\hbox{\strut}}\end{varwidth}%
\sphinxstopmulticolumn
&&
\sphinxAtStartPar
country, title
&
\sphinxAtStartPar
cities/\{code\}
&\\
\sphinxhline
\sphinxAtStartPar

&&&&
\sphinxAtStartPar
param
&
\sphinxAtStartPar
code
&
\sphinxAtStartPar
\sphinxstylestrong{Country}
&
\sphinxAtStartPar
read().code
\\
\sphinxhline
\sphinxAtStartPar

&&&
\sphinxAtStartPar
country
&
\sphinxAtStartPar
ref
&
\sphinxAtStartPar
\sphinxstylestrong{Country}
&
\sphinxAtStartPar
code
&
\sphinxAtStartPar
param()
\\
\sphinxhline
\sphinxAtStartPar

&&&
\sphinxAtStartPar
title
&
\sphinxAtStartPar
string
&&
\sphinxAtStartPar
TITLE
&\\
\sphinxbottomrule
\end{tabulary}
\sphinxtableafterendhook\par
\sphinxattableend\end{savenotes}
\index{ref (modulje param)@\spxentry{ref}\spxextra{modulje param}}

\begin{fulllineitems}
\phantomsection\label{\detokenize{dimensijos:param.ref}}
\pysigstartsignatures
\pysigline
{\sphinxcode{\sphinxupquote{param.}}\sphinxbfcode{\sphinxupquote{ref}}}
\pysigstopsignatures
\sphinxAtStartPar
Parametro {\hyperref[\detokenize{savokos:term-kodinis-pavadinimas}]{\sphinxtermref{\DUrole{xref}{\DUrole{std}{\DUrole{std-term}{kodinis pavadinimas}}}}}}.

\end{fulllineitems}

\index{prepare (modulje param)@\spxentry{prepare}\spxextra{modulje param}}

\begin{fulllineitems}
\phantomsection\label{\detokenize{dimensijos:param.prepare}}
\pysigstartsignatures
\pysigline
{\sphinxcode{\sphinxupquote{param.}}\sphinxbfcode{\sphinxupquote{prepare}}}
\pysigstopsignatures
\sphinxAtStartPar
Formulė, kuri grąžina sąrašą reikšmių aprašomam parametrui.

\end{fulllineitems}

\index{source (modulje param)@\spxentry{source}\spxextra{modulje param}}

\begin{fulllineitems}
\phantomsection\label{\detokenize{dimensijos:param.source}}
\pysigstartsignatures
\pysigline
{\sphinxcode{\sphinxupquote{param.}}\sphinxbfcode{\sphinxupquote{source}}}
\pysigstopsignatures
\sphinxAtStartPar
Nurodoma parametro reikšmė šaltinyje, kuri yra pateikiama kaip pirmas
{\hyperref[\detokenize{dimensijos:param.prepare}]{\sphinxcrossref{\sphinxcode{\sphinxupquote{param.prepare}}}}} funkcijos argumentas.

\sphinxAtStartPar
Jei {\hyperref[\detokenize{dimensijos:param.prepare}]{\sphinxcrossref{\sphinxcode{\sphinxupquote{param.prepare}}}}} nenurodyta jokia formulė, tada bus naudojam
konstanta nurodyta {\hyperref[\detokenize{dimensijos:param.source}]{\sphinxcrossref{\sphinxcode{\sphinxupquote{param.source}}}}} stulpelyje.

\begin{sphinxadmonition}{note}{Pavyzdys}

\sphinxAtStartPar
Jei {\hyperref[\detokenize{dimensijos:param.prepare}]{\sphinxcrossref{\sphinxcode{\sphinxupquote{param.prepare}}}}} pateikta formulė {\hyperref[\detokenize{dimensijos:param.read}]{\sphinxcrossref{\sphinxcode{\sphinxupquote{param.read()}}}}}, o
{\hyperref[\detokenize{dimensijos:param.source}]{\sphinxcrossref{\sphinxcode{\sphinxupquote{param.source}}}}} nurodyta \sphinxcode{\sphinxupquote{Country}}, tai formulė bus iškviesta kaip
\sphinxcode{\sphinxupquote{read("Country")}}.
\end{sphinxadmonition}

\end{fulllineitems}


\sphinxAtStartPar
Jei parametro reikšmė yra {\hyperref[\detokenize{savokos:term-iteratorius}]{\sphinxtermref{\DUrole{xref}{\DUrole{std}{\DUrole{std-term}{iteratorius}}}}}}, tada {\hyperref[\detokenize{savokos:term-dimensija}]{\sphinxtermref{\DUrole{xref}{\DUrole{std}{\DUrole{std-term}{dimensija}}}}}}, kurios
kontekste yra aprašytas {\hyperref[\detokenize{dimensijos:param}]{\sphinxcrossref{\DUrole{std}{\DUrole{std-ref}{parametras}}}}} yra kartojama tiek kartų,
kiek reikšmių grąžina {\hyperref[\detokenize{savokos:term-iteratorius}]{\sphinxtermref{\DUrole{xref}{\DUrole{std}{\DUrole{std-term}{iteratorius}}}}}}.

\sphinxAtStartPar
Jei yra keli {\hyperref[\detokenize{dimensijos:param}]{\sphinxcrossref{\DUrole{std}{\DUrole{std-ref}{param}}}}} grąžinantys {\hyperref[\detokenize{savokos:term-iteratorius}]{\sphinxtermref{\DUrole{xref}{\DUrole{std}{\DUrole{std-term}{iteratorius}}}}}}, tada iš
visų {\hyperref[\detokenize{savokos:term-iteratorius}]{\sphinxtermref{\DUrole{xref}{\DUrole{std}{\DUrole{std-term}{iteratorių}}}}}} sudaroma \sphinxhref{https://lt.wikipedia.org/wiki/Dekarto\_sandauga}{Dekarto sandauga} ir
{\hyperref[\detokenize{formatas:resource}]{\sphinxcrossref{\sphinxcode{\sphinxupquote{resource}}}}} dimensija vykdoma su kiekviena sandaugos rezultato reikšme.

\sphinxAtStartPar
Jei sekančioje {\hyperref[\detokenize{savokos:term-DSA}]{\sphinxtermref{\DUrole{xref}{\DUrole{std}{\DUrole{std-term}{DSA}}}}}} eilutėje, einančioje po eilutės, kurioje aprašytas
{\hyperref[\detokenize{dimensijos:param}]{\sphinxcrossref{\DUrole{std}{\DUrole{std-ref}{param}}}}}, nenurodytas {\hyperref[\detokenize{tipai:module-type}]{\sphinxcrossref{\sphinxcode{\sphinxupquote{type}}}}} ir neužpildytas joks kitas
{\hyperref[\detokenize{savokos:term-dimensija}]{\sphinxtermref{\DUrole{xref}{\DUrole{std}{\DUrole{std-term}{dimensijos}}}}}} stulpelis, tada parametras tampa
{\hyperref[\detokenize{savokos:term-iteratorius}]{\sphinxtermref{\DUrole{xref}{\DUrole{std}{\DUrole{std-term}{iteratoriumi}}}}}}, kurio reikšmių sąrašą sudaro sekančiose
eilutėse patektos {\hyperref[\detokenize{formatas:source}]{\sphinxcrossref{\sphinxcode{\sphinxupquote{source}}}}} ir {\hyperref[\detokenize{formatas:prepare}]{\sphinxcrossref{\sphinxcode{\sphinxupquote{prepare}}}}} reikšmės. Pavyzdžiui
anksčiau pateiktą pavyzdį galima būtų perdaryti taip:


\begin{savenotes}\sphinxattablestart
\sphinxthistablewithglobalstyle
\centering
\begin{tabulary}{\linewidth}[t]{TTTTTTTT}
\sphinxtoprule
\sphinxstyletheadfamily 
\sphinxAtStartPar
dataset
&\sphinxstyletheadfamily 
\sphinxAtStartPar
resource
&\sphinxstyletheadfamily 
\sphinxAtStartPar
model
&\sphinxstyletheadfamily 
\sphinxAtStartPar
property
&\sphinxstyletheadfamily 
\sphinxAtStartPar
type
&\sphinxstyletheadfamily 
\sphinxAtStartPar
ref
&\sphinxstyletheadfamily 
\sphinxAtStartPar
source
&\sphinxstyletheadfamily 
\sphinxAtStartPar
prepare
\\
\sphinxmidrule
\sphinxtableatstartofbodyhook\sphinxstartmulticolumn{4}%
\begin{varwidth}[t]{\sphinxcolwidth{4}{8}}
\sphinxAtStartPar
datasets/example/cities
\par
\vskip-\baselineskip\vbox{\hbox{\strut}}\end{varwidth}%
\sphinxstopmulticolumn
&&&&\\
\sphinxhline
\sphinxAtStartPar

&\sphinxstartmulticolumn{3}%
\begin{varwidth}[t]{\sphinxcolwidth{3}{8}}
\sphinxAtStartPar
places
\par
\vskip-\baselineskip\vbox{\hbox{\strut}}\end{varwidth}%
\sphinxstopmulticolumn
&
\sphinxAtStartPar
csv
&&
\sphinxAtStartPar
https://example.com/\{\}.csv
&\\
\sphinxhline
\sphinxAtStartPar

&&\sphinxstartmulticolumn{2}%
\begin{varwidth}[t]{\sphinxcolwidth{2}{8}}
\sphinxAtStartPar
\sphinxstylestrong{Country}
\par
\vskip-\baselineskip\vbox{\hbox{\strut}}\end{varwidth}%
\sphinxstopmulticolumn
&&
\sphinxAtStartPar
id
&
\sphinxAtStartPar
countries
&\\
\sphinxhline
\sphinxAtStartPar

&&&
\sphinxAtStartPar
code
&
\sphinxAtStartPar
string
&&
\sphinxAtStartPar
CODE
&\\
\sphinxhline
\sphinxAtStartPar

&&&
\sphinxAtStartPar
title
&
\sphinxAtStartPar
string
&&
\sphinxAtStartPar
TITLE
&\\
\sphinxhline
\sphinxAtStartPar

&&\sphinxstartmulticolumn{2}%
\begin{varwidth}[t]{\sphinxcolwidth{2}{8}}
\sphinxAtStartPar
\sphinxstylestrong{City}
\par
\vskip-\baselineskip\vbox{\hbox{\strut}}\end{varwidth}%
\sphinxstopmulticolumn
&&
\sphinxAtStartPar
country, title
&
\sphinxAtStartPar
cities/\{country\}
&\\
\sphinxhline
\sphinxAtStartPar

&&&&
\sphinxAtStartPar
param
&
\sphinxAtStartPar
country
&
\sphinxAtStartPar
lt
&\\
\sphinxhline
\sphinxAtStartPar

&&&&&&
\sphinxAtStartPar
lv
&\\
\sphinxhline
\sphinxAtStartPar

&&&&&&
\sphinxAtStartPar
ee
&\\
\sphinxhline
\sphinxAtStartPar

&&&
\sphinxAtStartPar
country
&
\sphinxAtStartPar
ref
&
\sphinxAtStartPar
\sphinxstylestrong{Country}
&&
\sphinxAtStartPar
\sphinxcode{\sphinxupquote{param(country)}}
\\
\sphinxhline
\sphinxAtStartPar

&&&
\sphinxAtStartPar
title
&
\sphinxAtStartPar
string
&&
\sphinxAtStartPar
TITLE
&\\
\sphinxbottomrule
\end{tabulary}
\sphinxtableafterendhook\par
\sphinxattableend\end{savenotes}

\sphinxAtStartPar
Šiame pavyzdyje, parametras \sphinxcode{\sphinxupquote{country}} grąžins tris šalies kodus: lt, lv ir
ee, kurie bus panaudojami \sphinxcode{\sphinxupquote{cities/\{country\}}} pavadinime, pakeičiant
\sphinxcode{\sphinxupquote{\{country\}}} dalį.

\sphinxAtStartPar
{\hyperref[\detokenize{dimensijos:param}]{\sphinxcrossref{\DUrole{std}{\DUrole{std-ref}{param}}}}} reikšmės pasiekiamos naudojant pavadinimą įrašytą
{\hyperref[\detokenize{dimensijos:param.ref}]{\sphinxcrossref{\sphinxcode{\sphinxupquote{param.ref}}}}} stulpelyje. Pavyzdžiui, jei {\hyperref[\detokenize{dimensijos:param.ref}]{\sphinxcrossref{\sphinxcode{\sphinxupquote{param.ref}}}}} stulpelyje
įrašyta \sphinxcode{\sphinxupquote{x}}, tada \sphinxcode{\sphinxupquote{x}} parametro reikšmę galima gauti taip:
\begin{description}
\sphinxlineitem{source}
\sphinxAtStartPar
\sphinxcode{\sphinxupquote{\{x\}}}.

\sphinxlineitem{prepare}
\sphinxAtStartPar
\sphinxcode{\sphinxupquote{x}} arba \sphinxcode{\sphinxupquote{param(x)}}.

\end{description}


\subsubsection{Funkcijos}
\label{\detokenize{dimensijos:id12}}
\sphinxAtStartPar
Parametrų generavimui galima naudoti tokias formules:
\index{module@\spxentry{module}!param@\spxentry{param}}\index{param@\spxentry{param}!module@\spxentry{module}}\index{read() (modulyje param)@\spxentry{read()}\spxextra{modulyje param}}\phantomsection\label{\detokenize{dimensijos:module-2}}

\begin{fulllineitems}
\phantomsection\label{\detokenize{dimensijos:param.read}}
\pysigstartsignatures
\pysiglinewithargsret
{\sphinxcode{\sphinxupquote{param.}}\sphinxbfcode{\sphinxupquote{read}}}
{\sphinxparam{\DUrole{n}{model}}}
{}
\pysigstopsignatures
\sphinxAtStartPar
Sukuriama priklausomybė nuo kito modelio, skaitomi duomenys iš kito modelio
ir su kiekvienu objektu, kreipiamasi į {\hyperref[\detokenize{dimensijos:resource.source}]{\sphinxcrossref{\sphinxcode{\sphinxupquote{resource.source}}}}}, panaudojant
nuskaitytą objektą kaip parametrą formuojant šaltinio užklausą.

\end{fulllineitems}

\index{range() (modulyje param)@\spxentry{range()}\spxextra{modulyje param}}

\begin{fulllineitems}
\phantomsection\label{\detokenize{dimensijos:param.range}}
\pysigstartsignatures
\pysiglinewithargsret
{\sphinxcode{\sphinxupquote{param.}}\sphinxbfcode{\sphinxupquote{range}}}
{\sphinxparam{\DUrole{n}{stop}}}
{}
\pysigstopsignatures
\sphinxAtStartPar
Sveikų skaičių generavimas nuo 0 iki \sphinxcode{\sphinxupquote{stop}}, \sphinxcode{\sphinxupquote{stop}} neįeina.

\end{fulllineitems}



\begin{fulllineitems}

\pysigstartsignatures
\pysiglinewithargsret
{\sphinxcode{\sphinxupquote{param.}}\sphinxbfcode{\sphinxupquote{range}}}
{\sphinxparam{\DUrole{n}{start}}\sphinxparamcomma \sphinxparam{\DUrole{n}{stop}}}
{}
\pysigstopsignatures
\sphinxAtStartPar
Sveikų skaičių generavimas nuo \sphinxcode{\sphinxupquote{start}} iki \sphinxcode{\sphinxupquote{stop}}, \sphinxcode{\sphinxupquote{stop}} neįeina.

\end{fulllineitems}

\index{path() (modulyje param)@\spxentry{path()}\spxextra{modulyje param}}

\begin{fulllineitems}
\phantomsection\label{\detokenize{dimensijos:param.path}}
\pysigstartsignatures
\pysiglinewithargsret
{\sphinxcode{\sphinxupquote{param.}}\sphinxbfcode{\sphinxupquote{path}}}
{\sphinxparam{\DUrole{n}{name}}\sphinxparamcomma \sphinxparam{\DUrole{n}{value}}}
{}
\pysigstopsignatures
\sphinxAtStartPar
Parametras pateikia URI path dalies parametrą.

\sphinxAtStartPar
{\hyperref[\detokenize{dimensijos:resource.source}]{\sphinxcrossref{\sphinxcode{\sphinxupquote{resource.source}}}}} URI path parametrų vieta pateikiama \sphinxcode{\sphinxupquote{\{\}}} skliaustų
viduje, pavyzdžiui \sphinxcode{\sphinxupquote{/cities/\{id\}}} nurodytas parametras \sphinxcode{\sphinxupquote{id}}.

\sphinxAtStartPar
Funkcija gali būti iškviestai tokiais būdais:
\begin{itemize}
\item {} 
\sphinxAtStartPar
\sphinxcode{\sphinxupquote{path(value)}}

\item {} 
\sphinxAtStartPar
\sphinxcode{\sphinxupquote{path(name, value)}}

\end{itemize}

\sphinxAtStartPar
\sphinxstylestrong{Argumentai}
\begin{description}
\sphinxlineitem{name}
\sphinxAtStartPar
Parametro pavadinimas nurodytas \sphinxcode{\sphinxupquote{\{\}}} riestinių skliaustelių viduje. Jei
nenurodyta, tada naudojamas {\hyperref[\detokenize{dimensijos:param.source}]{\sphinxcrossref{\sphinxcode{\sphinxupquote{param.source}}}}}, jei šis nenurodytas,
tada naudojamas {\hyperref[\detokenize{dimensijos:param.ref}]{\sphinxcrossref{\sphinxcode{\sphinxupquote{param.ref}}}}} pavadinimas.

\sphinxlineitem{value}
\sphinxAtStartPar
Parametro reikšmė, gali būti statinė arba dinaminė.

\end{description}

\begin{sphinxadmonition}{note}{Pavyzdys}


\begin{savenotes}\sphinxattablestart
\sphinxthistablewithglobalstyle
\centering
\begin{tabulary}{\linewidth}[t]{TTTTT}
\sphinxtoprule
\sphinxstyletheadfamily 
\sphinxAtStartPar
resource
&\sphinxstyletheadfamily 
\sphinxAtStartPar
type
&\sphinxstyletheadfamily 
\sphinxAtStartPar
ref
&\sphinxstyletheadfamily 
\sphinxAtStartPar
source
&\sphinxstyletheadfamily 
\sphinxAtStartPar
prepare
\\
\sphinxmidrule
\sphinxtableatstartofbodyhook
\sphinxAtStartPar
resource1
&
\sphinxAtStartPar
json
&&
\sphinxAtStartPar
/cities/\{id\}/
&\\
\sphinxhline
\sphinxAtStartPar

&
\sphinxAtStartPar
param
&
\sphinxAtStartPar
id
&&
\sphinxAtStartPar
\sphinxcode{\sphinxupquote{path(42)}}
\\
\sphinxbottomrule
\end{tabulary}
\sphinxtableafterendhook\par
\sphinxattableend\end{savenotes}

\sphinxAtStartPar
Bus konstruojamas toks URI:

\begin{sphinxVerbatim}[commandchars=\\\{\}]
/cities/42/
\end{sphinxVerbatim}
\end{sphinxadmonition}

\end{fulllineitems}

\index{query() (modulyje param)@\spxentry{query()}\spxextra{modulyje param}}

\begin{fulllineitems}
\phantomsection\label{\detokenize{dimensijos:param.query}}
\pysigstartsignatures
\pysiglinewithargsret
{\sphinxcode{\sphinxupquote{param.}}\sphinxbfcode{\sphinxupquote{query}}}
{\sphinxparam{\DUrole{n}{name}}\sphinxparamcomma \sphinxparam{\DUrole{n}{value}}}
{}
\pysigstopsignatures
\sphinxAtStartPar
Parametras pateikia URI query dalies parametrą.

\sphinxAtStartPar
Jei {\hyperref[\detokenize{dimensijos:resource.source}]{\sphinxcrossref{\sphinxcode{\sphinxupquote{resource.source}}}}} jau turi query parametrus, jei bus papildyti
arba perrašyti.

\sphinxAtStartPar
\sphinxstylestrong{Argumentai}
\begin{description}
\sphinxlineitem{name}
\sphinxAtStartPar
Nurodo URI query parametro pavadinimą, nurodomas {\hyperref[\detokenize{dimensijos:param.source}]{\sphinxcrossref{\sphinxcode{\sphinxupquote{param.source}}}}}
stulpelyje.

\sphinxlineitem{value}
\sphinxAtStartPar
URI query paramtro reikšmė.

\end{description}

\begin{sphinxadmonition}{note}{Pavyzdys}


\begin{savenotes}\sphinxattablestart
\sphinxthistablewithglobalstyle
\centering
\begin{tabulary}{\linewidth}[t]{TTTTT}
\sphinxtoprule
\sphinxstyletheadfamily 
\sphinxAtStartPar
resource
&\sphinxstyletheadfamily 
\sphinxAtStartPar
type
&\sphinxstyletheadfamily 
\sphinxAtStartPar
ref
&\sphinxstyletheadfamily 
\sphinxAtStartPar
source
&\sphinxstyletheadfamily 
\sphinxAtStartPar
prepare
\\
\sphinxmidrule
\sphinxtableatstartofbodyhook
\sphinxAtStartPar
resource1
&
\sphinxAtStartPar
json
&&
\sphinxAtStartPar
https://example.com/
&\\
\sphinxhline
\sphinxAtStartPar

&
\sphinxAtStartPar
param
&
\sphinxAtStartPar
name1
&
\sphinxAtStartPar
NAME1
&
\sphinxAtStartPar
\sphinxcode{\sphinxupquote{query("value1")}}
\\
\sphinxhline
\sphinxAtStartPar

&&
\sphinxAtStartPar
name2
&
\sphinxAtStartPar
NAME2
&
\sphinxAtStartPar
\sphinxcode{\sphinxupquote{query("value2")}}
\\
\sphinxbottomrule
\end{tabulary}
\sphinxtableafterendhook\par
\sphinxattableend\end{savenotes}

\sphinxAtStartPar
Bus konstruojamas toks URI:

\begin{sphinxVerbatim}[commandchars=\\\{\}]
https://example.com/?NAME1=value1\PYGZam{}NAME2=value2
\end{sphinxVerbatim}
\end{sphinxadmonition}

\end{fulllineitems}

\index{header() (modulyje param)@\spxentry{header()}\spxextra{modulyje param}}

\begin{fulllineitems}
\phantomsection\label{\detokenize{dimensijos:param.header}}
\pysigstartsignatures
\pysiglinewithargsret
{\sphinxcode{\sphinxupquote{param.}}\sphinxbfcode{\sphinxupquote{header}}}
{\sphinxparam{\DUrole{n}{name}}\sphinxparamcomma \sphinxparam{\DUrole{n}{value}}}
{}
\pysigstopsignatures
\sphinxAtStartPar
Parametras pateikiamas, kaip HTTP antraštė.

\sphinxAtStartPar
\sphinxstylestrong{Argumentai}
\begin{description}
\sphinxlineitem{name}
\sphinxAtStartPar
Nurodo HTTP antraštės pavadinimą, nurodomas {\hyperref[\detokenize{dimensijos:param.source}]{\sphinxcrossref{\sphinxcode{\sphinxupquote{param.source}}}}}
stulpelyje.

\sphinxlineitem{value}
\sphinxAtStartPar
HTTP antraštės reikšmė.

\end{description}

\begin{sphinxadmonition}{note}{Pavyzdys}


\begin{savenotes}\sphinxattablestart
\sphinxthistablewithglobalstyle
\centering
\begin{tabulary}{\linewidth}[t]{TTTTT}
\sphinxtoprule
\sphinxstyletheadfamily 
\sphinxAtStartPar
resource
&\sphinxstyletheadfamily 
\sphinxAtStartPar
type
&\sphinxstyletheadfamily 
\sphinxAtStartPar
ref
&\sphinxstyletheadfamily 
\sphinxAtStartPar
source
&\sphinxstyletheadfamily 
\sphinxAtStartPar
prepare
\\
\sphinxmidrule
\sphinxtableatstartofbodyhook
\sphinxAtStartPar
resource1
&
\sphinxAtStartPar
json
&&
\sphinxAtStartPar
https://example.com/
&\\
\sphinxhline
\sphinxAtStartPar

&
\sphinxAtStartPar
param
&
\sphinxAtStartPar
name1
&
\sphinxAtStartPar
X\sphinxhyphen{}Name1
&
\sphinxAtStartPar
\sphinxcode{\sphinxupquote{header("value1")}}
\\
\sphinxhline
\sphinxAtStartPar

&&
\sphinxAtStartPar
name2
&
\sphinxAtStartPar
X\sphinxhyphen{}Name2
&
\sphinxAtStartPar
\sphinxcode{\sphinxupquote{header("value2")}}
\\
\sphinxbottomrule
\end{tabulary}
\sphinxtableafterendhook\par
\sphinxattableend\end{savenotes}

\sphinxAtStartPar
Bus konstruojama tokia HTTP užklausa:

\begin{sphinxVerbatim}[commandchars=\\\{\}]
\PYG{n+nf}{GET} \PYG{n+nn}{/} \PYG{k+kr}{HTTP}\PYG{o}{/}\PYG{l+m}{1.1}
\PYG{n+na}{X\PYGZhy{}Name1}\PYG{o}{:} \PYG{l}{value1}
\PYG{n+na}{X\PYGZhy{}Name2}\PYG{o}{:} \PYG{l}{value2}
\end{sphinxVerbatim}
\end{sphinxadmonition}

\end{fulllineitems}

\index{body() (modulyje param)@\spxentry{body()}\spxextra{modulyje param}}

\begin{fulllineitems}
\phantomsection\label{\detokenize{dimensijos:param.body}}
\pysigstartsignatures
\pysiglinewithargsret
{\sphinxcode{\sphinxupquote{param.}}\sphinxbfcode{\sphinxupquote{body}}}
{\sphinxparam{\DUrole{n}{name}}\sphinxparamcomma \sphinxparam{\DUrole{n}{value}}\sphinxparamcomma \sphinxparam{\DUrole{n}{parent}}\sphinxparamcomma \sphinxparam{\DUrole{n}{type}}}
{}
\pysigstopsignatures
\sphinxAtStartPar
Generuoja XML, JSON ar kito formato dokumentą, kuris pateikiamas HTTP
užklausos metu.

\sphinxAtStartPar
\sphinxstylestrong{Argumentai}
\begin{description}
\sphinxlineitem{name}
\sphinxAtStartPar
\sphinxhref{https://www.ietf.org/archive/id/draft-goessner-dispatch-jsonpath-00.html}{JSONPath} arba {\color{red}\bfseries{}XPath\_} išraiška, priklauso nuoo \sphinxcode{\sphinxupquote{resource.prapare}}
nurodytos {\hyperref[\detokenize{dimensijos:resource.http}]{\sphinxcrossref{\sphinxcode{\sphinxupquote{resource.http()}}}}} \sphinxcode{\sphinxupquote{body}} tipo.

\sphinxAtStartPar
Nurodoma {\hyperref[\detokenize{dimensijos:param.source}]{\sphinxcrossref{\sphinxcode{\sphinxupquote{param.source}}}}} stulpelyje.

\sphinxlineitem{value}
\sphinxAtStartPar
Reikšmė suteikiama \sphinxcode{\sphinxupquote{name}} elementui http užklausos struktūroje.

\sphinxlineitem{parent (neprivalomas)}
\sphinxAtStartPar
Parametro pavadinimas, kurio pagrindu konstruojamas naujas dokumentas.

\sphinxlineitem{type (neprivalomas, vardinis)}
\sphinxAtStartPar
Naudojamas konstruojant naują dokumentą, jei nurodytas, kiti argumentai
turi būti nepateikti.

\end{description}

\begin{sphinxadmonition}{note}{Pavyzdys (JSON)}


\begin{savenotes}\sphinxattablestart
\sphinxthistablewithglobalstyle
\centering
\begin{tabulary}{\linewidth}[t]{TTTTT}
\sphinxtoprule
\sphinxstyletheadfamily 
\sphinxAtStartPar
resource
&\sphinxstyletheadfamily 
\sphinxAtStartPar
type
&\sphinxstyletheadfamily 
\sphinxAtStartPar
ref
&\sphinxstyletheadfamily 
\sphinxAtStartPar
source
&\sphinxstyletheadfamily 
\sphinxAtStartPar
prepare
\\
\sphinxmidrule
\sphinxtableatstartofbodyhook
\sphinxAtStartPar
resource1
&
\sphinxAtStartPar
json
&&
\sphinxAtStartPar
https://example.com/
&
\sphinxAtStartPar
\sphinxcode{\sphinxupquote{http(body: json)}}
\\
\sphinxhline
\sphinxAtStartPar

&
\sphinxAtStartPar
param
&
\sphinxAtStartPar
name1
&
\sphinxAtStartPar
NAME1
&
\sphinxAtStartPar
\sphinxcode{\sphinxupquote{body("value1")}}
\\
\sphinxhline
\sphinxAtStartPar

&&
\sphinxAtStartPar
name2
&
\sphinxAtStartPar
NAME2
&
\sphinxAtStartPar
\sphinxcode{\sphinxupquote{body("value2")}}
\\
\sphinxhline
\sphinxAtStartPar

&&
\sphinxAtStartPar
name3
&
\sphinxAtStartPar
NESTED.NAME3
&
\sphinxAtStartPar
\sphinxcode{\sphinxupquote{body("value3")}}
\\
\sphinxhline
\sphinxAtStartPar

&&
\sphinxAtStartPar
name4
&
\sphinxAtStartPar
ARRAY{[}{]}.NAME4
&
\sphinxAtStartPar
\sphinxcode{\sphinxupquote{body("value4")}}
\\
\sphinxhline
\sphinxAtStartPar

&&
\sphinxAtStartPar
name5
&
\sphinxAtStartPar
MATRIX{[}{]}
&
\sphinxAtStartPar
\sphinxcode{\sphinxupquote{body()}}
\\
\sphinxhline
\sphinxAtStartPar

&&&
\sphinxAtStartPar
MATRIX{[}{]}{[}{]}
&
\sphinxAtStartPar
\sphinxcode{\sphinxupquote{body("value5")}}
\\
\sphinxhline
\sphinxAtStartPar

&&&
\sphinxAtStartPar
MATRIX{[}{]}
&
\sphinxAtStartPar
\sphinxcode{\sphinxupquote{body()}}
\\
\sphinxhline
\sphinxAtStartPar

&&&
\sphinxAtStartPar
MATRIX{[}{]}{[}{]}
&
\sphinxAtStartPar
\sphinxcode{\sphinxupquote{body("value6")}}
\\
\sphinxhline
\sphinxAtStartPar

&&&
\sphinxAtStartPar
MATRIX{[}{]}{[}{]}
&
\sphinxAtStartPar
\sphinxcode{\sphinxupquote{body("value7")}}
\\
\sphinxbottomrule
\end{tabulary}
\sphinxtableafterendhook\par
\sphinxattableend\end{savenotes}

\begin{sphinxVerbatim}[commandchars=\\\{\}]
\PYG{p}{\PYGZob{}}
\PYG{+w}{    }\PYG{n+nt}{\PYGZdq{}NAME1\PYGZdq{}}\PYG{p}{:}\PYG{+w}{ }\PYG{l+s+s2}{\PYGZdq{}value1\PYGZdq{}}\PYG{p}{,}
\PYG{+w}{    }\PYG{n+nt}{\PYGZdq{}NAME2\PYGZdq{}}\PYG{p}{:}\PYG{+w}{ }\PYG{l+s+s2}{\PYGZdq{}value2\PYGZdq{}}\PYG{p}{,}
\PYG{+w}{    }\PYG{n+nt}{\PYGZdq{}NESTED\PYGZdq{}}\PYG{p}{:}\PYG{+w}{ }\PYG{p}{\PYGZob{}}
\PYG{+w}{        }\PYG{n+nt}{\PYGZdq{}NAME3\PYGZdq{}}\PYG{p}{:}\PYG{+w}{ }\PYG{l+s+s2}{\PYGZdq{}value3\PYGZdq{}}\PYG{p}{,}
\PYG{+w}{    }\PYG{p}{\PYGZcb{},}
\PYG{+w}{    }\PYG{n+nt}{\PYGZdq{}ARRAY\PYGZdq{}}\PYG{p}{:}\PYG{+w}{ }\PYG{p}{[}
\PYG{+w}{        }\PYG{p}{\PYGZob{}}\PYG{n+nt}{\PYGZdq{}NAME4\PYGZdq{}}\PYG{p}{:}\PYG{+w}{ }\PYG{l+s+s2}{\PYGZdq{}value4\PYGZdq{}}\PYG{p}{\PYGZcb{}}
\PYG{+w}{    }\PYG{p}{],}
\PYG{+w}{    }\PYG{n+nt}{\PYGZdq{}MATRIX\PYGZdq{}}\PYG{p}{:}\PYG{+w}{ }\PYG{p}{[}
\PYG{+w}{        }\PYG{p}{[}\PYG{l+s+s2}{\PYGZdq{}value5\PYGZdq{}}\PYG{p}{],}
\PYG{+w}{        }\PYG{p}{[}\PYG{l+s+s2}{\PYGZdq{}value6\PYGZdq{}}\PYG{p}{,}\PYG{+w}{ }\PYG{l+s+s2}{\PYGZdq{}value7\PYGZdq{}}\PYG{p}{]}
\PYG{+w}{    }\PYG{p}{]}
\PYG{p}{\PYGZcb{}}
\end{sphinxVerbatim}
\end{sphinxadmonition}

\begin{sphinxadmonition}{note}{Pavyzdys (XML)}


\begin{savenotes}\sphinxattablestart
\sphinxthistablewithglobalstyle
\centering
\begin{tabulary}{\linewidth}[t]{TTTTT}
\sphinxtoprule
\sphinxstyletheadfamily 
\sphinxAtStartPar
resource
&\sphinxstyletheadfamily 
\sphinxAtStartPar
type
&\sphinxstyletheadfamily 
\sphinxAtStartPar
ref
&\sphinxstyletheadfamily 
\sphinxAtStartPar
source
&\sphinxstyletheadfamily 
\sphinxAtStartPar
prepare
\\
\sphinxmidrule
\sphinxtableatstartofbodyhook
\sphinxAtStartPar
resource1
&
\sphinxAtStartPar
json
&&
\sphinxAtStartPar
https://example.com/
&
\sphinxAtStartPar
\sphinxcode{\sphinxupquote{http(body: xml)}}
\\
\sphinxhline
\sphinxAtStartPar

&
\sphinxAtStartPar
param
&
\sphinxAtStartPar
name1
&
\sphinxAtStartPar
DATA/@NAME1
&
\sphinxAtStartPar
\sphinxcode{\sphinxupquote{body("value1")}}
\\
\sphinxhline
\sphinxAtStartPar

&&
\sphinxAtStartPar
name2
&
\sphinxAtStartPar
DATA/NAME2
&
\sphinxAtStartPar
\sphinxcode{\sphinxupquote{body("value2")}}
\\
\sphinxhline
\sphinxAtStartPar

&&
\sphinxAtStartPar
name3
&
\sphinxAtStartPar
DATA/NESTED/NAME3
&
\sphinxAtStartPar
\sphinxcode{\sphinxupquote{body("value3")}}
\\
\sphinxhline
\sphinxAtStartPar

&&
\sphinxAtStartPar
name4
&
\sphinxAtStartPar
DATA/ARRAY/NAME4
&
\sphinxAtStartPar
\sphinxcode{\sphinxupquote{body("value4")}}
\\
\sphinxhline
\sphinxAtStartPar

&&
\sphinxAtStartPar
name5
&
\sphinxAtStartPar
DATA/ARRAY/NAME4
&
\sphinxAtStartPar
\sphinxcode{\sphinxupquote{body("value5")}}
\\
\sphinxbottomrule
\end{tabulary}
\sphinxtableafterendhook\par
\sphinxattableend\end{savenotes}

\begin{sphinxVerbatim}[commandchars=\\\{\}]
\PYG{c+cp}{\PYGZlt{}?xml version=\PYGZdq{}1.0\PYGZdq{} encoding=\PYGZdq{}utf\PYGZhy{}8\PYGZdq{}?\PYGZgt{}}
\PYG{n+nt}{\PYGZlt{}DATA}\PYG{+w}{ }\PYG{n+na}{NAME1=}\PYG{l+s}{\PYGZdq{}value1\PYGZdq{}}\PYG{n+nt}{\PYGZgt{}}
\PYG{+w}{    }\PYG{n+nt}{\PYGZlt{}NAME2}\PYG{n+nt}{\PYGZgt{}}value2\PYG{n+nt}{\PYGZlt{}/NAME2\PYGZgt{}}
\PYG{+w}{    }\PYG{n+nt}{\PYGZlt{}NESTED}\PYG{n+nt}{\PYGZgt{}}
\PYG{+w}{        }\PYG{n+nt}{\PYGZlt{}NAME3}\PYG{n+nt}{\PYGZgt{}}value3\PYG{n+nt}{\PYGZlt{}/NAME3\PYGZgt{}}
\PYG{+w}{    }\PYG{n+nt}{\PYGZlt{}/NESTED\PYGZgt{}}
\PYG{+w}{    }\PYG{n+nt}{\PYGZlt{}ARRAY}\PYG{n+nt}{\PYGZgt{}}
\PYG{+w}{        }\PYG{n+nt}{\PYGZlt{}NAME4}\PYG{n+nt}{\PYGZgt{}}value4\PYG{n+nt}{\PYGZlt{}/NAME4\PYGZgt{}}
\PYG{+w}{        }\PYG{n+nt}{\PYGZlt{}NAME4}\PYG{n+nt}{\PYGZgt{}}value5\PYG{n+nt}{\PYGZlt{}/NAME4\PYGZgt{}}
\PYG{+w}{    }\PYG{n+nt}{\PYGZlt{}/ARRAY\PYGZgt{}}
\PYG{n+nt}{\PYGZlt{}/DATA\PYGZgt{}}
\end{sphinxVerbatim}
\end{sphinxadmonition}

\begin{sphinxadmonition}{note}{Pavyzdys (maišytas)}


\begin{savenotes}\sphinxattablestart
\sphinxthistablewithglobalstyle
\centering
\begin{tabulary}{\linewidth}[t]{TTTTT}
\sphinxtoprule
\sphinxstyletheadfamily 
\sphinxAtStartPar
resource
&\sphinxstyletheadfamily 
\sphinxAtStartPar
type
&\sphinxstyletheadfamily 
\sphinxAtStartPar
ref
&\sphinxstyletheadfamily 
\sphinxAtStartPar
source
&\sphinxstyletheadfamily 
\sphinxAtStartPar
prepare
\\
\sphinxmidrule
\sphinxtableatstartofbodyhook
\sphinxAtStartPar
resource1
&
\sphinxAtStartPar
json
&&
\sphinxAtStartPar
https://example.com/
&
\sphinxAtStartPar
\sphinxcode{\sphinxupquote{http(body: xml)}}
\\
\sphinxhline
\sphinxAtStartPar

&
\sphinxAtStartPar
param
&
\sphinxAtStartPar
name1
&
\sphinxAtStartPar
DATA/@NAME1
&
\sphinxAtStartPar
\sphinxcode{\sphinxupquote{body("value1")}}
\\
\sphinxhline
\sphinxAtStartPar

&&
\sphinxAtStartPar
name2
&
\sphinxAtStartPar
DATA/NAME2
&
\sphinxAtStartPar
\sphinxcode{\sphinxupquote{body(name3)}}
\\
\sphinxhline
\sphinxAtStartPar

&&
\sphinxAtStartPar
name3
&&
\sphinxAtStartPar
\sphinxcode{\sphinxupquote{body(type: json)}}
\\
\sphinxhline
\sphinxAtStartPar

&&
\sphinxAtStartPar
name4
&
\sphinxAtStartPar
NAME4
&
\sphinxAtStartPar
\sphinxcode{\sphinxupquote{body("value4", name3)}}
\\
\sphinxhline
\sphinxAtStartPar

&&
\sphinxAtStartPar
name5
&
\sphinxAtStartPar
NAME5
&
\sphinxAtStartPar
\sphinxcode{\sphinxupquote{body("value5", name3)}}
\\
\sphinxbottomrule
\end{tabulary}
\sphinxtableafterendhook\par
\sphinxattableend\end{savenotes}

\begin{sphinxVerbatim}[commandchars=\\\{\}]
\PYG{c+cp}{\PYGZlt{}?xml version=\PYGZdq{}1.0\PYGZdq{} encoding=\PYGZdq{}utf\PYGZhy{}8\PYGZdq{}?\PYGZgt{}}
\PYG{n+nt}{\PYGZlt{}DATA}\PYG{+w}{ }\PYG{n+na}{NAME1=}\PYG{l+s}{\PYGZdq{}value1\PYGZdq{}}\PYG{n+nt}{\PYGZgt{}}
\PYG{+w}{    }\PYG{n+nt}{\PYGZlt{}NAME2}\PYG{n+nt}{\PYGZgt{}}\PYG{c+cp}{\PYGZlt{}![CDATA[}
\PYG{c+cp}{        \PYGZob{}}
\PYG{c+cp}{            \PYGZdq{}NAME4\PYGZdq{}: \PYGZdq{}value4\PYGZdq{},}
\PYG{c+cp}{            \PYGZdq{}NAME5\PYGZdq{}: \PYGZdq{}value5\PYGZdq{}}
\PYG{c+cp}{        \PYGZcb{}}
\PYG{c+cp}{    ]]\PYGZgt{}}\PYG{n+nt}{\PYGZlt{}/NAME2\PYGZgt{}}
\PYG{n+nt}{\PYGZlt{}/DATA\PYGZgt{}}
\end{sphinxVerbatim}
\end{sphinxadmonition}

\end{fulllineitems}


\sphinxAtStartPar
Jei užpildytas {\hyperref[\detokenize{dimensijos:param.source}]{\sphinxcrossref{\sphinxcode{\sphinxupquote{param.source}}}}} stulpelis, tada {\hyperref[\detokenize{dimensijos:param.prepare}]{\sphinxcrossref{\sphinxcode{\sphinxupquote{param.prepare}}}}}
stulpelyje galima naudoti filtrą nurodyto {\hyperref[\detokenize{dimensijos:param.source}]{\sphinxcrossref{\sphinxcode{\sphinxupquote{param.source}}}}} modelio duomenims
filtruoti, o naudojant parametrus galima nurodyti ir modelio laukų pavadinimus,
pavyzdžiui:
\begin{description}
\sphinxlineitem{source}
\sphinxAtStartPar
\sphinxcode{\sphinxupquote{\{x.field\}}}.

\sphinxlineitem{prepare}
\sphinxAtStartPar
\sphinxcode{\sphinxupquote{x.field}} arba \sphinxcode{\sphinxupquote{param(x).field}}.

\end{description}


\subsection{switch}
\label{\detokenize{dimensijos:switch}}\label{\detokenize{dimensijos:id13}}\index{module@\spxentry{module}!switch@\spxentry{switch}}\index{switch@\spxentry{switch}!module@\spxentry{module}}\phantomsection\label{\detokenize{dimensijos:module-switch}}
\sphinxAtStartPar
Tam tikrais atvejais duomenis tenka normalizuoti parenkant tam tikrą reikšmę jei
tenkinama nurodyta sąlyga. Tokias situacijas galima aprašyti pasitelkiant
{\hyperref[\detokenize{formules:switch}]{\sphinxcrossref{\sphinxcode{\sphinxupquote{switch}}}}} dimensiją.
\index{switch.source (modulje switch)@\spxentry{switch.source}\spxextra{modulje switch}}

\begin{fulllineitems}
\phantomsection\label{\detokenize{dimensijos:switch.switch.source}}
\pysigstartsignatures
\pysigline
{\sphinxcode{\sphinxupquote{switch.}}\sphinxbfcode{\sphinxupquote{source}}}
\pysigstopsignatures
\sphinxAtStartPar
Reikšmė, kuri bus atveriama.

\end{fulllineitems}

\index{switch.prepare (modulje switch)@\spxentry{switch.prepare}\spxextra{modulje switch}}

\begin{fulllineitems}
\phantomsection\label{\detokenize{dimensijos:switch.switch.prepare}}
\pysigstartsignatures
\pysigline
{\sphinxcode{\sphinxupquote{switch.}}\sphinxbfcode{\sphinxupquote{prepare}}}
\pysigstopsignatures
\sphinxAtStartPar
Sąlyga, naudojant einamojo modelio laukus. Jei sąlyga tenkinama, tada
laukui priskiriama {\hyperref[\detokenize{dimensijos:switch.switch.source}]{\sphinxcrossref{\sphinxcode{\sphinxupquote{switch.source}}}}} reikšmė. Jei sąlyga
netenkinama, tada bandoma tikrinti sekančią sąlygą. Parenkama ta
reikšmė, kurios pirmoji sąlyga tenkinama.

\sphinxAtStartPar
Jei {\hyperref[\detokenize{dimensijos:switch.switch.prepare}]{\sphinxcrossref{\sphinxcode{\sphinxupquote{switch.prepare}}}}} yra tuščias, tada sąlyga visada teigiama ir
visada grąžinama {\hyperref[\detokenize{dimensijos:switch.switch.source}]{\sphinxcrossref{\sphinxcode{\sphinxupquote{switch.source}}}}} reikšmė.

\end{fulllineitems}



\subsection{comment}
\label{\detokenize{dimensijos:comment}}\label{\detokenize{dimensijos:id14}}\index{module@\spxentry{module}!comment@\spxentry{comment}}\index{comment@\spxentry{comment}!module@\spxentry{module}}\phantomsection\label{\detokenize{dimensijos:module-comment}}
\sphinxAtStartPar
Dirbant su {\hyperref[\detokenize{savokos:term-DSA}]{\sphinxtermref{\DUrole{xref}{\DUrole{std}{\DUrole{std-term}{DSA}}}}}} yra galimybė komentuoti eilutes, naudojant papildomą
{\hyperref[\detokenize{dimensijos:module-comment}]{\sphinxcrossref{\sphinxcode{\sphinxupquote{comment}}}}} dimensiją, kurią galima naudoti bet kurios kitos dimensijos
kontekste.
\index{id (modulje comment)@\spxentry{id}\spxextra{modulje comment}}

\begin{fulllineitems}
\phantomsection\label{\detokenize{dimensijos:comment.id}}
\pysigstartsignatures
\pysigline
{\sphinxcode{\sphinxupquote{comment.}}\sphinxbfcode{\sphinxupquote{id}}}
\pysigstopsignatures
\sphinxAtStartPar
Komentaro numeris.

\end{fulllineitems}

\index{ref (modulje comment)@\spxentry{ref}\spxextra{modulje comment}}

\begin{fulllineitems}
\phantomsection\label{\detokenize{dimensijos:comment.ref}}
\pysigstartsignatures
\pysigline
{\sphinxcode{\sphinxupquote{comment.}}\sphinxbfcode{\sphinxupquote{ref}}}
\pysigstopsignatures
\sphinxAtStartPar
Komentuojamo vieno ar kelių kableliu atskirtų {\hyperref[\detokenize{formatas:property}]{\sphinxcrossref{\sphinxcode{\sphinxupquote{property}}}}}
pavadinimai. Galima nurodyti ne tik stulpelio pavadinimą, bet ir
dimensiją.

\end{fulllineitems}

\index{source (modulje comment)@\spxentry{source}\spxextra{modulje comment}}

\begin{fulllineitems}
\phantomsection\label{\detokenize{dimensijos:comment.source}}
\pysigstartsignatures
\pysigline
{\sphinxcode{\sphinxupquote{comment.}}\sphinxbfcode{\sphinxupquote{source}}}
\pysigstopsignatures
\sphinxAtStartPar
Komentaro autorius.

\end{fulllineitems}

\index{prepare (modulje comment)@\spxentry{prepare}\spxextra{modulje comment}}

\begin{fulllineitems}
\phantomsection\label{\detokenize{dimensijos:comment.prepare}}
\pysigstartsignatures
\pysigline
{\sphinxcode{\sphinxupquote{comment.}}\sphinxbfcode{\sphinxupquote{prepare}}}
\pysigstopsignatures
\sphinxAtStartPar
Keitimo pasiūlymas, naudojant \sphinxcode{\sphinxupquote{create()}}, \sphinxcode{\sphinxupquote{update}} ir \sphinxcode{\sphinxupquote{delete()}} funkcijas. Pavyzdžiui:

\begin{sphinxVerbatim}[commandchars=\\\{\}]
\PYG{n}{update}\PYG{p}{(}\PYG{n+nb}{property}\PYG{p}{:} \PYG{l+s+s2}{\PYGZdq{}}\PYG{l+s+s2}{pavadinimas@lt}\PYG{l+s+s2}{\PYGZdq{}}\PYG{p}{,} \PYG{n+nb}{type}\PYG{p}{:} \PYG{l+s+s2}{\PYGZdq{}}\PYG{l+s+s2}{text}\PYG{l+s+s2}{\PYGZdq{}}\PYG{p}{)}
\end{sphinxVerbatim}

\sphinxAtStartPar
Šiuo atveju nurodoma, kad siūloma keisti \sphinxcode{\sphinxupquote{property}} pavadnimą į
\sphinxcode{\sphinxupquote{pavadinimas@lt}}, o \sphinxcode{\sphinxupquote{type}} į \sphinxcode{\sphinxupquote{text}}.

\end{fulllineitems}

\index{level (modulje comment)@\spxentry{level}\spxextra{modulje comment}}

\begin{fulllineitems}
\phantomsection\label{\detokenize{dimensijos:comment.level}}
\pysigstartsignatures
\pysigline
{\sphinxcode{\sphinxupquote{comment.}}\sphinxbfcode{\sphinxupquote{level}}}
\pysigstopsignatures
\sphinxAtStartPar
Nurodoma, kad patenkinus keitimo siūlymą, kuris nurodytas
{\hyperref[\detokenize{dimensijos:comment.prepare}]{\sphinxcrossref{\sphinxcode{\sphinxupquote{comment.prepare}}}}} stulplyje, komentuojamai eilutei gali būti
suteiktas nurodytas brandos lygis.

\end{fulllineitems}

\index{access (modulje comment)@\spxentry{access}\spxextra{modulje comment}}

\begin{fulllineitems}
\phantomsection\label{\detokenize{dimensijos:comment.access}}
\pysigstartsignatures
\pysigline
{\sphinxcode{\sphinxupquote{comment.}}\sphinxbfcode{\sphinxupquote{access}}}
\pysigstopsignatures
\sphinxAtStartPar
Nurodoma, ar komentaras gali būti publikuojamas viešai.
\begin{description}
\sphinxlineitem{private}
\sphinxAtStartPar
Komentaras negali būti publikuojamas viešai. Šis prieigos lygis
naudojamas pagal nutylėjimą.

\sphinxlineitem{open}
\sphinxAtStartPar
Komentaras gali būti publikuojamas viešai.

\end{description}

\end{fulllineitems}

\index{uri (modulje comment)@\spxentry{uri}\spxextra{modulje comment}}

\begin{fulllineitems}
\phantomsection\label{\detokenize{dimensijos:comment.uri}}
\pysigstartsignatures
\pysigline
{\sphinxcode{\sphinxupquote{comment.}}\sphinxbfcode{\sphinxupquote{uri}}}
\pysigstopsignatures
\sphinxAtStartPar
Viena ar kelios kableliu atskirtos šaltinio nuorodos, kurios pateikia
daugiau informacijos apie tai, kas komentuojama. Taip pat gali būti
nurodytas kito komentaro {\hyperref[\detokenize{dimensijos:comment.id}]{\sphinxcrossref{\sphinxcode{\sphinxupquote{comment.id}}}}}, nurodant, kad tai yra
atsakymas į ankstesnį komentarą.

\sphinxAtStartPar
URI pateikiami sutrumpinta forma, naudojant prefikstus. Žiūrėti skrių
{\hyperref[\detokenize{zodynai:vocab}]{\sphinxcrossref{\DUrole{std}{\DUrole{std-ref}{Išoriniai žodynai}}}}}.

\end{fulllineitems}

\index{title (modulje comment)@\spxentry{title}\spxextra{modulje comment}}

\begin{fulllineitems}
\phantomsection\label{\detokenize{dimensijos:comment.title}}
\pysigstartsignatures
\pysigline
{\sphinxcode{\sphinxupquote{comment.}}\sphinxbfcode{\sphinxupquote{title}}}
\pysigstopsignatures
\sphinxAtStartPar
Komentaro data, \sphinxhref{https://en.wikipedia.org/wiki/ISO\_8601}{ISO 8601} formatu.

\end{fulllineitems}

\index{description (modulje comment)@\spxentry{description}\spxextra{modulje comment}}

\begin{fulllineitems}
\phantomsection\label{\detokenize{dimensijos:comment.description}}
\pysigstartsignatures
\pysigline
{\sphinxcode{\sphinxupquote{comment.}}\sphinxbfcode{\sphinxupquote{description}}}
\pysigstopsignatures
\sphinxAtStartPar
Komentaro tekstas.

\end{fulllineitems}


\sphinxAtStartPar
\sphinxstylestrong{Pavyzdys}


\begin{savenotes}\sphinxattablestart
\sphinxthistablewithglobalstyle
\centering
\begin{tabulary}{\linewidth}[t]{TTTTTTTTTTT}
\sphinxtoprule
\sphinxstyletheadfamily 
\sphinxAtStartPar
d
&\sphinxstyletheadfamily 
\sphinxAtStartPar
r
&\sphinxstyletheadfamily 
\sphinxAtStartPar
b
&\sphinxstyletheadfamily 
\sphinxAtStartPar
m
&\sphinxstyletheadfamily 
\sphinxAtStartPar
property
&\sphinxstyletheadfamily 
\sphinxAtStartPar
type
&\sphinxstyletheadfamily 
\sphinxAtStartPar
ref
&\sphinxstyletheadfamily 
\sphinxAtStartPar
prepare
&\sphinxstyletheadfamily 
\sphinxAtStartPar
level
&\sphinxstyletheadfamily 
\sphinxAtStartPar
access
&\sphinxstyletheadfamily 
\sphinxAtStartPar
uri
\\
\sphinxmidrule
\sphinxtableatstartofbodyhook\sphinxstartmulticolumn{5}%
\begin{varwidth}[t]{\sphinxcolwidth{5}{11}}
\sphinxAtStartPar
example
\par
\vskip-\baselineskip\vbox{\hbox{\strut}}\end{varwidth}%
\sphinxstopmulticolumn
&&&&&&\\
\sphinxhline
\sphinxAtStartPar

&&&&&
\sphinxAtStartPar
prefix
&
\sphinxAtStartPar
spinta
&&&&
\sphinxAtStartPar
https://github.com/atviriduomenys/spinta/issues/
\\
\sphinxhline
\sphinxAtStartPar

&&&&&&
\sphinxAtStartPar
dsa
&&&&
\sphinxAtStartPar
https://ivpk.github.io/dsa/
\\
\sphinxhline
\sphinxAtStartPar

&&&\sphinxstartmulticolumn{2}%
\begin{varwidth}[t]{\sphinxcolwidth{2}{11}}
\sphinxAtStartPar
Imone
\par
\vskip-\baselineskip\vbox{\hbox{\strut}}\end{varwidth}%
\sphinxstopmulticolumn
&&&&
\sphinxAtStartPar
2
&&\\
\sphinxhline
\sphinxAtStartPar

&&&&&
\sphinxAtStartPar
comment
&
\sphinxAtStartPar
base
&
\sphinxAtStartPar
update(base: "/jar/JuridinisAsmuo", ref: "id")
&
\sphinxAtStartPar
4
&
\sphinxAtStartPar
open
&
\sphinxAtStartPar
spinta:205, manifest:1290
\\
\sphinxhline
\sphinxAtStartPar

&&&&&
\sphinxAtStartPar
comment
&
\sphinxAtStartPar
ref
&
\sphinxAtStartPar
update(ref: "id")
&
\sphinxAtStartPar
4
&
\sphinxAtStartPar
open
&
\sphinxAtStartPar
vadovas:dsa/dimensijos.html\#model.ref
\\
\sphinxhline
\sphinxAtStartPar

&&&&
\sphinxAtStartPar
id
&
\sphinxAtStartPar
integer
&&&
\sphinxAtStartPar
4
&
\sphinxAtStartPar
open
&\\
\sphinxhline
\sphinxAtStartPar

&&&&
\sphinxAtStartPar
pavadinimas
&
\sphinxAtStartPar
string
&&&
\sphinxAtStartPar
2
&
\sphinxAtStartPar
open
&\\
\sphinxhline
\sphinxAtStartPar

&&&&&
\sphinxAtStartPar
comment
&
\sphinxAtStartPar
ref
&
\sphinxAtStartPar
update(property: "pavadinimas@lt", type: "text")
&
\sphinxAtStartPar
4
&
\sphinxAtStartPar
open
&
\sphinxAtStartPar
spinta:204
\\
\sphinxbottomrule
\end{tabulary}
\sphinxtableafterendhook\par
\sphinxattableend\end{savenotes}


\subsection{lang}
\label{\detokenize{dimensijos:lang}}\label{\detokenize{dimensijos:id15}}\index{module@\spxentry{module}!lang@\spxentry{lang}}\index{lang@\spxentry{lang}!module@\spxentry{module}}\phantomsection\label{\detokenize{dimensijos:module-lang}}
\sphinxAtStartPar
\DUrole{versionmodified}{\DUrole{deprecated}{Nebepalaikoma nuo 0.2 versijos.}}

\sphinxAtStartPar
{\hyperref[\detokenize{formatas:title}]{\sphinxcrossref{\sphinxcode{\sphinxupquote{title}}}}} ir {\hyperref[\detokenize{formatas:description}]{\sphinxcrossref{\sphinxcode{\sphinxupquote{description}}}}} stulpeliuose tekstas rašomas lietuvių
kalba, tačiau galima pateikti tekstą ir kita kalba, panaudojus papildomą
{\hyperref[\detokenize{dimensijos:module-lang}]{\sphinxcrossref{\sphinxcode{\sphinxupquote{lang}}}}} dimensiją, kurią reikia naudoti prieš eilutę, kuriai pateikiamas
tekstas kita kalba.
\index{ref (modulje lang)@\spxentry{ref}\spxextra{modulje lang}}

\begin{fulllineitems}
\phantomsection\label{\detokenize{dimensijos:lang.ref}}
\pysigstartsignatures
\pysigline
{\sphinxcode{\sphinxupquote{lang.}}\sphinxbfcode{\sphinxupquote{ref}}}
\pysigstopsignatures
\sphinxAtStartPar
\sphinxhref{https://en.wikipedia.org/wiki/List\_of\_ISO\_639-1\_codes}{ISO 639\sphinxhyphen{}1} dviejų simbolių kalbos kodas.

\end{fulllineitems}

\index{title (modulje lang)@\spxentry{title}\spxextra{modulje lang}}

\begin{fulllineitems}
\phantomsection\label{\detokenize{dimensijos:lang.title}}
\pysigstartsignatures
\pysigline
{\sphinxcode{\sphinxupquote{lang.}}\sphinxbfcode{\sphinxupquote{title}}}
\pysigstopsignatures
\sphinxAtStartPar
Pavadinimas {\hyperref[\detokenize{dimensijos:lang.ref}]{\sphinxcrossref{\sphinxcode{\sphinxupquote{lang.ref}}}}} stulpelyje nurodyta kalba.

\end{fulllineitems}

\index{description (modulje lang)@\spxentry{description}\spxextra{modulje lang}}

\begin{fulllineitems}
\phantomsection\label{\detokenize{dimensijos:lang.description}}
\pysigstartsignatures
\pysigline
{\sphinxcode{\sphinxupquote{lang.}}\sphinxbfcode{\sphinxupquote{description}}}
\pysigstopsignatures
\sphinxAtStartPar
Aprašymas {\hyperref[\detokenize{dimensijos:lang.ref}]{\sphinxcrossref{\sphinxcode{\sphinxupquote{lang.ref}}}}} stulpelyje nurodyta kalba.

\end{fulllineitems}



\subsection{migrate}
\label{\detokenize{dimensijos:migrate}}\label{\detokenize{dimensijos:id16}}\index{module@\spxentry{module}!migrate@\spxentry{migrate}}\index{migrate@\spxentry{migrate}!module@\spxentry{module}}\phantomsection\label{\detokenize{dimensijos:module-migrate}}
\sphinxAtStartPar
\DUrole{versionmodified}{\DUrole{deprecated}{Nebepalaikoma nuo 0.2 versijos.}}

\sphinxAtStartPar
Laikui einant, pirminių duomenų šaltinių arba jau atvertų duomenų struktūra
keičiasi, papildoma naujais {\hyperref[\detokenize{savokos:term-modelis}]{\sphinxtermref{\DUrole{xref}{\DUrole{std}{\DUrole{std-term}{modeliais}}}}}} ar {\hyperref[\detokenize{savokos:term-savybe}]{\sphinxtermref{\DUrole{xref}{\DUrole{std}{\DUrole{std-term}{savybėmis}}}}}}, keliant duomenų brandos lygį seni duomenys keičiami naujais,
aukštesnio brandos lygio duomenimis.

\sphinxAtStartPar
Visi šie struktūros ar pačių duomenų pasikeitimai fiksuojami papildomos
{\hyperref[\detokenize{dimensijos:module-migrate}]{\sphinxcrossref{\sphinxcode{\sphinxupquote{migrate}}}}} dimensijos pagalba, kuri gali būti naudojama, bet kurios kitos
dimensijos kontekste.

\begin{sphinxadmonition}{note}{Pastaba:}
\sphinxAtStartPar
Migracijos naudojamos tik tuo atveju, kai keičiasi duomenų struktūra arba
patys duomenys. Jei keičiasi tik metaduomenys, tai migracijų sąraše
neatsispindi.
\end{sphinxadmonition}


\begin{savenotes}\sphinxattablestart
\sphinxthistablewithglobalstyle
\centering
\begin{tabulary}{\linewidth}[t]{TTTTTTTTTTTT}
\sphinxtoprule
\sphinxstyletheadfamily 
\sphinxAtStartPar
id
&\sphinxstyletheadfamily 
\sphinxAtStartPar
d
&\sphinxstyletheadfamily 
\sphinxAtStartPar
r
&\sphinxstyletheadfamily 
\sphinxAtStartPar
b
&\sphinxstyletheadfamily 
\sphinxAtStartPar
m
&\sphinxstyletheadfamily 
\sphinxAtStartPar
property
&\sphinxstyletheadfamily 
\sphinxAtStartPar
type
&\sphinxstyletheadfamily 
\sphinxAtStartPar
ref
&\sphinxstyletheadfamily 
\sphinxAtStartPar
prepare
&\sphinxstyletheadfamily 
\sphinxAtStartPar
level
&\sphinxstyletheadfamily 
\sphinxAtStartPar
title
&\sphinxstyletheadfamily 
\sphinxAtStartPar
description
\\
\sphinxmidrule
\sphinxtableatstartofbodyhook
\sphinxAtStartPar
1
&&&&&&
\sphinxAtStartPar
migrate
&&&&
\sphinxAtStartPar
2021\sphinxhyphen{}12\sphinxhyphen{}21 16:29
&
\sphinxAtStartPar
Pirmoji migracija.
\\
\sphinxhline
\sphinxAtStartPar
2
&&&&&&
\sphinxAtStartPar
migrate
&
\sphinxAtStartPar
1
&&&
\sphinxAtStartPar
2021\sphinxhyphen{}12\sphinxhyphen{}21 16:33
&
\sphinxAtStartPar
Antroji migracija.
\\
\sphinxhline
\sphinxAtStartPar
3
&&&&&&
\sphinxAtStartPar
migrate
&
\sphinxAtStartPar
2
&&&
\sphinxAtStartPar
2022\sphinxhyphen{}06\sphinxhyphen{}21 16:41
&
\sphinxAtStartPar
Trečioji migracija.
\\
\sphinxhline
\sphinxAtStartPar

&\sphinxstartmulticolumn{5}%
\begin{varwidth}[t]{\sphinxcolwidth{5}{12}}
\sphinxAtStartPar
datasets/example/migrate
\par
\vskip-\baselineskip\vbox{\hbox{\strut}}\end{varwidth}%
\sphinxstopmulticolumn
&&&&&&\\
\sphinxhline
\sphinxAtStartPar

&&&&\sphinxstartmulticolumn{2}%
\begin{varwidth}[t]{\sphinxcolwidth{2}{12}}
\sphinxAtStartPar
Country
\par
\vskip-\baselineskip\vbox{\hbox{\strut}}\end{varwidth}%
\sphinxstopmulticolumn
&&
\sphinxAtStartPar
id
&&&&\\
\sphinxhline
\sphinxAtStartPar

&&&&&
\sphinxAtStartPar
id
&
\sphinxAtStartPar
integer
&&&
\sphinxAtStartPar
4
&&\\
\sphinxhline
\sphinxAtStartPar

&&&&&
\sphinxAtStartPar
code
&
\sphinxAtStartPar
string
&&&
\sphinxAtStartPar
3
&&\\
\sphinxhline
\sphinxAtStartPar

&&&&&&
\sphinxAtStartPar
migrate
&
\sphinxAtStartPar
1
&
\sphinxAtStartPar
create(level: 2)
&&&\\
\sphinxhline
\sphinxAtStartPar

&&&&&&
\sphinxAtStartPar
migrate
&
\sphinxAtStartPar
3
&
\sphinxAtStartPar
update(level: 3)
&&&\\
\sphinxhline
\sphinxAtStartPar

&&&&&
\sphinxAtStartPar
name
&
\sphinxAtStartPar
string
&&&&&\\
\sphinxhline
\sphinxAtStartPar

&&&&&&
\sphinxAtStartPar
migrate
&
\sphinxAtStartPar
2
&
\sphinxAtStartPar
create()
&&&\\
\sphinxbottomrule
\end{tabulary}
\sphinxtableafterendhook\par
\sphinxattableend\end{savenotes}

\sphinxAtStartPar
Pavyzdyje aukščiau matome, kad šis duomenų struktūros aprašas turi tris
migracijas:
\begin{enumerate}
\sphinxsetlistlabels{\arabic}{enumi}{enumii}{}{.}%
\item {} 
\sphinxAtStartPar
Pirmosios migracijos metu sukuriamas pradinis duomenų struktūros variantas.
Pirmoji migracija nežymima prie modelių ir duomenų laukų, nebent daromas
keitimas, tuomet įtraukiam ir pirmoji migracija, kad būtų matoma, kas
keitėsi. Būtent toks atvejis parodytas prie \sphinxcode{\sphinxupquote{Country.code}} lauko, kuri
trečiojo migracijoje keičiamas brandos lygis.

\item {} 
\sphinxAtStartPar
Antrosios migracijos metu buvo įtrauktas naujas duomenų laukas
\sphinxcode{\sphinxupquote{Country.name}}.

\item {} 
\sphinxAtStartPar
Trečiosios migracijos metu, buvo keičiami \sphinxcode{\sphinxupquote{Country.code}} lauko duomenys,
pakeitimo metu brandos lygis buvo pakeltas iki trečio. Atkreipkite dėmesį,
kad metaduomenų pasikeitimas, kaip šiuo atveju, žymimas migracijose tik tuo
atveju, jei tai yra susiję su pačių duomenų pasikeitimu.

\sphinxAtStartPar
Jei brandos lygis būtų pakeistas, nekeičiant pačių duomenų, tuomet tokio
pakeitimo nereikėtų įtraukti į migracijų sąrašą.

\sphinxAtStartPar
Kadangi trečiojoje migracijoje buvo atliktas su ankstesne versija
nesuderinamas pakeitimas, tai šios migracijos data yra 6 mėnesiai
ateityje, kadangi nesuderinamos migracijos pirmiausia paskelbiamos, o
įgyvendinamos tik praėjus 6 mėnesiams nuo paskelbimo.

\end{enumerate}
\index{id (modulje migrate)@\spxentry{id}\spxextra{modulje migrate}}

\begin{fulllineitems}
\phantomsection\label{\detokenize{dimensijos:migrate.id}}
\pysigstartsignatures
\pysigline
{\sphinxcode{\sphinxupquote{migrate.}}\sphinxbfcode{\sphinxupquote{id}}}
\pysigstopsignatures
\sphinxAtStartPar
Migracijos numeris (UUID). Kiekvienos migracijos metu gali būti
atliekama eilė operacijų, visos operacijos fiksuojamos naudojant
migracijos numerį.

\sphinxAtStartPar
Visų migracijų sąrašas pateikiamas, kai {\hyperref[\detokenize{dimensijos:module-migrate}]{\sphinxcrossref{\sphinxcode{\sphinxupquote{migrate}}}}} nepriklauso
jokiam dimensijos kontekstui.

\end{fulllineitems}

\index{ref (modulje migrate)@\spxentry{ref}\spxextra{modulje migrate}}

\begin{fulllineitems}
\phantomsection\label{\detokenize{dimensijos:migrate.ref}}
\pysigstartsignatures
\pysigline
{\sphinxcode{\sphinxupquote{migrate.}}\sphinxbfcode{\sphinxupquote{ref}}}
\pysigstopsignatures
\sphinxAtStartPar
Ankstesnės migracijos numeris, pateiktas {\hyperref[\detokenize{dimensijos:migrate.id}]{\sphinxcrossref{\sphinxcode{\sphinxupquote{migrate.id}}}}} stulpelyje,
arba tuščia, jei prieš tai jokių kitų migracijų nebuvo.

\sphinxAtStartPar
Naudojamas jei {\hyperref[\detokenize{dimensijos:module-migrate}]{\sphinxcrossref{\sphinxcode{\sphinxupquote{migrate}}}}} nepatenka į jokios dimensijos kontekstą.

\sphinxAtStartPar
Jei {\hyperref[\detokenize{dimensijos:module-migrate}]{\sphinxcrossref{\sphinxcode{\sphinxupquote{migrate}}}}} aprašomas dimensijos kontekste, tada šis stulpelis
nenaudojamas.

\end{fulllineitems}

\index{prepare (modulje migrate)@\spxentry{prepare}\spxextra{modulje migrate}}

\begin{fulllineitems}
\phantomsection\label{\detokenize{dimensijos:migrate.prepare}}
\pysigstartsignatures
\pysigline
{\sphinxcode{\sphinxupquote{migrate.}}\sphinxbfcode{\sphinxupquote{prepare}}}
\pysigstopsignatures
\sphinxAtStartPar
Migracijos operacija. Galimos tokios operacijos:
\index{create() (modulyje migrate)@\spxentry{create()}\spxextra{modulyje migrate}}

\begin{fulllineitems}
\phantomsection\label{\detokenize{dimensijos:migrate.create}}
\pysigstartsignatures
\pysiglinewithargsret
{\sphinxcode{\sphinxupquote{migrate.}}\sphinxbfcode{\sphinxupquote{create}}}
{}
{}
\pysigstopsignatures
\sphinxAtStartPar
Priklausomai nuo dimensijos konteksto, prideda naują modelį, arba
savybę.

\sphinxAtStartPar
Funkcijai galima perduoti \sphinxcode{\sphinxupquote{ref}} ir kitus vardinius argumentus,
kurie atitinka {\hyperref[\detokenize{savokos:term-DSA}]{\sphinxtermref{\DUrole{xref}{\DUrole{std}{\DUrole{std-term}{DSA}}}}}} lentelės metaduomenų stulpelių
pavadinimus.

\end{fulllineitems}

\index{update() (modulyje migrate)@\spxentry{update()}\spxextra{modulyje migrate}}

\begin{fulllineitems}
\phantomsection\label{\detokenize{dimensijos:migrate.update}}
\pysigstartsignatures
\pysiglinewithargsret
{\sphinxcode{\sphinxupquote{migrate.}}\sphinxbfcode{\sphinxupquote{update}}}
{}
{}
\pysigstopsignatures
\sphinxAtStartPar
Taikomas tik duomenų laukams ir nurodo, kad buvo pakeistos esamų
duomenų reikšmės, keičiant reikšmių dimensiją, matavimo vienetus,
formatą ir kita.

\sphinxAtStartPar
Funkcijai galima perduoti \sphinxcode{\sphinxupquote{ref}} ir kitus vardinius argumentus,
kurie atitinka {\hyperref[\detokenize{savokos:term-DSA}]{\sphinxtermref{\DUrole{xref}{\DUrole{std}{\DUrole{std-term}{DSA}}}}}} lentelės metaduomenų stulpelių
pavadinimus.

\sphinxAtStartPar
Perduodami tik tie vardiniai argumentai, kuriuos atitinkantys
metaduomenys keičiasi.

\end{fulllineitems}

\index{delete() (modulyje migrate)@\spxentry{delete()}\spxextra{modulyje migrate}}

\begin{fulllineitems}
\phantomsection\label{\detokenize{dimensijos:migrate.delete}}
\pysigstartsignatures
\pysiglinewithargsret
{\sphinxcode{\sphinxupquote{migrate.}}\sphinxbfcode{\sphinxupquote{delete}}}
{}
{}
\pysigstopsignatures
\sphinxAtStartPar
Priklausomai nuo dimensijos konteksto, šalina modelį ar savybę.

\sphinxAtStartPar
Pašalinto modelio ar savybės {\hyperref[\detokenize{tipai:module-type}]{\sphinxcrossref{\sphinxcode{\sphinxupquote{type}}}}} keičiamas į \sphinxcode{\sphinxupquote{absent}}
reikšmę.

\end{fulllineitems}

\index{filter() (modulyje migrate)@\spxentry{filter()}\spxextra{modulyje migrate}}

\begin{fulllineitems}
\phantomsection\label{\detokenize{dimensijos:migrate.filter}}
\pysigstartsignatures
\pysiglinewithargsret
{\sphinxcode{\sphinxupquote{migrate.}}\sphinxbfcode{\sphinxupquote{filter}}}
{\sphinxparam{\DUrole{n}{where}}}
{}
\pysigstopsignatures
\sphinxAtStartPar
Naudojamas {\hyperref[\detokenize{formatas:property}]{\sphinxcrossref{\sphinxcode{\sphinxupquote{property}}}}} kontekste, kai vykdoma duomenų
migracija. Nurodo, kad migracija taikoma tik \sphinxcode{\sphinxupquote{where}} sąlygą
tenkinantiems duomenims.

\end{fulllineitems}


\sphinxAtStartPar
Be šių pagrindinių migracijos operacijų, galima naudoti kitas duomenų
transformavimo operacijas, kurios vykdomos su kiekviena duomenų eilute
ir atlikus pateiktas transformacijos funkcijas, pakeista reikšmė
išsaugoma.

\end{fulllineitems}

\index{title (modulje migrate)@\spxentry{title}\spxextra{modulje migrate}}

\begin{fulllineitems}
\phantomsection\label{\detokenize{dimensijos:migrate.title}}
\pysigstartsignatures
\pysigline
{\sphinxcode{\sphinxupquote{migrate.}}\sphinxbfcode{\sphinxupquote{title}}}
\pysigstopsignatures
\sphinxAtStartPar
Migracijos įvykdymo data ir laikas. Migracijos laikas ir data gali
būti ir ateityje, tuo atveju, jei daromas nesuderinamas keitimas.

\sphinxAtStartPar
Naudojamas tik tada, kai {\hyperref[\detokenize{dimensijos:module-migrate}]{\sphinxcrossref{\sphinxcode{\sphinxupquote{migrate}}}}} nepatenka į jokios dimensijos
kontekstą.

\end{fulllineitems}

\index{description (modulje migrate)@\spxentry{description}\spxextra{modulje migrate}}

\begin{fulllineitems}
\phantomsection\label{\detokenize{dimensijos:migrate.description}}
\pysigstartsignatures
\pysigline
{\sphinxcode{\sphinxupquote{migrate.}}\sphinxbfcode{\sphinxupquote{description}}}
\pysigstopsignatures
\sphinxAtStartPar
Migracijos atliekamo pakeitimo trumpas aprašymas.

\end{fulllineitems}


\sphinxstepscope


\section{Duomenų tipai}
\label{\detokenize{tipai:module-type}}\label{\detokenize{tipai:id1}}\label{\detokenize{tipai:duomenu-tipai}}\label{\detokenize{tipai::doc}}\index{module@\spxentry{module}!type@\spxentry{type}}\index{type@\spxentry{type}!module@\spxentry{module}}\index{absent (modulje type)@\spxentry{absent}\spxextra{modulje type}}

\begin{fulllineitems}
\phantomsection\label{\detokenize{tipai:type.absent}}
\pysigstartsignatures
\pysigline
{\sphinxcode{\sphinxupquote{type.}}\sphinxbfcode{\sphinxupquote{absent}}}
\pysigstopsignatures
\sphinxAtStartPar
\DUrole{versionmodified}{\DUrole{deprecated}{Nebepalaikoma nuo 0.2 versijos: }}Šis tipas buvo naudojamas migracijoms ir versijavimui, tačiau nuo 0.2
versijos, versijavimo struktūros aprašuose atsisakyta.

\sphinxAtStartPar
Žymi {\hyperref[\detokenize{dimensijos:property}]{\sphinxcrossref{\DUrole{std}{\DUrole{std-ref}{savybę}}}}}, kuri buvo ištrinta ir nebenaudojama.


\begin{sphinxseealso}{Taip pat žiūrėkite:}

\sphinxAtStartPar
{\hyperref[\detokenize{dimensijos:migrate}]{\sphinxcrossref{\DUrole{std}{\DUrole{std-ref}{migrate}}}}}


\end{sphinxseealso}


\end{fulllineitems}

\index{boolean (modulje type)@\spxentry{boolean}\spxextra{modulje type}}

\begin{fulllineitems}
\phantomsection\label{\detokenize{tipai:type.boolean}}
\pysigstartsignatures
\pysigline
{\sphinxcode{\sphinxupquote{type.}}\sphinxbfcode{\sphinxupquote{boolean}}}
\pysigstopsignatures
\sphinxAtStartPar
Loginė reikšmė, pateikiama skaitine forma:


\begin{savenotes}\sphinxattablestart
\sphinxthistablewithglobalstyle
\centering
\begin{tabulary}{\linewidth}[t]{TT}
\sphinxtoprule
\sphinxtableatstartofbodyhook
\sphinxAtStartPar
\sphinxcode{\sphinxupquote{0}}
&
\sphinxAtStartPar
Neigiama reikšmė (\sphinxstyleemphasis{false}).
\\
\sphinxhline
\sphinxAtStartPar
\sphinxcode{\sphinxupquote{1}}
&
\sphinxAtStartPar
Teigiama reikšmė (\sphinxstyleemphasis{true}).
\\
\sphinxbottomrule
\end{tabulary}
\sphinxtableafterendhook\par
\sphinxattableend\end{savenotes}

\begin{sphinxtopic}
\sphinxstyletopictitle{Brandos lygis}
\begin{description}
\sphinxlineitem{{\hyperref[\detokenize{branda:l102}]{\sphinxcrossref{\DUrole{std}{\DUrole{std-ref}{L102: Nėra vientisumo}}}}}}
\sphinxAtStartPar
Duomenyse nėra vientisumo, kartais \sphinxcode{\sphinxupquote{true}} pateikta kaip \sphinxcode{\sphinxupquote{1}}, kartais
kaip \sphinxcode{\sphinxupquote{taip}}, arba \sphinxcode{\sphinxupquote{yes}}.

\sphinxlineitem{{\hyperref[\detokenize{branda:l202}]{\sphinxcrossref{\DUrole{std}{\DUrole{std-ref}{L202: Nestandartinis formatas}}}}}}
\sphinxAtStartPar
Visi duomenys pateikti vienoda, tačiau nestandartine forma. Tarkime
duomenys pateikti \sphinxcode{\sphinxupquote{true}}, \sphinxcode{\sphinxupquote{false}} reikšmėmis, tačiau turėtu būti
\sphinxcode{\sphinxupquote{1}}, \sphinxcode{\sphinxupquote{0}}.

\end{description}
\end{sphinxtopic}

\end{fulllineitems}

\index{integer (modulje type)@\spxentry{integer}\spxextra{modulje type}}

\begin{fulllineitems}
\phantomsection\label{\detokenize{tipai:type.integer}}
\pysigstartsignatures
\pysigline
{\sphinxcode{\sphinxupquote{type.}}\sphinxbfcode{\sphinxupquote{integer}}}
\pysigstopsignatures
\sphinxAtStartPar
Sveikas skaičius.

\sphinxAtStartPar
{\hyperref[\detokenize{dimensijos:property.ref}]{\sphinxcrossref{\sphinxcode{\sphinxupquote{property.ref}}}}} stulpelyje, nurodomi {\hyperref[\detokenize{vienetai:matavimo-vienetai}]{\sphinxcrossref{\DUrole{std}{\DUrole{std-ref}{Matavimo vienetai}}}}}.

\begin{sphinxtopic}
\sphinxstyletopictitle{Brandos lygis}
\begin{description}
\sphinxlineitem{{\hyperref[\detokenize{branda:l102}]{\sphinxcrossref{\DUrole{std}{\DUrole{std-ref}{L102: Nėra vientisumo}}}}}}
\sphinxAtStartPar
Duomenys pateikti skirtingais vienetais, pavyzdžui dalis duomenų
pateikti metrais, dalis kilometrais ir dalis milimetrais.

\sphinxlineitem{{\hyperref[\detokenize{branda:l105}]{\sphinxcrossref{\DUrole{std}{\DUrole{std-ref}{L105: Vienetų konvertavimo paklaida}}}}}}
\sphinxAtStartPar
Amžius pateiktas atskirai, nurodant metus, menesius ir dienas. Šiuo
atveju brandos lygis yra 1, kadangi neaišku, kiek yra dienų metuose
ir mėnesiuose, kadangi skirtingi metai ir skirtingi mėnesiai turi
skirtingą dienų skaičių. Tokiais atvejais duomenis reikia pateikti
dienomis.

\sphinxlineitem{{\hyperref[\detokenize{branda:l202}]{\sphinxcrossref{\DUrole{std}{\DUrole{std-ref}{L202: Nestandartinis formatas}}}}}}
\sphinxAtStartPar
Duomenys pateikti išskaidant vieną reikšmę į kelias reikšmes,
skirtingais vienetais. Duomenys turėtu būti pateikti mažiausiu
detalumu, viename duomenų lauke. Pavyzdžiui atstumas pateiktas
atskirais duomenų laukais, kur viename nurodomas atstumas
kilometrais, kitame metrais, trečiame milimetrais. Šiuo atveju,
duomenys turėtu būti pateikiami milimetrais.

\sphinxlineitem{{\hyperref[\detokenize{branda:l302}]{\sphinxcrossref{\DUrole{std}{\DUrole{std-ref}{L302: Nenurodyti matavimo vienetai}}}}}}
\sphinxAtStartPar
Duomenys yra kiekybiniai, tačiau {\hyperref[\detokenize{dimensijos:property.ref}]{\sphinxcrossref{\sphinxcode{\sphinxupquote{property.ref}}}}} stulpelyje
nenurodyti vienetai.

\end{description}
\end{sphinxtopic}

\end{fulllineitems}

\index{number (modulje type)@\spxentry{number}\spxextra{modulje type}}

\begin{fulllineitems}
\phantomsection\label{\detokenize{tipai:type.number}}
\pysigstartsignatures
\pysigline
{\sphinxcode{\sphinxupquote{type.}}\sphinxbfcode{\sphinxupquote{number}}}
\pysigstopsignatures
\sphinxAtStartPar
Realusis skaičius, apvalinamas naudojant \sphinxhref{https://en.wikipedia.org/wiki/IEEE\_754}{slankiojo kablelio aritmetiką}.

\sphinxAtStartPar
{\hyperref[\detokenize{dimensijos:property.ref}]{\sphinxcrossref{\sphinxcode{\sphinxupquote{property.ref}}}}} stulpelyje, nurodomi {\hyperref[\detokenize{vienetai:matavimo-vienetai}]{\sphinxcrossref{\DUrole{std}{\DUrole{std-ref}{Matavimo vienetai}}}}}.

\sphinxAtStartPar
Sveikoji dalis atskiriama \sphinxcode{\sphinxupquote{.}} simbolių.

\end{fulllineitems}

\index{binary (modulje type)@\spxentry{binary}\spxextra{modulje type}}

\begin{fulllineitems}
\phantomsection\label{\detokenize{tipai:type.binary}}
\pysigstartsignatures
\pysigline
{\sphinxcode{\sphinxupquote{type.}}\sphinxbfcode{\sphinxupquote{binary}}}
\pysigstopsignatures
\sphinxAtStartPar
Dvejetainiai duomenys. Bendras baitų skaičius turi būti ne didesnis nei 1G.

\sphinxAtStartPar
Jei reikšmė yra didesnė nei 1G reikųtu naudoti {\hyperref[\detokenize{tipai:type.file}]{\sphinxcrossref{\sphinxcode{\sphinxupquote{type.file}}}}}.

\end{fulllineitems}

\index{string (modulje type)@\spxentry{string}\spxextra{modulje type}}

\begin{fulllineitems}
\phantomsection\label{\detokenize{tipai:type.string}}
\pysigstartsignatures
\pysigline
{\sphinxcode{\sphinxupquote{type.}}\sphinxbfcode{\sphinxupquote{string}}}
\pysigstopsignatures
\sphinxAtStartPar
Simbolių eilutė. Neriboto dydžio, tačiau fiziškai simbolių eilutė turėtu
būti ne didesnė, nei 1G.

\sphinxAtStartPar
Simboliu eilutė turėtu būti pateikta UTF\sphinxhyphen{}8 koduote.

\sphinxAtStartPar
Šiuo tipu žymimi duomenų laukai, kuriuose tekstas pateiktas ne žmonių
kalba. Tai gali būti įvairūs kategoriniai duomenys, identifikatoriai ar
kito pobūdžio simbolių eilutės, kurios nėra užrašytos natūraliąja žmonių
kalba.

\sphinxAtStartPar
Jei {\hyperref[\detokenize{formatas:property}]{\sphinxcrossref{\sphinxcode{\sphinxupquote{property}}}}} pavadinimas turi kalbos žymę \sphinxcode{\sphinxupquote{@}}, tada \sphinxcode{\sphinxupquote{string}} tipas
tampa \sphinxcode{\sphinxupquote{text}} tipo dalimi. Kablos kodas nurodomas naudojant \sphinxhref{https://en.wikipedia.org/wiki/List\_of\_ISO\_639-1\_codes}{ISO 639\sphinxhyphen{}1}
kodų sąrašą.

\sphinxAtStartPar
Jei tekstas turi kalbos žyme, {\hyperref[\detokenize{dimensijos:property.ref}]{\sphinxcrossref{\sphinxcode{\sphinxupquote{property.ref}}}}} galima pateikti teksto
formatą, nadojant vieną iš šių formatų:


\begin{savenotes}\sphinxattablestart
\sphinxthistablewithglobalstyle
\centering
\begin{tabulary}{\linewidth}[t]{TT}
\sphinxtoprule
\sphinxtableatstartofbodyhook
\sphinxAtStartPar
\sphinxcode{\sphinxupquote{html}}
&
\sphinxAtStartPar
tekstas pateiktas \sphinxhref{https://en.wikipedia.org/wiki/HTML}{HTML} formatu.
\\
\sphinxhline
\sphinxAtStartPar
\sphinxcode{\sphinxupquote{md}}
&
\sphinxAtStartPar
tekstas pateiktas \sphinxhref{https://spec.commonmark.org/}{Markdown} formatu.
\\
\sphinxhline
\sphinxAtStartPar
\sphinxcode{\sphinxupquote{rst}}
&
\sphinxAtStartPar
tekstas pateitkas \sphinxhref{https://docutils.sourceforge.io/rst.html}{reStructuredText} formatu.
\\
\sphinxhline
\sphinxAtStartPar
\sphinxcode{\sphinxupquote{tei}}
&
\sphinxAtStartPar
tekstas pateiktas \sphinxhref{https://en.wikipedia.org/wiki/Text\_Encoding\_Initiative}{TEI} formatu.
\\
\sphinxbottomrule
\end{tabulary}
\sphinxtableafterendhook\par
\sphinxattableend\end{savenotes}

\begin{sphinxadmonition}{note}{Pavyzdys}


\begin{savenotes}\sphinxattablestart
\sphinxthistablewithglobalstyle
\centering
\begin{tabulary}{\linewidth}[t]{TTTTTTT}
\sphinxtoprule
\sphinxstyletheadfamily 
\sphinxAtStartPar
d
&\sphinxstyletheadfamily 
\sphinxAtStartPar
r
&\sphinxstyletheadfamily 
\sphinxAtStartPar
b
&\sphinxstyletheadfamily 
\sphinxAtStartPar
m
&\sphinxstyletheadfamily 
\sphinxAtStartPar
property
&\sphinxstyletheadfamily 
\sphinxAtStartPar
type
&\sphinxstyletheadfamily 
\sphinxAtStartPar
ref
\\
\sphinxmidrule
\sphinxtableatstartofbodyhook\sphinxstartmulticolumn{5}%
\begin{varwidth}[t]{\sphinxcolwidth{5}{7}}
\sphinxAtStartPar
example
\par
\vskip-\baselineskip\vbox{\hbox{\strut}}\end{varwidth}%
\sphinxstopmulticolumn
&&\\
\sphinxhline
\sphinxAtStartPar

&&&\sphinxstartmulticolumn{2}%
\begin{varwidth}[t]{\sphinxcolwidth{2}{7}}
\sphinxAtStartPar
Country
\par
\vskip-\baselineskip\vbox{\hbox{\strut}}\end{varwidth}%
\sphinxstopmulticolumn
&&\\
\sphinxhline
\sphinxAtStartPar

&&&&
\sphinxAtStartPar
name@lt
&
\sphinxAtStartPar
string
&\\
\sphinxhline
\sphinxAtStartPar

&&&&
\sphinxAtStartPar
description@lt
&
\sphinxAtStartPar
string
&
\sphinxAtStartPar
html
\\
\sphinxhline
\sphinxAtStartPar

&&&&
\sphinxAtStartPar
description@en
&
\sphinxAtStartPar
string
&
\sphinxAtStartPar
html
\\
\sphinxbottomrule
\end{tabulary}
\sphinxtableafterendhook\par
\sphinxattableend\end{savenotes}

\sphinxAtStartPar
Šiame pavyzdyje \sphinxcode{\sphinxupquote{@lt}} nurodo, kad šalies pavadinimai ir aprašymai
pateikti Lietuvių kalba, tačiau laukas \sphinxcode{\sphinxupquote{description}} papildomai turi
vertimą į anglų kalbą. Papildomai, šalies aprašymo teksto formatas yra
\sphinxhref{https://en.wikipedia.org/wiki/HTML}{HTML} tipo.
\end{sphinxadmonition}

\begin{sphinxtopic}
\sphinxstyletopictitle{Brandos lygis}
\begin{description}
\sphinxlineitem{{\hyperref[\detokenize{branda:l202}]{\sphinxcrossref{\DUrole{std}{\DUrole{std-ref}{L202: Nestandartinis formatas}}}}}}
\sphinxAtStartPar
Tekstas yra užrašytas natūralia žmonių kalba, tačiau neturi kalbos
žymės.

\sphinxlineitem{{\hyperref[\detokenize{branda:l202}]{\sphinxcrossref{\DUrole{std}{\DUrole{std-ref}{L202: Nestandartinis formatas}}}}}}
\sphinxAtStartPar
Duomenys pateikti nestandartine koduote. Standartinė koduotė yra
UTF\sphinxhyphen{}8.

\sphinxlineitem{{\hyperref[\detokenize{branda:l202}]{\sphinxcrossref{\DUrole{std}{\DUrole{std-ref}{L202: Nestandartinis formatas}}}}}}
\sphinxAtStartPar
Duomenys pateikti UTF\sphinxhyphen{}8 koduote, tačiau pats tekstas naudoja tam
tikrą formatavimo sintaksę, kuri nėra nurodyta {\hyperref[\detokenize{dimensijos:property.ref}]{\sphinxcrossref{\sphinxcode{\sphinxupquote{property.ref}}}}}
stulpelyje.

\end{description}
\end{sphinxtopic}

\end{fulllineitems}

\index{text (modulje type)@\spxentry{text}\spxextra{modulje type}}

\begin{fulllineitems}
\phantomsection\label{\detokenize{tipai:type.text}}
\pysigstartsignatures
\pysigline
{\sphinxcode{\sphinxupquote{type.}}\sphinxbfcode{\sphinxupquote{text}}}
\pysigstopsignatures
\sphinxAtStartPar
Natūraliaja žmonių kalba užrašytas tekstas, susidedantis iš vieno ar kelių
\sphinxcode{\sphinxupquote{string}} tipo duomenų laukų, pateikiant atskirą duomenų lauką, kiekvienai
kalbai.

\sphinxAtStartPar
Dažniausiai tiesiogiai \sphinxcode{\sphinxupquote{text}} tipas nenaudojamas, kadangi jei \sphinxcode{\sphinxupquote{string}}
tipas turi kalbos žymę, tai duomenų laukas yra interpretuojamas kaip \sphinxcode{\sphinxupquote{text}}
tipo.

\sphinxAtStartPar
Pavyzdžiui jei {\hyperref[\detokenize{formatas:property}]{\sphinxcrossref{\sphinxcode{\sphinxupquote{property}}}}} pavadinimas yra \sphinxcode{\sphinxupquote{title@lt}}, tada \sphinxcode{\sphinxupquote{title}}
duomenų laukas yra \sphinxcode{\sphinxupquote{text}} tipo.

\sphinxAtStartPar
Atskirai \sphinxcode{\sphinxupquote{text}} tipo duomenų laukas gali būti nurodomas tais atvejais, kai
reikia pateikti aprašymą ir {\hyperref[\detokenize{savokos:term-URI}]{\sphinxtermref{\DUrole{xref}{\DUrole{std}{\DUrole{std-term}{URI}}}}}} pačiam \sphinxcode{\sphinxupquote{text}} tipo laukui, o ne
vienam iš vertimų.

\begin{sphinxadmonition}{note}{Pavyzdys}


\begin{savenotes}\sphinxattablestart
\sphinxthistablewithglobalstyle
\centering
\begin{tabulary}{\linewidth}[t]{TTTTTTT}
\sphinxtoprule
\sphinxstyletheadfamily 
\sphinxAtStartPar
dataset
&\sphinxstyletheadfamily 
\sphinxAtStartPar
model
&\sphinxstyletheadfamily 
\sphinxAtStartPar
property
&\sphinxstyletheadfamily 
\sphinxAtStartPar
type
&\sphinxstyletheadfamily 
\sphinxAtStartPar
ref
&\sphinxstyletheadfamily 
\sphinxAtStartPar
uri
&\sphinxstyletheadfamily 
\sphinxAtStartPar
title
\\
\sphinxmidrule
\sphinxtableatstartofbodyhook\sphinxstartmulticolumn{3}%
\begin{varwidth}[t]{\sphinxcolwidth{3}{7}}
\sphinxAtStartPar
example
\par
\vskip-\baselineskip\vbox{\hbox{\strut}}\end{varwidth}%
\sphinxstopmulticolumn
&&&&\\
\sphinxhline
\sphinxAtStartPar

&&&
\sphinxAtStartPar
prefix
&
\sphinxAtStartPar
rdfs
&
\sphinxAtStartPar
http://www.w3.org/
2000/01/rdf\sphinxhyphen{}schema\#
&\\
\sphinxhline
\sphinxAtStartPar

&\sphinxstartmulticolumn{2}%
\begin{varwidth}[t]{\sphinxcolwidth{2}{7}}
\sphinxAtStartPar
Country
\par
\vskip-\baselineskip\vbox{\hbox{\strut}}\end{varwidth}%
\sphinxstopmulticolumn
&&
\sphinxAtStartPar
name@lt
&&\\
\sphinxhline
\sphinxAtStartPar

&&
\sphinxAtStartPar
name
&
\sphinxAtStartPar
text
&&
\sphinxAtStartPar
rdfs:label
&
\sphinxAtStartPar
Pavadinimas
\\
\sphinxhline
\sphinxAtStartPar

&&
\sphinxAtStartPar
name@lt
&
\sphinxAtStartPar
string
&&&\\
\sphinxbottomrule
\end{tabulary}
\sphinxtableafterendhook\par
\sphinxattableend\end{savenotes}
\end{sphinxadmonition}

\end{fulllineitems}

\index{datetime (modulje type)@\spxentry{datetime}\spxextra{modulje type}}

\begin{fulllineitems}
\phantomsection\label{\detokenize{tipai:type.datetime}}
\pysigstartsignatures
\pysigline
{\sphinxcode{\sphinxupquote{type.}}\sphinxbfcode{\sphinxupquote{datetime}}}
\pysigstopsignatures
\sphinxAtStartPar
Data ir laikas atitinkantis \sphinxhref{https://en.wikipedia.org/wiki/ISO\_8601}{ISO 8601}.

\sphinxAtStartPar
Mažiausia galima reikšmė: \sphinxcode{\sphinxupquote{0001\sphinxhyphen{}01\sphinxhyphen{}01T00:00:00}}.

\sphinxAtStartPar
Didžiausia galima reikšmė: \sphinxcode{\sphinxupquote{9999\sphinxhyphen{}12\sphinxhyphen{}31T23:59:59.999999}}.

\sphinxAtStartPar
Pagal \sphinxhref{https://en.wikipedia.org/wiki/ISO\_8601}{ISO 8601} standartą, data gali būti pateikta tokia forma:

\begin{sphinxVerbatim}[commandchars=\\\{\}]
\PYG{n}{YYYY}\PYG{o}{\PYGZhy{}}\PYG{n}{MM}\PYG{o}{\PYGZhy{}}\PYG{n}{DD}\PYG{p}{[}\PYG{o}{*}\PYG{n}{HH}\PYG{p}{[}\PYG{p}{:}\PYG{n}{MM}\PYG{p}{[}\PYG{p}{:}\PYG{n}{SS}\PYG{p}{[}\PYG{o}{.}\PYG{n}{fff}\PYG{p}{[}\PYG{n}{fff}\PYG{p}{]}\PYG{p}{]}\PYG{p}{]}\PYG{p}{]}\PYG{p}{[}\PYG{o}{+}\PYG{n}{HH}\PYG{p}{:}\PYG{n}{MM}\PYG{p}{[}\PYG{p}{:}\PYG{n}{SS}\PYG{p}{[}\PYG{o}{.}\PYG{n}{ffffff}\PYG{p}{]}\PYG{p}{]}\PYG{p}{]}\PYG{p}{]}
\end{sphinxVerbatim}

\sphinxAtStartPar
Simbolis \sphinxcode{\sphinxupquote{*}} reiškia, kad galima pateikti bet kokį vieną simbolį,
dažniausiai naudojamas tarpo simbolis, arba raidė \sphinxcode{\sphinxupquote{T}}.

\sphinxAtStartPar
{\hyperref[\detokenize{dimensijos:property.ref}]{\sphinxcrossref{\sphinxcode{\sphinxupquote{property.ref}}}}} stulpelyje, nurodomas \sphinxhref{https://www.w3.org/TR/vocab-dcat-2/\#Property:dataset\_temporal\_resolution}{datos ir laiko tikslumas}
sekundėmis. Tikslumą galima nurodyti laiko vienetais, pavyzdžiui \sphinxcode{\sphinxupquote{Y}},
\sphinxcode{\sphinxupquote{D}}, \sphinxcode{\sphinxupquote{S}}, arba \sphinxcode{\sphinxupquote{5Y}}, \sphinxcode{\sphinxupquote{10D}}, \sphinxcode{\sphinxupquote{30S}}. Visi duomenys turi atitikti vienodą
tikslumą, tikslumas negali varijuoti. Galimi vienetų variantai:


\begin{savenotes}\sphinxattablestart
\sphinxthistablewithglobalstyle
\centering
\begin{tabulary}{\linewidth}[t]{TT}
\sphinxtoprule
\sphinxstyletheadfamily 
\sphinxAtStartPar
Reikšmė
&\sphinxstyletheadfamily 
\sphinxAtStartPar
Prasmė
\\
\sphinxmidrule
\sphinxtableatstartofbodyhook
\sphinxAtStartPar
\sphinxcode{\sphinxupquote{Y}}
&
\sphinxAtStartPar
Metai
\\
\sphinxhline
\sphinxAtStartPar
\sphinxcode{\sphinxupquote{M}}
&
\sphinxAtStartPar
Mėnesiai
\\
\sphinxhline
\sphinxAtStartPar
\sphinxcode{\sphinxupquote{Q}}
&
\sphinxAtStartPar
Metų ketvirčiai
\\
\sphinxhline
\sphinxAtStartPar
\sphinxcode{\sphinxupquote{W}}
&
\sphinxAtStartPar
Savaitės
\\
\sphinxhline
\sphinxAtStartPar
\sphinxcode{\sphinxupquote{D}}
&
\sphinxAtStartPar
Dienos
\\
\sphinxhline
\sphinxAtStartPar
\sphinxcode{\sphinxupquote{H}}
&
\sphinxAtStartPar
Valandos
\\
\sphinxhline
\sphinxAtStartPar
\sphinxcode{\sphinxupquote{T}}
&
\sphinxAtStartPar
Minutės
\\
\sphinxhline
\sphinxAtStartPar
\sphinxcode{\sphinxupquote{S}}
&
\sphinxAtStartPar
Sekundės
\\
\sphinxhline
\sphinxAtStartPar
\sphinxcode{\sphinxupquote{L}}
&
\sphinxAtStartPar
Milisekundės
\\
\sphinxhline
\sphinxAtStartPar
\sphinxcode{\sphinxupquote{U}}
&
\sphinxAtStartPar
Mikrosekundės
\\
\sphinxhline
\sphinxAtStartPar
\sphinxcode{\sphinxupquote{N}}
&
\sphinxAtStartPar
Nanosekundės
\\
\sphinxbottomrule
\end{tabulary}
\sphinxtableafterendhook\par
\sphinxattableend\end{savenotes}

\begin{sphinxtopic}
\sphinxstyletopictitle{Brandos lygis}
\begin{description}
\sphinxlineitem{{\hyperref[\detokenize{branda:l101}]{\sphinxcrossref{\DUrole{std}{\DUrole{std-ref}{L101: Neaiški struktūra}}}}}}
\sphinxAtStartPar
Data ir laikas pateikti laisvu tekstu, pavyzdžiui \sphinxcode{\sphinxupquote{2020 paskutinę
pirmo mėnesio dieną}}.

\sphinxlineitem{{\hyperref[\detokenize{branda:l102}]{\sphinxcrossref{\DUrole{std}{\DUrole{std-ref}{L102: Nėra vientisumo}}}}}}
\sphinxAtStartPar
Data ir laikas pateikti naudojant skirtingus formatus, pavyzdžiui
\sphinxcode{\sphinxupquote{2020\sphinxhyphen{}01\sphinxhyphen{}31}}, \sphinxcode{\sphinxupquote{01/31/2020}}, \sphinxcode{\sphinxupquote{31.1.20}}.

\sphinxlineitem{{\hyperref[\detokenize{branda:l202}]{\sphinxcrossref{\DUrole{std}{\DUrole{std-ref}{L202: Nestandartinis formatas}}}}}}
\sphinxAtStartPar
Duomenys pateikti nestandartiniu formatu, tačiau visi duomenys
pateikti vienodu formatu. Pavyzdžiui visi duomenys pateikti
\sphinxcode{\sphinxupquote{01/31/2020}} formatu, tačiau datos turi būti pateiktos \sphinxhref{https://en.wikipedia.org/wiki/ISO\_8601}{ISO 8601}
formatu.

\sphinxlineitem{{\hyperref[\detokenize{branda:l210}]{\sphinxcrossref{\DUrole{std}{\DUrole{std-ref}{L210: Išskaidyta atskirais komponentais}}}}}}
\sphinxAtStartPar
Duomenys pateikti atskiruose laukuose, pavyzdžiui metai pateikti
viename \sphinxcode{\sphinxupquote{integer}} tipo lauke, o ketvirtis, kitame \sphinxcode{\sphinxupquote{integer}} tipo
lauke. Norint didesnio brandos lygio, duomenys turi būti viename
\sphinxcode{\sphinxupquote{date}} tipo lauke su {\hyperref[\detokenize{dimensijos:property.ref}]{\sphinxcrossref{\sphinxcode{\sphinxupquote{property.ref}}}}} = \sphinxcode{\sphinxupquote{Q}}.

\sphinxlineitem{{\hyperref[\detokenize{branda:l303}]{\sphinxcrossref{\DUrole{std}{\DUrole{std-ref}{L303: Nenurodytas duomenų tikslumas}}}}}}
\sphinxAtStartPar
Nenurodytas {\hyperref[\detokenize{dimensijos:property.ref}]{\sphinxcrossref{\sphinxcode{\sphinxupquote{property.ref}}}}}, kuriame turėtu būti pateiktas
duomenų tikslumas.

\end{description}
\end{sphinxtopic}

\end{fulllineitems}

\index{date (modulje type)@\spxentry{date}\spxextra{modulje type}}

\begin{fulllineitems}
\phantomsection\label{\detokenize{tipai:type.date}}
\pysigstartsignatures
\pysigline
{\sphinxcode{\sphinxupquote{type.}}\sphinxbfcode{\sphinxupquote{date}}}
\pysigstopsignatures
\sphinxAtStartPar
Tas pats kas \sphinxcode{\sphinxupquote{datetime}} tik dienos tikslumu. Šio tipo reikšmės taip pat
turi atitikti \sphinxhref{https://en.wikipedia.org/wiki/ISO\_8601}{ISO 8601}:

\begin{sphinxVerbatim}[commandchars=\\\{\}]
\PYG{n}{YYYY}\PYG{o}{\PYGZhy{}}\PYG{n}{MM}\PYG{o}{\PYGZhy{}}\PYG{n}{DD}
\end{sphinxVerbatim}

\sphinxAtStartPar
{\hyperref[\detokenize{dimensijos:property.ref}]{\sphinxcrossref{\sphinxcode{\sphinxupquote{property.ref}}}}} stulpeyje nurodomas datos tikslumas:


\begin{savenotes}\sphinxattablestart
\sphinxthistablewithglobalstyle
\centering
\begin{tabulary}{\linewidth}[t]{TT}
\sphinxtoprule
\sphinxstyletheadfamily 
\sphinxAtStartPar
Reikšmė
&\sphinxstyletheadfamily 
\sphinxAtStartPar
Prasmė
\\
\sphinxmidrule
\sphinxtableatstartofbodyhook
\sphinxAtStartPar
Y
&
\sphinxAtStartPar
Metai
\\
\sphinxhline
\sphinxAtStartPar
M
&
\sphinxAtStartPar
Mėnesiai
\\
\sphinxhline
\sphinxAtStartPar
Q
&
\sphinxAtStartPar
Metų ketvirčiai
\\
\sphinxhline
\sphinxAtStartPar
W
&
\sphinxAtStartPar
Savaitės
\\
\sphinxhline
\sphinxAtStartPar
D
&
\sphinxAtStartPar
Dienos
\\
\sphinxbottomrule
\end{tabulary}
\sphinxtableafterendhook\par
\sphinxattableend\end{savenotes}

\sphinxAtStartPar
Jei duomenys pateikti žemesniu nei dienos tikslumu, tada datos rekšmės turi
būti nurodytos \sphinxcode{\sphinxupquote{YYYY\sphinxhyphen{}MM\sphinxhyphen{}DD}} formatu, pakeičiant \sphinxcode{\sphinxupquote{MM}} ir arba \sphinxcode{\sphinxupquote{DD\textasciitilde{} į `01}}.

\begin{sphinxadmonition}{note}{Pavyzdys}

\sphinxAtStartPar
Turint tokį struktūros aprašą:


\begin{savenotes}\sphinxattablestart
\sphinxthistablewithglobalstyle
\centering
\begin{tabulary}{\linewidth}[t]{TTTT}
\sphinxtoprule
\sphinxstyletheadfamily 
\sphinxAtStartPar
model
&\sphinxstyletheadfamily 
\sphinxAtStartPar
property
&\sphinxstyletheadfamily 
\sphinxAtStartPar
type
&\sphinxstyletheadfamily 
\sphinxAtStartPar
ref
\\
\sphinxmidrule
\sphinxtableatstartofbodyhook\sphinxstartmulticolumn{2}%
\begin{varwidth}[t]{\sphinxcolwidth{2}{4}}
\sphinxAtStartPar
Country
\par
\vskip-\baselineskip\vbox{\hbox{\strut}}\end{varwidth}%
\sphinxstopmulticolumn
&&
\sphinxAtStartPar
id
\\
\sphinxhline
\sphinxAtStartPar

&
\sphinxAtStartPar
id
&
\sphinxAtStartPar
integer
&\\
\sphinxhline
\sphinxAtStartPar

&
\sphinxAtStartPar
independence
&
\sphinxAtStartPar
date
&
\sphinxAtStartPar
Y
\\
\sphinxbottomrule
\end{tabulary}
\sphinxtableafterendhook\par
\sphinxattableend\end{savenotes}

\sphinxAtStartPar
Nors \sphinxcode{\sphinxupquote{independence}} duomenų lauko tiksluas yra metų, tačiau pateikiant
duomenis būtina nurodyti mėnesį ir dieną taip pat:

\begin{sphinxVerbatim}[commandchars=\\\{\}]
\PYG{p}{\PYGZob{}}
\PYG{+w}{    }\PYG{n+nt}{\PYGZdq{}id\PYGZdq{}}\PYG{p}{:}\PYG{+w}{ }\PYG{l+m+mi}{1}\PYG{p}{,}
\PYG{+w}{    }\PYG{n+nt}{\PYGZdq{}independence\PYGZdq{}}\PYG{p}{:}\PYG{+w}{ }\PYG{l+s+s2}{\PYGZdq{}1990\PYGZhy{}01\PYGZhy{}01\PYGZdq{}}\PYG{p}{,}
\PYG{p}{\PYGZcb{}}
\end{sphinxVerbatim}

\sphinxAtStartPar
Šiuo atveju, kadangi datos tiksluas yra metai, \sphinxcode{\sphinxupquote{\sphinxhyphen{}01\sphinxhyphen{}01}} dalis datoje
neturi jokios reikšmės ir yra pateikiama tik tam, kad reikšmė atitiktu
\sphinxhref{https://en.wikipedia.org/wiki/ISO\_8601}{ISO 8601} reikalavimus.
\end{sphinxadmonition}

\end{fulllineitems}

\index{time (modulje type)@\spxentry{time}\spxextra{modulje type}}

\begin{fulllineitems}
\phantomsection\label{\detokenize{tipai:type.time}}
\pysigstartsignatures
\pysigline
{\sphinxcode{\sphinxupquote{type.}}\sphinxbfcode{\sphinxupquote{time}}}
\pysigstopsignatures
\sphinxAtStartPar
Dienos laikas, be konkrečios datos. Šio tipo reikšmės, kaip ir kiti
su laiku susiję tipai turi atitikti \sphinxhref{https://en.wikipedia.org/wiki/ISO\_8601}{ISO 8601}:

\begin{sphinxVerbatim}[commandchars=\\\{\}]
\PYG{n}{HH}\PYG{p}{[}\PYG{p}{:}\PYG{n}{MM}\PYG{p}{[}\PYG{p}{:}\PYG{n}{SS}\PYG{p}{[}\PYG{o}{.}\PYG{n}{fff}\PYG{p}{[}\PYG{n}{fff}\PYG{p}{]}\PYG{p}{]}\PYG{p}{]}\PYG{p}{]}\PYG{p}{[}\PYG{o}{+}\PYG{n}{HH}\PYG{p}{:}\PYG{n}{MM}\PYG{p}{[}\PYG{p}{:}\PYG{n}{SS}\PYG{p}{[}\PYG{o}{.}\PYG{n}{ffffff}\PYG{p}{]}\PYG{p}{]}\PYG{p}{]}
\end{sphinxVerbatim}

\sphinxAtStartPar
Jei norima nurodyti žemesnio nei sekundžių tikslumo laiką, tada
vietoj minučių ir/ar sekundžių galima naudoti \sphinxcode{\sphinxupquote{00}} ir
{\hyperref[\detokenize{dimensijos:property.ref}]{\sphinxcrossref{\sphinxcode{\sphinxupquote{property.ref}}}}} stulpelyje nurodyti tikslumą:


\begin{savenotes}\sphinxattablestart
\sphinxthistablewithglobalstyle
\centering
\begin{tabulary}{\linewidth}[t]{TT}
\sphinxtoprule
\sphinxstyletheadfamily 
\sphinxAtStartPar
Reikšmė
&\sphinxstyletheadfamily 
\sphinxAtStartPar
Prasmė
\\
\sphinxmidrule
\sphinxtableatstartofbodyhook
\sphinxAtStartPar
\sphinxcode{\sphinxupquote{H}}
&
\sphinxAtStartPar
Valandos
\\
\sphinxhline
\sphinxAtStartPar
\sphinxcode{\sphinxupquote{T}}
&
\sphinxAtStartPar
Minutės
\\
\sphinxhline
\sphinxAtStartPar
\sphinxcode{\sphinxupquote{S}}
&
\sphinxAtStartPar
Sekundės
\\
\sphinxhline
\sphinxAtStartPar
\sphinxcode{\sphinxupquote{L}}
&
\sphinxAtStartPar
Milisekundės
\\
\sphinxhline
\sphinxAtStartPar
\sphinxcode{\sphinxupquote{U}}
&
\sphinxAtStartPar
Mikrosekundės
\\
\sphinxhline
\sphinxAtStartPar
\sphinxcode{\sphinxupquote{N}}
&
\sphinxAtStartPar
Nanosekundės
\\
\sphinxbottomrule
\end{tabulary}
\sphinxtableafterendhook\par
\sphinxattableend\end{savenotes}

\end{fulllineitems}



\begin{fulllineitems}

\pysigstartsignatures
\pysigline
{\sphinxcode{\sphinxupquote{type.}}\sphinxbfcode{\sphinxupquote{temporal}}}
\pysigstopsignatures
\sphinxAtStartPar
\DUrole{versionmodified}{\DUrole{deprecated}{Nebepalaikoma nuo 0.2 versijos.}}

\sphinxAtStartPar
Apibrėžtis laike.

\sphinxAtStartPar
Šis tipas atitinka \sphinxcode{\sphinxupquote{datetime}}, tačiau nurodo, kad visas model yra
apibrėžtas laike, būtent pagal šią savybę. Tik viena model savybė gali
turėti \sphinxcode{\sphinxupquote{temporal}} tipą. Pagal šios savybės reikšmes apskaičiuojamas ir
įvertinamas \sphinxhref{https://www.w3.org/TR/vocab-dcat-2/\#Property:dataset\_temporal}{dct:temporal}.

\end{fulllineitems}

\index{geometry (modulje type)@\spxentry{geometry}\spxextra{modulje type}}

\begin{fulllineitems}
\phantomsection\label{\detokenize{tipai:type.geometry}}
\pysigstartsignatures
\pysigline
{\sphinxcode{\sphinxupquote{type.}}\sphinxbfcode{\sphinxupquote{geometry}}}
\pysigstopsignatures
\sphinxAtStartPar
Erdviniai duomenys. Duomenys pateikiami \sphinxhref{https://en.wikipedia.org/wiki/Well-known\_text\_representation\_of\_geometry}{WKT} formatu, naudojant \sphinxhref{https://epsg.org/home.html}{EPSG}
duomenų bazės parametrus, skirtingoms projekcijoms išreikšti.

\sphinxAtStartPar
{\hyperref[\detokenize{dimensijos:property.ref}]{\sphinxcrossref{\sphinxcode{\sphinxupquote{property.ref}}}}} stulpelyje nurodomas tikslumas metrais. Tikslumą
galima pateikti naudojanti SI vienetus, pavyzdžiui \sphinxcode{\sphinxupquote{m}}, \sphinxcode{\sphinxupquote{km}} arba \sphinxcode{\sphinxupquote{10m}},
\sphinxcode{\sphinxupquote{100km}}.

\sphinxAtStartPar
\sphinxcode{\sphinxupquote{geometry}} tipas gali turėti du argumentus \sphinxcode{\sphinxupquote{geometry(form, crs)}}:
\begin{itemize}
\item {} 
\sphinxAtStartPar
\sphinxcode{\sphinxupquote{form}} \sphinxhyphen{} geometrijos forma

\item {} 
\sphinxAtStartPar
\sphinxcode{\sphinxupquote{crs}} \sphinxhyphen{} koordinačių sistema

\end{itemize}

\sphinxAtStartPar
Pats tipas gali būti pateiktas vienu iš šių variantų:
\begin{itemize}
\item {} 
\sphinxAtStartPar
\sphinxcode{\sphinxupquote{geometry(form, crs)}} \sphinxhyphen{} nurodant formą ir koordinačių sistemą

\item {} 
\sphinxAtStartPar
\sphinxcode{\sphinxupquote{geometry(crs)}} \sphinxhyphen{} nurodant tik koordinačių sistemą

\item {} 
\sphinxAtStartPar
\sphinxcode{\sphinxupquote{geometry(form)}} \sphinxhyphen{} nurodant tik formą

\item {} 
\sphinxAtStartPar
\sphinxcode{\sphinxupquote{geometry}} \sphinxhyphen{} be argumentų.

\end{itemize}

\sphinxAtStartPar
\sphinxstylestrong{Geometrijos forma} (\sphinxcode{\sphinxupquote{form}})

\sphinxAtStartPar
Galimi tokie geometrijos tipai:
\begin{itemize}
\item {} 
\sphinxAtStartPar
\sphinxcode{\sphinxupquote{point}} \sphinxhyphen{} taškas.

\item {} 
\sphinxAtStartPar
\sphinxcode{\sphinxupquote{linestring}} \sphinxhyphen{} linija.

\item {} 
\sphinxAtStartPar
\sphinxcode{\sphinxupquote{polygon}} \sphinxhyphen{} daugiakampis (pradžios ir pabaigos taškai \sphinxstylestrong{turi} sutapti).

\item {} 
\sphinxAtStartPar
\sphinxcode{\sphinxupquote{multipoint}} \sphinxhyphen{} keli taškai.

\item {} 
\sphinxAtStartPar
\sphinxcode{\sphinxupquote{multilinestring}} \sphinxhyphen{} kelios linijos.

\item {} 
\sphinxAtStartPar
\sphinxcode{\sphinxupquote{multipolygon}} \sphinxhyphen{} keli daugiakampiai (kiekvieno daugiakampio pradžios ir pabaigos taškai \sphinxstylestrong{turi} sutapti).

\end{itemize}

\sphinxAtStartPar
Kiekviena iš formų gali turėti tokias galūnes nurodančias papildomą dimensiją:
\begin{itemize}
\item {} 
\sphinxAtStartPar
\sphinxcode{\sphinxupquote{z}} \sphinxhyphen{} aukštis.

\item {} 
\sphinxAtStartPar
\sphinxcode{\sphinxupquote{m}} \sphinxhyphen{} pasirinktas matmuo (pavyzdžiui laikas, atstumas, storis ir pan.)

\item {} 
\sphinxAtStartPar
\sphinxcode{\sphinxupquote{zm}} \sphinxhyphen{} aukštis ir pasirinktas matmuo.

\end{itemize}

\sphinxAtStartPar
Jei geometrijos forma nenurodyta, tada duomenys gali būti bet kokios
geometrinės formos. Jei forma nurodyta, tada visi duomenys turi būti tik
tokios formos, kokia nurodyta.

\sphinxAtStartPar
\sphinxstylestrong{Koordinačių sistema} (\sphinxcode{\sphinxupquote{crs}})

\sphinxAtStartPar
Antrasis \sphinxcode{\sphinxupquote{geometry}} argumentas nurodomas pateikiant \sphinxhref{https://en.wikipedia.org/wiki/Spatial\_reference\_system\#Identifier}{SRID} numerį, kuris yra
konkrečios koordinačių sistemos identifikacinis numeris \sphinxhref{https://epsg.org/home.html}{EPSG} duomenų
bazėje. Jei koordinačių sistemos numeris nenurodytas, tuomet daroma
prielaida, kad erdviniai duomenys atitinka \sphinxcode{\sphinxupquote{4326}} ({\color{red}\bfseries{}WGS84\_}) koordinačių
sistemą.

\sphinxAtStartPar
Svarbu, kad pateikiant duomenis, koordinačių ašių eiliškumas atitiktų tokį
eiliškumą, kuris nurodytas \sphinxhref{https://epsg.org/home.html}{EPSG} parametrų duomenų bazėje, konkrečiai
koordinačių sistemai, kuria pateikiami duomenys.

\sphinxAtStartPar
Svarbu, kad pateikiant duomenis, koordinačių ašių eiliškumas atitiktų tokį
eiliškumą, kuris nurodytas \sphinxhref{https://epsg.org/home.html}{EPSG} parametrų duomenų bazėje, konkrečiai
koordinačių sistemai, kuria pateikiami duomenys.

\begin{sphinxadmonition}{note}{Pavyzdys}
\begin{enumerate}
\sphinxsetlistlabels{\alph}{enumi}{enumii}{}{)}%
\item {} 
\sphinxAtStartPar
pateikiant duomenis  LKS 94 (SRID:3346) ir WGS84 (SRID:4326)
koordinačių sistemose į ADP Saugyklą turi būti laikomasi eiliškumo:
pirmiausia pateikiama X (į šiaurę/platumos/latitude), o po to Y (į
rytus/ilgumos/longitute) reikšmės;

\item {} 
\sphinxAtStartPar
tačiau pateikiant duomenis  WGS84/Pseudo\sphinxhyphen{}Merkator (SRID:3857)
koordinačių sistemoje jau atvirkščiai – pirmiausia pateikiama  rytų
ilgumos, o po to šiaurės platumos reikšmės.

\end{enumerate}
\end{sphinxadmonition}

\sphinxAtStartPar
Pilną \sphinxhref{https://en.wikipedia.org/wiki/Spatial\_reference\_system\#Identifier}{SRID} kodų sąrašą galite rasti \sphinxhref{https://epsg.io/}{epsg.io} svetainėje. Keletas
dažniau naudojamų \sphinxhref{https://en.wikipedia.org/wiki/Spatial\_reference\_system\#Identifier}{SRID} kodų:


\begin{savenotes}\sphinxattablestart
\sphinxthistablewithglobalstyle
\centering
\begin{tabulary}{\linewidth}[t]{TTTTT}
\sphinxtoprule
\sphinxstyletheadfamily 
\sphinxAtStartPar
SRID
&\sphinxstyletheadfamily 
\sphinxAtStartPar
CRS
&\sphinxstyletheadfamily 
\sphinxAtStartPar
Pavyzdys
&\sphinxstyletheadfamily 
\sphinxAtStartPar
ašys
&\sphinxstyletheadfamily 
\sphinxAtStartPar
orientacija
\\
\sphinxmidrule
\sphinxtableatstartofbodyhook
\sphinxAtStartPar
\sphinxhref{https://epsg.io/4326}{4326}
&
\sphinxAtStartPar
WGS84
&
\sphinxAtStartPar
\sphinxhref{https://get.data.gov.lt/\_srid/4326/54.6981/25.2738}{POINT(54.6981 25.2738)}
&
\sphinxAtStartPar
lat, lon
&
\sphinxAtStartPar
north, east
\\
\sphinxhline
\sphinxAtStartPar
\sphinxhref{https://epsg.io/3346}{3346}
&
\sphinxAtStartPar
LKS94
&
\sphinxAtStartPar
\sphinxhref{https://get.data.gov.lt/\_srid/3346/6063156/582111}{POINT(6063156 582111)}
&
\sphinxAtStartPar
north, east (x, y)
&
\sphinxAtStartPar
north, east
\\
\sphinxhline
\sphinxAtStartPar
\sphinxhref{https://epsg.io/3857}{3857}
&
\sphinxAtStartPar
WGS84 / Pseudo\sphinxhyphen{}Merctor
&
\sphinxAtStartPar
\sphinxhref{https://get.data.gov.lt/\_srid/3857/2813472/7303494}{POINT(2813472 7303494)}
&
\sphinxAtStartPar
lon, lat
&
\sphinxAtStartPar
east, north
\\
\sphinxhline
\sphinxAtStartPar
\sphinxhref{https://epsg.io/4258}{4258}
&
\sphinxAtStartPar
ETRS89
&
\sphinxAtStartPar
\sphinxhref{https://get.data.gov.lt/\_srid/4258/54.6981/25.2738}{POINT(54.6981 25.2738)}
&
\sphinxAtStartPar
lat, lon
&
\sphinxAtStartPar
north, east
\\
\sphinxbottomrule
\end{tabulary}
\sphinxtableafterendhook\par
\sphinxattableend\end{savenotes}

\begin{sphinxadmonition}{note}{Pastaba:}
\sphinxAtStartPar
Atkreipkite dėmesį, kad LKS94 koordinačių sistemoje geometrinės ašys
neatitinka įprastinio ašių eiliškumo naudojamo GIS sistemose. LKS94 pirmas
skaičius yra apytiksliai 6,000,000 metrų nuo pusiaujo į šiaurę, o antrasis
skaičius apytiksliai 500,000 metrų į rytus, skaičiuojant nuo 24º rytų
meridiano, atėmus 500km. Teikiant duomenis, taškai turėtu atrodyti taip:
\sphinxcode{\sphinxupquote{6000000 500000}}, pirmas ilgesnis, antras trumpesnis.
\begin{quote}

\sphinxAtStartPar
Ašinio meridiano projekcija yra abscisių (x) ašis. Šios ašies
teigiamoji kryptis nukreipta į šiaurę. Ordinačių (y) ašies
teigiamoji kryptis nukreipta į rytus.

\sphinxAtStartPar
Išvyniojus cilindrą, gaunamos stačiakampės koordinatės su x
šiaurinės abscisės pradžia pusiaujuje ir y rytinės ordinatės
reikšme 24°C meridiane 500 000 metrų.

\begin{flushright}
---\sphinxurl{https://www.e-tar.lt/portal/lt/legalAct/TAR.6D575923F94A}
\end{flushright}
\end{quote}
\end{sphinxadmonition}

\sphinxAtStartPar
Prieš publikuojant duomenis, galite pasitikrinti, ar koordinačių ašys
pateikiamos teisinga tvarka, naudotami taško atvaizdavimo įrankį.

\sphinxAtStartPar
Pavyzdžiui, norint patikrinti Vilniaus Katedros varpinės bokšto taško
koordinates, LKS94 (EPSG:3346) sistemoje, galite naršyklės adreso juostoje
pateikti šį adresą:

\sphinxAtStartPar
\sphinxurl{https://get.data.gov.lt/\_srid/3346/6061789/582964}

\sphinxAtStartPar
Jei ašių eiliškumas teisingas, gausite tašką ten kur tikėjotės, jei ašys
sukeistos vietomis, tada taškas žemėlapyje gali būti visai kitoje vietoje,
nei tikėjotės.

\sphinxAtStartPar
Adreso formatas:

\begin{sphinxVerbatim}[commandchars=\\\{\}]
\PYG{o}{/}\PYG{n}{\PYGZus{}srid}\PYG{o}{/}\PYG{p}{\PYGZob{}}\PYG{n}{srid}\PYG{p}{\PYGZcb{}}\PYG{o}{/}\PYG{p}{\PYGZob{}}\PYG{n}{ašis1}\PYG{p}{\PYGZcb{}}\PYG{o}{/}\PYG{p}{\PYGZob{}}\PYG{n}{ašis2}\PYG{p}{\PYGZcb{}}
\end{sphinxVerbatim}
\begin{itemize}
\item {} 
\sphinxAtStartPar
\sphinxcode{\sphinxupquote{\{srid\}}} \sphinxhyphen{} \sphinxhref{https://epsg.org/home.html}{EPSG} duomenų bazėje esančios koordinačių sistemos \sphinxhref{https://en.wikipedia.org/wiki/Spatial\_reference\_system\#Identifier}{SRID} kodas

\item {} 
\sphinxAtStartPar
\sphinxcode{\sphinxupquote{\{ašis1\}}} \sphinxhyphen{} pirmosios ašies reikšmė (kryptis priklauso nuo \sphinxcode{\sphinxupquote{\{srid\}}})

\item {} 
\sphinxAtStartPar
\sphinxcode{\sphinxupquote{\{ašis2\}}} \sphinxhyphen{} antrosios ašies reikšmė (kryptis priklauso nuo \sphinxcode{\sphinxupquote{\{srid\}}})

\end{itemize}

\begin{sphinxadmonition}{note}{Pavyzdinės \sphinxstyleliteralintitle{\sphinxupquote{property.type}} reikšmės}
\begin{itemize}
\item {} 
\sphinxAtStartPar
\sphinxcode{\sphinxupquote{geometry}} \sphinxhyphen{} WGS84 projekcijos, bet kokio  tipo geometriniai
objektai.

\item {} 
\sphinxAtStartPar
\sphinxcode{\sphinxupquote{geometry(3346)}} \sphinxhyphen{} LKS94 projekcijos, bet kokio tipo geometriniai
objektai.

\item {} 
\sphinxAtStartPar
\sphinxcode{\sphinxupquote{geometry(point)}} \sphinxhyphen{} GWS84 projekcijos, bet \sphinxcode{\sphinxupquote{point}} tipo geometriniai
objektai.

\item {} 
\sphinxAtStartPar
\sphinxcode{\sphinxupquote{geometry(linestringm, 3345)}} \sphinxhyphen{} LKS94 projekcijos, \sphinxcode{\sphinxupquote{linestringm}} tipo
geometriniai objektai su pasirinktu matmeniu, kaip trečia dimensija.

\end{itemize}
\end{sphinxadmonition}

\begin{sphinxadmonition}{note}{Pavyzdys (duomenys)}

\sphinxAtStartPar
Vilniaus Katedros varpinės bokšto taškas, LKS94 (EPSG:3346) koordinačių
sistemoje:

\begin{sphinxVerbatim}[commandchars=\\\{\}]
\PYG{p}{\PYGZob{}}
\PYG{+w}{    }\PYG{n+nt}{\PYGZdq{}koordinates\PYGZdq{}}\PYG{p}{:}\PYG{+w}{ }\PYG{l+s+s2}{\PYGZdq{}POINT (6061789 582964)\PYGZdq{}}
\PYG{p}{\PYGZcb{}}
\end{sphinxVerbatim}
\end{sphinxadmonition}

\begin{sphinxtopic}
\sphinxstyletopictitle{Brandos lygis}
\begin{description}
\sphinxlineitem{{\hyperref[\detokenize{branda:l101}]{\sphinxcrossref{\DUrole{std}{\DUrole{std-ref}{L101: Neaiški struktūra}}}}}}
\sphinxAtStartPar
Pateiktas adresas, nenurodant adreso koordinačių.

\sphinxlineitem{{\hyperref[\detokenize{branda:l102}]{\sphinxcrossref{\DUrole{std}{\DUrole{std-ref}{L102: Nėra vientisumo}}}}}}
\sphinxAtStartPar
Nenurodytas koordinačių sistema ir duomenys pateikti skirtingomis
koordinatėmis.

\sphinxlineitem{{\hyperref[\detokenize{branda:l102}]{\sphinxcrossref{\DUrole{std}{\DUrole{std-ref}{L102: Nėra vientisumo}}}}}}
\sphinxAtStartPar
Sumaišytos ašys, pavyzdžiui vieni duomenys pateikiami x, y, kiti y,
x.

\sphinxlineitem{{\hyperref[\detokenize{branda:l102}]{\sphinxcrossref{\DUrole{std}{\DUrole{std-ref}{L102: Nėra vientisumo}}}}}}
\sphinxAtStartPar
Sumaišyti vienetai, pavyzdžiui vieni duomenys pateikti metrais,
kiti laipsniais.

\sphinxlineitem{{\hyperref[\detokenize{branda:l201}]{\sphinxcrossref{\DUrole{std}{\DUrole{std-ref}{L201: Nestandartiniai duomenų tipai}}}}}}
\sphinxAtStartPar
Nenurodyta koordinačių sistema, tačiau visi duomenys pateikti
naudojant vienodą koordinačių sistemą.

\sphinxlineitem{{\hyperref[\detokenize{branda:l210}]{\sphinxcrossref{\DUrole{std}{\DUrole{std-ref}{L210: Išskaidyta atskirais komponentais}}}}}}
\sphinxAtStartPar
Taško koordinatės pateiktos, kaip du atskiri duomenų laukai.

\sphinxlineitem{{\hyperref[\detokenize{branda:l303}]{\sphinxcrossref{\DUrole{std}{\DUrole{std-ref}{L303: Nenurodytas duomenų tikslumas}}}}}}
\sphinxAtStartPar
Nenurodytas {\hyperref[\detokenize{dimensijos:property.ref}]{\sphinxcrossref{\sphinxcode{\sphinxupquote{property.ref}}}}}, kuriame turėtu būti pateiktas
duomenų tikslumas metrais.

\end{description}
\end{sphinxtopic}

\end{fulllineitems}



\begin{fulllineitems}

\pysigstartsignatures
\pysigline
{\sphinxcode{\sphinxupquote{type.}}\sphinxbfcode{\sphinxupquote{spatial}}}
\pysigstopsignatures
\sphinxAtStartPar
\DUrole{versionmodified}{\DUrole{deprecated}{Nebepalaikoma nuo 0.2 versijos.}}

\sphinxAtStartPar
Apibrėžtis erdvėje.

\sphinxAtStartPar
Šis tipas atitinka \sphinxcode{\sphinxupquote{geometry}}, tačiau nurodo, kad visas model yra
apibrėžtas erdvėje, būtent pagal šią savybę.  Tik viena model savybė
gali turėti \sphinxcode{\sphinxupquote{spatial}} tipą. Pagal šios savybės reikšmes apskaičiuojamas ir
įvertinamas \sphinxhref{https://www.w3.org/TR/vocab-dcat-2/\#Property:dataset\_spatial}{dct:spatial}.

\end{fulllineitems}

\index{money (modulje type)@\spxentry{money}\spxextra{modulje type}}

\begin{fulllineitems}
\phantomsection\label{\detokenize{tipai:type.money}}
\pysigstartsignatures
\pysigline
{\sphinxcode{\sphinxupquote{type.}}\sphinxbfcode{\sphinxupquote{money}}}
\pysigstopsignatures
\sphinxAtStartPar
Valiuta. Saugomas valiutos kiekis, nurodant tiek sumą, tiek valiutos
kodą naudojant \sphinxhref{https://en.wikipedia.org/wiki/ISO\_4217}{ISO 4217} kodus.

\sphinxAtStartPar
Valiutos kodas nurodomas {\hyperref[\detokenize{dimensijos:property.ref}]{\sphinxcrossref{\sphinxcode{\sphinxupquote{property.ref}}}}} stulpelyje.

\sphinxAtStartPar
Pavyzdys:


\begin{savenotes}\sphinxattablestart
\sphinxthistablewithglobalstyle
\centering
\begin{tabulary}{\linewidth}[t]{TTTTTTTT}
\sphinxtoprule
\sphinxstyletheadfamily 
\sphinxAtStartPar
d
&\sphinxstyletheadfamily 
\sphinxAtStartPar
r
&\sphinxstyletheadfamily 
\sphinxAtStartPar
b
&\sphinxstyletheadfamily 
\sphinxAtStartPar
m
&\sphinxstyletheadfamily 
\sphinxAtStartPar
property
&\sphinxstyletheadfamily 
\sphinxAtStartPar
type
&\sphinxstyletheadfamily 
\sphinxAtStartPar
ref
&\sphinxstyletheadfamily 
\sphinxAtStartPar
source
\\
\sphinxmidrule
\sphinxtableatstartofbodyhook\sphinxstartmulticolumn{5}%
\begin{varwidth}[t]{\sphinxcolwidth{5}{8}}
\sphinxAtStartPar
example
\par
\vskip-\baselineskip\vbox{\hbox{\strut}}\end{varwidth}%
\sphinxstopmulticolumn
&&&\\
\sphinxhline
\sphinxAtStartPar

&&&\sphinxstartmulticolumn{2}%
\begin{varwidth}[t]{\sphinxcolwidth{2}{8}}
\sphinxAtStartPar
Product
\par
\vskip-\baselineskip\vbox{\hbox{\strut}}\end{varwidth}%
\sphinxstopmulticolumn
&&&
\sphinxAtStartPar
PRODUCT
\\
\sphinxhline
\sphinxAtStartPar

&&&&
\sphinxAtStartPar
price
&
\sphinxAtStartPar
money
&
\sphinxAtStartPar
EUR
&
\sphinxAtStartPar
PRICE
\\
\sphinxbottomrule
\end{tabulary}
\sphinxtableafterendhook\par
\sphinxattableend\end{savenotes}

\sphinxAtStartPar
Jei valiutos suma ir pavadinimas saugomi atskirai, tuomet valiutą galima
aprašyti taip:


\begin{savenotes}\sphinxattablestart
\sphinxthistablewithglobalstyle
\centering
\begin{tabulary}{\linewidth}[t]{TTTTTTTTT}
\sphinxtoprule
\sphinxstyletheadfamily 
\sphinxAtStartPar
d
&\sphinxstyletheadfamily 
\sphinxAtStartPar
r
&\sphinxstyletheadfamily 
\sphinxAtStartPar
b
&\sphinxstyletheadfamily 
\sphinxAtStartPar
m
&\sphinxstyletheadfamily 
\sphinxAtStartPar
property
&\sphinxstyletheadfamily 
\sphinxAtStartPar
type
&\sphinxstyletheadfamily 
\sphinxAtStartPar
ref
&\sphinxstyletheadfamily 
\sphinxAtStartPar
source
&\sphinxstyletheadfamily 
\sphinxAtStartPar
prepare
\\
\sphinxmidrule
\sphinxtableatstartofbodyhook\sphinxstartmulticolumn{5}%
\begin{varwidth}[t]{\sphinxcolwidth{5}{9}}
\sphinxAtStartPar
example
\par
\vskip-\baselineskip\vbox{\hbox{\strut}}\end{varwidth}%
\sphinxstopmulticolumn
&&&&\\
\sphinxhline
\sphinxAtStartPar

&&&\sphinxstartmulticolumn{2}%
\begin{varwidth}[t]{\sphinxcolwidth{2}{9}}
\sphinxAtStartPar
Product
\par
\vskip-\baselineskip\vbox{\hbox{\strut}}\end{varwidth}%
\sphinxstopmulticolumn
&&&
\sphinxAtStartPar
PRODUCT
&\\
\sphinxhline
\sphinxAtStartPar

&&&&
\sphinxAtStartPar
amount
&&&
\sphinxAtStartPar
PRICE
&\\
\sphinxhline
\sphinxAtStartPar

&&&&
\sphinxAtStartPar
currency
&&&
\sphinxAtStartPar
CURRENCY\_CODE
&\\
\sphinxhline
\sphinxAtStartPar

&&&&
\sphinxAtStartPar
price
&
\sphinxAtStartPar
money
&&&
\sphinxAtStartPar
money(amount, currency)
\\
\sphinxbottomrule
\end{tabulary}
\sphinxtableafterendhook\par
\sphinxattableend\end{savenotes}

\sphinxAtStartPar
Šio tipo duomenys pateikiami viena iš šių formų:

\begin{sphinxVerbatim}[commandchars=\\\{\}]
\PYG{l+m+mi}{123}
\PYG{l+m+mf}{123.45}
\PYG{l+m+mi}{123} \PYG{n}{EUR}
\PYG{l+m+mf}{123.45} \PYG{n}{EUR}
\end{sphinxVerbatim}

\end{fulllineitems}

\index{file (modulje type)@\spxentry{file}\spxextra{modulje type}}

\begin{fulllineitems}
\phantomsection\label{\detokenize{tipai:type.file}}
\pysigstartsignatures
\pysigline
{\sphinxcode{\sphinxupquote{type.}}\sphinxbfcode{\sphinxupquote{file}}}
\pysigstopsignatures
\sphinxAtStartPar
Šis duomenų tipas yra sudėtinis, susidedantis iš tokių duomenų:
\begin{description}
\sphinxlineitem{id}
\sphinxAtStartPar
Laukas, kuris unikaliai identifikuoja failą, šis laukas duomenų
saugojimo metu pavirs failo identifikatoriumi, jam suteikiant unikalų
UUID.

\sphinxlineitem{name}
\sphinxAtStartPar
Failo pavadinimas.

\sphinxlineitem{type}
\sphinxAtStartPar
Failo \sphinxhref{https://en.wikipedia.org/wiki/Media\_type}{media tipas}.

\sphinxlineitem{size}
\sphinxAtStartPar
Failo turinio dydis baitais.

\sphinxlineitem{content}
\sphinxAtStartPar
Failo turinys.

\end{description}

\sphinxAtStartPar
Šiuos metaduomenis galima perduoti \sphinxcode{\sphinxupquote{file()}} funkcijai, kaip vardinius
argumentus.

\begin{sphinxadmonition}{note}{Pavyzdys}


\begin{savenotes}\sphinxattablestart
\sphinxthistablewithglobalstyle
\centering
\begin{tabulary}{\linewidth}[t]{TTTTTTTTT}
\sphinxtoprule
\sphinxstyletheadfamily 
\sphinxAtStartPar
d
&\sphinxstyletheadfamily 
\sphinxAtStartPar
r
&\sphinxstyletheadfamily 
\sphinxAtStartPar
b
&\sphinxstyletheadfamily 
\sphinxAtStartPar
m
&\sphinxstyletheadfamily 
\sphinxAtStartPar
property
&\sphinxstyletheadfamily 
\sphinxAtStartPar
type
&\sphinxstyletheadfamily 
\sphinxAtStartPar
source
&\sphinxstyletheadfamily 
\sphinxAtStartPar
prepare
&\sphinxstyletheadfamily 
\sphinxAtStartPar
access
\\
\sphinxmidrule
\sphinxtableatstartofbodyhook\sphinxstartmulticolumn{5}%
\begin{varwidth}[t]{\sphinxcolwidth{5}{9}}
\sphinxAtStartPar
datasets/example
\par
\vskip-\baselineskip\vbox{\hbox{\strut}}\end{varwidth}%
\sphinxstopmulticolumn
&&&&\\
\sphinxhline
\sphinxAtStartPar

&&&\sphinxstartmulticolumn{2}%
\begin{varwidth}[t]{\sphinxcolwidth{2}{9}}
\sphinxAtStartPar
Country
\par
\vskip-\baselineskip\vbox{\hbox{\strut}}\end{varwidth}%
\sphinxstopmulticolumn
&&&&\\
\sphinxhline
\sphinxAtStartPar

&&&&
\sphinxAtStartPar
name
&
\sphinxAtStartPar
string
&
\sphinxAtStartPar
NAME
&&
\sphinxAtStartPar
open
\\
\sphinxhline
\sphinxAtStartPar

&&&&
\sphinxAtStartPar
flag\_file\_name
&
\sphinxAtStartPar
string
&
\sphinxAtStartPar
FLAG\_FILE\_NAME
&&
\sphinxAtStartPar
private
\\
\sphinxhline
\sphinxAtStartPar

&&&&
\sphinxAtStartPar
flag\_file\_data
&
\sphinxAtStartPar
binary
&
\sphinxAtStartPar
FLAG\_FILE\_DATA
&&
\sphinxAtStartPar
private
\\
\sphinxhline
\sphinxAtStartPar

&&&&
\sphinxAtStartPar
flag
&
\sphinxAtStartPar
file
&&
\sphinxAtStartPar
file(name: flag\_file\_name, content: flag\_file\_data)
&
\sphinxAtStartPar
open
\\
\sphinxbottomrule
\end{tabulary}
\sphinxtableafterendhook\par
\sphinxattableend\end{savenotes}

\sphinxAtStartPar
Šiame pavyzdyje, iš \sphinxcode{\sphinxupquote{flag\_file\_name}} ir \sphinxcode{\sphinxupquote{flag\_file\_data}} laukų
padaromas vienas \sphinxcode{\sphinxupquote{flag}} laukas, kuriame panaudojami duomenys iš dviejų
laukų. Šiuo atveju, \sphinxcode{\sphinxupquote{flag\_file\_name}} ir \sphinxcode{\sphinxupquote{flag\_file\_data}} laukai tampa
pertekliniais, todėl {\hyperref[\detokenize{prieiga:id0}]{\sphinxcrossref{\sphinxcode{\sphinxupquote{access}}}}} stulpelyje jie pažymėti \sphinxcode{\sphinxupquote{private}}.

\sphinxAtStartPar
Analogiškai, tokius pačius duomenis galima aprašyti ir nenaudojant
formulių:


\begin{savenotes}\sphinxattablestart
\sphinxthistablewithglobalstyle
\centering
\begin{tabulary}{\linewidth}[t]{TTTTTTTTT}
\sphinxtoprule
\sphinxstyletheadfamily 
\sphinxAtStartPar
d
&\sphinxstyletheadfamily 
\sphinxAtStartPar
r
&\sphinxstyletheadfamily 
\sphinxAtStartPar
b
&\sphinxstyletheadfamily 
\sphinxAtStartPar
m
&\sphinxstyletheadfamily 
\sphinxAtStartPar
property
&\sphinxstyletheadfamily 
\sphinxAtStartPar
type
&\sphinxstyletheadfamily 
\sphinxAtStartPar
source
&\sphinxstyletheadfamily 
\sphinxAtStartPar
prepare
&\sphinxstyletheadfamily 
\sphinxAtStartPar
access
\\
\sphinxmidrule
\sphinxtableatstartofbodyhook\sphinxstartmulticolumn{5}%
\begin{varwidth}[t]{\sphinxcolwidth{5}{9}}
\sphinxAtStartPar
datasets/example
\par
\vskip-\baselineskip\vbox{\hbox{\strut}}\end{varwidth}%
\sphinxstopmulticolumn
&&&&\\
\sphinxhline
\sphinxAtStartPar

&&&\sphinxstartmulticolumn{2}%
\begin{varwidth}[t]{\sphinxcolwidth{2}{9}}
\sphinxAtStartPar
Country
\par
\vskip-\baselineskip\vbox{\hbox{\strut}}\end{varwidth}%
\sphinxstopmulticolumn
&&&&\\
\sphinxhline
\sphinxAtStartPar

&&&&
\sphinxAtStartPar
name
&
\sphinxAtStartPar
string
&
\sphinxAtStartPar
NAME
&&
\sphinxAtStartPar
open
\\
\sphinxhline
\sphinxAtStartPar

&&&&
\sphinxAtStartPar
flag
&
\sphinxAtStartPar
file
&&&
\sphinxAtStartPar
open
\\
\sphinxhline
\sphinxAtStartPar

&&&&
\sphinxAtStartPar
flag.\_name
&&
\sphinxAtStartPar
FLAG\_FILE\_NAME
&&
\sphinxAtStartPar
open
\\
\sphinxhline
\sphinxAtStartPar

&&&&
\sphinxAtStartPar
flag.\_content
&&
\sphinxAtStartPar
FLAG\_FILE\_DATA
&&
\sphinxAtStartPar
open
\\
\sphinxbottomrule
\end{tabulary}
\sphinxtableafterendhook\par
\sphinxattableend\end{savenotes}
\end{sphinxadmonition}

\end{fulllineitems}

\index{image (modulje type)@\spxentry{image}\spxextra{modulje type}}

\begin{fulllineitems}
\phantomsection\label{\detokenize{tipai:type.image}}
\pysigstartsignatures
\pysigline
{\sphinxcode{\sphinxupquote{type.}}\sphinxbfcode{\sphinxupquote{image}}}
\pysigstopsignatures
\sphinxAtStartPar
Paveiksliukas. \sphinxcode{\sphinxupquote{image}} tipas turi tokias pačias savybes kaip \sphinxcode{\sphinxupquote{file}}
tipas.

\end{fulllineitems}

\index{ref (modulje type)@\spxentry{ref}\spxextra{modulje type}}

\begin{fulllineitems}
\phantomsection\label{\detokenize{tipai:type.ref}}
\pysigstartsignatures
\pysigline
{\sphinxcode{\sphinxupquote{type.}}\sphinxbfcode{\sphinxupquote{ref}}}
\pysigstopsignatures
\sphinxAtStartPar
Ryšys su modeliu. Šis tipas naudojamas norint pažymėti, kad lauko
reikšmė yra {\hyperref[\detokenize{dimensijos:property.ref}]{\sphinxcrossref{\sphinxcode{\sphinxupquote{property.ref}}}}} stulpelyje nurodyto modelio objektas.

\sphinxAtStartPar
Pagal nutylėjimą, jungimas su kito modelio objektais daromas per siejamo
pirminį raktą ({\hyperref[\detokenize{dimensijos:model.ref}]{\sphinxcrossref{\sphinxcode{\sphinxupquote{model.ref}}}}}), tačiau yra galimybė nurodyti ir kitą,
nebūtinai pirminį raktą.

\sphinxAtStartPar
Jei jungimas daromas, ne per pirminį raktą, tuomet, laukai per kuriuos
daromas jungimas nurodomi {\hyperref[\detokenize{dimensijos:property.ref}]{\sphinxcrossref{\sphinxcode{\sphinxupquote{property.ref}}}}} stulpelyje laužtiniuose
sklaustuose, pavyzdžiui:

\begin{sphinxVerbatim}[commandchars=\\\{\}]
\PYG{n}{Country}\PYG{p}{[}\PYG{n}{code}\PYG{p}{]}
\end{sphinxVerbatim}

\sphinxAtStartPar
Čia jungiama su \sphinxcode{\sphinxupquote{Country}} modeliu, per \sphinxcode{\sphinxupquote{Country}} modelio \sphinxcode{\sphinxupquote{code}} duomenų
lauką.

\sphinxAtStartPar
Jei laukas, per kurį daromas jungimas nenurodytas, pavyzdžiui:

\begin{sphinxVerbatim}[commandchars=\\\{\}]
\PYG{n}{Country}
\end{sphinxVerbatim}

\sphinxAtStartPar
Tada, jungimas daromas per \sphinxcode{\sphinxupquote{Country}} modelio pirminį raktą, kuris nurodytas
{\hyperref[\detokenize{dimensijos:model.ref}]{\sphinxcrossref{\sphinxcode{\sphinxupquote{model.ref}}}}} stulpelyje.

\sphinxAtStartPar
Šio objekto reikšmės yra pateikiamos, kaip dalis objekto į kurį rodoma. Jei
\sphinxcode{\sphinxupquote{ref}} tipo lauko brandos lygis ({\hyperref[\detokenize{dimensijos:property.level}]{\sphinxcrossref{\sphinxcode{\sphinxupquote{property.level}}}}}) yra 4 ar didesnis,
tuomet šio duomenų tipo reikšmės atrodo taip:

\begin{sphinxVerbatim}[commandchars=\\\{\}]
\PYG{p}{\PYGZob{}}\PYG{n+nt}{\PYGZdq{}\PYGZus{}id\PYGZdq{}}\PYG{p}{:}\PYG{+w}{ }\PYG{l+s+s2}{\PYGZdq{}69c98b0f\PYGZhy{}9e4e\PYGZhy{}424b\PYGZhy{}9575\PYGZhy{}9f601d79b68e\PYGZdq{}}\PYG{p}{\PYGZcb{}}
\end{sphinxVerbatim}

\sphinxAtStartPar
Jei brandos lygis ({\hyperref[\detokenize{dimensijos:property.level}]{\sphinxcrossref{\sphinxcode{\sphinxupquote{property.level}}}}}) yra žemesnis nei 4, tada reikšmė
atrodo taip:

\begin{sphinxVerbatim}[commandchars=\\\{\}]
\PYG{p}{\PYGZob{}}\PYG{n+nt}{\PYGZdq{}id\PYGZdq{}}\PYG{p}{:}\PYG{+w}{ }\PYG{l+s+s2}{\PYGZdq{}69c98b0f\PYGZhy{}9e4e\PYGZhy{}424b\PYGZhy{}9575\PYGZhy{}9f601d79b68e\PYGZdq{}}\PYG{p}{\PYGZcb{}}
\end{sphinxVerbatim}

\sphinxAtStartPar
Čia \sphinxcode{\sphinxupquote{id}} yra {\hyperref[\detokenize{dimensijos:model.ref}]{\sphinxcrossref{\sphinxcode{\sphinxupquote{model.ref}}}}} arba {\hyperref[\detokenize{identifikatoriai:ref-fkey}]{\sphinxcrossref{\DUrole{std}{\DUrole{std-ref}{kitas laukas}}}}}, per
kurį daromas jungimas. Jei nenurodytas nei {\hyperref[\detokenize{dimensijos:model.ref}]{\sphinxcrossref{\sphinxcode{\sphinxupquote{model.ref}}}}}, nei
{\hyperref[\detokenize{identifikatoriai:ref-fkey}]{\sphinxcrossref{\DUrole{std}{\DUrole{std-ref}{kitas laukas}}}}}, tada jungimas daromas per \sphinxcode{\sphinxupquote{\_id}}, tačiau
netikrinama ar toks \sphinxcode{\sphinxupquote{\_id}} egzistuoja jungiamame modelyje.


\begin{sphinxseealso}{Taip pat žiūrėkite:}

\sphinxAtStartPar
{\hyperref[\detokenize{identifikatoriai:rysiai}]{\sphinxcrossref{\DUrole{std}{\DUrole{std-ref}{Asociacija}}}}}


\end{sphinxseealso}


\end{fulllineitems}

\index{backref (modulje type)@\spxentry{backref}\spxextra{modulje type}}

\begin{fulllineitems}
\phantomsection\label{\detokenize{tipai:type.backref}}
\pysigstartsignatures
\pysigline
{\sphinxcode{\sphinxupquote{type.}}\sphinxbfcode{\sphinxupquote{backref}}}
\pysigstopsignatures
\sphinxAtStartPar
Atgalinis ryšys su modeliu.

\sphinxAtStartPar
Jei ryšys tarp dviejų modlių yra daug su vienu, tada \sphinxcode{\sphinxupquote{property}} pavadinimas
nurodomas su \sphinxcode{\sphinxupquote{{[}{]}}} simboliu.

\begin{sphinxadmonition}{note}{Pavyzdys}

\sphinxAtStartPar
Koncepcinis modelis

\begin{DUlineblock}{0em}
\item[] 
\end{DUlineblock}

\sphinxAtStartPar
Struktūros aprašas


\begin{savenotes}\sphinxattablestart
\sphinxthistablewithglobalstyle
\centering
\begin{tabulary}{\linewidth}[t]{TTTT}
\sphinxtoprule
\sphinxstyletheadfamily 
\sphinxAtStartPar
model
&\sphinxstyletheadfamily 
\sphinxAtStartPar
property
&\sphinxstyletheadfamily 
\sphinxAtStartPar
type
&\sphinxstyletheadfamily 
\sphinxAtStartPar
ref
\\
\sphinxmidrule
\sphinxtableatstartofbodyhook\sphinxstartmulticolumn{2}%
\begin{varwidth}[t]{\sphinxcolwidth{2}{4}}
\sphinxAtStartPar
\sphinxstylestrong{Country}
\par
\vskip-\baselineskip\vbox{\hbox{\strut}}\end{varwidth}%
\sphinxstopmulticolumn
&&
\sphinxAtStartPar
id
\\
\sphinxhline
\sphinxAtStartPar

&
\sphinxAtStartPar
id
&
\sphinxAtStartPar
integer
&\\
\sphinxhline
\sphinxAtStartPar

&
\sphinxAtStartPar
name@lt
&
\sphinxAtStartPar
string
&\\
\sphinxhline
\sphinxAtStartPar

&
\sphinxAtStartPar
cities{[}{]}
&
\sphinxAtStartPar
backref
&
\sphinxAtStartPar
\sphinxstylestrong{City}
\\
\sphinxhline\sphinxstartmulticolumn{2}%
\begin{varwidth}[t]{\sphinxcolwidth{2}{4}}
\sphinxAtStartPar
\sphinxstylestrong{City}
\par
\vskip-\baselineskip\vbox{\hbox{\strut}}\end{varwidth}%
\sphinxstopmulticolumn
&&
\sphinxAtStartPar
id
\\
\sphinxhline
\sphinxAtStartPar

&
\sphinxAtStartPar
id
&
\sphinxAtStartPar
integer
&\\
\sphinxhline
\sphinxAtStartPar

&
\sphinxAtStartPar
name@lt
&
\sphinxAtStartPar
string
&\\
\sphinxhline
\sphinxAtStartPar

&
\sphinxAtStartPar
country
&
\sphinxAtStartPar
ref
&
\sphinxAtStartPar
\sphinxstylestrong{Country}
\\
\sphinxbottomrule
\end{tabulary}
\sphinxtableafterendhook\par
\sphinxattableend\end{savenotes}
\end{sphinxadmonition}


\begin{sphinxseealso}{Taip pat žiūrėkite:}

\sphinxAtStartPar
{\hyperref[\detokenize{identifikatoriai:atgalinis-rysys}]{\sphinxcrossref{\DUrole{std}{\DUrole{std-ref}{Atgalinis ryšys}}}}}


\end{sphinxseealso}


\end{fulllineitems}

\index{generic (modulje type)@\spxentry{generic}\spxextra{modulje type}}

\begin{fulllineitems}
\phantomsection\label{\detokenize{tipai:type.generic}}
\pysigstartsignatures
\pysigline
{\sphinxcode{\sphinxupquote{type.}}\sphinxbfcode{\sphinxupquote{generic}}}
\pysigstopsignatures
\sphinxAtStartPar
Dinaminis ryšys su modeliu.

\sphinxAtStartPar
Šis tipas naudojamas tada, kai yra poreikis perteikti dinaminį ryšį, t.
y. duomenys siejami ne tik pagal id, bet ir pagal modelio pavadinimą.
Tokiu būdu, vieno modelio laukas gali būti siejamas su keliais
modeliais.


\begin{sphinxseealso}{Taip pat žiūrėkite:}

\sphinxAtStartPar
{\hyperref[\detokenize{identifikatoriai:polimorfinis-rysys}]{\sphinxcrossref{\DUrole{std}{\DUrole{std-ref}{Polimorfinis jungimas}}}}}


\end{sphinxseealso}


\sphinxAtStartPar
Šis duomenų tipas yra sudėtinis, susidedantis iš tokių duomenų:
\begin{description}
\sphinxlineitem{object\_model}
\sphinxAtStartPar
Pilnas modelio pavadinimas, su kuriuo yra siejamas objektas.

\sphinxlineitem{object\_id}
\sphinxAtStartPar
\sphinxcode{\sphinxupquote{object\_model}} modelio objekto id.

\end{description}

\end{fulllineitems}

\index{object (modulje type)@\spxentry{object}\spxextra{modulje type}}

\begin{fulllineitems}
\phantomsection\label{\detokenize{tipai:type.object}}
\pysigstartsignatures
\pysigline
{\sphinxcode{\sphinxupquote{type.}}\sphinxbfcode{\sphinxupquote{object}}}
\pysigstopsignatures
\sphinxAtStartPar
{\hyperref[\detokenize{savokos:term-sudetinis-tipas}]{\sphinxtermref{\DUrole{xref}{\DUrole{std}{\DUrole{std-term}{Sudėtinis tipas}}}}}}, apjungiantis kelias savybes į
grupę, po vienu pavadinimu.

\sphinxAtStartPar
Šis tipas naudojamas apibrėžti sudėtiniams duomenims, kurie aprašyti
naudojant kelis skirtingus tipas. Kompozicinio tipo atveju property
stulpelyje komponuojami pavadinimai atskiriami taško simboliu.

\sphinxAtStartPar
Sudarant duomenų modelį, rekomenduojama laikytis plokščios struktūros ir
komponavimą įgyvendinti siejant modelius per \sphinxcode{\sphinxupquote{ref}} ar \sphinxcode{\sphinxupquote{generic}} tipus.

\end{fulllineitems}

\index{array (modulje type)@\spxentry{array}\spxextra{modulje type}}

\begin{fulllineitems}
\phantomsection\label{\detokenize{tipai:type.array}}
\pysigstartsignatures
\pysigline
{\sphinxcode{\sphinxupquote{type.}}\sphinxbfcode{\sphinxupquote{array}}}
\pysigstopsignatures
\sphinxAtStartPar
{\hyperref[\detokenize{savokos:term-sudetinis-tipas}]{\sphinxtermref{\DUrole{xref}{\DUrole{std}{\DUrole{std-term}{Sudėtinis tipas}}}}}}, nurodo reikšmių masyvą.

\sphinxAtStartPar
Šis tipas naudojamas apibrėžti duomenų masyvams. Jei masyvo elementai
turi vienodus tipus, tada elemento tipas pateikiamas property pavadinimo
gale prirašant {[}{]} sufiksą, kuris nurodo, kad aprašomas ne pats masyvas,
o masyvo elementas.

\end{fulllineitems}

\index{url (modulje type)@\spxentry{url}\spxextra{modulje type}}

\begin{fulllineitems}
\phantomsection\label{\detokenize{tipai:type.url}}
\pysigstartsignatures
\pysigline
{\sphinxcode{\sphinxupquote{type.}}\sphinxbfcode{\sphinxupquote{url}}}
\pysigstopsignatures
\sphinxAtStartPar
Unikali resurso vieta (URL) (angl. \sphinxstyleemphasis{Uniform Resource
Locator}).

\sphinxAtStartPar
Šis tipas naudojamas pateikiant nuorodas į išorinius šaltinius.

\sphinxAtStartPar
\sphinxurl{https://en.wikipedia.org/wiki/Uniform\_Resource\_Locator}

\end{fulllineitems}

\index{uri (modulje type)@\spxentry{uri}\spxextra{modulje type}}

\begin{fulllineitems}
\phantomsection\label{\detokenize{tipai:type.uri}}
\pysigstartsignatures
\pysigline
{\sphinxcode{\sphinxupquote{type.}}\sphinxbfcode{\sphinxupquote{uri}}}
\pysigstopsignatures
\sphinxAtStartPar
Universalus resurso identifikatorius (URI) (angl. \sphinxstyleemphasis{Universal Resource
Identifier}).

\sphinxAtStartPar
Šis tipas naudojamas tais atvejais, kai pateikiamas išorinio resurso
identifikatorius, RDF duomenų modelyje tai yra subjeto identifikatorius.

\sphinxAtStartPar
\sphinxurl{https://en.wikipedia.org/wiki/Uniform\_Resource\_Identifier}

\end{fulllineitems}


\sphinxstepscope


\section{Kodiniai pavadinimai}
\label{\detokenize{pavadinimai:kodiniai-pavadinimai}}\label{\detokenize{pavadinimai:id1}}\label{\detokenize{pavadinimai::doc}}
\sphinxAtStartPar
Kadangi {\hyperref[\detokenize{savokos:term-DSA}]{\sphinxtermref{\DUrole{xref}{\DUrole{std}{\DUrole{std-term}{DSA}}}}}} lentelė skirta naudoti tiek žmonėms tiek automatizuotoms
priemonėms, tam tikros lentelės dalys privalo naudoti sutartinius kodinius
pavadinimus. Kodiniams pavadinimams keliami griežtesni reikalavimai, kadangi
šiuos pavadinimus interpretuos automatizuotos priemonės.

\sphinxAtStartPar
Visi {\hyperref[\detokenize{savokos:term-DSA}]{\sphinxtermref{\DUrole{xref}{\DUrole{std}{\DUrole{std-term}{DSA}}}}}} lentelės stulpelių pavadinimai turi būti užrašyti tiksliai
taip, kaip nurodyta, kad kompiuterio programos galėtų juos atpažinti.

\sphinxAtStartPar
Kodiniai pavadinimai rašomi naudojant tik lotyniškas raidas. Lietuviškų
raidžių naudoti negalima, todėl geriausia pavadinimus užrašyti anglų kalba,
arba pakeičiant lietuviškas raides į lotyniškos raidės analogą.

\sphinxAtStartPar
Deja, vis dar pasitaiko vietų, kuriose palaikoma tik lotyniška abėcėlė, todėl
ir keliamas toks reikalavimas, siekiant užtikrinti maksimalų suderinamumą
tarp skirtingų sistemų.

\sphinxAtStartPar
Pavadinimai turėtu būti rašomi laikantis tokio stiliaus:


\subsection{Vardų erdvių}
\label{\detokenize{pavadinimai:vardu-erdviu}}
\sphinxAtStartPar
Pavyzdys: \sphinxcode{\sphinxupquote{datasets/gov/abbr/short/word}}

\sphinxAtStartPar
Visos mažosios raidės, stengiantis naudoti vieno žodžio trumpus
pavadinimus arba žodžio trumpinius. Kadangi vardų erdvė rašoma prie
kiekvieno modelio pavadinimo, todėl reikia stengtis vardų erdvių ir
duomenų rinkinių pavadinimus išlaikyti kiek įmanoma trumpesnius.

\sphinxAtStartPar
Vardų erdvės pavadinimai užrašomi daugiskaita ir turi prasidėti mažąja raide.


\subsection{Modeliai}
\label{\detokenize{pavadinimai:modeliai}}
\sphinxAtStartPar
Pavyzdys: \sphinxcode{\sphinxupquote{UpperCamelCase}}

\sphinxAtStartPar
Kiekvieno modelio pavadinimo pirma raidė didžioji, kitos mažosios.
Pavadinimo žodžiai atskiriami juos užrašant iš didžiosios raidės. Tarp
žodžių neturi būti nei tarpų, nei kitų skyrybos ženklų.

\sphinxAtStartPar
Modelio pavadinimas įprastai užrašomas veinaskaitos forma.

\sphinxAtStartPar
Modelio kodinius pavadinimus užrašome taip, kaip pavadintume vieną objektą,
kuriam yra taikomas duomenų modelis. Tarkime jei aprašome pastatus, tai vienas
pastatas būtų vadinamas vienaskaitos forma \sphinxcode{\sphinxupquote{Pastatas}}. Tačiau, jei vienas
objektas yra patatų grupė, kuriuos jungia bendra paskirtis, tada galima
pavadinti \sphinxcode{\sphinxupquote{Pastatai}} arba \sphinxcode{\sphinxupquote{PastatuKompleksas}}.

\sphinxAtStartPar
Modelio pavadinimas turi atspindėti \sphinxhref{https://en.wikipedia.org/wiki/Entity\%E2\%80\%93relationship\_model\#Entity\%E2\%80\%93relationship\_model}{duomenų subjekto} tipą.
Duomenų subjektas yra dalykas turintis pavadinimą ar unikalų identifikatorių.
Duomenų subjekto tipas yra subjektų grupė priklausančių tai pačiai kategorijai
ar \sphinxhref{https://en.wikipedia.org/wiki/Class\_(knowledge\_representation)}{klasei}.


\subsubsection{Nekartojame vardų erdvės}
\label{\detokenize{pavadinimai:nekartojame-vardu-erdves}}
\sphinxAtStartPar
Modelio pavadinime nekartojamos vardų erdvės, kurioje yra modelis.

\sphinxAtStartPar
Pavyzdys, kaip nereikėtų daryti: \sphinxcode{\sphinxupquote{example/planets/EarthPlanet}}. Šioje
vietoje nereikia kartoti \sphinxcode{\sphinxupquote{Planet}}, kadangi tai atsispindi vardų erdvės
pavadinime \sphinxcode{\sphinxupquote{planets}}.


\subsection{Duomenų laukai}
\label{\detokenize{pavadinimai:duomenu-laukai}}
\sphinxAtStartPar
Pavyzdys: \sphinxcode{\sphinxupquote{snake\_case}}

\sphinxAtStartPar
Visi duomenų lauko žodžiai rašomi mažosiomis raidėmis, atskiriami pabraukimo
ženklu \sphinxcode{\sphinxupquote{\_}}.

\sphinxAtStartPar
Duomenų lauko pavadinimas turi prasidėti mažąja raide.


\subsubsection{Ryšiai tarp modelių}
\label{\detokenize{pavadinimai:rysiai-tarp-modeliu}}
\sphinxAtStartPar
{\hyperref[\detokenize{formatas:ref}]{\sphinxcrossref{\sphinxcode{\sphinxupquote{ref}}}}} tipo laukai rašomi be \sphinxcode{\sphinxupquote{id}} ar \sphinxcode{\sphinxupquote{\_id}} galūnės, kadangi jis yra
perteklinis.

\sphinxAtStartPar
{\hyperref[\detokenize{formatas:ref}]{\sphinxcrossref{\sphinxcode{\sphinxupquote{ref}}}}} tipo laukai atspindi ne konkretų identifikatorių, o visą
objektą. Konkretus identifikatorius yra rezervuotas pavadinimas ir
duomenų struktūros apraše nenurodomas.

\sphinxAtStartPar
Pavyzdžiui vietoje \sphinxcode{\sphinxupquote{country\_id}}, kurio tipas yra \sphinxcode{\sphinxupquote{ref}}, reikėtų rašyti
\sphinxcode{\sphinxupquote{country}}.


\begin{savenotes}\sphinxattablestart
\sphinxthistablewithglobalstyle
\centering
\begin{tabulary}{\linewidth}[t]{TTTT}
\sphinxtoprule
\sphinxstyletheadfamily 
\sphinxAtStartPar
m
&\sphinxstyletheadfamily 
\sphinxAtStartPar
property
&\sphinxstyletheadfamily 
\sphinxAtStartPar
type
&\sphinxstyletheadfamily 
\sphinxAtStartPar
ref
\\
\sphinxmidrule
\sphinxtableatstartofbodyhook\sphinxstartmulticolumn{2}%
\begin{varwidth}[t]{\sphinxcolwidth{2}{4}}
\sphinxAtStartPar
Country
\par
\vskip-\baselineskip\vbox{\hbox{\strut}}\end{varwidth}%
\sphinxstopmulticolumn
&&\\
\sphinxhline
\sphinxAtStartPar

&
\sphinxAtStartPar
name@lt
&
\sphinxAtStartPar
text
&\\
\sphinxhline\sphinxstartmulticolumn{2}%
\begin{varwidth}[t]{\sphinxcolwidth{2}{4}}
\sphinxAtStartPar
City
\par
\vskip-\baselineskip\vbox{\hbox{\strut}}\end{varwidth}%
\sphinxstopmulticolumn
&&\\
\sphinxhline
\sphinxAtStartPar

&
\sphinxAtStartPar
name@lt
&
\sphinxAtStartPar
text
&\\
\sphinxhline
\sphinxAtStartPar

&
\sphinxAtStartPar
country
&
\sphinxAtStartPar
ref
&
\sphinxAtStartPar
Country
\\
\sphinxbottomrule
\end{tabulary}
\sphinxtableafterendhook\par
\sphinxattableend\end{savenotes}

\sphinxAtStartPar
Tais atvejais, kai duomenys yra denormalizuoti, duomenų lauko
pavadinimas užrašomas su tašku, nurodant duomenų lauką iš siejamo
modelio. Plačiau apie tai {\hyperref[\detokenize{identifikatoriai:ref-denorm}]{\sphinxcrossref{\DUrole{std}{\DUrole{std-ref}{Jungtinis modelis}}}}}.


\subsubsection{Nekartojame modelio pavadinimo}
\label{\detokenize{pavadinimai:nekartojame-modelio-pavadinimo}}
\sphinxAtStartPar
Visi modelio duomenų laukai yra konkretaus modelio laukai, todėl
nereikia kartoti duomenų laukuose modelio pavadinimo, pavyzdžiui vietoje
tokių pavadinimų:


\begin{savenotes}\sphinxattablestart
\sphinxthistablewithglobalstyle
\centering
\begin{tabulary}{\linewidth}[t]{TT}
\sphinxtoprule
\sphinxstyletheadfamily 
\sphinxAtStartPar
m
&\sphinxstyletheadfamily 
\sphinxAtStartPar
property
\\
\sphinxmidrule
\sphinxtableatstartofbodyhook\sphinxstartmulticolumn{2}%
\begin{varwidth}[t]{\sphinxcolwidth{2}{2}}
\sphinxAtStartPar
City
\par
\vskip-\baselineskip\vbox{\hbox{\strut}}\end{varwidth}%
\sphinxstopmulticolumn
\\
\sphinxhline
\sphinxAtStartPar

&
\sphinxAtStartPar
city\_id
\\
\sphinxhline
\sphinxAtStartPar

&
\sphinxAtStartPar
city\_name
\\
\sphinxbottomrule
\end{tabulary}
\sphinxtableafterendhook\par
\sphinxattableend\end{savenotes}

\sphinxAtStartPar
Reikėtų rašyti taip:


\begin{savenotes}\sphinxattablestart
\sphinxthistablewithglobalstyle
\centering
\begin{tabulary}{\linewidth}[t]{TT}
\sphinxtoprule
\sphinxstyletheadfamily 
\sphinxAtStartPar
m
&\sphinxstyletheadfamily 
\sphinxAtStartPar
property
\\
\sphinxmidrule
\sphinxtableatstartofbodyhook\sphinxstartmulticolumn{2}%
\begin{varwidth}[t]{\sphinxcolwidth{2}{2}}
\sphinxAtStartPar
City
\par
\vskip-\baselineskip\vbox{\hbox{\strut}}\end{varwidth}%
\sphinxstopmulticolumn
\\
\sphinxhline
\sphinxAtStartPar

&
\sphinxAtStartPar
id
\\
\sphinxhline
\sphinxAtStartPar

&
\sphinxAtStartPar
name
\\
\sphinxbottomrule
\end{tabulary}
\sphinxtableafterendhook\par
\sphinxattableend\end{savenotes}

\sphinxAtStartPar
Jei kiti modeliai siejami su \sphinxcode{\sphinxupquote{City}}, tada nurodant tarkim \sphinxcode{\sphinxupquote{city\_name}} iš
kito modelio, reikėtų rašyti \sphinxcode{\sphinxupquote{city.city\_name}}. Todėl \sphinxcode{\sphinxupquote{city.name}} yra
aiškesnis pavadinimas, kuriame nesikartoja modelio pavadinimas.


\subsubsection{Nekartojame duomenų tipo pavadinimo}
\label{\detokenize{pavadinimai:nekartojame-duomenu-tipo-pavadinimo}}
\sphinxAtStartPar
Duomenų lauko pavadinime nereikia kartoti duomenų tipo pavadinimo.

\sphinxAtStartPar
Pavyzdžiui taip nereikėtų daryti:


\begin{savenotes}\sphinxattablestart
\sphinxthistablewithglobalstyle
\centering
\begin{tabulary}{\linewidth}[t]{TTT}
\sphinxtoprule
\sphinxstyletheadfamily 
\sphinxAtStartPar
m
&\sphinxstyletheadfamily 
\sphinxAtStartPar
property
&\sphinxstyletheadfamily 
\sphinxAtStartPar
type
\\
\sphinxmidrule
\sphinxtableatstartofbodyhook\sphinxstartmulticolumn{2}%
\begin{varwidth}[t]{\sphinxcolwidth{2}{3}}
\sphinxAtStartPar
City
\par
\vskip-\baselineskip\vbox{\hbox{\strut}}\end{varwidth}%
\sphinxstopmulticolumn
&\\
\sphinxhline
\sphinxAtStartPar

&
\sphinxAtStartPar
founded\_date
&
\sphinxAtStartPar
date
\\
\sphinxbottomrule
\end{tabulary}
\sphinxtableafterendhook\par
\sphinxattableend\end{savenotes}

\sphinxAtStartPar
Reikėtų rašyti taip:


\begin{savenotes}\sphinxattablestart
\sphinxthistablewithglobalstyle
\centering
\begin{tabulary}{\linewidth}[t]{TTT}
\sphinxtoprule
\sphinxstyletheadfamily 
\sphinxAtStartPar
m
&\sphinxstyletheadfamily 
\sphinxAtStartPar
property
&\sphinxstyletheadfamily 
\sphinxAtStartPar
type
\\
\sphinxmidrule
\sphinxtableatstartofbodyhook\sphinxstartmulticolumn{2}%
\begin{varwidth}[t]{\sphinxcolwidth{2}{3}}
\sphinxAtStartPar
City
\par
\vskip-\baselineskip\vbox{\hbox{\strut}}\end{varwidth}%
\sphinxstopmulticolumn
&\\
\sphinxhline
\sphinxAtStartPar

&
\sphinxAtStartPar
founded
&
\sphinxAtStartPar
date
\\
\sphinxbottomrule
\end{tabulary}
\sphinxtableafterendhook\par
\sphinxattableend\end{savenotes}

\sphinxAtStartPar
Nėra prasmės kartoti duomenų tipo, lauko pavadinime.

\sphinxstepscope


\section{Matavimo vienetai}
\label{\detokenize{vienetai:matavimo-vienetai}}\label{\detokenize{vienetai:id1}}\label{\detokenize{vienetai::doc}}

\subsection{Apibrėžtis laike}
\label{\detokenize{vienetai:apibreztis-laike}}
\sphinxAtStartPar
\sphinxcode{\sphinxupquote{date}} ir \sphinxcode{\sphinxupquote{datetime}} tipo duomenų laukams gali būti žymimas ir \DUrole{xref}{\DUrole{std}{\DUrole{std-ref}{laiko}}} duomenų tikslumas, pavyzdžiui:


\begin{savenotes}\sphinxattablestart
\sphinxthistablewithglobalstyle
\centering
\begin{tabulary}{\linewidth}[t]{TTTTTTTT}
\sphinxtoprule
\sphinxstyletheadfamily 
\sphinxAtStartPar
d
&\sphinxstyletheadfamily 
\sphinxAtStartPar
r
&\sphinxstyletheadfamily 
\sphinxAtStartPar
b
&\sphinxstyletheadfamily 
\sphinxAtStartPar
m
&\sphinxstyletheadfamily 
\sphinxAtStartPar
property
&\sphinxstyletheadfamily 
\sphinxAtStartPar
type
&\sphinxstyletheadfamily 
\sphinxAtStartPar
ref
&\sphinxstyletheadfamily 
\sphinxAtStartPar
level
\\
\sphinxmidrule
\sphinxtableatstartofbodyhook\sphinxstartmulticolumn{5}%
\begin{varwidth}[t]{\sphinxcolwidth{5}{8}}
\sphinxAtStartPar
datasets/example
\par
\vskip-\baselineskip\vbox{\hbox{\strut}}\end{varwidth}%
\sphinxstopmulticolumn
&&&\\
\sphinxhline
\sphinxAtStartPar

&&&\sphinxstartmulticolumn{2}%
\begin{varwidth}[t]{\sphinxcolwidth{2}{8}}
\sphinxAtStartPar
Matavimas
\par
\vskip-\baselineskip\vbox{\hbox{\strut}}\end{varwidth}%
\sphinxstopmulticolumn
&&
\sphinxAtStartPar
id
&\\
\sphinxhline
\sphinxAtStartPar

&&&&
\sphinxAtStartPar
id
&
\sphinxAtStartPar
integer
&&
\sphinxAtStartPar
4
\\
\sphinxhline
\sphinxAtStartPar

&&&&
\sphinxAtStartPar
laikas
&
\sphinxAtStartPar
datetime
&
\sphinxAtStartPar
1S
&
\sphinxAtStartPar
4
\\
\sphinxhline
\sphinxAtStartPar

&&&&
\sphinxAtStartPar
vieta
&
\sphinxAtStartPar
geometry(point, 4326)
&
\sphinxAtStartPar
1m
&
\sphinxAtStartPar
4
\\
\sphinxbottomrule
\end{tabulary}
\sphinxtableafterendhook\par
\sphinxattableend\end{savenotes}

\sphinxAtStartPar
Šiuo atveju, nurodyta, kad laukas \sphinxcode{\sphinxupquote{laikas}} yra 1 sekundės tikslumu, o \sphinxcode{\sphinxupquote{vieta}} 1
metro tikslumu.

\sphinxAtStartPar
Žymint laiko tikslumą, galite naudoti tokius sutartinius simbolius (atkreipkite
dėmesį, kad šie vienetai veikia tik su \sphinxcode{\sphinxupquote{date}} ir \sphinxcode{\sphinxupquote{datetime}} tipais):


\begin{savenotes}\sphinxattablestart
\sphinxthistablewithglobalstyle
\centering
\begin{tabulary}{\linewidth}[t]{TT}
\sphinxtoprule
\sphinxstyletheadfamily 
\sphinxAtStartPar
Simbolis
&\sphinxstyletheadfamily 
\sphinxAtStartPar
Prasmė
\\
\sphinxmidrule
\sphinxtableatstartofbodyhook
\sphinxAtStartPar
Y
&
\sphinxAtStartPar
Metai
\\
\sphinxhline
\sphinxAtStartPar
Q
&
\sphinxAtStartPar
Metų ketvirčiai
\\
\sphinxhline
\sphinxAtStartPar
M
&
\sphinxAtStartPar
Mėnesiai
\\
\sphinxhline
\sphinxAtStartPar
W
&
\sphinxAtStartPar
Savaitės
\\
\sphinxhline
\sphinxAtStartPar
D
&
\sphinxAtStartPar
Dienos
\\
\sphinxhline
\sphinxAtStartPar
H
&
\sphinxAtStartPar
Valandos
\\
\sphinxhline
\sphinxAtStartPar
T
&
\sphinxAtStartPar
Minutės
\\
\sphinxhline
\sphinxAtStartPar
S
&
\sphinxAtStartPar
Sekundės
\\
\sphinxhline
\sphinxAtStartPar
L
&
\sphinxAtStartPar
Milisekundės
\\
\sphinxhline
\sphinxAtStartPar
U
&
\sphinxAtStartPar
Mikrosekundės
\\
\sphinxhline
\sphinxAtStartPar
N
&
\sphinxAtStartPar
Nanosekundės
\\
\sphinxbottomrule
\end{tabulary}
\sphinxtableafterendhook\par
\sphinxattableend\end{savenotes}


\subsection{Apibrėžtis erdvėje}
\label{\detokenize{vienetai:apibreztis-erdveje}}
\sphinxAtStartPar
\sphinxcode{\sphinxupquote{geometry}} tipo duomenų laukams gali būti žymimas \DUrole{xref}{\DUrole{std}{\DUrole{std-ref}{erdvinių}}} duomenų tikslumas, pavyzdžiui:


\begin{savenotes}\sphinxattablestart
\sphinxthistablewithglobalstyle
\centering
\begin{tabulary}{\linewidth}[t]{TT}
\sphinxtoprule
\sphinxstyletheadfamily 
\sphinxAtStartPar
Simbolis
&\sphinxstyletheadfamily 
\sphinxAtStartPar
Prasmė
\\
\sphinxmidrule
\sphinxtableatstartofbodyhook
\sphinxAtStartPar
nm
&
\sphinxAtStartPar
Nanometrai (10⁻⁹ m)
\\
\sphinxhline
\sphinxAtStartPar
mm
&
\sphinxAtStartPar
Milimetrai (10⁻³ m)
\\
\sphinxhline
\sphinxAtStartPar
cm
&
\sphinxAtStartPar
Centimetrai (10⁻² m)
\\
\sphinxhline
\sphinxAtStartPar
m
&
\sphinxAtStartPar
Metrai
\\
\sphinxhline
\sphinxAtStartPar
km
&
\sphinxAtStartPar
Kilometrai (10³ m)
\\
\sphinxbottomrule
\end{tabulary}
\sphinxtableafterendhook\par
\sphinxattableend\end{savenotes}


\subsection{Kokybiniai duomenys}
\label{\detokenize{vienetai:kokybiniai-duomenys}}
\sphinxAtStartPar
Kokybiniai duomenys skirstomi į dvi kategorijas:
\begin{itemize}
\item {} 
\sphinxAtStartPar
pavadinimai ir identifikatoriai

\item {} 
\sphinxAtStartPar
kategoriniai duomenys

\end{itemize}

\sphinxAtStartPar
Pavadinimai ir identifikatoriai {\hyperref[\detokenize{dimensijos:property.ref}]{\sphinxcrossref{\sphinxcode{\sphinxupquote{property.ref}}}}} stulpelyje neturi jokio
žymėjimo ir jiems suteikiamas \sphinxcode{\sphinxupquote{4}} brandos lygis.

\sphinxAtStartPar
Kategoriniai duomenys žymimi naudojant papildomą \sphinxcode{\sphinxupquote{enum}} dimensiją, kurioje
išvardinamos visos galimos kategorinių duomenų reikšmės.

\sphinxAtStartPar
Jei kategoriniai duomenys yra palyginami, pavyzdžiui:
\begin{itemize}
\item {} 
\sphinxAtStartPar
puikiai

\item {} 
\sphinxAtStartPar
gerai

\item {} 
\sphinxAtStartPar
vidutiniškai

\item {} 
\sphinxAtStartPar
blogai

\end{itemize}

\sphinxAtStartPar
Tada, tokiems duomenims, turi būti naudojamas \sphinxcode{\sphinxupquote{integer}} tipas, kad nebūtų
prarastos palyginamosios savybės.

\sphinxAtStartPar
Jei duomenys yra nepalyginami, pavyzdžiui:
\begin{itemize}
\item {} 
\sphinxAtStartPar
raudona

\item {} 
\sphinxAtStartPar
geltona

\item {} 
\sphinxAtStartPar
žalia

\item {} 
\sphinxAtStartPar
mėlyna

\end{itemize}

\sphinxAtStartPar
tada nebūtina naudoti \sphinxcode{\sphinxupquote{integer}} tipą.


\subsection{Kiekybiniai duomenys}
\label{\detokenize{vienetai:kiekybiniai-duomenys}}
\sphinxAtStartPar
Matavimo vienetai naudojant \sphinxhref{https://en.wikipedia.org/wiki/International\_System\_of\_Units}{SI simbolius}, \sphinxhref{https://en.wikipedia.org/wiki/SI\_derived\_unit}{išvestinius SI simbolius} ir
\sphinxhref{https://en.wikipedia.org/wiki/Non-SI\_units\_mentioned\_in\_the\_SI}{simbolius patvirtintus naudojimui su SI},  pateikiami {\hyperref[\detokenize{dimensijos:property.ref}]{\sphinxcrossref{\sphinxcode{\sphinxupquote{property.ref}}}}}
stulpelyje.

\sphinxAtStartPar
Pateikus vienetus, laukui gali būti suteikiamas 4\sphinxhyphen{}as brandos lygis.

\sphinxAtStartPar
Pavyzdys:


\begin{savenotes}\sphinxattablestart
\sphinxthistablewithglobalstyle
\centering
\begin{tabulary}{\linewidth}[t]{TTTTTTTT}
\sphinxtoprule
\sphinxstyletheadfamily 
\sphinxAtStartPar
d
&\sphinxstyletheadfamily 
\sphinxAtStartPar
r
&\sphinxstyletheadfamily 
\sphinxAtStartPar
b
&\sphinxstyletheadfamily 
\sphinxAtStartPar
m
&\sphinxstyletheadfamily 
\sphinxAtStartPar
property
&\sphinxstyletheadfamily 
\sphinxAtStartPar
type
&\sphinxstyletheadfamily 
\sphinxAtStartPar
ref
&\sphinxstyletheadfamily 
\sphinxAtStartPar
level
\\
\sphinxmidrule
\sphinxtableatstartofbodyhook\sphinxstartmulticolumn{5}%
\begin{varwidth}[t]{\sphinxcolwidth{5}{8}}
\sphinxAtStartPar
datasets/example
\par
\vskip-\baselineskip\vbox{\hbox{\strut}}\end{varwidth}%
\sphinxstopmulticolumn
&&&\\
\sphinxhline
\sphinxAtStartPar

&&&\sphinxstartmulticolumn{2}%
\begin{varwidth}[t]{\sphinxcolwidth{2}{8}}
\sphinxAtStartPar
Matavimas
\par
\vskip-\baselineskip\vbox{\hbox{\strut}}\end{varwidth}%
\sphinxstopmulticolumn
&&
\sphinxAtStartPar
id
&\\
\sphinxhline
\sphinxAtStartPar

&&&&
\sphinxAtStartPar
id
&
\sphinxAtStartPar
integer
&&
\sphinxAtStartPar
4
\\
\sphinxhline
\sphinxAtStartPar

&&&&
\sphinxAtStartPar
temperatura
&
\sphinxAtStartPar
number
&
\sphinxAtStartPar
°C
&
\sphinxAtStartPar
4
\\
\sphinxhline
\sphinxAtStartPar

&&&&
\sphinxAtStartPar
svorlis
&
\sphinxAtStartPar
number
&
\sphinxAtStartPar
kg
&
\sphinxAtStartPar
4
\\
\sphinxhline
\sphinxAtStartPar

&&&&
\sphinxAtStartPar
plotas
&
\sphinxAtStartPar
number
&
\sphinxAtStartPar
m²
&
\sphinxAtStartPar
4
\\
\sphinxhline
\sphinxAtStartPar

&&&&
\sphinxAtStartPar
turis
&
\sphinxAtStartPar
number
&
\sphinxAtStartPar
m³
&
\sphinxAtStartPar
4
\\
\sphinxhline
\sphinxAtStartPar

&&&&
\sphinxAtStartPar
greitis
&
\sphinxAtStartPar
number
&
\sphinxAtStartPar
km/h
&
\sphinxAtStartPar
4
\\
\sphinxbottomrule
\end{tabulary}
\sphinxtableafterendhook\par
\sphinxattableend\end{savenotes}

\sphinxAtStartPar
Vienetai užrašomi naudojant matematinę notaciją, kurioje galima naudoti
skaičius, daugybos ir dalybos simbolius, kėlimą laipsniu ir atskirų vienetų
sudėtį:


\begin{savenotes}\sphinxattablestart
\sphinxthistablewithglobalstyle
\centering
\begin{tabulary}{\linewidth}[t]{TT}
\sphinxtoprule
\sphinxstyletheadfamily 
\sphinxAtStartPar
Žymėjimas
&\sphinxstyletheadfamily 
\sphinxAtStartPar
Reikšmė
\\
\sphinxmidrule
\sphinxtableatstartofbodyhook
\sphinxAtStartPar
⋅ · *
&
\sphinxAtStartPar
Daugyba
\\
\sphinxhline
\sphinxAtStartPar
/
&
\sphinxAtStartPar
Dalyba
\\
\sphinxhline
\sphinxAtStartPar
(tarpas)
&
\sphinxAtStartPar
Sudėtis
\\
\sphinxhline
\sphinxAtStartPar
\textasciicircum{}$^{\text{(+\sphinxhyphen{})(skaičius)}}$ arba ⁺ ⁻ ⁰ ¹ ² ³ ⁴ ⁵ ⁶ ⁷ ⁸ ⁹
&
\sphinxAtStartPar
Kėlimas laipsniu
\\
\sphinxbottomrule
\end{tabulary}
\sphinxtableafterendhook\par
\sphinxattableend\end{savenotes}

\sphinxAtStartPar
Pavyzdžiai:
\begin{quote}

\begin{DUlineblock}{0em}
\item[] m
\item[] 1m
\item[] 10m
\item[] m\textasciicircum{}2
\item[] m²
\item[] km¹⁰
\item[] kg⋅m²⋅s⁻³⋅A⁻¹
\item[] kg*m\textasciicircum{}2*s\textasciicircum{}\sphinxhyphen{}3⋅A\textasciicircum{}\sphinxhyphen{}1
\item[] 8kg⋅m²⋅s⁻³⋅A⁻¹
\item[] mg/l
\item[] g/m\textasciicircum{}2
\item[] mg/m\textasciicircum{}3
\item[] mm
\item[] U/m\textasciicircum{}2
\item[] U/m\textasciicircum{}3
\item[] \%
\item[] ha
\item[] min
\item[] h
\item[] bar
\item[] U
\item[] 10\textasciicircum{}6s
\item[] 10⁶s
\item[] μ/m³
\item[] yr
\item[] 3mo
\item[] yr 2mo 4wk
\item[] °C
\item[] °
\end{DUlineblock}
\end{quote}


\subsubsection{Prefiksai}
\label{\detokenize{vienetai:prefiksai}}
\sphinxAtStartPar
Kiekybiniai matavimo vienetai gali turėti tokius prefiksus:


\begin{savenotes}\sphinxattablestart
\sphinxthistablewithglobalstyle
\centering
\begin{tabulary}{\linewidth}[t]{TTT}
\sphinxtoprule
\sphinxstyletheadfamily 
\sphinxAtStartPar
Žymėjimas
&\sphinxstyletheadfamily 
\sphinxAtStartPar
10$^{\text{n}}$
&\sphinxstyletheadfamily 
\sphinxAtStartPar
Priešdėlis
\\
\sphinxmidrule
\sphinxtableatstartofbodyhook
\sphinxAtStartPar
Y
&
\sphinxAtStartPar
10$^{\text{24}}$
&
\sphinxAtStartPar
yotta
\\
\sphinxhline
\sphinxAtStartPar
Z
&
\sphinxAtStartPar
10$^{\text{21}}$
&
\sphinxAtStartPar
zetta
\\
\sphinxhline
\sphinxAtStartPar
E
&
\sphinxAtStartPar
10$^{\text{18}}$
&
\sphinxAtStartPar
exa
\\
\sphinxhline
\sphinxAtStartPar
P
&
\sphinxAtStartPar
10$^{\text{15}}$
&
\sphinxAtStartPar
peta
\\
\sphinxhline
\sphinxAtStartPar
T
&
\sphinxAtStartPar
10$^{\text{12}}$
&
\sphinxAtStartPar
tera
\\
\sphinxhline
\sphinxAtStartPar
G
&
\sphinxAtStartPar
10$^{\text{9}}$
&
\sphinxAtStartPar
giga
\\
\sphinxhline
\sphinxAtStartPar
M
&
\sphinxAtStartPar
10$^{\text{6}}$
&
\sphinxAtStartPar
mega
\\
\sphinxhline
\sphinxAtStartPar
k
&
\sphinxAtStartPar
10$^{\text{3}}$
&
\sphinxAtStartPar
kilo
\\
\sphinxhline
\sphinxAtStartPar
h
&
\sphinxAtStartPar
10$^{\text{2}}$
&
\sphinxAtStartPar
hecto
\\
\sphinxhline
\sphinxAtStartPar
da
&
\sphinxAtStartPar
10$^{\text{1}}$
&
\sphinxAtStartPar
deca
\\
\sphinxhline
\sphinxAtStartPar
d
&
\sphinxAtStartPar
10$^{\text{\sphinxhyphen{}1}}$
&
\sphinxAtStartPar
deci
\\
\sphinxhline
\sphinxAtStartPar
c
&
\sphinxAtStartPar
10$^{\text{\sphinxhyphen{}2}}$
&
\sphinxAtStartPar
centi
\\
\sphinxhline
\sphinxAtStartPar
m
&
\sphinxAtStartPar
10$^{\text{\sphinxhyphen{}3}}$
&
\sphinxAtStartPar
milli
\\
\sphinxhline
\sphinxAtStartPar
µ
&
\sphinxAtStartPar
10$^{\text{\sphinxhyphen{}6}}$
&
\sphinxAtStartPar
micro
\\
\sphinxhline
\sphinxAtStartPar
n
&
\sphinxAtStartPar
10$^{\text{\sphinxhyphen{}9}}$
&
\sphinxAtStartPar
nano
\\
\sphinxhline
\sphinxAtStartPar
p
&
\sphinxAtStartPar
10$^{\text{\sphinxhyphen{}12}}$
&
\sphinxAtStartPar
pico
\\
\sphinxhline
\sphinxAtStartPar
f
&
\sphinxAtStartPar
10$^{\text{\sphinxhyphen{}15}}$
&
\sphinxAtStartPar
femto
\\
\sphinxhline
\sphinxAtStartPar
a
&
\sphinxAtStartPar
10$^{\text{\sphinxhyphen{}18}}$
&
\sphinxAtStartPar
atto
\\
\sphinxhline
\sphinxAtStartPar
z
&
\sphinxAtStartPar
10$^{\text{\sphinxhyphen{}21}}$
&
\sphinxAtStartPar
zepto
\\
\sphinxhline
\sphinxAtStartPar
y
&
\sphinxAtStartPar
10$^{\text{\sphinxhyphen{}24}}$
&
\sphinxAtStartPar
yocto
\\
\sphinxbottomrule
\end{tabulary}
\sphinxtableafterendhook\par
\sphinxattableend\end{savenotes}


\subsubsection{Vienetai}
\label{\detokenize{vienetai:vienetai}}

\paragraph{Specialiejie vienetai}
\label{\detokenize{vienetai:specialiejie-vienetai}}

\begin{savenotes}\sphinxattablestart
\sphinxthistablewithglobalstyle
\centering
\begin{tabulary}{\linewidth}[t]{TT}
\sphinxtoprule
\sphinxstyletheadfamily 
\sphinxAtStartPar
Žymėjimas
&\sphinxstyletheadfamily 
\sphinxAtStartPar
Pavadinimas
\\
\sphinxmidrule
\sphinxtableatstartofbodyhook
\sphinxAtStartPar
U
&
\sphinxAtStartPar
vienetai (keikis vienetais)
\\
\sphinxhline
\sphinxAtStartPar
\%
&
\sphinxAtStartPar
procentai
\\
\sphinxbottomrule
\end{tabulary}
\sphinxtableafterendhook\par
\sphinxattableend\end{savenotes}


\paragraph{Laiko vienetai}
\label{\detokenize{vienetai:laiko-vienetai}}
\sphinxAtStartPar
Naudojami tik tais atvejais, kai matuojamas laiko kiekis, o ne data ir laikas.
Datos ir laiko (\sphinxcode{\sphinxupquote{date}} ir \sphinxcode{\sphinxupquote{datetime}} tipai) tikslumui žymėti, naudojamos kitos
žymės.


\begin{savenotes}\sphinxattablestart
\sphinxthistablewithglobalstyle
\centering
\begin{tabulary}{\linewidth}[t]{TT}
\sphinxtoprule
\sphinxstyletheadfamily 
\sphinxAtStartPar
Žymėjimas
&\sphinxstyletheadfamily 
\sphinxAtStartPar
Pavadinimas
\\
\sphinxmidrule
\sphinxtableatstartofbodyhook
\sphinxAtStartPar
s
&
\sphinxAtStartPar
sekundė
\\
\sphinxhline
\sphinxAtStartPar
min
&
\sphinxAtStartPar
minutė
\\
\sphinxhline
\sphinxAtStartPar
h
&
\sphinxAtStartPar
valanda
\\
\sphinxhline
\sphinxAtStartPar
d
&
\sphinxAtStartPar
diena (24 valandos)
\\
\sphinxhline
\sphinxAtStartPar
wk
&
\sphinxAtStartPar
savaitė (7 dienos)
\\
\sphinxhline
\sphinxAtStartPar
mo
&
\sphinxAtStartPar
mėnuo (28\sphinxhyphen{}31 diena arba 4 savaitės)
\\
\sphinxhline
\sphinxAtStartPar
yr
&
\sphinxAtStartPar
metai (354.37 dienos arba 12 mėnesių)
\\
\sphinxbottomrule
\end{tabulary}
\sphinxtableafterendhook\par
\sphinxattableend\end{savenotes}


\paragraph{SI Baziniai vienetai}
\label{\detokenize{vienetai:si-baziniai-vienetai}}

\begin{savenotes}\sphinxattablestart
\sphinxthistablewithglobalstyle
\centering
\begin{tabulary}{\linewidth}[t]{TT}
\sphinxtoprule
\sphinxstyletheadfamily 
\sphinxAtStartPar
Žymėjimas
&\sphinxstyletheadfamily 
\sphinxAtStartPar
Pavadinimas
\\
\sphinxmidrule
\sphinxtableatstartofbodyhook
\sphinxAtStartPar
m
&
\sphinxAtStartPar
metre
\\
\sphinxhline
\sphinxAtStartPar
g
&
\sphinxAtStartPar
gram
\\
\sphinxhline
\sphinxAtStartPar
s
&
\sphinxAtStartPar
second
\\
\sphinxhline
\sphinxAtStartPar
A
&
\sphinxAtStartPar
ampere
\\
\sphinxhline
\sphinxAtStartPar
K
&
\sphinxAtStartPar
kelvin
\\
\sphinxhline
\sphinxAtStartPar
mol
&
\sphinxAtStartPar
mole
\\
\sphinxhline
\sphinxAtStartPar
cd
&
\sphinxAtStartPar
candela
\\
\sphinxbottomrule
\end{tabulary}
\sphinxtableafterendhook\par
\sphinxattableend\end{savenotes}


\paragraph{SI Išvestiniai vienetai}
\label{\detokenize{vienetai:si-isvestiniai-vienetai}}

\begin{savenotes}\sphinxattablestart
\sphinxthistablewithglobalstyle
\centering
\begin{tabulary}{\linewidth}[t]{TT}
\sphinxtoprule
\sphinxstyletheadfamily 
\sphinxAtStartPar
Žymėjimas
&\sphinxstyletheadfamily 
\sphinxAtStartPar
Pavadinimas
\\
\sphinxmidrule
\sphinxtableatstartofbodyhook
\sphinxAtStartPar
Hz
&
\sphinxAtStartPar
hertz
\\
\sphinxhline
\sphinxAtStartPar
rad
&
\sphinxAtStartPar
radian
\\
\sphinxhline
\sphinxAtStartPar
sr
&
\sphinxAtStartPar
steradian
\\
\sphinxhline
\sphinxAtStartPar
N
&
\sphinxAtStartPar
newton
\\
\sphinxhline
\sphinxAtStartPar
Pa
&
\sphinxAtStartPar
pascal
\\
\sphinxhline
\sphinxAtStartPar
J
&
\sphinxAtStartPar
joule
\\
\sphinxhline
\sphinxAtStartPar
W
&
\sphinxAtStartPar
watt
\\
\sphinxhline
\sphinxAtStartPar
C
&
\sphinxAtStartPar
coulomb
\\
\sphinxhline
\sphinxAtStartPar
V
&
\sphinxAtStartPar
volt
\\
\sphinxhline
\sphinxAtStartPar
F
&
\sphinxAtStartPar
farad
\\
\sphinxhline
\sphinxAtStartPar
Ω
&
\sphinxAtStartPar
ohm
\\
\sphinxhline
\sphinxAtStartPar
S
&
\sphinxAtStartPar
siemens
\\
\sphinxhline
\sphinxAtStartPar
Wb
&
\sphinxAtStartPar
weber
\\
\sphinxhline
\sphinxAtStartPar
T
&
\sphinxAtStartPar
tesla
\\
\sphinxhline
\sphinxAtStartPar
H
&
\sphinxAtStartPar
henry
\\
\sphinxhline
\sphinxAtStartPar
°C
&
\sphinxAtStartPar
degree Celsius
\\
\sphinxhline
\sphinxAtStartPar
lm
&
\sphinxAtStartPar
lumen
\\
\sphinxhline
\sphinxAtStartPar
lx
&
\sphinxAtStartPar
lux
\\
\sphinxhline
\sphinxAtStartPar
Bq
&
\sphinxAtStartPar
becquerel
\\
\sphinxhline
\sphinxAtStartPar
Gy
&
\sphinxAtStartPar
gray
\\
\sphinxhline
\sphinxAtStartPar
Sv
&
\sphinxAtStartPar
sievert
\\
\sphinxhline
\sphinxAtStartPar
kat
&
\sphinxAtStartPar
katal
\\
\sphinxbottomrule
\end{tabulary}
\sphinxtableafterendhook\par
\sphinxattableend\end{savenotes}


\paragraph{Kiti vienetai}
\label{\detokenize{vienetai:kiti-vienetai}}

\begin{savenotes}
\sphinxatlongtablestart
\sphinxthistablewithglobalstyle
\makeatletter
  \LTleft \@totalleftmargin plus1fill
  \LTright\dimexpr\columnwidth-\@totalleftmargin-\linewidth\relax plus1fill
\makeatother
\begin{longtable}{ll}
\sphinxtoprule
\sphinxstyletheadfamily 
\sphinxAtStartPar
Žymėjimas
&\sphinxstyletheadfamily 
\sphinxAtStartPar
Pavadinimas
\\
\sphinxmidrule
\endfirsthead

\multicolumn{2}{c}{\sphinxnorowcolor
    \makebox[0pt]{\sphinxtablecontinued{\tablename\ \thetable{} \textendash{} tęsinys iš praeito puslapio}}%
}\\
\sphinxtoprule
\sphinxstyletheadfamily 
\sphinxAtStartPar
Žymėjimas
&\sphinxstyletheadfamily 
\sphinxAtStartPar
Pavadinimas
\\
\sphinxmidrule
\endhead

\sphinxbottomrule
\multicolumn{2}{r}{\sphinxnorowcolor
    \makebox[0pt][r]{\sphinxtablecontinued{continues on next page}}%
}\\
\endfoot

\endlastfoot
\sphinxtableatstartofbodyhook

\sphinxAtStartPar
au
&
\sphinxAtStartPar
astronomical unit
\\
\sphinxhline
\sphinxAtStartPar
°
&
\sphinxAtStartPar
degree
\\
\sphinxhline
\sphinxAtStartPar
′
&
\sphinxAtStartPar
arcminute
\\
\sphinxhline
\sphinxAtStartPar
″
&
\sphinxAtStartPar
arcsecond
\\
\sphinxhline
\sphinxAtStartPar
ha
&
\sphinxAtStartPar
hectare
\\
\sphinxhline
\sphinxAtStartPar
l
&
\sphinxAtStartPar
litre
\\
\sphinxhline
\sphinxAtStartPar
L
&
\sphinxAtStartPar
litre
\\
\sphinxhline
\sphinxAtStartPar
t
&
\sphinxAtStartPar
tonne
\\
\sphinxhline
\sphinxAtStartPar
Da
&
\sphinxAtStartPar
dalton
\\
\sphinxhline
\sphinxAtStartPar
eV
&
\sphinxAtStartPar
electronvolt
\\
\sphinxhline
\sphinxAtStartPar
Np
&
\sphinxAtStartPar
neper
\\
\sphinxhline
\sphinxAtStartPar
B
&
\sphinxAtStartPar
bel
\\
\sphinxhline
\sphinxAtStartPar
dB
&
\sphinxAtStartPar
decibel
\\
\sphinxhline
\sphinxAtStartPar
Gal
&
\sphinxAtStartPar
gal (acceleration)
\\
\sphinxhline
\sphinxAtStartPar
u
&
\sphinxAtStartPar
unified atomic mass unit
\\
\sphinxhline
\sphinxAtStartPar
var
&
\sphinxAtStartPar
volt\sphinxhyphen{}ampere reactive
\\
\sphinxhline
\sphinxAtStartPar
pc
&
\sphinxAtStartPar
parsec
\\
\sphinxhline
\sphinxAtStartPar
c₀ arba c\_0
&
\sphinxAtStartPar
natural unit of speed
\\
\sphinxhline
\sphinxAtStartPar
ħ
&
\sphinxAtStartPar
natural unit of action
\\
\sphinxhline
\sphinxAtStartPar
mₑ arba m\_e
&
\sphinxAtStartPar
natural unit of mass
\\
\sphinxhline
\sphinxAtStartPar
e
&
\sphinxAtStartPar
atomic unit of charge
\\
\sphinxhline
\sphinxAtStartPar
a₀ arba a\_0
&
\sphinxAtStartPar
atomic unit of length
\\
\sphinxhline
\sphinxAtStartPar
E\_h
&
\sphinxAtStartPar
atomic unit of energy
\\
\sphinxhline
\sphinxAtStartPar
M
&
\sphinxAtStartPar
nautical mile
\\
\sphinxhline
\sphinxAtStartPar
kn
&
\sphinxAtStartPar
knot
\\
\sphinxhline
\sphinxAtStartPar
Å
&
\sphinxAtStartPar
ångström
\\
\sphinxhline
\sphinxAtStartPar
a
&
\sphinxAtStartPar
are
\\
\sphinxhline
\sphinxAtStartPar
b
&
\sphinxAtStartPar
barn
\\
\sphinxhline
\sphinxAtStartPar
bar
&
\sphinxAtStartPar
bar
\\
\sphinxhline
\sphinxAtStartPar
atm
&
\sphinxAtStartPar
standard atmosphere
\\
\sphinxhline
\sphinxAtStartPar
Ci
&
\sphinxAtStartPar
curie
\\
\sphinxhline
\sphinxAtStartPar
R
&
\sphinxAtStartPar
roentgen
\\
\sphinxhline
\sphinxAtStartPar
rem
&
\sphinxAtStartPar
rem
\\
\sphinxhline
\sphinxAtStartPar
erg
&
\sphinxAtStartPar
erg
\\
\sphinxhline
\sphinxAtStartPar
dyn
&
\sphinxAtStartPar
dyne
\\
\sphinxhline
\sphinxAtStartPar
P
&
\sphinxAtStartPar
poise
\\
\sphinxhline
\sphinxAtStartPar
st
&
\sphinxAtStartPar
stokes
\\
\sphinxhline
\sphinxAtStartPar
Mx
&
\sphinxAtStartPar
maxwell
\\
\sphinxhline
\sphinxAtStartPar
G
&
\sphinxAtStartPar
gauss
\\
\sphinxhline
\sphinxAtStartPar
Oe
&
\sphinxAtStartPar
ørsted
\\
\sphinxhline
\sphinxAtStartPar
sb
&
\sphinxAtStartPar
stilb
\\
\sphinxhline
\sphinxAtStartPar
ph
&
\sphinxAtStartPar
phot
\\
\sphinxhline
\sphinxAtStartPar
Torr
&
\sphinxAtStartPar
torr
\\
\sphinxhline
\sphinxAtStartPar
kgf
&
\sphinxAtStartPar
kilogram\sphinxhyphen{}force
\\
\sphinxhline
\sphinxAtStartPar
cal
&
\sphinxAtStartPar
calorie
\\
\sphinxhline
\sphinxAtStartPar
μ
&
\sphinxAtStartPar
micron
\\
\sphinxhline
\sphinxAtStartPar
xu
&
\sphinxAtStartPar
x\sphinxhyphen{}unit
\\
\sphinxhline
\sphinxAtStartPar
γ
&
\sphinxAtStartPar
gamma (mass, magnetic flux density)
\\
\sphinxhline
\sphinxAtStartPar
λ
&
\sphinxAtStartPar
lambda
\\
\sphinxhline
\sphinxAtStartPar
Jy
&
\sphinxAtStartPar
jansky
\\
\sphinxhline
\sphinxAtStartPar
mmHg
&
\sphinxAtStartPar
millimetre of mercury
\\
\sphinxbottomrule
\end{longtable}
\sphinxtableafterendhook
\sphinxatlongtableend
\end{savenotes}

\sphinxstepscope


\section{Apibendrinimas}
\label{\detokenize{apibendrinimas:apibendrinimas}}\label{\detokenize{apibendrinimas:generalization}}\label{\detokenize{apibendrinimas::doc}}
\sphinxAtStartPar
Duomenų apibendrinimas, generalizavimas arba paveldimumas yra būdas nurodyti,
kad keli skirtingi duomenų modeliai priklauso tai pačiai vienai koncepcinio
modelio klasei.


\begin{sphinxseealso}{Taip pat žiūrėkite:}
\begin{itemize}
\item {} 
\sphinxAtStartPar
{\hyperref[\detokenize{modelis:uml-index}]{\sphinxcrossref{\DUrole{std}{\DUrole{std-ref}{Koncepcinis modelis}}}}} / {\hyperref[\detokenize{modelis:uml-generalization}]{\sphinxcrossref{\DUrole{std}{\DUrole{std-ref}{Apibendrinimas}}}}}

\item {} 
\sphinxAtStartPar
{\hyperref[\detokenize{dimensijos:dimensijos}]{\sphinxcrossref{\DUrole{std}{\DUrole{std-ref}{Dimensijos}}}}} / {\hyperref[\detokenize{dimensijos:base}]{\sphinxcrossref{\DUrole{std}{\DUrole{std-ref}{base}}}}}

\end{itemize}


\end{sphinxseealso}


\sphinxstepscope


\section{Asociacija}
\label{\detokenize{identifikatoriai:asociacija}}\label{\detokenize{identifikatoriai:rysiai}}\label{\detokenize{identifikatoriai::doc}}
\sphinxAtStartPar
Pateikiant metaduomenis apie ryšius tarp modelių, duomenų {\hyperref[\detokenize{branda:level}]{\sphinxcrossref{\DUrole{std}{\DUrole{std-ref}{brandos lygis}}}}} pakeliamas iki ketvirto lygio.

\sphinxAtStartPar
Ryšiai tarp modelių aprašomi tais atvejais, kai vienoje duomenų lentelėje
naudojami identifikatoriai iš kitos lentelės.


\subsection{Kompozicija}
\label{\detokenize{identifikatoriai:kompozicija}}\label{\detokenize{identifikatoriai:composition}}
\sphinxAtStartPar
Kompozicija arba duomenų normalizavimas yra duomenų modeliavimo principas, kai
atskiros duomenų klasės yra pateikiamos kaip atskiri duomenų modeliai, kurie
gali būti jungiami tarpusavyje, taikant įvairius {\hyperref[\detokenize{identifikatoriai:ref-types}]{\sphinxcrossref{\DUrole{std}{\DUrole{std-ref}{duomenų jungimo būdus}}}}}.

\sphinxAtStartPar
Pavyzdžiui turint tokią koncepcinio modelio klasių diagramą:

\sphinxAtStartPar
Duomenų modelis, taikant kompoziciją atrodytų taip:


\begin{savenotes}\sphinxattablestart
\sphinxthistablewithglobalstyle
\centering
\sphinxcapstartof{table}
\sphinxthecaptionisattop
\sphinxcaption{Kompozicijos pavyzdys}\label{\detokenize{identifikatoriai:id2}}\label{\detokenize{identifikatoriai:norm-dsa-example-1}}
\sphinxaftertopcaption
\begin{tabulary}{\linewidth}[t]{TTTTTT}
\sphinxtoprule
\sphinxstyletheadfamily 
\sphinxAtStartPar
dataset
&\sphinxstyletheadfamily 
\sphinxAtStartPar
model
&\sphinxstyletheadfamily 
\sphinxAtStartPar
property
&\sphinxstyletheadfamily 
\sphinxAtStartPar
type
&\sphinxstyletheadfamily 
\sphinxAtStartPar
ref
&\sphinxstyletheadfamily 
\sphinxAtStartPar
level
\\
\sphinxmidrule
\sphinxtableatstartofbodyhook\sphinxstartmulticolumn{3}%
\begin{varwidth}[t]{\sphinxcolwidth{3}{6}}
\sphinxAtStartPar
datasets/gov/rc/ar/ws
\par
\vskip-\baselineskip\vbox{\hbox{\strut}}\end{varwidth}%
\sphinxstopmulticolumn
&&&\\
\sphinxhline
\sphinxAtStartPar

&\sphinxstartmulticolumn{2}%
\begin{varwidth}[t]{\sphinxcolwidth{2}{6}}
\sphinxAtStartPar
Country
\par
\vskip-\baselineskip\vbox{\hbox{\strut}}\end{varwidth}%
\sphinxstopmulticolumn
&&
\sphinxAtStartPar
code
&
\sphinxAtStartPar
4
\\
\sphinxhline
\sphinxAtStartPar

&&
\sphinxAtStartPar
code
&
\sphinxAtStartPar
string
&&
\sphinxAtStartPar
4
\\
\sphinxhline
\sphinxAtStartPar

&&
\sphinxAtStartPar
name@en
&
\sphinxAtStartPar
string
&&
\sphinxAtStartPar
4
\\
\sphinxhline
\sphinxAtStartPar

&\sphinxstartmulticolumn{2}%
\begin{varwidth}[t]{\sphinxcolwidth{2}{6}}
\sphinxAtStartPar
City
\par
\vskip-\baselineskip\vbox{\hbox{\strut}}\end{varwidth}%
\sphinxstopmulticolumn
&&&
\sphinxAtStartPar
3
\\
\sphinxhline
\sphinxAtStartPar

&&
\sphinxAtStartPar
name@en
&
\sphinxAtStartPar
string
&&
\sphinxAtStartPar
4
\\
\sphinxhline
\sphinxAtStartPar

&&
\sphinxAtStartPar
country
&
\sphinxAtStartPar
ref
&
\sphinxAtStartPar
Country
&
\sphinxAtStartPar
4
\\
\sphinxbottomrule
\end{tabulary}
\sphinxtableafterendhook\par
\sphinxattableend\end{savenotes}

\sphinxAtStartPar
Taikant kompozicijos principą, kiekvienai klasei kuriamas atskiras
nepriklausomas duomenų modelis, o atskiri modeliai jungiami tarpusavyje
ryšiais.


\subsection{Jungtinis modelis}
\label{\detokenize{identifikatoriai:jungtinis-modelis}}\label{\detokenize{identifikatoriai:ref-denorm}}
\sphinxAtStartPar
Jungtinis modelis yra toks duomenų modelis, kuriame apjungiamos savybės iš
daugiau nei vienos koncepcinio modelio klasės. Jungtinis modelis yra priešingas
dalykas {\hyperref[\detokenize{identifikatoriai:composition}]{\sphinxcrossref{\DUrole{std}{\DUrole{std-ref}{kompozicijai}}}}}, kur kompozicija išskaido klases į
atskirus modelius, jungtinis modelis atvirkščiai apjungia klases į vieną
modelį.

\sphinxAtStartPar
Jungtinis modelis dar yra vadinamas agregatu arba denormalizuotu duomenų
modeliu.

\sphinxAtStartPar
Jungtinis modelis turi vieną šakninį modelį (angl. \sphinxstyleemphasis{aggregate root}), prie
kurio prijungiami kiti modeliai naudojant vieną iš {\hyperref[\detokenize{identifikatoriai:ref-types}]{\sphinxcrossref{\DUrole{std}{\DUrole{std-ref}{duomenų jungimo būdų}}}}}.

\sphinxAtStartPar
Žiamiau pateikiamas jungtinio modelio pavyzdys.


\begin{savenotes}\sphinxattablestart
\sphinxthistablewithglobalstyle
\centering
\sphinxcapstartof{table}
\sphinxthecaptionisattop
\sphinxcaption{Jungtinio modelio pavyzdys}\label{\detokenize{identifikatoriai:id3}}\label{\detokenize{identifikatoriai:denorm-dsa-example-1}}
\sphinxaftertopcaption
\begin{tabulary}{\linewidth}[t]{TTTTTT}
\sphinxtoprule
\sphinxstyletheadfamily 
\sphinxAtStartPar
dataset
&\sphinxstyletheadfamily 
\sphinxAtStartPar
model
&\sphinxstyletheadfamily 
\sphinxAtStartPar
property
&\sphinxstyletheadfamily 
\sphinxAtStartPar
type
&\sphinxstyletheadfamily 
\sphinxAtStartPar
ref
&\sphinxstyletheadfamily 
\sphinxAtStartPar
level
\\
\sphinxmidrule
\sphinxtableatstartofbodyhook\sphinxstartmulticolumn{3}%
\begin{varwidth}[t]{\sphinxcolwidth{3}{6}}
\sphinxAtStartPar
datasets/gov/rc/ar/ws
\par
\vskip-\baselineskip\vbox{\hbox{\strut}}\end{varwidth}%
\sphinxstopmulticolumn
&&&\\
\sphinxhline
\sphinxAtStartPar

&\sphinxstartmulticolumn{2}%
\begin{varwidth}[t]{\sphinxcolwidth{2}{6}}
\sphinxAtStartPar
Country
\par
\vskip-\baselineskip\vbox{\hbox{\strut}}\end{varwidth}%
\sphinxstopmulticolumn
&&
\sphinxAtStartPar
code
&
\sphinxAtStartPar
4
\\
\sphinxhline
\sphinxAtStartPar

&&
\sphinxAtStartPar
code
&
\sphinxAtStartPar
string
&&
\sphinxAtStartPar
4
\\
\sphinxhline
\sphinxAtStartPar

&&
\sphinxAtStartPar
name@en
&
\sphinxAtStartPar
string
&&
\sphinxAtStartPar
4
\\
\sphinxhline
\sphinxAtStartPar

&\sphinxstartmulticolumn{2}%
\begin{varwidth}[t]{\sphinxcolwidth{2}{6}}
\sphinxAtStartPar
City
\par
\vskip-\baselineskip\vbox{\hbox{\strut}}\end{varwidth}%
\sphinxstopmulticolumn
&&&
\sphinxAtStartPar
3
\\
\sphinxhline
\sphinxAtStartPar

&&
\sphinxAtStartPar
name@en
&
\sphinxAtStartPar
string
&&
\sphinxAtStartPar
4
\\
\sphinxhline
\sphinxAtStartPar

&&
\sphinxAtStartPar
country
&
\sphinxAtStartPar
ref
&
\sphinxAtStartPar
Country
&
\sphinxAtStartPar
4
\\
\sphinxhline
\sphinxAtStartPar

&&
\sphinxAtStartPar
country.code
&&&
\sphinxAtStartPar
4
\\
\sphinxhline
\sphinxAtStartPar

&&
\sphinxAtStartPar
country.name@en
&&&
\sphinxAtStartPar
4
\\
\sphinxhline
\sphinxAtStartPar

&&
\sphinxAtStartPar
country.name@lt
&
\sphinxAtStartPar
string
&&
\sphinxAtStartPar
4
\\
\sphinxbottomrule
\end{tabulary}
\sphinxtableafterendhook\par
\sphinxattableend\end{savenotes}

\sphinxAtStartPar
Pavyzdyje \sphinxcode{\sphinxupquote{City}} yra jungtinis modelis, kadangi \sphinxcode{\sphinxupquote{City}} išvardintos ne tik
\sphinxcode{\sphinxupquote{City}} klasei priklausančios savybės, tačiau įtraukiamos ir kitos klasės
\sphinxcode{\sphinxupquote{Country}} savybės.

\sphinxAtStartPar
\sphinxcode{\sphinxupquote{City}} šiame pavyzdyje yra šakninis modelis, o \sphinxcode{\sphinxupquote{Country}} yra modelis,
prijungtas per \sphinxcode{\sphinxupquote{City/country}} savybę.


\begin{sphinxseealso}{Taip pat žiūrėkite:}

\begin{DUlineblock}{0em}
\item[] {\hyperref[\detokenize{modeliai/funkciniai:func-model-part}]{\sphinxcrossref{\DUrole{std}{\DUrole{std-ref}{Daliniai modeliai}}}}} (\sphinxcode{\sphinxupquote{/:part}})
\end{DUlineblock}


\end{sphinxseealso}



\subsubsection{Savybių įtraukimas}
\label{\detokenize{identifikatoriai:savybiu-itraukimas}}\label{\detokenize{identifikatoriai:prop-expand}}
\sphinxAtStartPar
Kad jungtiniame modelyje nereikėtu kartoti prijungiamo modelio savybių, galima
pateikti \sphinxcode{\sphinxupquote{expand()}} funkciją {\hyperref[\detokenize{dimensijos:model.prepare}]{\sphinxcrossref{\sphinxcode{\sphinxupquote{model.prepare}}}}} arba
{\hyperref[\detokenize{dimensijos:property.prepare}]{\sphinxcrossref{\sphinxcode{\sphinxupquote{property.prepare}}}}} stulpeliuose.

\sphinxAtStartPar
Perrašant {\hyperref[\detokenize{identifikatoriai:denorm-dsa-example-1}]{\sphinxcrossref{\DUrole{std}{\DUrole{std-ref}{Jungtinio modelio pavyzdys}}}}} lentelę su \sphinxcode{\sphinxupquote{expand()}}, gautume, tokią
trumpesnę struktūros aprašo lentelę:


\begin{savenotes}\sphinxattablestart
\sphinxthistablewithglobalstyle
\centering
\sphinxcapstartof{table}
\sphinxthecaptionisattop
\sphinxcaption{Jungtinio modelio pavyzdys su expand()}\label{\detokenize{identifikatoriai:id4}}
\sphinxaftertopcaption
\begin{tabulary}{\linewidth}[t]{TTTTTTT}
\sphinxtoprule
\sphinxstyletheadfamily 
\sphinxAtStartPar
dataset
&\sphinxstyletheadfamily 
\sphinxAtStartPar
model
&\sphinxstyletheadfamily 
\sphinxAtStartPar
property
&\sphinxstyletheadfamily 
\sphinxAtStartPar
type
&\sphinxstyletheadfamily 
\sphinxAtStartPar
ref
&\sphinxstyletheadfamily 
\sphinxAtStartPar
prepare
&\sphinxstyletheadfamily 
\sphinxAtStartPar
level
\\
\sphinxmidrule
\sphinxtableatstartofbodyhook\sphinxstartmulticolumn{3}%
\begin{varwidth}[t]{\sphinxcolwidth{3}{7}}
\sphinxAtStartPar
datasets/gov/rc/ar/ws
\par
\vskip-\baselineskip\vbox{\hbox{\strut}}\end{varwidth}%
\sphinxstopmulticolumn
&&&&\\
\sphinxhline
\sphinxAtStartPar

&\sphinxstartmulticolumn{2}%
\begin{varwidth}[t]{\sphinxcolwidth{2}{7}}
\sphinxAtStartPar
Country
\par
\vskip-\baselineskip\vbox{\hbox{\strut}}\end{varwidth}%
\sphinxstopmulticolumn
&&
\sphinxAtStartPar
code
&&
\sphinxAtStartPar
4
\\
\sphinxhline
\sphinxAtStartPar

&&
\sphinxAtStartPar
code
&
\sphinxAtStartPar
string
&&&
\sphinxAtStartPar
4
\\
\sphinxhline
\sphinxAtStartPar

&&
\sphinxAtStartPar
name@en
&
\sphinxAtStartPar
string
&&&
\sphinxAtStartPar
4
\\
\sphinxhline
\sphinxAtStartPar

&\sphinxstartmulticolumn{2}%
\begin{varwidth}[t]{\sphinxcolwidth{2}{7}}
\sphinxAtStartPar
City
\par
\vskip-\baselineskip\vbox{\hbox{\strut}}\end{varwidth}%
\sphinxstopmulticolumn
&&&&
\sphinxAtStartPar
3
\\
\sphinxhline
\sphinxAtStartPar

&&
\sphinxAtStartPar
name@en
&
\sphinxAtStartPar
string
&&&
\sphinxAtStartPar
4
\\
\sphinxhline
\sphinxAtStartPar

&&
\sphinxAtStartPar
country
&
\sphinxAtStartPar
ref
&
\sphinxAtStartPar
Country
&
\sphinxAtStartPar
expand()
&
\sphinxAtStartPar
4
\\
\sphinxhline
\sphinxAtStartPar

&&
\sphinxAtStartPar
country.name@lt
&
\sphinxAtStartPar
string
&&&
\sphinxAtStartPar
4
\\
\sphinxbottomrule
\end{tabulary}
\sphinxtableafterendhook\par
\sphinxattableend\end{savenotes}

\sphinxAtStartPar
Ši lentelė yra lygiai tokia pati kaip ir {\hyperref[\detokenize{identifikatoriai:denorm-dsa-example-1}]{\sphinxcrossref{\DUrole{std}{\DUrole{std-ref}{Jungtinio modelio pavyzdys}}}}}.

\sphinxAtStartPar
Prie \sphinxcode{\sphinxupquote{City/country}} nurodyta \sphinxcode{\sphinxupquote{expand()}} funkcija į \sphinxcode{\sphinxupquote{City}} modelį įtraukia
visas \sphinxcode{\sphinxupquote{Country}} savybes.

\sphinxAtStartPar
Jei norima įtraukti ne visas \sphinxcode{\sphinxupquote{Country}} savybes, reikia naudoti \sphinxcode{\sphinxupquote{include()}}
funkciją, pateikiant sąrašą savybių, kurias norima įtraukti, pavyzdžiui
\sphinxcode{\sphinxupquote{include(code)}} \sphinxhyphen{} bus įtraukta tik viena \sphinxcode{\sphinxupquote{Country/code}} savybė. Kelias savybes
galima išvardinti, atskiriant savybių pavadinimus kableliu.

\sphinxAtStartPar
Prie \sphinxcode{\sphinxupquote{City}} modelio yra įtraukta ir \sphinxcode{\sphinxupquote{City/country.name@lt}} savybė, kurios nėra
\sphinxcode{\sphinxupquote{Country}} modelyje.

\sphinxAtStartPar
\sphinxcode{\sphinxupquote{expand()}} įtraukia visas savybes, kurios išvardintos prie modelio, įskaitant
ir jungtinio modelio savybes iš kitų klasių.


\subsection{Daugiareikšmiškumas}
\label{\detokenize{identifikatoriai:daugiareiksmiskumas}}

\subsection{Duomenų kilmė}
\label{\detokenize{identifikatoriai:duomenu-kilme}}
\sphinxAtStartPar
Įprastai duomenys yra registruojami vieną kartą pirminiame šaltinyje ir daug
kartų pernaudojami išvestiniuose šaltiniuose. Informacija apie tai iš kokio
pirminio šaldinio duomenys pateko į išvestinį šaltinį, vadinama duomenų kilme.

\sphinxAtStartPar
Struktūros aprašuose, duomenų kilmė pažymima nepildant {\hyperref[\detokenize{dimensijos:property.type}]{\sphinxcrossref{\sphinxcode{\sphinxupquote{property.type}}}}}
stulpleio. Jei {\hyperref[\detokenize{dimensijos:property.type}]{\sphinxcrossref{\sphinxcode{\sphinxupquote{property.type}}}}} yra neužpildytas, nurodoma, kad modelis
kuriame pateikta savybė, nėra pirminis šios savybės šaltinis.

\sphinxAtStartPar
Jei modelis yra kito jungtinio modelio dalis arba pirminio modelio dalis,
{\hyperref[\detokenize{dimensijos:property.type}]{\sphinxcrossref{\sphinxcode{\sphinxupquote{property.type}}}}} stulpelis yra nepildomas, jei savybės modelyje yra
pateikiamos tik skaitymui, be galimybės keisti savybių reikšmių.
\begin{itemize}
\item {} 
\sphinxAtStartPar
jei {\hyperref[\detokenize{dimensijos:property.type}]{\sphinxcrossref{\sphinxcode{\sphinxupquote{property.type}}}}} užpildytas, tada nurodoma, kad ši savybė yra
{\hyperref[\detokenize{savokos:term-pirminis-duomenu-saltinis}]{\sphinxtermref{\DUrole{xref}{\DUrole{std}{\DUrole{std-term}{pirminis duomenų šaltinis}}}}}}, tos savybės duomenys gali būti keičiami
duomenų modelyje prie kurios savybė yra pateikta,

\item {} 
\sphinxAtStartPar
jei \sphinxcode{\sphinxupquote{type}} nenurodytas, tada nurodoma, kad ši savybė nėra {\hyperref[\detokenize{savokos:term-pirminis-duomenu-saltinis}]{\sphinxtermref{\DUrole{xref}{\DUrole{std}{\DUrole{std-term}{pirminis
duomenų šaltinis}}}}}} ir gali būti naudojama tik skaitymui, be galimybės keisti
savybės reikšias per modelį, kuriame savybė pateikita.

\end{itemize}

\sphinxAtStartPar
Pavyzdžiui turime jungtinį \sphinxcode{\sphinxupquote{City}} modelį:


\begin{savenotes}\sphinxattablestart
\sphinxthistablewithglobalstyle
\centering
\begin{tabulary}{\linewidth}[t]{TTTTTT}
\sphinxtoprule
\sphinxstyletheadfamily 
\sphinxAtStartPar
dataset
&\sphinxstyletheadfamily 
\sphinxAtStartPar
model
&\sphinxstyletheadfamily 
\sphinxAtStartPar
property
&\sphinxstyletheadfamily 
\sphinxAtStartPar
type
&\sphinxstyletheadfamily 
\sphinxAtStartPar
ref
&\sphinxstyletheadfamily 
\sphinxAtStartPar
level
\\
\sphinxmidrule
\sphinxtableatstartofbodyhook\sphinxstartmulticolumn{3}%
\begin{varwidth}[t]{\sphinxcolwidth{3}{6}}
\sphinxAtStartPar
datasets/gov/rc/ar/ws
\par
\vskip-\baselineskip\vbox{\hbox{\strut}}\end{varwidth}%
\sphinxstopmulticolumn
&&&\\
\sphinxhline
\sphinxAtStartPar

&\sphinxstartmulticolumn{2}%
\begin{varwidth}[t]{\sphinxcolwidth{2}{6}}
\sphinxAtStartPar
Country
\par
\vskip-\baselineskip\vbox{\hbox{\strut}}\end{varwidth}%
\sphinxstopmulticolumn
&&
\sphinxAtStartPar
code
&
\sphinxAtStartPar
4
\\
\sphinxhline
\sphinxAtStartPar

&&
\sphinxAtStartPar
code
&
\sphinxAtStartPar
string
&&
\sphinxAtStartPar
4
\\
\sphinxhline
\sphinxAtStartPar

&&
\sphinxAtStartPar
name@en
&
\sphinxAtStartPar
string
&&
\sphinxAtStartPar
4
\\
\sphinxhline
\sphinxAtStartPar

&\sphinxstartmulticolumn{2}%
\begin{varwidth}[t]{\sphinxcolwidth{2}{6}}
\sphinxAtStartPar
City
\par
\vskip-\baselineskip\vbox{\hbox{\strut}}\end{varwidth}%
\sphinxstopmulticolumn
&&&
\sphinxAtStartPar
3
\\
\sphinxhline
\sphinxAtStartPar

&&
\sphinxAtStartPar
name@en
&
\sphinxAtStartPar
string
&&
\sphinxAtStartPar
4
\\
\sphinxhline
\sphinxAtStartPar

&&
\sphinxAtStartPar
country
&
\sphinxAtStartPar
ref
&
\sphinxAtStartPar
Country
&
\sphinxAtStartPar
4
\\
\sphinxhline
\sphinxAtStartPar

&&
\sphinxAtStartPar
country.code
&&&
\sphinxAtStartPar
4
\\
\sphinxhline
\sphinxAtStartPar

&&
\sphinxAtStartPar
country.name@en
&&&
\sphinxAtStartPar
4
\\
\sphinxhline
\sphinxAtStartPar

&&
\sphinxAtStartPar
country.name@lt
&
\sphinxAtStartPar
string
&&
\sphinxAtStartPar
4
\\
\sphinxbottomrule
\end{tabulary}
\sphinxtableafterendhook\par
\sphinxattableend\end{savenotes}

\sphinxAtStartPar
Kuriame prie \sphinxcode{\sphinxupquote{City}} prijungiama \sphinxcode{\sphinxupquote{Country}} klasė.`country.code` ir
\sphinxcode{\sphinxupquote{country.name@en}} neturi {\hyperref[\detokenize{dimensijos:property.type}]{\sphinxcrossref{\sphinxcode{\sphinxupquote{property.type}}}}}, nurodant, kad \sphinxcode{\sphinxupquote{City}} jungtinis
modelis nėra šių savybių pirminis šaltinis ir šios savybės gali būti naudojamos
tik skaitymo tikslais.

\sphinxAtStartPar
Tačiau \sphinxcode{\sphinxupquote{City/country.name@lt}} turi {\hyperref[\detokenize{dimensijos:property.type}]{\sphinxcrossref{\sphinxcode{\sphinxupquote{property.type}}}}}, todėl \sphinxcode{\sphinxupquote{City}}
jungtinis modelis yra šios savybės pirminis šaltinis.

\sphinxAtStartPar
Jei ta pati savybė turi daugiau nei vieną pirminį šaltinį, tada savybės, kuri
nurodo {\hyperref[\detokenize{dimensijos:property.type}]{\sphinxcrossref{\sphinxcode{\sphinxupquote{property.type}}}}} ir yra pateikta prie išvestinio arba jungtinio
modelio, brandos lygis yra \sphinxcode{\sphinxupquote{2}}, kadangi negali būti du pirminiai duomenų
šaltiniai viename objektui.


\subsection{Jungimo būdai}
\label{\detokenize{identifikatoriai:jungimo-budai}}\label{\detokenize{identifikatoriai:ref-types}}

\subsubsection{Per pirminį raktą}
\label{\detokenize{identifikatoriai:per-pirmini-rakta}}
\sphinxAtStartPar
Pavyzdžiui, jei turime tokias dvi duomenų lenteles:


\begin{savenotes}\sphinxattablestart
\sphinxthistablewithglobalstyle
\centering
\begin{tabulary}{\linewidth}[t]{TTT}
\sphinxtoprule
\sphinxstartmulticolumn{3}%
\begin{varwidth}[t]{\sphinxcolwidth{3}{3}}
\sphinxstyletheadfamily \sphinxAtStartPar
Country
\par
\vskip-\baselineskip\vbox{\hbox{\strut}}\end{varwidth}%
\sphinxstopmulticolumn
\\
\sphinxhline\sphinxstyletheadfamily 
\sphinxAtStartPar
id
&\sphinxstyletheadfamily 
\sphinxAtStartPar
name
&\sphinxstyletheadfamily 
\sphinxAtStartPar
code
\\
\sphinxmidrule
\sphinxtableatstartofbodyhook
\sphinxAtStartPar
1
&
\sphinxAtStartPar
Lietuva
&
\sphinxAtStartPar
lt
\\
\sphinxhline
\sphinxAtStartPar
2
&
\sphinxAtStartPar
Latvija
&
\sphinxAtStartPar
lv
\\
\sphinxbottomrule
\end{tabulary}
\sphinxtableafterendhook\par
\sphinxattableend\end{savenotes}


\begin{savenotes}\sphinxattablestart
\sphinxthistablewithglobalstyle
\centering
\begin{tabulary}{\linewidth}[t]{TTT}
\sphinxtoprule
\sphinxstartmulticolumn{3}%
\begin{varwidth}[t]{\sphinxcolwidth{3}{3}}
\sphinxstyletheadfamily \sphinxAtStartPar
City
\par
\vskip-\baselineskip\vbox{\hbox{\strut}}\end{varwidth}%
\sphinxstopmulticolumn
\\
\sphinxhline\sphinxstyletheadfamily 
\sphinxAtStartPar
id
&\sphinxstyletheadfamily 
\sphinxAtStartPar
name
&\sphinxstyletheadfamily 
\sphinxAtStartPar
country
\\
\sphinxmidrule
\sphinxtableatstartofbodyhook
\sphinxAtStartPar
1
&
\sphinxAtStartPar
Vilnius
&
\sphinxAtStartPar
lt
\\
\sphinxhline
\sphinxAtStartPar
2
&
\sphinxAtStartPar
Kaunas
&
\sphinxAtStartPar
lt
\\
\sphinxhline
\sphinxAtStartPar
3
&
\sphinxAtStartPar
Ryga
&
\sphinxAtStartPar
lv
\\
\sphinxbottomrule
\end{tabulary}
\sphinxtableafterendhook\par
\sphinxattableend\end{savenotes}

\sphinxAtStartPar
Šiuo atveju, jei norime parengti aukščiau pateiktų duomenų struktūros aprašą,
jis atrodytų taip:


\begin{savenotes}\sphinxattablestart
\sphinxthistablewithglobalstyle
\centering
\begin{tabulary}{\linewidth}[t]{TTTTTTTTT}
\sphinxtoprule
\sphinxstyletheadfamily 
\sphinxAtStartPar
id
&\sphinxstyletheadfamily 
\sphinxAtStartPar
d
&\sphinxstyletheadfamily 
\sphinxAtStartPar
r
&\sphinxstyletheadfamily 
\sphinxAtStartPar
b
&\sphinxstyletheadfamily 
\sphinxAtStartPar
m
&\sphinxstyletheadfamily 
\sphinxAtStartPar
property
&\sphinxstyletheadfamily 
\sphinxAtStartPar
type
&\sphinxstyletheadfamily 
\sphinxAtStartPar
ref
&\sphinxstyletheadfamily 
\sphinxAtStartPar
level
\\
\sphinxmidrule
\sphinxtableatstartofbodyhook
\sphinxAtStartPar
1
&\sphinxstartmulticolumn{5}%
\begin{varwidth}[t]{\sphinxcolwidth{5}{9}}
\sphinxAtStartPar
datasets/gov/example/countries
\par
\vskip-\baselineskip\vbox{\hbox{\strut}}\end{varwidth}%
\sphinxstopmulticolumn
&&&\\
\sphinxhline
\sphinxAtStartPar
2
&&&&\sphinxstartmulticolumn{2}%
\begin{varwidth}[t]{\sphinxcolwidth{2}{9}}
\sphinxAtStartPar
Country
\par
\vskip-\baselineskip\vbox{\hbox{\strut}}\end{varwidth}%
\sphinxstopmulticolumn
&&
\sphinxAtStartPar
code
&
\sphinxAtStartPar
4
\\
\sphinxhline
\sphinxAtStartPar
3
&&&&&
\sphinxAtStartPar
id
&
\sphinxAtStartPar
integer
&&
\sphinxAtStartPar
4
\\
\sphinxhline
\sphinxAtStartPar
4
&&&&&
\sphinxAtStartPar
name
&
\sphinxAtStartPar
string
&&
\sphinxAtStartPar
4
\\
\sphinxhline
\sphinxAtStartPar
5
&&&&&
\sphinxAtStartPar
code
&
\sphinxAtStartPar
string
&&
\sphinxAtStartPar
4
\\
\sphinxhline
\sphinxAtStartPar
6
&&&&\sphinxstartmulticolumn{2}%
\begin{varwidth}[t]{\sphinxcolwidth{2}{9}}
\sphinxAtStartPar
City
\par
\vskip-\baselineskip\vbox{\hbox{\strut}}\end{varwidth}%
\sphinxstopmulticolumn
&&
\sphinxAtStartPar
id
&
\sphinxAtStartPar
4
\\
\sphinxhline
\sphinxAtStartPar
7
&&&&&
\sphinxAtStartPar
id
&
\sphinxAtStartPar
integer
&&
\sphinxAtStartPar
4
\\
\sphinxhline
\sphinxAtStartPar
8
&&&&&
\sphinxAtStartPar
name
&
\sphinxAtStartPar
string
&&
\sphinxAtStartPar
4
\\
\sphinxhline
\sphinxAtStartPar
9
&&&&&
\sphinxAtStartPar
country
&
\sphinxAtStartPar
ref
&
\sphinxAtStartPar
Country
&
\sphinxAtStartPar
4
\\
\sphinxbottomrule
\end{tabulary}
\sphinxtableafterendhook\par
\sphinxattableend\end{savenotes}

\sphinxAtStartPar
Šiame duomenų struktūros apraše, 9\sphinxhyphen{}oje eilutėje \sphinxcode{\sphinxupquote{country}} stulpelio tipas yra
\sphinxcode{\sphinxupquote{ref}}, tai reiškia, kad šis stulpelis yra kito modelio išorinis raktas.
\sphinxcode{\sphinxupquote{property.ref}} stulpelyje nurodyta kurio modelio išorinis raktas šis
stulpelis yra. Šiuo atveju, tai yra \sphinxcode{\sphinxupquote{Country}} modelis, kuris apibrėžtas 2\sphinxhyphen{}oje
eilutėje.

\sphinxAtStartPar
Pagal nutylėjimą, ryšys su kitu modeliu nustatomas naudojant kitos lentelės
pirminį raktą nurodytą {\hyperref[\detokenize{dimensijos:model.ref}]{\sphinxcrossref{\sphinxcode{\sphinxupquote{model.ref}}}}} stulpelyje. Šiuo atveju, \sphinxcode{\sphinxupquote{City
.country}} yra jungiamas per \sphinxcode{\sphinxupquote{Country.code}}. Tai reiškia, kad \sphinxcode{\sphinxupquote{City.country}}
duomenų tipas turi sutapti su \sphinxcode{\sphinxupquote{Country.code}} duomenų tipu, kuris yra \sphinxcode{\sphinxupquote{string}}.

\sphinxAtStartPar
\sphinxcode{\sphinxupquote{property.ref}} reikšmė gali būti pateikiama vienu iš šių variantų:


\begin{fulllineitems}

\pysigstartsignatures
\pysigline
{\sphinxbfcode{\sphinxupquote{property.ref}}}
\pysigstopsignatures

\begin{fulllineitems}

\pysigstartsignatures
\pysigline
{\sphinxbfcode{\sphinxupquote{model}}}
\pysigstopsignatures
\sphinxAtStartPar
\sphinxcode{\sphinxupquote{model}} nurodo kito {\hyperref[\detokenize{formatas:model}]{\sphinxcrossref{\sphinxcode{\sphinxupquote{model}}}}} pavadinimą kurio {\hyperref[\detokenize{dimensijos:model.ref}]{\sphinxcrossref{\sphinxcode{\sphinxupquote{model.ref}}}}}
siejamas su {\hyperref[\detokenize{formatas:property}]{\sphinxcrossref{\sphinxcode{\sphinxupquote{property}}}}}.

\sphinxAtStartPar
Jei {\hyperref[\detokenize{dimensijos:model.ref}]{\sphinxcrossref{\sphinxcode{\sphinxupquote{model.ref}}}}} pirminiam raktui naudoja daugiau nei vieną lauką,
tada {\hyperref[\detokenize{dimensijos:property.source}]{\sphinxcrossref{\sphinxcode{\sphinxupquote{property.source}}}}} laukas turi būti tuščias, o
{\hyperref[\detokenize{dimensijos:property.prepare}]{\sphinxcrossref{\sphinxcode{\sphinxupquote{property.prepare}}}}} turi būti pateikiamos kableliu atskirtos
property reikšmės, kurios bus naudojamos susiejimui.

\end{fulllineitems}



\begin{fulllineitems}

\pysigstartsignatures
\pysigline
{\sphinxbfcode{\sphinxupquote{model{[}property{]}}}}
\pysigstopsignatures
\sphinxAtStartPar
Tais atvejais, kai {\hyperref[\detokenize{formatas:property}]{\sphinxcrossref{\sphinxcode{\sphinxupquote{property}}}}} duomenys nesutampa su siejamo
{\hyperref[\detokenize{dimensijos:model.ref}]{\sphinxcrossref{\sphinxcode{\sphinxupquote{model.ref}}}}}, galima nurodyti {\hyperref[\detokenize{formatas:property}]{\sphinxcrossref{\sphinxcode{\sphinxupquote{property}}}}} iš {\hyperref[\detokenize{formatas:model}]{\sphinxcrossref{\sphinxcode{\sphinxupquote{model}}}}}.

\end{fulllineitems}



\begin{fulllineitems}

\pysigstartsignatures
\pysigline
{\sphinxbfcode{\sphinxupquote{model{[}*property{]}}}}
\pysigstopsignatures
\sphinxAtStartPar
Jei susiejimui reikia daugiau nei vieno duomenų lauko ir jie nesutampa
su model.ref, tada galima nurodyti kelias property reikšmes atskirtas
kableliu. Tačiau šiuo atveju taip pat būtina nurodyti ir
{\hyperref[\detokenize{dimensijos:property.prepare}]{\sphinxcrossref{\sphinxcode{\sphinxupquote{property.prepare}}}}} kelias reikšmes atskirtas kableliu, o
{\hyperref[\detokenize{dimensijos:property.source}]{\sphinxcrossref{\sphinxcode{\sphinxupquote{property.source}}}}} reikšmė turi būti tuščia.
{\hyperref[\detokenize{dimensijos:property.prepare}]{\sphinxcrossref{\sphinxcode{\sphinxupquote{property.prepare}}}}} stulpelyje nurodomi kiti modelio
{\hyperref[\detokenize{formatas:property}]{\sphinxcrossref{\sphinxcode{\sphinxupquote{property}}}}} pavadinimai iš kurių duomenų reikšmių turi būti
formuojamas sudėtinis raktas.

\end{fulllineitems}


\end{fulllineitems}



\subsubsection{Per nepirminį raktą}
\label{\detokenize{identifikatoriai:per-nepirmini-rakta}}\label{\detokenize{identifikatoriai:ref-fkey}}
\sphinxAtStartPar
Jei modelius reikia jungti ne per pirminį raktą, o per kitus laukus, tada
naudojama \sphinxcode{\sphinxupquote{model{[}property{]}}} forma.

\sphinxAtStartPar
Pavyzdžiui, jei turime tokius duomenis:


\begin{savenotes}\sphinxattablestart
\sphinxthistablewithglobalstyle
\centering
\begin{tabulary}{\linewidth}[t]{TTT}
\sphinxtoprule
\sphinxstartmulticolumn{3}%
\begin{varwidth}[t]{\sphinxcolwidth{3}{3}}
\sphinxstyletheadfamily \sphinxAtStartPar
Country
\par
\vskip-\baselineskip\vbox{\hbox{\strut}}\end{varwidth}%
\sphinxstopmulticolumn
\\
\sphinxhline\sphinxstyletheadfamily 
\sphinxAtStartPar
id
&\sphinxstyletheadfamily 
\sphinxAtStartPar
name
&\sphinxstyletheadfamily 
\sphinxAtStartPar
code
\\
\sphinxmidrule
\sphinxtableatstartofbodyhook
\sphinxAtStartPar
1
&
\sphinxAtStartPar
Lietuva
&
\sphinxAtStartPar
lt
\\
\sphinxhline
\sphinxAtStartPar
2
&
\sphinxAtStartPar
Latvija
&
\sphinxAtStartPar
lv
\\
\sphinxbottomrule
\end{tabulary}
\sphinxtableafterendhook\par
\sphinxattableend\end{savenotes}


\begin{savenotes}\sphinxattablestart
\sphinxthistablewithglobalstyle
\centering
\begin{tabulary}{\linewidth}[t]{TTT}
\sphinxtoprule
\sphinxstartmulticolumn{3}%
\begin{varwidth}[t]{\sphinxcolwidth{3}{3}}
\sphinxstyletheadfamily \sphinxAtStartPar
City
\par
\vskip-\baselineskip\vbox{\hbox{\strut}}\end{varwidth}%
\sphinxstopmulticolumn
\\
\sphinxhline\sphinxstyletheadfamily 
\sphinxAtStartPar
id
&\sphinxstyletheadfamily 
\sphinxAtStartPar
name
&\sphinxstyletheadfamily 
\sphinxAtStartPar
country
\\
\sphinxmidrule
\sphinxtableatstartofbodyhook
\sphinxAtStartPar
1
&
\sphinxAtStartPar
Vilnius
&
\sphinxAtStartPar
lt
\\
\sphinxhline
\sphinxAtStartPar
2
&
\sphinxAtStartPar
Kaunas
&
\sphinxAtStartPar
lt
\\
\sphinxhline
\sphinxAtStartPar
3
&
\sphinxAtStartPar
Ryga
&
\sphinxAtStartPar
lv
\\
\sphinxbottomrule
\end{tabulary}
\sphinxtableafterendhook\par
\sphinxattableend\end{savenotes}

\sphinxAtStartPar
Kur \sphinxcode{\sphinxupquote{Country}} pirminis raktas yra \sphinxcode{\sphinxupquote{id}} ir norime jungti \sphinxcode{\sphinxupquote{City.country}} per
\sphinxcode{\sphinxupquote{Country.code}}, tuomet duomenų struktūros aprašas atrodys taip:


\begin{savenotes}\sphinxattablestart
\sphinxthistablewithglobalstyle
\centering
\begin{tabulary}{\linewidth}[t]{TTTTTTTTT}
\sphinxtoprule
\sphinxstyletheadfamily 
\sphinxAtStartPar
d
&\sphinxstyletheadfamily 
\sphinxAtStartPar
d
&\sphinxstyletheadfamily 
\sphinxAtStartPar
r
&\sphinxstyletheadfamily 
\sphinxAtStartPar
b
&\sphinxstyletheadfamily 
\sphinxAtStartPar
m
&\sphinxstyletheadfamily 
\sphinxAtStartPar
property
&\sphinxstyletheadfamily 
\sphinxAtStartPar
type
&\sphinxstyletheadfamily 
\sphinxAtStartPar
ref
&\sphinxstyletheadfamily 
\sphinxAtStartPar
level
\\
\sphinxmidrule
\sphinxtableatstartofbodyhook
\sphinxAtStartPar
1
&\sphinxstartmulticolumn{5}%
\begin{varwidth}[t]{\sphinxcolwidth{5}{9}}
\sphinxAtStartPar
datasets/gov/example/countries
\par
\vskip-\baselineskip\vbox{\hbox{\strut}}\end{varwidth}%
\sphinxstopmulticolumn
&&&\\
\sphinxhline
\sphinxAtStartPar
2
&&&&\sphinxstartmulticolumn{2}%
\begin{varwidth}[t]{\sphinxcolwidth{2}{9}}
\sphinxAtStartPar
Country
\par
\vskip-\baselineskip\vbox{\hbox{\strut}}\end{varwidth}%
\sphinxstopmulticolumn
&&
\sphinxAtStartPar
id
&
\sphinxAtStartPar
4
\\
\sphinxhline
\sphinxAtStartPar
3
&&&&&
\sphinxAtStartPar
id
&
\sphinxAtStartPar
integer
&&
\sphinxAtStartPar
4
\\
\sphinxhline
\sphinxAtStartPar
4
&&&&&
\sphinxAtStartPar
name
&
\sphinxAtStartPar
string
&&
\sphinxAtStartPar
4
\\
\sphinxhline
\sphinxAtStartPar
5
&&&&&
\sphinxAtStartPar
code
&
\sphinxAtStartPar
string
&&
\sphinxAtStartPar
4
\\
\sphinxhline
\sphinxAtStartPar
6
&&&&\sphinxstartmulticolumn{2}%
\begin{varwidth}[t]{\sphinxcolwidth{2}{9}}
\sphinxAtStartPar
City
\par
\vskip-\baselineskip\vbox{\hbox{\strut}}\end{varwidth}%
\sphinxstopmulticolumn
&&
\sphinxAtStartPar
id
&
\sphinxAtStartPar
4
\\
\sphinxhline
\sphinxAtStartPar
7
&&&&&
\sphinxAtStartPar
id
&
\sphinxAtStartPar
integer
&&
\sphinxAtStartPar
4
\\
\sphinxhline
\sphinxAtStartPar
8
&&&&&
\sphinxAtStartPar
name
&
\sphinxAtStartPar
string
&&
\sphinxAtStartPar
4
\\
\sphinxhline
\sphinxAtStartPar
9
&&&&&
\sphinxAtStartPar
country
&
\sphinxAtStartPar
ref
&
\sphinxAtStartPar
Country{[}code{]}
&
\sphinxAtStartPar
4
\\
\sphinxbottomrule
\end{tabulary}
\sphinxtableafterendhook\par
\sphinxattableend\end{savenotes}

\sphinxAtStartPar
9\sphinxhyphen{}oje eilutėje \sphinxcode{\sphinxupquote{property.ref}} stulpelyje pateikta \sphinxcode{\sphinxupquote{Country{[}code{]}}} reikšmė, kuri
\sphinxcode{\sphinxupquote{Country}} nurodo su kokiu modeliu jungiame, o \sphinxcode{\sphinxupquote{code}} nurodo su kokiu \sphinxcode{\sphinxupquote{Country}}
stulpeliu jungiame. Jei pateiktas tik modelis, tada jungiama per to modelio
pirminį raktą, jei pateiktas stulpelis laužtiniuose skliausteliuose, tada
jungiama per nurodytą stulpelį.


\subsubsection{Per kompozicinį raktą}
\label{\detokenize{identifikatoriai:per-kompozicini-rakta}}
\sphinxAtStartPar
Jei modelius reikia jungti per kelis laukus, tada naudojama
\sphinxcode{\sphinxupquote{model{[}*property{]}}} forma, kur laužtiniuose skliaustuose pateikiami keli
stulpeliai atskirti kableliais.

\sphinxAtStartPar
Pavyzdžiui, jei turime tokius duomenis:


\begin{savenotes}\sphinxattablestart
\sphinxthistablewithglobalstyle
\centering
\begin{tabulary}{\linewidth}[t]{TTT}
\sphinxtoprule
\sphinxstartmulticolumn{3}%
\begin{varwidth}[t]{\sphinxcolwidth{3}{3}}
\sphinxstyletheadfamily \sphinxAtStartPar
Country
\par
\vskip-\baselineskip\vbox{\hbox{\strut}}\end{varwidth}%
\sphinxstopmulticolumn
\\
\sphinxhline\sphinxstyletheadfamily 
\sphinxAtStartPar
id
&\sphinxstyletheadfamily 
\sphinxAtStartPar
name
&\sphinxstyletheadfamily 
\sphinxAtStartPar
code
\\
\sphinxmidrule
\sphinxtableatstartofbodyhook
\sphinxAtStartPar
1
&
\sphinxAtStartPar
Lietuva
&
\sphinxAtStartPar
lt
\\
\sphinxhline
\sphinxAtStartPar
2
&
\sphinxAtStartPar
Latvija
&
\sphinxAtStartPar
lv
\\
\sphinxbottomrule
\end{tabulary}
\sphinxtableafterendhook\par
\sphinxattableend\end{savenotes}


\begin{savenotes}\sphinxattablestart
\sphinxthistablewithglobalstyle
\centering
\begin{tabulary}{\linewidth}[t]{TTTT}
\sphinxtoprule
\sphinxstartmulticolumn{4}%
\begin{varwidth}[t]{\sphinxcolwidth{4}{4}}
\sphinxstyletheadfamily \sphinxAtStartPar
City
\par
\vskip-\baselineskip\vbox{\hbox{\strut}}\end{varwidth}%
\sphinxstopmulticolumn
\\
\sphinxhline\sphinxstyletheadfamily 
\sphinxAtStartPar
id
&\sphinxstyletheadfamily 
\sphinxAtStartPar
name
&\sphinxstyletheadfamily 
\sphinxAtStartPar
country
&\sphinxstyletheadfamily 
\sphinxAtStartPar
country\_id
\\
\sphinxmidrule
\sphinxtableatstartofbodyhook
\sphinxAtStartPar
1
&
\sphinxAtStartPar
Vilnius
&
\sphinxAtStartPar
lt
&
\sphinxAtStartPar
1
\\
\sphinxhline
\sphinxAtStartPar
2
&
\sphinxAtStartPar
Kaunas
&
\sphinxAtStartPar
lt
&
\sphinxAtStartPar
1
\\
\sphinxhline
\sphinxAtStartPar
3
&
\sphinxAtStartPar
Ryga
&
\sphinxAtStartPar
lv
&
\sphinxAtStartPar
2
\\
\sphinxbottomrule
\end{tabulary}
\sphinxtableafterendhook\par
\sphinxattableend\end{savenotes}

\sphinxAtStartPar
Kur \sphinxcode{\sphinxupquote{City}} su \sphinxcode{\sphinxupquote{Country}} yra jungiamas per du \sphinxcode{\sphinxupquote{country}} ir \sphinxcode{\sphinxupquote{country\_id}}
stulpelius, tuomet reikia įtraukti išvestinį duomenų lauką, kuriame formulės
įrašomos į {\hyperref[\detokenize{dimensijos:property.prepare}]{\sphinxcrossref{\sphinxcode{\sphinxupquote{property.prepare}}}}} pagalba apjungiami keli laukai į vieną
kompozicinį raktą. Šiuo atveju duomenų struktūros aprašas atrodys taip:


\begin{savenotes}\sphinxattablestart
\sphinxthistablewithglobalstyle
\centering
\begin{tabulary}{\linewidth}[t]{TTTTTTTTTT}
\sphinxtoprule
\sphinxstyletheadfamily 
\sphinxAtStartPar
d
&\sphinxstyletheadfamily 
\sphinxAtStartPar
d
&\sphinxstyletheadfamily 
\sphinxAtStartPar
r
&\sphinxstyletheadfamily 
\sphinxAtStartPar
b
&\sphinxstyletheadfamily 
\sphinxAtStartPar
m
&\sphinxstyletheadfamily 
\sphinxAtStartPar
property
&\sphinxstyletheadfamily 
\sphinxAtStartPar
type
&\sphinxstyletheadfamily 
\sphinxAtStartPar
ref
&\sphinxstyletheadfamily 
\sphinxAtStartPar
prepare
&\sphinxstyletheadfamily 
\sphinxAtStartPar
level
\\
\sphinxmidrule
\sphinxtableatstartofbodyhook
\sphinxAtStartPar
1
&\sphinxstartmulticolumn{5}%
\begin{varwidth}[t]{\sphinxcolwidth{5}{10}}
\sphinxAtStartPar
datasets/gov/example/countries
\par
\vskip-\baselineskip\vbox{\hbox{\strut}}\end{varwidth}%
\sphinxstopmulticolumn
&&&&\\
\sphinxhline
\sphinxAtStartPar
2
&&&&\sphinxstartmulticolumn{2}%
\begin{varwidth}[t]{\sphinxcolwidth{2}{10}}
\sphinxAtStartPar
Country
\par
\vskip-\baselineskip\vbox{\hbox{\strut}}\end{varwidth}%
\sphinxstopmulticolumn
&&
\sphinxAtStartPar
id
&&
\sphinxAtStartPar
4
\\
\sphinxhline
\sphinxAtStartPar
3
&&&&&
\sphinxAtStartPar
id
&
\sphinxAtStartPar
integer
&&&
\sphinxAtStartPar
4
\\
\sphinxhline
\sphinxAtStartPar
4
&&&&&
\sphinxAtStartPar
name
&
\sphinxAtStartPar
string
&&&
\sphinxAtStartPar
4
\\
\sphinxhline
\sphinxAtStartPar
5
&&&&&
\sphinxAtStartPar
code
&
\sphinxAtStartPar
string
&&&
\sphinxAtStartPar
4
\\
\sphinxhline
\sphinxAtStartPar
6
&&&&\sphinxstartmulticolumn{2}%
\begin{varwidth}[t]{\sphinxcolwidth{2}{10}}
\sphinxAtStartPar
City
\par
\vskip-\baselineskip\vbox{\hbox{\strut}}\end{varwidth}%
\sphinxstopmulticolumn
&&
\sphinxAtStartPar
id
&&
\sphinxAtStartPar
4
\\
\sphinxhline
\sphinxAtStartPar
7
&&&&&
\sphinxAtStartPar
id
&
\sphinxAtStartPar
integer
&&&
\sphinxAtStartPar
4
\\
\sphinxhline
\sphinxAtStartPar
8
&&&&&
\sphinxAtStartPar
name
&
\sphinxAtStartPar
string
&&&
\sphinxAtStartPar
4
\\
\sphinxhline
\sphinxAtStartPar
9
&&&&&
\sphinxAtStartPar
country\_code
&
\sphinxAtStartPar
string
&&&
\sphinxAtStartPar
4
\\
\sphinxhline
\sphinxAtStartPar
10
&&&&&
\sphinxAtStartPar
country\_id
&
\sphinxAtStartPar
integer
&&&
\sphinxAtStartPar
4
\\
\sphinxhline
\sphinxAtStartPar
11
&&&&&
\sphinxAtStartPar
country
&
\sphinxAtStartPar
ref
&
\sphinxAtStartPar
Country{[}id,code{]}
&
\sphinxAtStartPar
country\_id, country\_code
&
\sphinxAtStartPar
4
\\
\sphinxbottomrule
\end{tabulary}
\sphinxtableafterendhook\par
\sphinxattableend\end{savenotes}

\sphinxAtStartPar
Čia matome, kad 11\sphinxhyphen{}oje eilutėje buvo įtrauktas išvestinis laukas \sphinxcode{\sphinxupquote{country}},
kuris išskaičiuojamas apjungiant \sphinxcode{\sphinxupquote{country\_id}} ir \sphinxcode{\sphinxupquote{country\_code}}. O ryšiui su
\sphinxcode{\sphinxupquote{Country}}, laužtiniuose skliausteliuose nurodyti du laukai iš jungiamo
\sphinxcode{\sphinxupquote{Country}} modelio. Abiejų jungiamų pusių pateiktas laukų sąrašas turi būti
vienodo eiliškumo, o jungiami laukai turi turėti vienodus tipus.

\sphinxAtStartPar
Jei \sphinxcode{\sphinxupquote{Country}} pirminis raktas būtų {\hyperref[\detokenize{savokos:term-kompozicinis-raktas}]{\sphinxtermref{\DUrole{xref}{\DUrole{std}{\DUrole{std-term}{kompozicinis}}}}}},
pavyzdžiui \sphinxcode{\sphinxupquote{id, code}}, tuomet, 11\sphinxhyphen{}oje eilutėje \sphinxcode{\sphinxupquote{property.ref}} užtektu nurodyti
tik \sphinxcode{\sphinxupquote{Country}}.


\subsubsection{Atgalinis ryšys}
\label{\detokenize{identifikatoriai:atgalinis-rysys}}\label{\detokenize{identifikatoriai:id1}}
\sphinxAtStartPar
Jungiant modelius atgaliniu ryšiu kuriamas išvestinis arba virtualus laukas,
kuriame analogiškai kaip ir paprasto ryšio atveju, apjungiami du modeliai,
tik šiuo atveju kuriamas daug su vienas tipo ryšys.

\sphinxAtStartPar
Pavyzdžiui, jei turime tokius duomenis:


\begin{savenotes}\sphinxattablestart
\sphinxthistablewithglobalstyle
\centering
\begin{tabulary}{\linewidth}[t]{TT}
\sphinxtoprule
\sphinxstartmulticolumn{2}%
\begin{varwidth}[t]{\sphinxcolwidth{2}{2}}
\sphinxstyletheadfamily \sphinxAtStartPar
Country
\par
\vskip-\baselineskip\vbox{\hbox{\strut}}\end{varwidth}%
\sphinxstopmulticolumn
\\
\sphinxhline\sphinxstyletheadfamily 
\sphinxAtStartPar
id
&\sphinxstyletheadfamily 
\sphinxAtStartPar
name
\\
\sphinxmidrule
\sphinxtableatstartofbodyhook
\sphinxAtStartPar
1
&
\sphinxAtStartPar
Lietuva
\\
\sphinxhline
\sphinxAtStartPar
2
&
\sphinxAtStartPar
Latvija
\\
\sphinxbottomrule
\end{tabulary}
\sphinxtableafterendhook\par
\sphinxattableend\end{savenotes}


\begin{savenotes}\sphinxattablestart
\sphinxthistablewithglobalstyle
\centering
\begin{tabulary}{\linewidth}[t]{TTT}
\sphinxtoprule
\sphinxstartmulticolumn{3}%
\begin{varwidth}[t]{\sphinxcolwidth{3}{3}}
\sphinxstyletheadfamily \sphinxAtStartPar
City
\par
\vskip-\baselineskip\vbox{\hbox{\strut}}\end{varwidth}%
\sphinxstopmulticolumn
\\
\sphinxhline\sphinxstyletheadfamily 
\sphinxAtStartPar
id
&\sphinxstyletheadfamily 
\sphinxAtStartPar
name
&\sphinxstyletheadfamily 
\sphinxAtStartPar
country
\\
\sphinxmidrule
\sphinxtableatstartofbodyhook
\sphinxAtStartPar
1
&
\sphinxAtStartPar
Vilnius
&
\sphinxAtStartPar
1
\\
\sphinxhline
\sphinxAtStartPar
2
&
\sphinxAtStartPar
Kaunas
&
\sphinxAtStartPar
1
\\
\sphinxhline
\sphinxAtStartPar
3
&
\sphinxAtStartPar
Ryga
&
\sphinxAtStartPar
2
\\
\sphinxbottomrule
\end{tabulary}
\sphinxtableafterendhook\par
\sphinxattableend\end{savenotes}

\sphinxAtStartPar
Ir šiuos duomenis atitinkantį duomenų modelį:

\begin{DUlineblock}{0em}
\item[] 
\end{DUlineblock}

\sphinxAtStartPar
Tai norint sukurti atgalinį ryšį iš \sphinxcode{\sphinxupquote{City}} modelio į \sphinxcode{\sphinxupquote{Country}} modelį, duomenų
struktūros aprašas atrodys taip:


\begin{savenotes}\sphinxattablestart
\sphinxthistablewithglobalstyle
\centering
\begin{tabulary}{\linewidth}[t]{TTTTT}
\sphinxtoprule
\sphinxstyletheadfamily 
\sphinxAtStartPar
model
&\sphinxstyletheadfamily 
\sphinxAtStartPar
property
&\sphinxstyletheadfamily 
\sphinxAtStartPar
type
&\sphinxstyletheadfamily 
\sphinxAtStartPar
ref
&\sphinxstyletheadfamily 
\sphinxAtStartPar
level
\\
\sphinxmidrule
\sphinxtableatstartofbodyhook\sphinxstartmulticolumn{2}%
\begin{varwidth}[t]{\sphinxcolwidth{2}{5}}
\sphinxAtStartPar
\sphinxstylestrong{Country}
\par
\vskip-\baselineskip\vbox{\hbox{\strut}}\end{varwidth}%
\sphinxstopmulticolumn
&&
\sphinxAtStartPar
id
&
\sphinxAtStartPar
4
\\
\sphinxhline
\sphinxAtStartPar

&
\sphinxAtStartPar
id
&
\sphinxAtStartPar
integer
&&
\sphinxAtStartPar
4
\\
\sphinxhline
\sphinxAtStartPar

&
\sphinxAtStartPar
name@lt
&
\sphinxAtStartPar
string
&&
\sphinxAtStartPar
4
\\
\sphinxhline
\sphinxAtStartPar

&
\sphinxAtStartPar
cities{[}{]}
&
\sphinxAtStartPar
backref
&
\sphinxAtStartPar
\sphinxstylestrong{City}
&
\sphinxAtStartPar
4
\\
\sphinxhline\sphinxstartmulticolumn{2}%
\begin{varwidth}[t]{\sphinxcolwidth{2}{5}}
\sphinxAtStartPar
\sphinxstylestrong{City}
\par
\vskip-\baselineskip\vbox{\hbox{\strut}}\end{varwidth}%
\sphinxstopmulticolumn
&&
\sphinxAtStartPar
id
&
\sphinxAtStartPar
4
\\
\sphinxhline
\sphinxAtStartPar

&
\sphinxAtStartPar
id
&
\sphinxAtStartPar
integer
&&
\sphinxAtStartPar
4
\\
\sphinxhline
\sphinxAtStartPar

&
\sphinxAtStartPar
name@lt
&
\sphinxAtStartPar
string
&&
\sphinxAtStartPar
4
\\
\sphinxhline
\sphinxAtStartPar

&
\sphinxAtStartPar
country
&
\sphinxAtStartPar
ref
&
\sphinxAtStartPar
\sphinxstylestrong{Country}
&
\sphinxAtStartPar
4
\\
\sphinxbottomrule
\end{tabulary}
\sphinxtableafterendhook\par
\sphinxattableend\end{savenotes}

\sphinxAtStartPar
Čia atgalinis ryšys nurodytas 5\sphinxhyphen{}oje eilutėje, pateikiant virtualų
\sphinxcode{\sphinxupquote{Country.cities}} lauką, kuris jungiamas per \sphinxcode{\sphinxupquote{City.country}} lauką, kadangi
\sphinxcode{\sphinxupquote{City.country}} turi ryšį su \sphinxcode{\sphinxupquote{Country}}.

\sphinxAtStartPar
Jei \sphinxcode{\sphinxupquote{City}} modelyje būtų pateikti keli stulpeliai susieti su \sphinxcode{\sphinxupquote{Country}}, tada
5\sphinxhyphen{}oje eilutėje \sphinxcode{\sphinxupquote{property.ref}} reikšmė turėtų nurodyti konkretų lauką, per
kurį jungiama, pavyzdžiui \sphinxcode{\sphinxupquote{City{[}country{]}}}.


\subsubsection{Polimorfinis jungimas}
\label{\detokenize{identifikatoriai:polimorfinis-jungimas}}\label{\detokenize{identifikatoriai:polimorfinis-rysys}}
\begin{sphinxadmonition}{note}{Pastaba:}
\sphinxAtStartPar
Tokio tipo jungimas kol kas dar nėra įgyvendintas.
\end{sphinxadmonition}

\sphinxAtStartPar
Polimorfinis jungimas yra toks ryšys tarp modelių, kai vieno modelio laukas
yra siejamas su daugiau nei vienu kitu modeliu. Tokiam ryšiui nurodyti
polimorfinis laukas turi dvi reikšmes, išorinio modelio pavadinimą ir to
modelio stulpelio per kurį jungiama reikšmę.


\begin{savenotes}\sphinxattablestart
\sphinxthistablewithglobalstyle
\centering
\begin{tabulary}{\linewidth}[t]{TT}
\sphinxtoprule
\sphinxstartmulticolumn{2}%
\begin{varwidth}[t]{\sphinxcolwidth{2}{2}}
\sphinxstyletheadfamily \sphinxAtStartPar
Country
\par
\vskip-\baselineskip\vbox{\hbox{\strut}}\end{varwidth}%
\sphinxstopmulticolumn
\\
\sphinxhline\sphinxstyletheadfamily 
\sphinxAtStartPar
id
&\sphinxstyletheadfamily 
\sphinxAtStartPar
name
\\
\sphinxmidrule
\sphinxtableatstartofbodyhook
\sphinxAtStartPar
1
&
\sphinxAtStartPar
Lietuva
\\
\sphinxhline
\sphinxAtStartPar
2
&
\sphinxAtStartPar
Latvija
\\
\sphinxbottomrule
\end{tabulary}
\sphinxtableafterendhook\par
\sphinxattableend\end{savenotes}


\begin{savenotes}\sphinxattablestart
\sphinxthistablewithglobalstyle
\centering
\begin{tabulary}{\linewidth}[t]{TTT}
\sphinxtoprule
\sphinxstartmulticolumn{3}%
\begin{varwidth}[t]{\sphinxcolwidth{3}{3}}
\sphinxstyletheadfamily \sphinxAtStartPar
City
\par
\vskip-\baselineskip\vbox{\hbox{\strut}}\end{varwidth}%
\sphinxstopmulticolumn
\\
\sphinxhline\sphinxstyletheadfamily 
\sphinxAtStartPar
id
&\sphinxstyletheadfamily 
\sphinxAtStartPar
name
&\sphinxstyletheadfamily 
\sphinxAtStartPar
country
\\
\sphinxmidrule
\sphinxtableatstartofbodyhook
\sphinxAtStartPar
1
&
\sphinxAtStartPar
Vilnius
&
\sphinxAtStartPar
1
\\
\sphinxhline
\sphinxAtStartPar
2
&
\sphinxAtStartPar
Ryga
&
\sphinxAtStartPar
2
\\
\sphinxbottomrule
\end{tabulary}
\sphinxtableafterendhook\par
\sphinxattableend\end{savenotes}


\begin{savenotes}\sphinxattablestart
\sphinxthistablewithglobalstyle
\centering
\begin{tabulary}{\linewidth}[t]{TTTT}
\sphinxtoprule
\sphinxstartmulticolumn{4}%
\begin{varwidth}[t]{\sphinxcolwidth{4}{4}}
\sphinxstyletheadfamily \sphinxAtStartPar
Event
\par
\vskip-\baselineskip\vbox{\hbox{\strut}}\end{varwidth}%
\sphinxstopmulticolumn
\\
\sphinxhline\sphinxstyletheadfamily 
\sphinxAtStartPar
id
&\sphinxstyletheadfamily 
\sphinxAtStartPar
name
&\sphinxstyletheadfamily 
\sphinxAtStartPar
object\_id
&\sphinxstyletheadfamily 
\sphinxAtStartPar
object\_model
\\
\sphinxmidrule
\sphinxtableatstartofbodyhook
\sphinxAtStartPar
1
&
\sphinxAtStartPar
Įkūrimas
&
\sphinxAtStartPar
1
&
\sphinxAtStartPar
datasets/gov/example/countries/Country
\\
\sphinxhline
\sphinxAtStartPar
2
&
\sphinxAtStartPar
Įkūrimas
&
\sphinxAtStartPar
2
&
\sphinxAtStartPar
datasets/gov/example/countries/Country
\\
\sphinxhline
\sphinxAtStartPar
3
&
\sphinxAtStartPar
Įkūrimas
&
\sphinxAtStartPar
1
&
\sphinxAtStartPar
datasets/gov/example/countries/City
\\
\sphinxhline
\sphinxAtStartPar
4
&
\sphinxAtStartPar
Įkūrimas
&
\sphinxAtStartPar
2
&
\sphinxAtStartPar
datasets/gov/example/countries/City
\\
\sphinxbottomrule
\end{tabulary}
\sphinxtableafterendhook\par
\sphinxattableend\end{savenotes}

\sphinxAtStartPar
Pavyzdyje aukščiau matome, kad yra du modeliai \sphinxcode{\sphinxupquote{Country}} ir \sphinxcode{\sphinxupquote{City}}, kuriuos
jungia \sphinxcode{\sphinxupquote{Event}} modelis per \sphinxcode{\sphinxupquote{object\_id}} ir \sphinxcode{\sphinxupquote{object\_model}} laukus. Pavyzdžiui
\sphinxcode{\sphinxupquote{Event}} kurio \sphinxcode{\sphinxupquote{id}} yra 1, siejamas su \sphinxcode{\sphinxupquote{Country}} modeliu, kurio \sphinxcode{\sphinxupquote{id}} yra 1.

\sphinxAtStartPar
Tokių duomenų struktūros aprašas atrodys taip:


\begin{savenotes}\sphinxattablestart
\sphinxthistablewithglobalstyle
\centering
\begin{tabulary}{\linewidth}[t]{TTTTTTTTTT}
\sphinxtoprule
\sphinxstyletheadfamily 
\sphinxAtStartPar
d
&\sphinxstyletheadfamily 
\sphinxAtStartPar
d
&\sphinxstyletheadfamily 
\sphinxAtStartPar
r
&\sphinxstyletheadfamily 
\sphinxAtStartPar
b
&\sphinxstyletheadfamily 
\sphinxAtStartPar
m
&\sphinxstyletheadfamily 
\sphinxAtStartPar
property
&\sphinxstyletheadfamily 
\sphinxAtStartPar
type
&\sphinxstyletheadfamily 
\sphinxAtStartPar
ref
&\sphinxstyletheadfamily 
\sphinxAtStartPar
prepare
&\sphinxstyletheadfamily 
\sphinxAtStartPar
level
\\
\sphinxmidrule
\sphinxtableatstartofbodyhook
\sphinxAtStartPar
1
&\sphinxstartmulticolumn{5}%
\begin{varwidth}[t]{\sphinxcolwidth{5}{10}}
\sphinxAtStartPar
datasets/gov/example/countries
\par
\vskip-\baselineskip\vbox{\hbox{\strut}}\end{varwidth}%
\sphinxstopmulticolumn
&&&&\\
\sphinxhline
\sphinxAtStartPar
2
&&&&\sphinxstartmulticolumn{2}%
\begin{varwidth}[t]{\sphinxcolwidth{2}{10}}
\sphinxAtStartPar
Country
\par
\vskip-\baselineskip\vbox{\hbox{\strut}}\end{varwidth}%
\sphinxstopmulticolumn
&&
\sphinxAtStartPar
id
&&
\sphinxAtStartPar
4
\\
\sphinxhline
\sphinxAtStartPar
3
&&&&&
\sphinxAtStartPar
id
&
\sphinxAtStartPar
integer
&&&
\sphinxAtStartPar
4
\\
\sphinxhline
\sphinxAtStartPar
4
&&&&&
\sphinxAtStartPar
name
&
\sphinxAtStartPar
string
&&&
\sphinxAtStartPar
4
\\
\sphinxhline
\sphinxAtStartPar
5
&&&&&
\sphinxAtStartPar
cities{[}{]}
&
\sphinxAtStartPar
backref
&
\sphinxAtStartPar
City
&&
\sphinxAtStartPar
4
\\
\sphinxhline
\sphinxAtStartPar
6
&&&&\sphinxstartmulticolumn{2}%
\begin{varwidth}[t]{\sphinxcolwidth{2}{10}}
\sphinxAtStartPar
City
\par
\vskip-\baselineskip\vbox{\hbox{\strut}}\end{varwidth}%
\sphinxstopmulticolumn
&&
\sphinxAtStartPar
id
&&
\sphinxAtStartPar
4
\\
\sphinxhline
\sphinxAtStartPar
7
&&&&&
\sphinxAtStartPar
id
&
\sphinxAtStartPar
integer
&&&
\sphinxAtStartPar
4
\\
\sphinxhline
\sphinxAtStartPar
8
&&&&&
\sphinxAtStartPar
name
&
\sphinxAtStartPar
string
&&&
\sphinxAtStartPar
4
\\
\sphinxhline
\sphinxAtStartPar
9
&&&&&
\sphinxAtStartPar
country
&
\sphinxAtStartPar
ref
&
\sphinxAtStartPar
Country
&&
\sphinxAtStartPar
4
\\
\sphinxhline
\sphinxAtStartPar
10
&&&&\sphinxstartmulticolumn{2}%
\begin{varwidth}[t]{\sphinxcolwidth{2}{10}}
\sphinxAtStartPar
Event
\par
\vskip-\baselineskip\vbox{\hbox{\strut}}\end{varwidth}%
\sphinxstopmulticolumn
&&
\sphinxAtStartPar
id
&&
\sphinxAtStartPar
4
\\
\sphinxhline
\sphinxAtStartPar
11
&&&&&
\sphinxAtStartPar
id
&
\sphinxAtStartPar
integer
&&&
\sphinxAtStartPar
4
\\
\sphinxhline
\sphinxAtStartPar
12
&&&&&
\sphinxAtStartPar
name
&
\sphinxAtStartPar
string
&&&
\sphinxAtStartPar
4
\\
\sphinxhline
\sphinxAtStartPar
13
&&&&&
\sphinxAtStartPar
object\_id
&
\sphinxAtStartPar
integer
&&&
\sphinxAtStartPar
4
\\
\sphinxhline
\sphinxAtStartPar
14
&&&&&
\sphinxAtStartPar
object\_model
&
\sphinxAtStartPar
string
&&&
\sphinxAtStartPar
4
\\
\sphinxhline
\sphinxAtStartPar
15
&&&&&
\sphinxAtStartPar
object
&
\sphinxAtStartPar
generic
&
\sphinxAtStartPar
Country
&
\sphinxAtStartPar
object\_model, object\_id
&
\sphinxAtStartPar
4
\\
\sphinxhline
\sphinxAtStartPar
16
&&&&&&&
\sphinxAtStartPar
City
&&\\
\sphinxbottomrule
\end{tabulary}
\sphinxtableafterendhook\par
\sphinxattableend\end{savenotes}

\sphinxAtStartPar
15\sphinxhyphen{}oje eilutėje įtrauktas virtualus \sphinxcode{\sphinxupquote{Event.object}} laukas, kuris 15\sphinxhyphen{}oje ir
16\sphinxhyphen{}oje eilutėse, {\hyperref[\detokenize{dimensijos:property.ref}]{\sphinxcrossref{\sphinxcode{\sphinxupquote{property.ref}}}}} stulpelyje išvardina du modelius
\sphinxcode{\sphinxupquote{Country}} ir \sphinxcode{\sphinxupquote{City}}, su kuriais jungiamas šis laukas, per \sphinxcode{\sphinxupquote{object\_model}} ir
\sphinxcode{\sphinxupquote{object\_id}} laukus, kurie aprašyti atskirai.

\sphinxAtStartPar
\sphinxcode{\sphinxupquote{object\_id}} ir \sphinxcode{\sphinxupquote{object\_model}} aprašomi atskirai tik todėl, kad duomenys
ateina iš išorinio šaltinio. Jei duomenys rašomi tiesiogiai į \DUrole{xref}{\DUrole{std}{\DUrole{std-ref}{Saugyklą}}}, tada atskirai \sphinxcode{\sphinxupquote{generic}} laukų apsirašyti nereikia.


\subsection{Brandos lygis}
\label{\detokenize{identifikatoriai:brandos-lygis}}\label{\detokenize{identifikatoriai:ref-level}}
\sphinxAtStartPar
Apibrėžiant ryšius tarp modelių, brandos lygis įrašomas {\hyperref[\detokenize{formatas:level}]{\sphinxcrossref{\sphinxcode{\sphinxupquote{level}}}}}
stulpelyje atlieka svarbų vaidmenį. Nuo brandos lygio, priklauso, kaip turi būti
interpretuojamas išorinis raktas, siejamas su kitu modeliu.
\begin{description}
\sphinxlineitem{1 brandos lygis: Susiejimas neįmanomas}
\sphinxAtStartPar
Duomenys pateikti tokia forma, kurios pagalba dviejų modelių jungimas nėra
įmanomas.

\sphinxAtStartPar
Pavyzdžiui, pateikta adreso tekstinė forma, kuri nesutampa su tekstine
forma pateikiama oficialiame adresų registre arba naudojamas toks tam
tikras identifikatorius, kuris nėra surištas su siejamo modelio pirminiu
raktu.

\sphinxlineitem{2 brandos lygis: Susiejimas nepatikimas}
\sphinxAtStartPar
Duomenys pateikiami tam tikra forma, kuri neužtikrina patikimo duomenų
susiejimo, tačiau siejimas atliekamas pagal siejamo modelio atributą, kuris
negarantuoja unikalaus objekto identifikavimo.

\sphinxAtStartPar
Pavyzdžiui siejimas atliekamas pagal pavadinimą, kuris gali keistis arba ne
visais atvejais sutampa.

\sphinxlineitem{3 brandos lygis: Susiejimas ne per pirminį raktą}
\sphinxAtStartPar
Duomenims susieti naudojamas patikimas identifikatorius, kuris yra surištas su
siejamo modelio pirminiu raktu, tačiau naudojamas ne pirminis raktas, o
kitas identifikatorius.

\sphinxlineitem{4 brandos lygis: Susiejimas per pirminį raktą}
\sphinxAtStartPar
Susiejimas daromas per pirminį raktą.

\end{description}


\subsubsection{Susiejimas neįmanomas}
\label{\detokenize{identifikatoriai:susiejimas-neimanomas}}
\sphinxAtStartPar
Jei \sphinxcode{\sphinxupquote{ref}} tipui nurodytas 1 arba žemesnis brandos lygis, tai reiškia, duomenų
jungimas nėra įmanomas. Tokiu atveju, atveriant duomenis, \sphinxcode{\sphinxupquote{property}} įgaus tokį
tipą, koks yra lauko su kuriuo siejamas ryšys tipas.

\sphinxAtStartPar
Pavyzdžiui:


\begin{savenotes}\sphinxattablestart
\sphinxthistablewithglobalstyle
\centering
\begin{tabulary}{\linewidth}[t]{TTTTTTTT}
\sphinxtoprule
\sphinxstyletheadfamily 
\sphinxAtStartPar
d
&\sphinxstyletheadfamily 
\sphinxAtStartPar
r
&\sphinxstyletheadfamily 
\sphinxAtStartPar
b
&\sphinxstyletheadfamily 
\sphinxAtStartPar
m
&\sphinxstyletheadfamily 
\sphinxAtStartPar
property
&\sphinxstyletheadfamily 
\sphinxAtStartPar
type
&\sphinxstyletheadfamily 
\sphinxAtStartPar
ref
&\sphinxstyletheadfamily 
\sphinxAtStartPar
level
\\
\sphinxmidrule
\sphinxtableatstartofbodyhook\sphinxstartmulticolumn{5}%
\begin{varwidth}[t]{\sphinxcolwidth{5}{8}}
\sphinxAtStartPar
example
\par
\vskip-\baselineskip\vbox{\hbox{\strut}}\end{varwidth}%
\sphinxstopmulticolumn
&&&\\
\sphinxhline
\sphinxAtStartPar

&&&\sphinxstartmulticolumn{2}%
\begin{varwidth}[t]{\sphinxcolwidth{2}{8}}
\sphinxAtStartPar
Country
\par
\vskip-\baselineskip\vbox{\hbox{\strut}}\end{varwidth}%
\sphinxstopmulticolumn
&&
\sphinxAtStartPar
name@lt
&
\sphinxAtStartPar
4
\\
\sphinxhline
\sphinxAtStartPar

&&&&
\sphinxAtStartPar
name@lt
&
\sphinxAtStartPar
string
&&
\sphinxAtStartPar
4
\\
\sphinxhline
\sphinxAtStartPar

&&&\sphinxstartmulticolumn{2}%
\begin{varwidth}[t]{\sphinxcolwidth{2}{8}}
\sphinxAtStartPar
City
\par
\vskip-\baselineskip\vbox{\hbox{\strut}}\end{varwidth}%
\sphinxstopmulticolumn
&&
\sphinxAtStartPar
name
&
\sphinxAtStartPar
4
\\
\sphinxhline
\sphinxAtStartPar

&&&&
\sphinxAtStartPar
name@lt
&
\sphinxAtStartPar
string
&&
\sphinxAtStartPar
4
\\
\sphinxhline
\sphinxAtStartPar

&&&&
\sphinxAtStartPar
country
&
\sphinxAtStartPar
ref
&
\sphinxAtStartPar
Country
&
\sphinxAtStartPar
1
\\
\sphinxbottomrule
\end{tabulary}
\sphinxtableafterendhook\par
\sphinxattableend\end{savenotes}

\sphinxAtStartPar
Šiuo atveju, \sphinxcode{\sphinxupquote{City.country}} yra siejamas su \sphinxcode{\sphinxupquote{Country.name}}. Kadangi
\sphinxcode{\sphinxupquote{City.country}} brandos lygis yra 2, tai reiškia, kad \sphinxcode{\sphinxupquote{City.country}} ir
\sphinxcode{\sphinxupquote{Country.name}} pavadinimai nesutampa ir jungimo atlikti neįmanoma. Tokiu
atveju, \sphinxcode{\sphinxupquote{City.country}} tipas bus ne \sphinxcode{\sphinxupquote{ref}}, o toks pat, kaip \sphinxcode{\sphinxupquote{Country.name}},
t.y. \sphinxcode{\sphinxupquote{string}}.

\sphinxAtStartPar
Tačiau, metaduomenyse išliks informacija, apie tai, kad šios lentelės yra
susijusios. Dėl prasto duomenų brandos lygios, realus susiejimas nėra
įmanomas.

\sphinxAtStartPar
Jei modeliai yra susiję, tačiau, tokio duomenų lauko, per kurį galima būtų
atlikti susiejimą iš vis nėra, tuomet, tokį lauką galima sukurti, nurodant
brandos lygį 0. Pavyzdžiui:


\begin{savenotes}\sphinxattablestart
\sphinxthistablewithglobalstyle
\centering
\begin{tabulary}{\linewidth}[t]{TTTTTTTT}
\sphinxtoprule
\sphinxstyletheadfamily 
\sphinxAtStartPar
d
&\sphinxstyletheadfamily 
\sphinxAtStartPar
r
&\sphinxstyletheadfamily 
\sphinxAtStartPar
b
&\sphinxstyletheadfamily 
\sphinxAtStartPar
m
&\sphinxstyletheadfamily 
\sphinxAtStartPar
property
&\sphinxstyletheadfamily 
\sphinxAtStartPar
type
&\sphinxstyletheadfamily 
\sphinxAtStartPar
ref
&\sphinxstyletheadfamily 
\sphinxAtStartPar
level
\\
\sphinxmidrule
\sphinxtableatstartofbodyhook\sphinxstartmulticolumn{5}%
\begin{varwidth}[t]{\sphinxcolwidth{5}{8}}
\sphinxAtStartPar
example
\par
\vskip-\baselineskip\vbox{\hbox{\strut}}\end{varwidth}%
\sphinxstopmulticolumn
&&&\\
\sphinxhline
\sphinxAtStartPar

&&&\sphinxstartmulticolumn{2}%
\begin{varwidth}[t]{\sphinxcolwidth{2}{8}}
\sphinxAtStartPar
Country
\par
\vskip-\baselineskip\vbox{\hbox{\strut}}\end{varwidth}%
\sphinxstopmulticolumn
&&
\sphinxAtStartPar
name@lt
&
\sphinxAtStartPar
4
\\
\sphinxhline
\sphinxAtStartPar

&&&&
\sphinxAtStartPar
name@lt
&
\sphinxAtStartPar
string
&&
\sphinxAtStartPar
4
\\
\sphinxhline
\sphinxAtStartPar

&&&&
\sphinxAtStartPar
name@en
&
\sphinxAtStartPar
string
&&
\sphinxAtStartPar
0
\\
\sphinxhline
\sphinxAtStartPar

&&&\sphinxstartmulticolumn{2}%
\begin{varwidth}[t]{\sphinxcolwidth{2}{8}}
\sphinxAtStartPar
City
\par
\vskip-\baselineskip\vbox{\hbox{\strut}}\end{varwidth}%
\sphinxstopmulticolumn
&&
\sphinxAtStartPar
name
&
\sphinxAtStartPar
4
\\
\sphinxhline
\sphinxAtStartPar

&&&&
\sphinxAtStartPar
name@en
&
\sphinxAtStartPar
string
&&
\sphinxAtStartPar
4
\\
\sphinxhline
\sphinxAtStartPar

&&&&
\sphinxAtStartPar
country
&
\sphinxAtStartPar
ref
&
\sphinxAtStartPar
Country{[}name@en{]}
&
\sphinxAtStartPar
1
\\
\sphinxbottomrule
\end{tabulary}
\sphinxtableafterendhook\par
\sphinxattableend\end{savenotes}

\sphinxAtStartPar
Šioje vietoje \sphinxcode{\sphinxupquote{City.country}} tampa \sphinxcode{\sphinxupquote{country@en}}, kurio tipas yra \sphinxcode{\sphinxupquote{string}}. O į
\sphinxcode{\sphinxupquote{Country}} yra įtrauktas papildomas laukas \sphinxcode{\sphinxupquote{name@en}}, per kurį ir atliekamas
susiejimas, t.y. per kurį galėtų būti atliktas susiejimas, jei toks laukas
egzistuotų ne tik \sphinxcode{\sphinxupquote{City.country}}, bet ir \sphinxcode{\sphinxupquote{Country.name@en}}.


\subsubsection{Susiejimas nepatikimas}
\label{\detokenize{identifikatoriai:susiejimas-nepatikimas}}
\sphinxAtStartPar
Jei \sphinxcode{\sphinxupquote{ref}} tipui suteiktas 2 brandos lygis, tai reiškia, kad susiejimas yra
įmanomas, tačiau nėra garantijos, kad jis veiks visais atvejais.

\sphinxAtStartPar
Susiejimas laikomas nepatikimu, tada, kai siejimas atliekamas ne patikimo
unikalaus identifikatoriaus pagalba, o per pavadinimą ar panašiais būdais.

\sphinxAtStartPar
Pavadinimai gali keistis, gali dubliuotis, gali skirtis jų užrašymo forma, todėl
toks jungimas laikomas nepatikimu.

\sphinxAtStartPar
Toks jungimas ir 2 brandos lygio žymėjimas taikomas tik tais atvejais, kai
jungimas daromas, per jungiamo modelio atributą. Pavyzdžiui:


\begin{savenotes}\sphinxattablestart
\sphinxthistablewithglobalstyle
\centering
\begin{tabulary}{\linewidth}[t]{TTTTTTTT}
\sphinxtoprule
\sphinxstyletheadfamily 
\sphinxAtStartPar
d
&\sphinxstyletheadfamily 
\sphinxAtStartPar
r
&\sphinxstyletheadfamily 
\sphinxAtStartPar
b
&\sphinxstyletheadfamily 
\sphinxAtStartPar
m
&\sphinxstyletheadfamily 
\sphinxAtStartPar
property
&\sphinxstyletheadfamily 
\sphinxAtStartPar
type
&\sphinxstyletheadfamily 
\sphinxAtStartPar
ref
&\sphinxstyletheadfamily 
\sphinxAtStartPar
level
\\
\sphinxmidrule
\sphinxtableatstartofbodyhook\sphinxstartmulticolumn{5}%
\begin{varwidth}[t]{\sphinxcolwidth{5}{8}}
\sphinxAtStartPar
example
\par
\vskip-\baselineskip\vbox{\hbox{\strut}}\end{varwidth}%
\sphinxstopmulticolumn
&&&\\
\sphinxhline
\sphinxAtStartPar

&&&\sphinxstartmulticolumn{2}%
\begin{varwidth}[t]{\sphinxcolwidth{2}{8}}
\sphinxAtStartPar
Country
\par
\vskip-\baselineskip\vbox{\hbox{\strut}}\end{varwidth}%
\sphinxstopmulticolumn
&&
\sphinxAtStartPar
name@lt
&
\sphinxAtStartPar
4
\\
\sphinxhline
\sphinxAtStartPar

&&&&
\sphinxAtStartPar
name@lt
&
\sphinxAtStartPar
string
&&
\sphinxAtStartPar
4
\\
\sphinxhline
\sphinxAtStartPar

&&&\sphinxstartmulticolumn{2}%
\begin{varwidth}[t]{\sphinxcolwidth{2}{8}}
\sphinxAtStartPar
City
\par
\vskip-\baselineskip\vbox{\hbox{\strut}}\end{varwidth}%
\sphinxstopmulticolumn
&&
\sphinxAtStartPar
name
&
\sphinxAtStartPar
4
\\
\sphinxhline
\sphinxAtStartPar

&&&&
\sphinxAtStartPar
name@lt
&
\sphinxAtStartPar
string
&&
\sphinxAtStartPar
4
\\
\sphinxhline
\sphinxAtStartPar

&&&&
\sphinxAtStartPar
country
&
\sphinxAtStartPar
ref
&
\sphinxAtStartPar
Country
&
\sphinxAtStartPar
2
\\
\sphinxbottomrule
\end{tabulary}
\sphinxtableafterendhook\par
\sphinxattableend\end{savenotes}

\sphinxAtStartPar
Šiuo atveju, kadangi \sphinxcode{\sphinxupquote{City.country}} brandos lygis yra \sphinxcode{\sphinxupquote{2}}, tai reiškia, kad
\sphinxcode{\sphinxupquote{City.country}} duomenys yra realiai paimti iš \sphinxcode{\sphinxupquote{Country.name@lt}}. Jei
\sphinxcode{\sphinxupquote{City.country}} būtų paimti ne iš \sphinxcode{\sphinxupquote{Country.name@lt}}, o iš kokio nors kito
šaltinio ir gali nesutapti, tada brandos lygis turėtu būti \sphinxcode{\sphinxupquote{1}}.

\sphinxAtStartPar
Tai reiškia, kad \sphinxcode{\sphinxupquote{2}} brandos lygis žymimas tik tais atvejais, kai išorinis
raktas yra paimtas iš siejamo modelio atributo.


\subsubsection{Susiejimas ne per pirminį raktą}
\label{\detokenize{identifikatoriai:susiejimas-ne-per-pirmini-rakta}}
\sphinxAtStartPar
Jei \sphinxcode{\sphinxupquote{ref}} tipui suteiktas 3 ar didesnis brandos lygis, vadinasi susiejimas yra
patikimas. Duomenys siejami naudojant patikimus unikalius identifikatorius,
kurie nesidubliuoja, nesikeičia ir užrašomi visada vienodai.

\sphinxAtStartPar
Dažniausiai patikimais identifikatoriais laikomi sveiki skaičiai, tam tikri
sutartiniai kodai ir kiti specializuoti identifikatoriai, tokie kaip UUID.

\sphinxAtStartPar
Tačiau, naudojamas ne pirminis raktas, o kitas duomenų laukas. Pavyzdžiui:


\begin{savenotes}\sphinxattablestart
\sphinxthistablewithglobalstyle
\centering
\begin{tabulary}{\linewidth}[t]{TTTTTTTT}
\sphinxtoprule
\sphinxstyletheadfamily 
\sphinxAtStartPar
d
&\sphinxstyletheadfamily 
\sphinxAtStartPar
r
&\sphinxstyletheadfamily 
\sphinxAtStartPar
b
&\sphinxstyletheadfamily 
\sphinxAtStartPar
m
&\sphinxstyletheadfamily 
\sphinxAtStartPar
property
&\sphinxstyletheadfamily 
\sphinxAtStartPar
type
&\sphinxstyletheadfamily 
\sphinxAtStartPar
ref
&\sphinxstyletheadfamily 
\sphinxAtStartPar
level
\\
\sphinxmidrule
\sphinxtableatstartofbodyhook\sphinxstartmulticolumn{5}%
\begin{varwidth}[t]{\sphinxcolwidth{5}{8}}
\sphinxAtStartPar
example
\par
\vskip-\baselineskip\vbox{\hbox{\strut}}\end{varwidth}%
\sphinxstopmulticolumn
&&&\\
\sphinxhline
\sphinxAtStartPar

&&&\sphinxstartmulticolumn{2}%
\begin{varwidth}[t]{\sphinxcolwidth{2}{8}}
\sphinxAtStartPar
Country
\par
\vskip-\baselineskip\vbox{\hbox{\strut}}\end{varwidth}%
\sphinxstopmulticolumn
&&
\sphinxAtStartPar
id
&
\sphinxAtStartPar
4
\\
\sphinxhline
\sphinxAtStartPar

&&&&
\sphinxAtStartPar
id
&
\sphinxAtStartPar
integer
&&
\sphinxAtStartPar
4
\\
\sphinxhline
\sphinxAtStartPar

&&&&
\sphinxAtStartPar
code
&
\sphinxAtStartPar
string
&&
\sphinxAtStartPar
4
\\
\sphinxhline
\sphinxAtStartPar

&&&&
\sphinxAtStartPar
name@lt
&
\sphinxAtStartPar
string
&&
\sphinxAtStartPar
4
\\
\sphinxhline
\sphinxAtStartPar

&&&\sphinxstartmulticolumn{2}%
\begin{varwidth}[t]{\sphinxcolwidth{2}{8}}
\sphinxAtStartPar
City
\par
\vskip-\baselineskip\vbox{\hbox{\strut}}\end{varwidth}%
\sphinxstopmulticolumn
&&
\sphinxAtStartPar
name
&
\sphinxAtStartPar
4
\\
\sphinxhline
\sphinxAtStartPar

&&&&
\sphinxAtStartPar
name@lt
&
\sphinxAtStartPar
string
&&
\sphinxAtStartPar
4
\\
\sphinxhline
\sphinxAtStartPar

&&&&
\sphinxAtStartPar
country
&
\sphinxAtStartPar
ref
&
\sphinxAtStartPar
Country{[}code{]}
&
\sphinxAtStartPar
3
\\
\sphinxbottomrule
\end{tabulary}
\sphinxtableafterendhook\par
\sphinxattableend\end{savenotes}

\sphinxAtStartPar
Skirtumas tarp \sphinxcode{\sphinxupquote{3}} ir \sphinxcode{\sphinxupquote{4}} brandos lygio iš esmės susijęs su duomenų saugojimu
Saugykloje ar kitoje vietoje, kur pirminiai raktai yra generuojami ir jų
negalima keisti. Jei naudojamas \sphinxcode{\sphinxupquote{3}} brandos lygis, tuomet saugykloje saugomas,
ne išorinis saugyklos identifikatorius UUID, o vidinis duomenų rinkinio
identifikatorius.

\sphinxAtStartPar
Publikuojant duomenis iš tam tikro šaltinio, išorinis raktas visada turėtu
būti konvertuojamas į išorinį pirminį raktą, tačiau tais atvejais, jei dėl tam
tikrų priežasčių tas nėra daroma, tuomet žymimas 3 brandos lygis ir
publikuojami ne išoriniai pirminiai raktai, o šaltinio vidiniai.

\sphinxAtStartPar
Pavyzdžiui, jei turime tokius duomenis:


\begin{savenotes}\sphinxattablestart
\sphinxthistablewithglobalstyle
\centering
\begin{tabulary}{\linewidth}[t]{TTTT}
\sphinxtoprule
\sphinxstartmulticolumn{4}%
\begin{varwidth}[t]{\sphinxcolwidth{4}{4}}
\sphinxstyletheadfamily \sphinxAtStartPar
example/Country
\par
\vskip-\baselineskip\vbox{\hbox{\strut}}\end{varwidth}%
\sphinxstopmulticolumn
\\
\sphinxhline\sphinxstyletheadfamily 
\sphinxAtStartPar
\_id
&\sphinxstyletheadfamily 
\sphinxAtStartPar
id
&\sphinxstyletheadfamily 
\sphinxAtStartPar
code
&\sphinxstyletheadfamily 
\sphinxAtStartPar
name@lt
\\
\sphinxmidrule
\sphinxtableatstartofbodyhook
\sphinxAtStartPar
4dbb1b77\sphinxhyphen{}a930\sphinxhyphen{}4f2a\sphinxhyphen{}8ef4\sphinxhyphen{}f05b89f0fcfe
&
\sphinxAtStartPar
1
&
\sphinxAtStartPar
lt
&
\sphinxAtStartPar
Lietuva
\\
\sphinxbottomrule
\end{tabulary}
\sphinxtableafterendhook\par
\sphinxattableend\end{savenotes}

\sphinxAtStartPar
Ir jei \sphinxcode{\sphinxupquote{City.country}} turi brandos lygį \sphinxcode{\sphinxupquote{3}}, tada \sphinxcode{\sphinxupquote{City}} duomenys atrodys taip:


\begin{savenotes}\sphinxattablestart
\sphinxthistablewithglobalstyle
\centering
\begin{tabulary}{\linewidth}[t]{TTT}
\sphinxtoprule
\sphinxstartmulticolumn{3}%
\begin{varwidth}[t]{\sphinxcolwidth{3}{3}}
\sphinxstyletheadfamily \sphinxAtStartPar
example/City
\par
\vskip-\baselineskip\vbox{\hbox{\strut}}\end{varwidth}%
\sphinxstopmulticolumn
\\
\sphinxhline\sphinxstyletheadfamily 
\sphinxAtStartPar
\_id
&\sphinxstyletheadfamily 
\sphinxAtStartPar
name@lt
&\sphinxstyletheadfamily 
\sphinxAtStartPar
country.\_id
\\
\sphinxmidrule
\sphinxtableatstartofbodyhook
\sphinxAtStartPar
096e054e\sphinxhyphen{}7a4c\sphinxhyphen{}44cc\sphinxhyphen{}8f27\sphinxhyphen{}98af815080d5
&
\sphinxAtStartPar
Vilnius
&
\sphinxAtStartPar
lt
\\
\sphinxbottomrule
\end{tabulary}
\sphinxtableafterendhook\par
\sphinxattableend\end{savenotes}


\subsubsection{Susiejimas per pirminį raktą}
\label{\detokenize{identifikatoriai:susiejimas-per-pirmini-rakta}}
\sphinxAtStartPar
Šiuo atveju, brandos lygis žymimas \sphinxcode{\sphinxupquote{4}} ir skirtumas nuo \sphinxcode{\sphinxupquote{3}} brandos lygio yra
toks, kad duomenyse naudojamas išorinis pirminis raktas. Pavyzdžiui:


\begin{savenotes}\sphinxattablestart
\sphinxthistablewithglobalstyle
\centering
\begin{tabulary}{\linewidth}[t]{TTTTTTTT}
\sphinxtoprule
\sphinxstyletheadfamily 
\sphinxAtStartPar
d
&\sphinxstyletheadfamily 
\sphinxAtStartPar
r
&\sphinxstyletheadfamily 
\sphinxAtStartPar
b
&\sphinxstyletheadfamily 
\sphinxAtStartPar
m
&\sphinxstyletheadfamily 
\sphinxAtStartPar
property
&\sphinxstyletheadfamily 
\sphinxAtStartPar
type
&\sphinxstyletheadfamily 
\sphinxAtStartPar
ref
&\sphinxstyletheadfamily 
\sphinxAtStartPar
level
\\
\sphinxmidrule
\sphinxtableatstartofbodyhook\sphinxstartmulticolumn{5}%
\begin{varwidth}[t]{\sphinxcolwidth{5}{8}}
\sphinxAtStartPar
example
\par
\vskip-\baselineskip\vbox{\hbox{\strut}}\end{varwidth}%
\sphinxstopmulticolumn
&&&\\
\sphinxhline
\sphinxAtStartPar

&&&\sphinxstartmulticolumn{2}%
\begin{varwidth}[t]{\sphinxcolwidth{2}{8}}
\sphinxAtStartPar
Country
\par
\vskip-\baselineskip\vbox{\hbox{\strut}}\end{varwidth}%
\sphinxstopmulticolumn
&&
\sphinxAtStartPar
id
&
\sphinxAtStartPar
4
\\
\sphinxhline
\sphinxAtStartPar

&&&&
\sphinxAtStartPar
id
&
\sphinxAtStartPar
integer
&&
\sphinxAtStartPar
4
\\
\sphinxhline
\sphinxAtStartPar

&&&&
\sphinxAtStartPar
code
&
\sphinxAtStartPar
string
&&
\sphinxAtStartPar
4
\\
\sphinxhline
\sphinxAtStartPar

&&&&
\sphinxAtStartPar
name@lt
&
\sphinxAtStartPar
string
&&
\sphinxAtStartPar
4
\\
\sphinxhline
\sphinxAtStartPar

&&&\sphinxstartmulticolumn{2}%
\begin{varwidth}[t]{\sphinxcolwidth{2}{8}}
\sphinxAtStartPar
City
\par
\vskip-\baselineskip\vbox{\hbox{\strut}}\end{varwidth}%
\sphinxstopmulticolumn
&&
\sphinxAtStartPar
name
&
\sphinxAtStartPar
4
\\
\sphinxhline
\sphinxAtStartPar

&&&&
\sphinxAtStartPar
name@lt
&
\sphinxAtStartPar
string
&&
\sphinxAtStartPar
4
\\
\sphinxhline
\sphinxAtStartPar

&&&&
\sphinxAtStartPar
country
&
\sphinxAtStartPar
ref
&
\sphinxAtStartPar
Country
&
\sphinxAtStartPar
4
\\
\sphinxbottomrule
\end{tabulary}
\sphinxtableafterendhook\par
\sphinxattableend\end{savenotes}

\sphinxAtStartPar
Turint tokį struktūros aprašą, kur \sphinxcode{\sphinxupquote{City.country}} brandos lygis yra \sphinxcode{\sphinxupquote{4}},
duomenys atrodys taip:


\begin{savenotes}\sphinxattablestart
\sphinxthistablewithglobalstyle
\centering
\begin{tabulary}{\linewidth}[t]{TTTT}
\sphinxtoprule
\sphinxstartmulticolumn{4}%
\begin{varwidth}[t]{\sphinxcolwidth{4}{4}}
\sphinxstyletheadfamily \sphinxAtStartPar
example/Country
\par
\vskip-\baselineskip\vbox{\hbox{\strut}}\end{varwidth}%
\sphinxstopmulticolumn
\\
\sphinxhline\sphinxstyletheadfamily 
\sphinxAtStartPar
\_id
&\sphinxstyletheadfamily 
\sphinxAtStartPar
id
&\sphinxstyletheadfamily 
\sphinxAtStartPar
code
&\sphinxstyletheadfamily 
\sphinxAtStartPar
name@lt
\\
\sphinxmidrule
\sphinxtableatstartofbodyhook
\sphinxAtStartPar
4dbb1b77\sphinxhyphen{}a930\sphinxhyphen{}4f2a\sphinxhyphen{}8ef4\sphinxhyphen{}f05b89f0fcfe
&
\sphinxAtStartPar
1
&
\sphinxAtStartPar
lt
&
\sphinxAtStartPar
Lietuva
\\
\sphinxbottomrule
\end{tabulary}
\sphinxtableafterendhook\par
\sphinxattableend\end{savenotes}


\begin{savenotes}\sphinxattablestart
\sphinxthistablewithglobalstyle
\centering
\begin{tabulary}{\linewidth}[t]{TTT}
\sphinxtoprule
\sphinxstartmulticolumn{3}%
\begin{varwidth}[t]{\sphinxcolwidth{3}{3}}
\sphinxstyletheadfamily \sphinxAtStartPar
example/City
\par
\vskip-\baselineskip\vbox{\hbox{\strut}}\end{varwidth}%
\sphinxstopmulticolumn
\\
\sphinxhline\sphinxstyletheadfamily 
\sphinxAtStartPar
\_id
&\sphinxstyletheadfamily 
\sphinxAtStartPar
name@lt
&\sphinxstyletheadfamily 
\sphinxAtStartPar
country.\_id
\\
\sphinxmidrule
\sphinxtableatstartofbodyhook
\sphinxAtStartPar
096e054e\sphinxhyphen{}7a4c\sphinxhyphen{}44cc\sphinxhyphen{}8f27\sphinxhyphen{}98af815080d5
&
\sphinxAtStartPar
Vilnius
&
\sphinxAtStartPar
4dbb1b77\sphinxhyphen{}a930\sphinxhyphen{}4f2a\sphinxhyphen{}8ef4\sphinxhyphen{}f05b89f0fcfe
\\
\sphinxbottomrule
\end{tabulary}
\sphinxtableafterendhook\par
\sphinxattableend\end{savenotes}

\sphinxAtStartPar
Matome, kad \sphinxcode{\sphinxupquote{City.country.\_id}} yra \sphinxcode{\sphinxupquote{Country}} pirminis raktas. Tai reiškia, kad
vidiniai duomenų rinkinio raktai konvertuojami į išorinius.

\sphinxstepscope


\section{Funkciniai modeliai}
\label{\detokenize{modeliai/funkciniai:funkciniai-modeliai}}\label{\detokenize{modeliai/funkciniai:functional-models}}\label{\detokenize{modeliai/funkciniai::doc}}
\sphinxAtStartPar
Loginis duomenų modelis formuojamas {\hyperref[\detokenize{formatas:model}]{\sphinxcrossref{\sphinxcode{\sphinxupquote{model}}}}} stulpelyje įvardinant
{\hyperref[\detokenize{modelis:uml-index}]{\sphinxcrossref{\DUrole{std}{\DUrole{std-ref}{koncepcinio modelio}}}}} klasės pavadinimą.

\sphinxAtStartPar
{\hyperref[\detokenize{modelis:uml-index}]{\sphinxcrossref{\DUrole{std}{\DUrole{std-ref}{Koncepcinis modelis}}}}} nėra siejamas su jokiu konkrečiu duomenų šaltiniu, tačiau
loginis modelis yra siejamas su konkrečiu duomenų šaltiniu, tačiau išlaiko
koncepciniame modelyje apibrėžtą duomenų struktūrą.

\sphinxAtStartPar
Dažnai viena koncepcinio modelio klasė, loginio modelio pagalba yra siejama su
keliais skirtingais duomenų šaltinio prieigos taškais, kurie įgyvendina
skirtingas duomenų gavimo funkcijas arba veiksmus.

\sphinxAtStartPar
Tarkime, jei turime tokį koncepcinį modelį:

\sphinxAtStartPar
Ir turime duomenų šaltinį, kuris leidžia duomenis gauti skirtingais metodais:


\begin{savenotes}\sphinxattablestart
\sphinxthistablewithglobalstyle
\centering
\begin{tabulary}{\linewidth}[t]{TTTTTTTT}
\sphinxtoprule
\sphinxstyletheadfamily 
\sphinxAtStartPar
dataset
&\sphinxstyletheadfamily 
\sphinxAtStartPar
resource
&\sphinxstyletheadfamily 
\sphinxAtStartPar
model
&\sphinxstyletheadfamily 
\sphinxAtStartPar
property
&\sphinxstyletheadfamily 
\sphinxAtStartPar
type
&\sphinxstyletheadfamily 
\sphinxAtStartPar
ref
&\sphinxstyletheadfamily 
\sphinxAtStartPar
source
&\sphinxstyletheadfamily 
\sphinxAtStartPar
prepare
\\
\sphinxmidrule
\sphinxtableatstartofbodyhook\sphinxstartmulticolumn{4}%
\begin{varwidth}[t]{\sphinxcolwidth{4}{8}}
\sphinxAtStartPar
datasets/gov/example
\par
\vskip-\baselineskip\vbox{\hbox{\strut}}\end{varwidth}%
\sphinxstopmulticolumn
&&&&\\
\sphinxhline
\sphinxAtStartPar

&&\sphinxstartmulticolumn{2}%
\begin{varwidth}[t]{\sphinxcolwidth{2}{8}}
\sphinxAtStartPar
\sphinxstylestrong{City}
\par
\vskip-\baselineskip\vbox{\hbox{\strut}}\end{varwidth}%
\sphinxstopmulticolumn
&&
\sphinxAtStartPar
code
&&\\
\sphinxhline
\sphinxAtStartPar

&&&
\sphinxAtStartPar
code
&
\sphinxAtStartPar
integer
&&&\\
\sphinxhline
\sphinxAtStartPar

&&&
\sphinxAtStartPar
name@lt
&
\sphinxAtStartPar
string
&&&\\
\sphinxhline
\sphinxAtStartPar

&\sphinxstartmulticolumn{3}%
\begin{varwidth}[t]{\sphinxcolwidth{3}{8}}
\sphinxAtStartPar
get\_all\_cities
\par
\vskip-\baselineskip\vbox{\hbox{\strut}}\end{varwidth}%
\sphinxstopmulticolumn
&
\sphinxAtStartPar
json
&&
\sphinxAtStartPar
/cities
&\\
\sphinxhline
\sphinxAtStartPar

&\sphinxstartmulticolumn{3}%
\begin{varwidth}[t]{\sphinxcolwidth{3}{8}}
\sphinxAtStartPar
get\_city\_by\_code
\par
\vskip-\baselineskip\vbox{\hbox{\strut}}\end{varwidth}%
\sphinxstopmulticolumn
&
\sphinxAtStartPar
json
&&
\sphinxAtStartPar
/cities/\{code\}
&\\
\sphinxbottomrule
\end{tabulary}
\sphinxtableafterendhook\par
\sphinxattableend\end{savenotes}

\sphinxAtStartPar
Šiame pavyzdyje turime duomenų struktūros aprašą, kuriame yra \sphinxcode{\sphinxupquote{City}} modelis,
atitinkantis koncepcinį modelį, nesusietas su jokiu duomenų šaltinius ir du
duomenų šaltiniai \sphinxcode{\sphinxupquote{get\_all\_cities}} ir \sphinxcode{\sphinxupquote{get\_city\_by\_code}}, nesusieti su
loginiu modeliu.

\sphinxAtStartPar
Norint susieti \sphinxcode{\sphinxupquote{get\_all\_cities}} ir \sphinxcode{\sphinxupquote{get\_city\_by\_code}} duomenų šaltinius su
loginiu modeliu, mums reikia panaudoti funkcinius modelius, kadangi duomenų
šaltinis įgyvendina tik dalį funkcionalumo duomenims gauti.

\sphinxAtStartPar
Galutinis pilnai susietas struktūros aprašas atrodys taip:


\begin{savenotes}\sphinxattablestart
\sphinxthistablewithglobalstyle
\centering
\begin{tabulary}{\linewidth}[t]{TTTTTTTT}
\sphinxtoprule
\sphinxstyletheadfamily 
\sphinxAtStartPar
dataset
&\sphinxstyletheadfamily 
\sphinxAtStartPar
resource
&\sphinxstyletheadfamily 
\sphinxAtStartPar
model
&\sphinxstyletheadfamily 
\sphinxAtStartPar
property
&\sphinxstyletheadfamily 
\sphinxAtStartPar
type
&\sphinxstyletheadfamily 
\sphinxAtStartPar
ref
&\sphinxstyletheadfamily 
\sphinxAtStartPar
source
&\sphinxstyletheadfamily 
\sphinxAtStartPar
prepare
\\
\sphinxmidrule
\sphinxtableatstartofbodyhook\sphinxstartmulticolumn{4}%
\begin{varwidth}[t]{\sphinxcolwidth{4}{8}}
\sphinxAtStartPar
datasets/gov/example
\par
\vskip-\baselineskip\vbox{\hbox{\strut}}\end{varwidth}%
\sphinxstopmulticolumn
&&&&\\
\sphinxhline
\sphinxAtStartPar

&&\sphinxstartmulticolumn{2}%
\begin{varwidth}[t]{\sphinxcolwidth{2}{8}}
\sphinxAtStartPar
\sphinxstylestrong{City}
\par
\vskip-\baselineskip\vbox{\hbox{\strut}}\end{varwidth}%
\sphinxstopmulticolumn
&&
\sphinxAtStartPar
code
&&\\
\sphinxhline
\sphinxAtStartPar

&&&
\sphinxAtStartPar
code
&
\sphinxAtStartPar
integer
&&
\sphinxAtStartPar
code
&\\
\sphinxhline
\sphinxAtStartPar

&&&
\sphinxAtStartPar
name@lt
&
\sphinxAtStartPar
string
&&
\sphinxAtStartPar
title
&\\
\sphinxhline
\sphinxAtStartPar

&\sphinxstartmulticolumn{3}%
\begin{varwidth}[t]{\sphinxcolwidth{3}{8}}
\sphinxAtStartPar
get\_all\_cities
\par
\vskip-\baselineskip\vbox{\hbox{\strut}}\end{varwidth}%
\sphinxstopmulticolumn
&
\sphinxAtStartPar
json
&&
\sphinxAtStartPar
/cities
&\\
\sphinxhline
\sphinxAtStartPar

&&\sphinxstartmulticolumn{2}%
\begin{varwidth}[t]{\sphinxcolwidth{2}{8}}
\sphinxAtStartPar
City/:getall
\par
\vskip-\baselineskip\vbox{\hbox{\strut}}\end{varwidth}%
\sphinxstopmulticolumn
&&&
\sphinxAtStartPar
data
&\\
\sphinxhline
\sphinxAtStartPar

&\sphinxstartmulticolumn{3}%
\begin{varwidth}[t]{\sphinxcolwidth{3}{8}}
\sphinxAtStartPar
get\_city\_by\_code
\par
\vskip-\baselineskip\vbox{\hbox{\strut}}\end{varwidth}%
\sphinxstopmulticolumn
&
\sphinxAtStartPar
json
&&
\sphinxAtStartPar
/cities/\{code\}
&\\
\sphinxhline
\sphinxAtStartPar

&&\sphinxstartmulticolumn{2}%
\begin{varwidth}[t]{\sphinxcolwidth{2}{8}}
\sphinxAtStartPar
City/:getone
\par
\vskip-\baselineskip\vbox{\hbox{\strut}}\end{varwidth}%
\sphinxstopmulticolumn
&&&&\\
\sphinxhline
\sphinxAtStartPar

&\sphinxstartmulticolumn{2}%
\begin{varwidth}[t]{\sphinxcolwidth{2}{8}}
\par
\vskip-\baselineskip\vbox{\hbox{\strut}}\end{varwidth}%
\sphinxstopmulticolumn
&
\sphinxAtStartPar
code
&
\sphinxAtStartPar
integer
&&&
\sphinxAtStartPar
path()
\\
\sphinxbottomrule
\end{tabulary}
\sphinxtableafterendhook\par
\sphinxattableend\end{savenotes}

\sphinxAtStartPar
\sphinxcode{\sphinxupquote{City/:getall}} ir \sphinxcode{\sphinxupquote{City/:getone}} yra funkciniai modeliai, nurodantys, kad
duomenų modelis \sphinxcode{\sphinxupquote{City}} yra siejamas su duomenų šaltiniu, kuris įgyvendina tam
tikras duomenų skaitymo funkcijas.

\sphinxAtStartPar
Duomenų skaitymo funkcijos sutampa su \sphinxhref{https://ivpk.github.io/uapi}{UDTS} specifikacijoje aprašytais veiksmais,
kuriuos galima atlikti su duomenimis.

\sphinxAtStartPar
Funkciniai modeliai paveldi visas savybes iš pagrindinio modelio, tačiau gali
papildyti pagrindinį modelį naujomis savybėmis, arba pateikti pagrindinio
modelio savybes su kitais metaduomenis, tarkim nurodant kitokią
{\hyperref[\detokenize{dimensijos:property.source}]{\sphinxcrossref{\sphinxcode{\sphinxupquote{property.source}}}}} reikšmę.

\sphinxAtStartPar
Jei nenurodytas joks funkcinis modelis, daroma prielaida, kad šaltinis palaiko
visas \sphinxhref{https://ivpk.github.io/uapi}{UDTS} funkcijas.


\subsection{Funkcijos}
\label{\detokenize{modeliai/funkciniai:funkcijos}}

\subsubsection{getall}
\label{\detokenize{modeliai/funkciniai:getall}}
\sphinxAtStartPar
Nurodo, kad duomenų šaltinis leidžia gauti visus klasės objektus, netaikant jokių filtrų.

\begin{sphinxadmonition}{note}{Pavyzdys}


\begin{savenotes}\sphinxattablestart
\sphinxthistablewithglobalstyle
\centering
\begin{tabulary}{\linewidth}[t]{TTTTTTTT}
\sphinxtoprule
\sphinxstyletheadfamily 
\sphinxAtStartPar
dataset
&\sphinxstyletheadfamily 
\sphinxAtStartPar
resource
&\sphinxstyletheadfamily 
\sphinxAtStartPar
model
&\sphinxstyletheadfamily 
\sphinxAtStartPar
property
&\sphinxstyletheadfamily 
\sphinxAtStartPar
type
&\sphinxstyletheadfamily 
\sphinxAtStartPar
ref
&\sphinxstyletheadfamily 
\sphinxAtStartPar
source
&\sphinxstyletheadfamily 
\sphinxAtStartPar
prepare
\\
\sphinxmidrule
\sphinxtableatstartofbodyhook\sphinxstartmulticolumn{4}%
\begin{varwidth}[t]{\sphinxcolwidth{4}{8}}
\sphinxAtStartPar
datasets/gov/example
\par
\vskip-\baselineskip\vbox{\hbox{\strut}}\end{varwidth}%
\sphinxstopmulticolumn
&&&&\\
\sphinxhline
\sphinxAtStartPar

&&\sphinxstartmulticolumn{2}%
\begin{varwidth}[t]{\sphinxcolwidth{2}{8}}
\sphinxAtStartPar
\sphinxstylestrong{City}
\par
\vskip-\baselineskip\vbox{\hbox{\strut}}\end{varwidth}%
\sphinxstopmulticolumn
&&
\sphinxAtStartPar
code
&&\\
\sphinxhline
\sphinxAtStartPar

&&&
\sphinxAtStartPar
code
&
\sphinxAtStartPar
integer
&&
\sphinxAtStartPar
code
&\\
\sphinxhline
\sphinxAtStartPar

&&&
\sphinxAtStartPar
name@lt
&
\sphinxAtStartPar
string
&&
\sphinxAtStartPar
title
&\\
\sphinxhline
\sphinxAtStartPar

&\sphinxstartmulticolumn{3}%
\begin{varwidth}[t]{\sphinxcolwidth{3}{8}}
\sphinxAtStartPar
get\_all\_cities
\par
\vskip-\baselineskip\vbox{\hbox{\strut}}\end{varwidth}%
\sphinxstopmulticolumn
&
\sphinxAtStartPar
json
&&
\sphinxAtStartPar
/cities
&\\
\sphinxhline
\sphinxAtStartPar

&&\sphinxstartmulticolumn{2}%
\begin{varwidth}[t]{\sphinxcolwidth{2}{8}}
\sphinxAtStartPar
City/:getall
\par
\vskip-\baselineskip\vbox{\hbox{\strut}}\end{varwidth}%
\sphinxstopmulticolumn
&&&
\sphinxAtStartPar
data
&\\
\sphinxbottomrule
\end{tabulary}
\sphinxtableafterendhook\par
\sphinxattableend\end{savenotes}
\end{sphinxadmonition}


\subsubsection{getone}
\label{\detokenize{modeliai/funkciniai:getone}}
\sphinxAtStartPar
Nurodo, kad duomenų šaltinis leidžia gauti vieną klasės objektą nurodžius
objekto identifikatorių.

\begin{sphinxadmonition}{note}{Pavyzdys}


\begin{savenotes}\sphinxattablestart
\sphinxthistablewithglobalstyle
\centering
\begin{tabulary}{\linewidth}[t]{TTTTTTTT}
\sphinxtoprule
\sphinxstyletheadfamily 
\sphinxAtStartPar
dataset
&\sphinxstyletheadfamily 
\sphinxAtStartPar
resource
&\sphinxstyletheadfamily 
\sphinxAtStartPar
model
&\sphinxstyletheadfamily 
\sphinxAtStartPar
property
&\sphinxstyletheadfamily 
\sphinxAtStartPar
type
&\sphinxstyletheadfamily 
\sphinxAtStartPar
ref
&\sphinxstyletheadfamily 
\sphinxAtStartPar
source
&\sphinxstyletheadfamily 
\sphinxAtStartPar
prepare
\\
\sphinxmidrule
\sphinxtableatstartofbodyhook\sphinxstartmulticolumn{4}%
\begin{varwidth}[t]{\sphinxcolwidth{4}{8}}
\sphinxAtStartPar
datasets/gov/example
\par
\vskip-\baselineskip\vbox{\hbox{\strut}}\end{varwidth}%
\sphinxstopmulticolumn
&&&&\\
\sphinxhline
\sphinxAtStartPar

&&\sphinxstartmulticolumn{2}%
\begin{varwidth}[t]{\sphinxcolwidth{2}{8}}
\sphinxAtStartPar
\sphinxstylestrong{City}
\par
\vskip-\baselineskip\vbox{\hbox{\strut}}\end{varwidth}%
\sphinxstopmulticolumn
&&
\sphinxAtStartPar
code
&&\\
\sphinxhline
\sphinxAtStartPar

&&&
\sphinxAtStartPar
code
&
\sphinxAtStartPar
integer
&&
\sphinxAtStartPar
code
&\\
\sphinxhline
\sphinxAtStartPar

&&&
\sphinxAtStartPar
name@lt
&
\sphinxAtStartPar
string
&&
\sphinxAtStartPar
title
&\\
\sphinxhline
\sphinxAtStartPar

&\sphinxstartmulticolumn{3}%
\begin{varwidth}[t]{\sphinxcolwidth{3}{8}}
\sphinxAtStartPar
get\_city\_by\_code
\par
\vskip-\baselineskip\vbox{\hbox{\strut}}\end{varwidth}%
\sphinxstopmulticolumn
&
\sphinxAtStartPar
json
&&
\sphinxAtStartPar
/cities/\{code\}
&\\
\sphinxhline
\sphinxAtStartPar

&&\sphinxstartmulticolumn{2}%
\begin{varwidth}[t]{\sphinxcolwidth{2}{8}}
\sphinxAtStartPar
City/:getone
\par
\vskip-\baselineskip\vbox{\hbox{\strut}}\end{varwidth}%
\sphinxstopmulticolumn
&&&&\\
\sphinxhline
\sphinxAtStartPar

&\sphinxstartmulticolumn{2}%
\begin{varwidth}[t]{\sphinxcolwidth{2}{8}}
\par
\vskip-\baselineskip\vbox{\hbox{\strut}}\end{varwidth}%
\sphinxstopmulticolumn
&
\sphinxAtStartPar
code
&
\sphinxAtStartPar
integer
&&&
\sphinxAtStartPar
path()
\\
\sphinxbottomrule
\end{tabulary}
\sphinxtableafterendhook\par
\sphinxattableend\end{savenotes}
\end{sphinxadmonition}

\sphinxAtStartPar
\sphinxcode{\sphinxupquote{getone}} veiksmo atveju, modelis turi turėti pirminį raktą, nurodytą
{\hyperref[\detokenize{dimensijos:model.ref}]{\sphinxcrossref{\sphinxcode{\sphinxupquote{model.ref}}}}} stulpelyje. Pirminis raktas gali būti paveldimas iš
pagrindinio modelio arba gali būti nurodomas kitas pirminis raktas prie
funkcinio modelio.

\sphinxAtStartPar
Sąsajai su duomenų šaltiniu, pirminio rakto savybės turėtų nurodyti funkciją,
kuri siejama su duomenų šaltinio parametrais. Konkrečiai, pavyzdyje aukščiau
prie \sphinxcode{\sphinxupquote{City/:getone/code}} savybės yra nurodyta \sphinxcode{\sphinxupquote{path()}} funkcija
\sphinxcode{\sphinxupquote{proprty.prepare}} stulpelyje, kuri nurodo, kad \sphinxcode{\sphinxupquote{code}} savybė yra
naudojama kaip duomenų šaltinio URI path parametras, pažymėtas
{\hyperref[\detokenize{dimensijos:resource.source}]{\sphinxcrossref{\sphinxcode{\sphinxupquote{resource.source}}}}} stulpelyje \sphinxcode{\sphinxupquote{\{\}}} riestiniuose skliausteliuose, tuo pačiu
pavadinimu, kaip ir {\hyperref[\detokenize{formatas:property}]{\sphinxcrossref{\sphinxcode{\sphinxupquote{property}}}}} pavadinimas.

\sphinxAtStartPar
Per \sphinxhref{https://ivpk.github.io/uapi}{UDTS} protokolą, bus tikimasi gauti tokią užklausią:

\begin{sphinxVerbatim}[commandchars=\\\{\}]
\PYG{n+nf}{GET} \PYG{n+nn}{/datasets/gov/example/City/87a1d91a\PYGZhy{}e00d\PYGZhy{}4981\PYGZhy{}8287\PYGZhy{}d1810243d160} \PYG{k+kr}{HTTP}\PYG{o}{/}\PYG{l+m}{1.1}
\end{sphinxVerbatim}

\sphinxAtStartPar
Tokia \sphinxhref{https://ivpk.github.io/uapi}{UDTS} užklausa, pagal pateiktą duomenų struktūros aprašo pavyzdį, bus
konvertuota į tokią duomenų šaltiniui skirtą užklausą:

\begin{sphinxVerbatim}[commandchars=\\\{\}]
\PYG{n+nf}{GET} \PYG{n+nn}{/cities/42} \PYG{k+kr}{HTTP}\PYG{o}{/}\PYG{l+m}{1.1}
\end{sphinxVerbatim}

\sphinxAtStartPar
Kad tai veiktu, duomenų agentas, turi saugoti išorinių ir vidinių
identifikatorių lentelę, kurioje yra susietas išorinis
\sphinxcode{\sphinxupquote{87a1d91a\sphinxhyphen{}e00d\sphinxhyphen{}4981\sphinxhyphen{}8287\sphinxhyphen{}d1810243d160}} su vidiniu \sphinxcode{\sphinxupquote{42}}, kas leidžia konvertuoti
tarp vidinių ir išorinių identifikatorių.


\begin{savenotes}\sphinxattablestart
\sphinxthistablewithglobalstyle
\centering
\sphinxcapstartof{table}
\sphinxthecaptionisattop
\sphinxcaption{Identifikatorių susiejimas}\label{\detokenize{modeliai/funkciniai:id1}}
\sphinxaftertopcaption
\begin{tabulary}{\linewidth}[t]{TT}
\sphinxtoprule
\sphinxstyletheadfamily 
\sphinxAtStartPar
\_id (išorinis)
&\sphinxstyletheadfamily 
\sphinxAtStartPar
id (vidinis)
\\
\sphinxmidrule
\sphinxtableatstartofbodyhook
\sphinxAtStartPar
87a1d91a\sphinxhyphen{}e00d\sphinxhyphen{}4981\sphinxhyphen{}8287\sphinxhyphen{}d1810243d160
&
\sphinxAtStartPar
42
\\
\sphinxbottomrule
\end{tabulary}
\sphinxtableafterendhook\par
\sphinxattableend\end{savenotes}

\sphinxAtStartPar
Visų duomenų šaltinių lokalūs identifikatoriai, siejami su vienu esybės
išoriniu identifikatoriumi.


\subsubsection{search}
\label{\detokenize{modeliai/funkciniai:search}}
\sphinxAtStartPar
Nurodo, kad duomenų šaltinis leidžia gauti ne visus klasės objektus, o tam
tikrą objektų imtį, pagal nurodytą duomenų filtrą.


\paragraph{Statiniai filtrai}
\label{\detokenize{modeliai/funkciniai:statiniai-filtrai}}
\sphinxAtStartPar
Statiniai filtrai nurodo, kad duomenys pateikiami naudojant konkrečias filtrų
reikšmes nurodytas struktūros apraše.

\begin{sphinxadmonition}{note}{Pavyzdys}


\begin{savenotes}\sphinxattablestart
\sphinxthistablewithglobalstyle
\centering
\begin{tabulary}{\linewidth}[t]{TTTTTTTT}
\sphinxtoprule
\sphinxstyletheadfamily 
\sphinxAtStartPar
dataset
&\sphinxstyletheadfamily 
\sphinxAtStartPar
resource
&\sphinxstyletheadfamily 
\sphinxAtStartPar
model
&\sphinxstyletheadfamily 
\sphinxAtStartPar
property
&\sphinxstyletheadfamily 
\sphinxAtStartPar
type
&\sphinxstyletheadfamily 
\sphinxAtStartPar
ref
&\sphinxstyletheadfamily 
\sphinxAtStartPar
source
&\sphinxstyletheadfamily 
\sphinxAtStartPar
prepare
\\
\sphinxmidrule
\sphinxtableatstartofbodyhook\sphinxstartmulticolumn{4}%
\begin{varwidth}[t]{\sphinxcolwidth{4}{8}}
\sphinxAtStartPar
datasets/gov/example
\par
\vskip-\baselineskip\vbox{\hbox{\strut}}\end{varwidth}%
\sphinxstopmulticolumn
&&&&\\
\sphinxhline
\sphinxAtStartPar

&&\sphinxstartmulticolumn{2}%
\begin{varwidth}[t]{\sphinxcolwidth{2}{8}}
\sphinxAtStartPar
\sphinxstylestrong{Country}
\par
\vskip-\baselineskip\vbox{\hbox{\strut}}\end{varwidth}%
\sphinxstopmulticolumn
&&
\sphinxAtStartPar
code
&&\\
\sphinxhline
\sphinxAtStartPar

&&&
\sphinxAtStartPar
code
&
\sphinxAtStartPar
string
&&&\\
\sphinxhline
\sphinxAtStartPar

&&\sphinxstartmulticolumn{2}%
\begin{varwidth}[t]{\sphinxcolwidth{2}{8}}
\sphinxAtStartPar
\sphinxstylestrong{City}
\par
\vskip-\baselineskip\vbox{\hbox{\strut}}\end{varwidth}%
\sphinxstopmulticolumn
&&
\sphinxAtStartPar
code
&&\\
\sphinxhline
\sphinxAtStartPar

&&&
\sphinxAtStartPar
code
&
\sphinxAtStartPar
integer
&&
\sphinxAtStartPar
city\_code
&\\
\sphinxhline
\sphinxAtStartPar

&&&
\sphinxAtStartPar
name@lt
&
\sphinxAtStartPar
string
&&
\sphinxAtStartPar
city\_name
&\\
\sphinxhline
\sphinxAtStartPar

&&&
\sphinxAtStartPar
country
&
\sphinxAtStartPar
ref
&
\sphinxAtStartPar
Country
&&\\
\sphinxhline
\sphinxAtStartPar

&&&
\sphinxAtStartPar
country.code
&
\sphinxAtStartPar
string
&&&
\sphinxAtStartPar
"lt"
\\
\sphinxhline
\sphinxAtStartPar

&\sphinxstartmulticolumn{3}%
\begin{varwidth}[t]{\sphinxcolwidth{3}{8}}
\sphinxAtStartPar
get\_all\_cities
\par
\vskip-\baselineskip\vbox{\hbox{\strut}}\end{varwidth}%
\sphinxstopmulticolumn
&
\sphinxAtStartPar
json
&&
\sphinxAtStartPar
/cities
&\\
\sphinxhline
\sphinxAtStartPar

&&\sphinxstartmulticolumn{2}%
\begin{varwidth}[t]{\sphinxcolwidth{2}{8}}
\sphinxAtStartPar
City?country.code="lt"
\par
\vskip-\baselineskip\vbox{\hbox{\strut}}\end{varwidth}%
\sphinxstopmulticolumn
&&&&\\
\sphinxbottomrule
\end{tabulary}
\sphinxtableafterendhook\par
\sphinxattableend\end{savenotes}
\end{sphinxadmonition}

\sphinxAtStartPar
Pavyzdyje nurodyta, kad funkcinis \sphinxcode{\sphinxupquote{City?country.code="lt"}} modelis grąžina ne
visus duomenis, o tik Lietuvos miestų duomenis.


\paragraph{Dinaminiai filtrai}
\label{\detokenize{modeliai/funkciniai:dinaminiai-filtrai}}
\sphinxAtStartPar
Dinaminiai filtrai nurodo, kad duomenys pateikiami naudojanti filtrų reikšmes,
kurias pateikia duomenų naudotojas, per \sphinxhref{https://ivpk.github.io/uapi}{UDTS} užklausą, pateikti duomenys
konvertuojami ir perduodami duomenų šaltiniui.

\sphinxAtStartPar
Dinaminiai filtrai veikia lygiai taip pat, kaip ir statiniai filtrai, tik
dinaminio filtro atveju, nenurodoma statinė reikšmė.

\sphinxAtStartPar
Tarkime statinis \sphinxcode{\sphinxupquote{country.code="lt"}} filtras gali būti dinaminis pašalinus
\sphinxcode{\sphinxupquote{="lt"}} dalį ir paliekant tik \sphinxcode{\sphinxupquote{country.code}}.

\begin{sphinxadmonition}{note}{Pavyzdys}


\begin{savenotes}\sphinxattablestart
\sphinxthistablewithglobalstyle
\centering
\begin{tabulary}{\linewidth}[t]{TTTTTTTT}
\sphinxtoprule
\sphinxstyletheadfamily 
\sphinxAtStartPar
dataset
&\sphinxstyletheadfamily 
\sphinxAtStartPar
resource
&\sphinxstyletheadfamily 
\sphinxAtStartPar
model
&\sphinxstyletheadfamily 
\sphinxAtStartPar
property
&\sphinxstyletheadfamily 
\sphinxAtStartPar
type
&\sphinxstyletheadfamily 
\sphinxAtStartPar
ref
&\sphinxstyletheadfamily 
\sphinxAtStartPar
source
&\sphinxstyletheadfamily 
\sphinxAtStartPar
prepare
\\
\sphinxmidrule
\sphinxtableatstartofbodyhook\sphinxstartmulticolumn{4}%
\begin{varwidth}[t]{\sphinxcolwidth{4}{8}}
\sphinxAtStartPar
datasets/gov/example
\par
\vskip-\baselineskip\vbox{\hbox{\strut}}\end{varwidth}%
\sphinxstopmulticolumn
&&&&\\
\sphinxhline
\sphinxAtStartPar

&&\sphinxstartmulticolumn{2}%
\begin{varwidth}[t]{\sphinxcolwidth{2}{8}}
\sphinxAtStartPar
\sphinxstylestrong{Country}
\par
\vskip-\baselineskip\vbox{\hbox{\strut}}\end{varwidth}%
\sphinxstopmulticolumn
&&
\sphinxAtStartPar
code
&&\\
\sphinxhline
\sphinxAtStartPar

&&&
\sphinxAtStartPar
code
&
\sphinxAtStartPar
string
&&&\\
\sphinxhline
\sphinxAtStartPar

&&\sphinxstartmulticolumn{2}%
\begin{varwidth}[t]{\sphinxcolwidth{2}{8}}
\sphinxAtStartPar
\sphinxstylestrong{City}
\par
\vskip-\baselineskip\vbox{\hbox{\strut}}\end{varwidth}%
\sphinxstopmulticolumn
&&
\sphinxAtStartPar
code
&&\\
\sphinxhline
\sphinxAtStartPar

&&&
\sphinxAtStartPar
code
&
\sphinxAtStartPar
integer
&&
\sphinxAtStartPar
city\_code
&\\
\sphinxhline
\sphinxAtStartPar

&&&
\sphinxAtStartPar
name@lt
&
\sphinxAtStartPar
string
&&
\sphinxAtStartPar
city\_name
&\\
\sphinxhline
\sphinxAtStartPar

&&&
\sphinxAtStartPar
country
&
\sphinxAtStartPar
ref
&
\sphinxAtStartPar
Country
&&\\
\sphinxhline
\sphinxAtStartPar

&&&
\sphinxAtStartPar
country.code
&
\sphinxAtStartPar
string
&&
\sphinxAtStartPar
country\_code
&\\
\sphinxhline
\sphinxAtStartPar

&\sphinxstartmulticolumn{3}%
\begin{varwidth}[t]{\sphinxcolwidth{3}{8}}
\sphinxAtStartPar
get\_cities\_by\_country
\par
\vskip-\baselineskip\vbox{\hbox{\strut}}\end{varwidth}%
\sphinxstopmulticolumn
&
\sphinxAtStartPar
json
&&
\sphinxAtStartPar
/cities
&\\
\sphinxhline
\sphinxAtStartPar

&&\sphinxstartmulticolumn{2}%
\begin{varwidth}[t]{\sphinxcolwidth{2}{8}}
\sphinxAtStartPar
City?country.code
\par
\vskip-\baselineskip\vbox{\hbox{\strut}}\end{varwidth}%
\sphinxstopmulticolumn
&&&&\\
\sphinxhline
\sphinxAtStartPar

&&&
\sphinxAtStartPar
country.code
&
\sphinxAtStartPar
string
&&
\sphinxAtStartPar
country
&
\sphinxAtStartPar
query()
\\
\sphinxbottomrule
\end{tabulary}
\sphinxtableafterendhook\par
\sphinxattableend\end{savenotes}
\end{sphinxadmonition}

\sphinxAtStartPar
Šiame pavyzdyje turime \sphinxcode{\sphinxupquote{City?country.code}} funkcinį modelį, kuriame naudojamas
dinaminis filtras pagal \sphinxcode{\sphinxupquote{country.code}}.

\sphinxAtStartPar
Kadangi \sphinxcode{\sphinxupquote{City?country.code}} modelis nurodo \sphinxcode{\sphinxupquote{country.code}} savybę su {\hyperref[\detokenize{formules:query}]{\sphinxcrossref{\sphinxcode{\sphinxupquote{query()}}}}}
formule {\hyperref[\detokenize{dimensijos:property.prepare}]{\sphinxcrossref{\sphinxcode{\sphinxupquote{property.prepare}}}}} stulpelyje, tai reiškia, kad konvertuojant
užklausą į duomenų šaltinio protokolą, \sphinxcode{\sphinxupquote{country.code}} reikšmė bus perduota kaip
URI query parametras.

\sphinxAtStartPar
Per \sphinxhref{https://ivpk.github.io/uapi}{UDTS} protokolą, bus tikimasi gauti tokią užklausią:

\begin{sphinxVerbatim}[commandchars=\\\{\}]
\PYG{n+nf}{GET} \PYG{n+nn}{/datasets/gov/example/City?country.code=\PYGZdq{}lt\PYGZdq{}} \PYG{k+kr}{HTTP}\PYG{o}{/}\PYG{l+m}{1.1}
\end{sphinxVerbatim}

\sphinxAtStartPar
Tokia \sphinxhref{https://ivpk.github.io/uapi}{UDTS} užklausa, pagal pateiktą duomenų struktūros aprašo pavyzdį, bus
konvertuota į tokią duomenų šaltiniui skirtą užklausą:

\begin{sphinxVerbatim}[commandchars=\\\{\}]
\PYG{n+nf}{GET} \PYG{n+nn}{/cities?country=\PYGZdq{}lt\PYGZdq{}} \PYG{k+kr}{HTTP}\PYG{o}{/}\PYG{l+m}{1.1}
\end{sphinxVerbatim}


\paragraph{Operatoriai}
\label{\detokenize{modeliai/funkciniai:operatoriai}}
\sphinxAtStartPar
Funkciniam modeliui galima perduoti daugiau nei vieną filtro parametrą,
naudojant skirtingus filtravimui skirtus operatorius.


\begin{sphinxseealso}{Taip pat žiūrėkite:}

\begin{DUlineblock}{0em}
\item[] {\hyperref[\detokenize{formules:duomenu-atranka}]{\sphinxcrossref{\DUrole{std}{\DUrole{std-ref}{Duomenų atranka}}}}}
\end{DUlineblock}


\end{sphinxseealso}


\sphinxAtStartPar
Pavyzdžiui funkcinis modelis naudojantis dinaminį filtravimą pagal du
kriterijus atrodytų taip:

\begin{sphinxVerbatim}[commandchars=\\\{\}]
City?country.code\PYGZam{}code
\end{sphinxVerbatim}

\sphinxAtStartPar
Šiuo atveju, duomenys pateikiami naudojant filtrą pagal šalies ir miesto kodus.


\subsubsection{select}
\label{\detokenize{modeliai/funkciniai:select}}
\sphinxAtStartPar
Nurodo, kad duomenų šaltinis grąžina ne visas klasės savybes, o tik tam tikras.

\begin{sphinxadmonition}{note}{Pavyzdys}


\begin{savenotes}\sphinxattablestart
\sphinxthistablewithglobalstyle
\centering
\begin{tabulary}{\linewidth}[t]{TTTTTTTT}
\sphinxtoprule
\sphinxstyletheadfamily 
\sphinxAtStartPar
dataset
&\sphinxstyletheadfamily 
\sphinxAtStartPar
resource
&\sphinxstyletheadfamily 
\sphinxAtStartPar
model
&\sphinxstyletheadfamily 
\sphinxAtStartPar
property
&\sphinxstyletheadfamily 
\sphinxAtStartPar
type
&\sphinxstyletheadfamily 
\sphinxAtStartPar
ref
&\sphinxstyletheadfamily 
\sphinxAtStartPar
source
&\sphinxstyletheadfamily 
\sphinxAtStartPar
prepare
\\
\sphinxmidrule
\sphinxtableatstartofbodyhook\sphinxstartmulticolumn{4}%
\begin{varwidth}[t]{\sphinxcolwidth{4}{8}}
\sphinxAtStartPar
datasets/gov/example
\par
\vskip-\baselineskip\vbox{\hbox{\strut}}\end{varwidth}%
\sphinxstopmulticolumn
&&&&\\
\sphinxhline
\sphinxAtStartPar

&&\sphinxstartmulticolumn{2}%
\begin{varwidth}[t]{\sphinxcolwidth{2}{8}}
\sphinxAtStartPar
\sphinxstylestrong{City}
\par
\vskip-\baselineskip\vbox{\hbox{\strut}}\end{varwidth}%
\sphinxstopmulticolumn
&&
\sphinxAtStartPar
code
&&\\
\sphinxhline
\sphinxAtStartPar

&&&
\sphinxAtStartPar
code
&
\sphinxAtStartPar
integer
&&
\sphinxAtStartPar
city\_code
&\\
\sphinxhline
\sphinxAtStartPar

&&&
\sphinxAtStartPar
name@lt
&
\sphinxAtStartPar
string
&&
\sphinxAtStartPar
city\_name
&\\
\sphinxhline
\sphinxAtStartPar

&\sphinxstartmulticolumn{3}%
\begin{varwidth}[t]{\sphinxcolwidth{3}{8}}
\sphinxAtStartPar
get\_cities
\par
\vskip-\baselineskip\vbox{\hbox{\strut}}\end{varwidth}%
\sphinxstopmulticolumn
&
\sphinxAtStartPar
json
&&
\sphinxAtStartPar
/cities
&\\
\sphinxhline
\sphinxAtStartPar

&&\sphinxstartmulticolumn{2}%
\begin{varwidth}[t]{\sphinxcolwidth{2}{8}}
\sphinxAtStartPar
City?select(code)
\par
\vskip-\baselineskip\vbox{\hbox{\strut}}\end{varwidth}%
\sphinxstopmulticolumn
&&&&\\
\sphinxbottomrule
\end{tabulary}
\sphinxtableafterendhook\par
\sphinxattableend\end{savenotes}
\end{sphinxadmonition}

\sphinxAtStartPar
Pavyzdyje nurodoma, kad \sphinxcode{\sphinxupquote{get\_cities}} duomenų šaltinis grąžina ne visas \sphinxcode{\sphinxupquote{City}}
klasės savybes (\sphinxcode{\sphinxupquote{code}} ir \sphinxcode{\sphinxupquote{name}}), o tik vieną \sphinxcode{\sphinxupquote{code}}.

\sphinxAtStartPar
{\hyperref[\detokenize{formules:select}]{\sphinxcrossref{\sphinxcode{\sphinxupquote{select()}}}}} funkcijai galima nurodyti kelias savybes, atskiriant jas kableliu:

\begin{sphinxVerbatim}[commandchars=\\\{\}]
\PYG{k}{select}\PYG{p}{(}\PYG{err}{c}\PYG{err}{o}\PYG{err}{d}\PYG{err}{e}\PYG{p}{,} \PYG{err}{n}\PYG{err}{a}\PYG{err}{m}\PYG{err}{e}\PYG{err}{@}\PYG{err}{l}\PYG{err}{t}\PYG{p}{)}
\end{sphinxVerbatim}

\sphinxAtStartPar
Taip pat galima naudoti ir kitas savybių atrankos funkcijas.


\begin{sphinxseealso}{Taip pat žiūrėkite:}

\begin{DUlineblock}{0em}
\item[] {\hyperref[\detokenize{formules:select}]{\sphinxcrossref{\sphinxcode{\sphinxupquote{select()}}}}}
\end{DUlineblock}

\begin{DUlineblock}{0em}
\item[] Kitos savybių atrankos funkcijos:
\item[] {\hyperref[\detokenize{formules:include}]{\sphinxcrossref{\sphinxcode{\sphinxupquote{include()}}}}}
\item[] {\hyperref[\detokenize{formules:exclude}]{\sphinxcrossref{\sphinxcode{\sphinxupquote{exclude()}}}}}
\item[] {\hyperref[\detokenize{formules:expand}]{\sphinxcrossref{\sphinxcode{\sphinxupquote{expand()}}}}}
\end{DUlineblock}


\end{sphinxseealso}



\subsubsection{sort}
\label{\detokenize{modeliai/funkciniai:sort}}
\sphinxAtStartPar
Nurodo, kad duomenų šaltinis grąžina surūšiuotus duomenis, pagal tam tikras
savybių reikšmes.


\begin{sphinxseealso}{Taip pat žiūrėkite:}

\begin{DUlineblock}{0em}
\item[] {\hyperref[\detokenize{formules:sort}]{\sphinxcrossref{\sphinxcode{\sphinxupquote{sort()}}}}}
\end{DUlineblock}


\end{sphinxseealso}



\subsubsection{part}
\label{\detokenize{modeliai/funkciniai:part}}\label{\detokenize{modeliai/funkciniai:func-model-part}}
\sphinxAtStartPar
Nurodo, kad duomenų šaltinis neleidžia tiesiogiai pasiekti modelio duomenų ir
šis modelis yra naudojamas tik kaip sudėtinė, vieno ar kelių kitų
{\hyperref[\detokenize{identifikatoriai:ref-denorm}]{\sphinxcrossref{\DUrole{std}{\DUrole{std-ref}{jungtinių modelių}}}}}.

\begin{sphinxadmonition}{note}{Pavyzdys}


\begin{savenotes}\sphinxattablestart
\sphinxthistablewithglobalstyle
\centering
\begin{tabulary}{\linewidth}[t]{TTTTTTTT}
\sphinxtoprule
\sphinxstyletheadfamily 
\sphinxAtStartPar
dataset
&\sphinxstyletheadfamily 
\sphinxAtStartPar
resource
&\sphinxstyletheadfamily 
\sphinxAtStartPar
model
&\sphinxstyletheadfamily 
\sphinxAtStartPar
property
&\sphinxstyletheadfamily 
\sphinxAtStartPar
type
&\sphinxstyletheadfamily 
\sphinxAtStartPar
ref
&\sphinxstyletheadfamily 
\sphinxAtStartPar
source
&\sphinxstyletheadfamily 
\sphinxAtStartPar
prepare
\\
\sphinxmidrule
\sphinxtableatstartofbodyhook\sphinxstartmulticolumn{4}%
\begin{varwidth}[t]{\sphinxcolwidth{4}{8}}
\sphinxAtStartPar
datasets/gov/example
\par
\vskip-\baselineskip\vbox{\hbox{\strut}}\end{varwidth}%
\sphinxstopmulticolumn
&&&&\\
\sphinxhline
\sphinxAtStartPar

&&\sphinxstartmulticolumn{2}%
\begin{varwidth}[t]{\sphinxcolwidth{2}{8}}
\sphinxAtStartPar
\sphinxstylestrong{City}
\par
\vskip-\baselineskip\vbox{\hbox{\strut}}\end{varwidth}%
\sphinxstopmulticolumn
&&
\sphinxAtStartPar
code
&&\\
\sphinxhline
\sphinxAtStartPar

&&&
\sphinxAtStartPar
code
&
\sphinxAtStartPar
integer
&&
\sphinxAtStartPar
city\_code
&\\
\sphinxhline
\sphinxAtStartPar

&&&
\sphinxAtStartPar
name@lt
&
\sphinxAtStartPar
string
&&
\sphinxAtStartPar
city\_name
&\\
\sphinxhline
\sphinxAtStartPar

&\sphinxstartmulticolumn{3}%
\begin{varwidth}[t]{\sphinxcolwidth{3}{8}}
\sphinxAtStartPar
get\_cities
\par
\vskip-\baselineskip\vbox{\hbox{\strut}}\end{varwidth}%
\sphinxstopmulticolumn
&
\sphinxAtStartPar
json
&&
\sphinxAtStartPar
/cities
&\\
\sphinxhline
\sphinxAtStartPar

&&\sphinxstartmulticolumn{2}%
\begin{varwidth}[t]{\sphinxcolwidth{2}{8}}
\sphinxAtStartPar
City?select(code)
\par
\vskip-\baselineskip\vbox{\hbox{\strut}}\end{varwidth}%
\sphinxstopmulticolumn
&&&&\\
\sphinxhline
\sphinxAtStartPar

&&&
\sphinxAtStartPar
id
&
\sphinxAtStartPar
integer
&&
\sphinxAtStartPar
page
&
\sphinxAtStartPar
query()
\\
\sphinxbottomrule
\end{tabulary}
\sphinxtableafterendhook\par
\sphinxattableend\end{savenotes}
\end{sphinxadmonition}

\sphinxstepscope


\section{Brandos lygiai}
\label{\detokenize{branda:brandos-lygiai}}\label{\detokenize{branda:level}}\label{\detokenize{branda::doc}}
\sphinxAtStartPar
Duomenų brandos lygis nurodomas {\hyperref[\detokenize{formatas:level}]{\sphinxcrossref{\sphinxcode{\sphinxupquote{level}}}}} stulpelyje.

\sphinxAtStartPar
Duomenų brandos lygis atitinka \sphinxhref{https://5stardata.info/}{5 ★ Open Data} skalę, tačiau ji yra adaptuota duomenų
struktūros aprašo kontekstui, brandos lygius pritaikant ir uždariems duomenims.

\sphinxAtStartPar
Pirmasis brandos lygis, kuris nurodo, kad duomenys turi būti publikuojami pagal
atvirą licenciją, DSA kontekste, šis reikalavimas išplėstas ir apima ne tik
atviras licencijas, bet ir bet kokias duomenų teikimo sąlygas.

\sphinxAtStartPar
Todėl reikia atkreipti dėmesį, kad duomenų brandos lygis ar formatas nėra
susijęs su duomenų atvirumu. Duomenys gali būti pateikti aukščiausiu 5 brandos
lygiu, tačiau pats duomenų prieinamumas gali būti visiškai uždaras, siauram
naudotojų ratui, kurie turi ribotą ir saugų prieigos kanalą prie duomenų.

\sphinxAtStartPar
Papildomai įtrauktas nulinis lygis, kai duomenys nekaupiami, tačiau yra
reikalingi ir yra parengtas jų {\hyperref[\detokenize{savokos:term-DSA}]{\sphinxtermref{\DUrole{xref}{\DUrole{std}{\DUrole{std-term}{duomenų struktūros aprašas}}}}}}.

\sphinxAtStartPar
Tim Berners Lee brandos lygius aprašo, kaip pavyzdį pasitelkiant duomenų
distribucijų formatus. Tačiau duomenų distribucijų formatai yra labai
netikslus pavyzdys. Rengiant duomenų struktūros aprašą, brandos lygis
vertinamas kiekvienam duomenų laukui atskirai, todėl tarkime CSV failas, gali
būti didesnio arba mažesnio nei trečias brandos lygio, priklausomai nuo CSV
faile esančių duomenų turinio. Tačiau grubiai vertinant, vidutiniškai CSV
failai turi daugiau ar mažiau 3 brandos lygį, koks ir yra nurodytas Tim
Berners Lee pavyzdžiuose.

\sphinxAtStartPar
Rengiant duomenų struktūros aprašą, reikėtų nurodyti ne šaltinio duomenų
brandos lygį, o galutinį brandos lygį, kuris yra gaunamas atlikus visas
duomenų struktūros apraše nurodytas transformacijas.


\subsection{L000: Duomenų nėra}
\label{\detokenize{branda:l000-duomenu-nera}}\label{\detokenize{branda:l000}}
\sphinxAtStartPar
\sphinxstylestrong{Nekaupiama}

\sphinxAtStartPar
Duomenys nekaupiami. Duomenų rinkinys užregistruotas duomenų portale. Užpildyta
{\hyperref[\detokenize{formatas:dataset}]{\sphinxcrossref{\sphinxcode{\sphinxupquote{dataset}}}}} eilutė.

\sphinxAtStartPar
Plačiau apie brandos lygio kėlimą skaitykite skyriuje \DUrole{xref}{\DUrole{std}{\DUrole{std-ref}{to\sphinxhyphen{}level\sphinxhyphen{}0}}}.

\sphinxAtStartPar
\sphinxstylestrong{Pavyzdžiai}


\begin{savenotes}\sphinxattablestart
\sphinxthistablewithglobalstyle
\centering
\begin{tabulary}{\linewidth}[t]{TTT}
\sphinxtoprule
\sphinxstartmulticolumn{3}%
\begin{varwidth}[t]{\sphinxcolwidth{3}{3}}
\sphinxstyletheadfamily \sphinxAtStartPar
Imone
\par
\vskip-\baselineskip\vbox{\hbox{\strut}}\end{varwidth}%
\sphinxstopmulticolumn
\\
\sphinxhline\sphinxstyletheadfamily 
\sphinxAtStartPar
imones\_id
&\sphinxstyletheadfamily 
\sphinxAtStartPar
imones\_pavadinimas
&\sphinxstyletheadfamily 
\sphinxAtStartPar
rusis
\\
\sphinxmidrule
\sphinxtableatstartofbodyhook
\sphinxAtStartPar
42
&
\sphinxAtStartPar
UAB "Įmonė"
&
\sphinxAtStartPar
1
\\
\sphinxbottomrule
\end{tabulary}
\sphinxtableafterendhook\par
\sphinxattableend\end{savenotes}


\begin{savenotes}\sphinxattablestart
\sphinxthistablewithglobalstyle
\centering
\begin{tabulary}{\linewidth}[t]{TTTTT}
\sphinxtoprule
\sphinxstartmulticolumn{5}%
\begin{varwidth}[t]{\sphinxcolwidth{5}{5}}
\sphinxstyletheadfamily \sphinxAtStartPar
Filialas
\par
\vskip-\baselineskip\vbox{\hbox{\strut}}\end{varwidth}%
\sphinxstopmulticolumn
\\
\sphinxhline\sphinxstyletheadfamily 
\sphinxAtStartPar
ikurimo\_data
&\sphinxstyletheadfamily 
\sphinxAtStartPar
atstumas
&\sphinxstyletheadfamily 
\sphinxAtStartPar
imones\_id
&\sphinxstyletheadfamily 
\sphinxAtStartPar
imones\_pavadinimas.\_id
&\sphinxstyletheadfamily 
\sphinxAtStartPar
tel\_nr
\\
\sphinxmidrule
\sphinxtableatstartofbodyhook&&&&\\
\sphinxbottomrule
\end{tabulary}
\sphinxtableafterendhook\par
\sphinxattableend\end{savenotes}


\begin{savenotes}\sphinxattablestart
\sphinxthistablewithglobalstyle
\centering
\begin{tabulary}{\linewidth}[t]{TTTTTTTT}
\sphinxtoprule
\sphinxstartmulticolumn{8}%
\begin{varwidth}[t]{\sphinxcolwidth{8}{8}}
\sphinxstyletheadfamily \sphinxAtStartPar
Struktūros aprašas
\par
\vskip-\baselineskip\vbox{\hbox{\strut}}\end{varwidth}%
\sphinxstopmulticolumn
\\
\sphinxhline\sphinxstyletheadfamily 
\sphinxAtStartPar
d
&\sphinxstyletheadfamily 
\sphinxAtStartPar
r
&\sphinxstyletheadfamily 
\sphinxAtStartPar
b
&\sphinxstyletheadfamily 
\sphinxAtStartPar
m
&\sphinxstyletheadfamily 
\sphinxAtStartPar
property
&\sphinxstyletheadfamily 
\sphinxAtStartPar
type
&\sphinxstyletheadfamily 
\sphinxAtStartPar
ref
&\sphinxstyletheadfamily 
\sphinxAtStartPar
level
\\
\sphinxmidrule
\sphinxtableatstartofbodyhook\sphinxstartmulticolumn{5}%
\begin{varwidth}[t]{\sphinxcolwidth{5}{8}}
\sphinxAtStartPar
example
\par
\vskip-\baselineskip\vbox{\hbox{\strut}}\end{varwidth}%
\sphinxstopmulticolumn
&&&\\
\sphinxhline
\sphinxAtStartPar

&&&\sphinxstartmulticolumn{2}%
\begin{varwidth}[t]{\sphinxcolwidth{2}{8}}
\sphinxAtStartPar
Imone
\par
\vskip-\baselineskip\vbox{\hbox{\strut}}\end{varwidth}%
\sphinxstopmulticolumn
&&
\sphinxAtStartPar
imones\_id
&
\sphinxAtStartPar
4
\\
\sphinxhline
\sphinxAtStartPar

&&&&
\sphinxAtStartPar
imones\_id
&
\sphinxAtStartPar
integer
&&
\sphinxAtStartPar
4
\\
\sphinxhline
\sphinxAtStartPar

&&&&
\sphinxAtStartPar
imones\_pavadinimas
&
\sphinxAtStartPar
string
&&
\sphinxAtStartPar
2
\\
\sphinxhline
\sphinxAtStartPar

&&&&
\sphinxAtStartPar
rusis
&
\sphinxAtStartPar
integer
&&
\sphinxAtStartPar
2
\\
\sphinxhline
\sphinxAtStartPar

&&&\sphinxstartmulticolumn{2}%
\begin{varwidth}[t]{\sphinxcolwidth{2}{8}}
\sphinxAtStartPar
Filialas
\par
\vskip-\baselineskip\vbox{\hbox{\strut}}\end{varwidth}%
\sphinxstopmulticolumn
&&&
\sphinxAtStartPar
0
\\
\sphinxhline
\sphinxAtStartPar

&&&&
\sphinxAtStartPar
ikurimo\_data
&
\sphinxAtStartPar
string
&&
\sphinxAtStartPar
0
\\
\sphinxhline
\sphinxAtStartPar

&&&&
\sphinxAtStartPar
atstumas
&
\sphinxAtStartPar
string
&&
\sphinxAtStartPar
0
\\
\sphinxhline
\sphinxAtStartPar

&&&&
\sphinxAtStartPar
imone
&
\sphinxAtStartPar
ref
&
\sphinxAtStartPar
Imone
&
\sphinxAtStartPar
0
\\
\sphinxhline
\sphinxAtStartPar

&&&&
\sphinxAtStartPar
tel\_nr
&
\sphinxAtStartPar
string
&&
\sphinxAtStartPar
0
\\
\sphinxbottomrule
\end{tabulary}
\sphinxtableafterendhook\par
\sphinxattableend\end{savenotes}


\subsubsection{L001: Duomenys nepublikuojami}
\label{\detokenize{branda:l001-duomenys-nepublikuojami}}\label{\detokenize{branda:l001}}
\sphinxAtStartPar
Nuliniu brandos lygiu žymimi duomenys, kurie fiziškai egzistuoja, tačiau nėra
publikuojami jokia forma.


\subsubsection{L002: Ribojamas duomenų naudojimas}
\label{\detokenize{branda:l002-ribojamas-duomenu-naudojimas}}\label{\detokenize{branda:l002}}
\sphinxAtStartPar
Nuliniu brandos lygiu yra žymimi duomenys, kurie yra saugomi duomenų bazėse, ir
yra poreikis juos naudoti, tačiau duomenys nėra teikiami už informacinės
sistemos ribų.


\subsubsection{L003: Nėra identifikatoriaus}
\label{\detokenize{branda:l003-nera-identifikatoriaus}}\label{\detokenize{branda:l003}}
\sphinxAtStartPar
Duomenų šaltinis neturi jokio unikalaus objekto identifikatoriaus.


\subsubsection{L004: Duomenų nėra}
\label{\detokenize{branda:l004-duomenu-nera}}\label{\detokenize{branda:l004}}
\sphinxAtStartPar
Apibrėžtas duomenų modelis, tačiau pačių duomenų kol kas nėra.


\subsection{L100: Nenuskaitoma mašininiu būdu}
\label{\detokenize{branda:l100-nenuskaitoma-masininiu-budu}}\label{\detokenize{branda:l100}}
\sphinxAtStartPar
Duomenys publikuojami bet kokia forma. Užpildyta {\hyperref[\detokenize{formatas:resource}]{\sphinxcrossref{\sphinxcode{\sphinxupquote{resource}}}}}
eilutė.

\sphinxAtStartPar
Plačiau apie brandos lygio kėlimą skaitykite skyriuje \DUrole{xref}{\DUrole{std}{\DUrole{std-ref}{to\sphinxhyphen{}level\sphinxhyphen{}1}}}.

\sphinxAtStartPar
Pirmu brandos lygiu žymimi duomenų laukai, kurių reikšmės neturi
vientisumo, tarkime ta pati reikšmė gali būti pateikta keliais
skirtingais variantais.

\sphinxAtStartPar
\sphinxstylestrong{Pavyzdžiai}


\begin{savenotes}\sphinxattablestart
\sphinxthistablewithglobalstyle
\centering
\begin{tabulary}{\linewidth}[t]{TTT}
\sphinxtoprule
\sphinxstartmulticolumn{3}%
\begin{varwidth}[t]{\sphinxcolwidth{3}{3}}
\sphinxstyletheadfamily \sphinxAtStartPar
Imone
\par
\vskip-\baselineskip\vbox{\hbox{\strut}}\end{varwidth}%
\sphinxstopmulticolumn
\\
\sphinxhline\sphinxstyletheadfamily 
\sphinxAtStartPar
imones\_id
&\sphinxstyletheadfamily 
\sphinxAtStartPar
imones\_pavadinimas
&\sphinxstyletheadfamily 
\sphinxAtStartPar
rusis
\\
\sphinxmidrule
\sphinxtableatstartofbodyhook
\sphinxAtStartPar
42
&
\sphinxAtStartPar
UAB "Įmonė"
&
\sphinxAtStartPar
1
\\
\sphinxbottomrule
\end{tabulary}
\sphinxtableafterendhook\par
\sphinxattableend\end{savenotes}


\begin{savenotes}\sphinxattablestart
\sphinxthistablewithglobalstyle
\centering
\begin{tabular}[t]{*{5}{\X{1}{5}}}
\sphinxtoprule
\sphinxstartmulticolumn{5}%
\begin{varwidth}[t]{\sphinxcolwidth{5}{5}}
\sphinxstyletheadfamily \sphinxAtStartPar
Filialas
\par
\vskip-\baselineskip\vbox{\hbox{\strut}}\end{varwidth}%
\sphinxstopmulticolumn
\\
\sphinxhline\sphinxstyletheadfamily 
\sphinxAtStartPar
ikurimo\_data
&\sphinxstyletheadfamily 
\sphinxAtStartPar
atstumas
&\sphinxstyletheadfamily 
\sphinxAtStartPar
imones\_id.\_id
&\sphinxstyletheadfamily 
\sphinxAtStartPar
imones\_pavadinimas
&\sphinxstyletheadfamily 
\sphinxAtStartPar
tel\_nr
\\
\sphinxmidrule
\sphinxtableatstartofbodyhook
\sphinxAtStartPar
vakar
&
\sphinxAtStartPar
1 m.
&
\sphinxAtStartPar
1
&
\sphinxAtStartPar
Įmonė 1
&
\sphinxAtStartPar
+370 345 36522
\\
\sphinxhline
\sphinxAtStartPar
2021 rugpjūčio 1 d.
&
\sphinxAtStartPar
1 m
&
\sphinxAtStartPar
1
&
\sphinxAtStartPar
UAB Įmonė 1
&
\sphinxAtStartPar
8 345 36 522
\\
\sphinxhline
\sphinxAtStartPar
1/9/21
&
\sphinxAtStartPar
1 metras
&
\sphinxAtStartPar
1
&
\sphinxAtStartPar
Įmonė 1, UAB
&\begin{enumerate}
\sphinxsetlistlabels{\arabic}{enumi}{enumii}{(}{)}%
\setcounter{enumi}{82}
\item {} 
\sphinxAtStartPar
45 34522

\end{enumerate}
\\
\sphinxhline
\sphinxAtStartPar
21/9/1
&
\sphinxAtStartPar
0.001 km
&
\sphinxAtStartPar
1
&
\sphinxAtStartPar
„ĮMONĖ 1“, UAB
&
\sphinxAtStartPar
037034536522
\\
\sphinxbottomrule
\end{tabular}
\sphinxtableafterendhook\par
\sphinxattableend\end{savenotes}


\begin{savenotes}\sphinxattablestart
\sphinxthistablewithglobalstyle
\centering
\begin{tabulary}{\linewidth}[t]{TTTTTTTT}
\sphinxtoprule
\sphinxstartmulticolumn{8}%
\begin{varwidth}[t]{\sphinxcolwidth{8}{8}}
\sphinxstyletheadfamily \sphinxAtStartPar
Struktūros aprašas
\par
\vskip-\baselineskip\vbox{\hbox{\strut}}\end{varwidth}%
\sphinxstopmulticolumn
\\
\sphinxhline\sphinxstyletheadfamily 
\sphinxAtStartPar
d
&\sphinxstyletheadfamily 
\sphinxAtStartPar
r
&\sphinxstyletheadfamily 
\sphinxAtStartPar
b
&\sphinxstyletheadfamily 
\sphinxAtStartPar
m
&\sphinxstyletheadfamily 
\sphinxAtStartPar
property
&\sphinxstyletheadfamily 
\sphinxAtStartPar
type
&\sphinxstyletheadfamily 
\sphinxAtStartPar
ref
&\sphinxstyletheadfamily 
\sphinxAtStartPar
level
\\
\sphinxmidrule
\sphinxtableatstartofbodyhook\sphinxstartmulticolumn{5}%
\begin{varwidth}[t]{\sphinxcolwidth{5}{8}}
\sphinxAtStartPar
example
\par
\vskip-\baselineskip\vbox{\hbox{\strut}}\end{varwidth}%
\sphinxstopmulticolumn
&&&\\
\sphinxhline
\sphinxAtStartPar

&&&\sphinxstartmulticolumn{2}%
\begin{varwidth}[t]{\sphinxcolwidth{2}{8}}
\sphinxAtStartPar
Imone
\par
\vskip-\baselineskip\vbox{\hbox{\strut}}\end{varwidth}%
\sphinxstopmulticolumn
&&
\sphinxAtStartPar
id
&
\sphinxAtStartPar
4
\\
\sphinxhline
\sphinxAtStartPar

&&&&
\sphinxAtStartPar
imones\_id
&
\sphinxAtStartPar
integer
&&
\sphinxAtStartPar
2
\\
\sphinxhline
\sphinxAtStartPar

&&&&
\sphinxAtStartPar
imones\_pavadinimas
&
\sphinxAtStartPar
string
&&
\sphinxAtStartPar
2
\\
\sphinxhline
\sphinxAtStartPar

&&&&
\sphinxAtStartPar
rusis
&
\sphinxAtStartPar
integer
&&
\sphinxAtStartPar
2
\\
\sphinxhline
\sphinxAtStartPar

&&&\sphinxstartmulticolumn{2}%
\begin{varwidth}[t]{\sphinxcolwidth{2}{8}}
\sphinxAtStartPar
Filialas
\par
\vskip-\baselineskip\vbox{\hbox{\strut}}\end{varwidth}%
\sphinxstopmulticolumn
&&&
\sphinxAtStartPar
3
\\
\sphinxhline
\sphinxAtStartPar

&&&&
\sphinxAtStartPar
ikurimo\_data
&
\sphinxAtStartPar
string
&&
\sphinxAtStartPar
1
\\
\sphinxhline
\sphinxAtStartPar

&&&&
\sphinxAtStartPar
atstumas
&
\sphinxAtStartPar
string
&&
\sphinxAtStartPar
1
\\
\sphinxhline
\sphinxAtStartPar

&&&&
\sphinxAtStartPar
imones\_id
&
\sphinxAtStartPar
ref
&
\sphinxAtStartPar
Imone
&
\sphinxAtStartPar
1
\\
\sphinxhline
\sphinxAtStartPar

&&&&
\sphinxAtStartPar
imones\_pavadinimas
&
\sphinxAtStartPar
string
&&
\sphinxAtStartPar
1
\\
\sphinxhline
\sphinxAtStartPar

&&&&
\sphinxAtStartPar
tel\_nr
&
\sphinxAtStartPar
string
&&
\sphinxAtStartPar
1
\\
\sphinxbottomrule
\end{tabulary}
\sphinxtableafterendhook\par
\sphinxattableend\end{savenotes}


\subsubsection{L101: Neaiški struktūra}
\label{\detokenize{branda:l101-neaiski-struktura}}\label{\detokenize{branda:l101}}
\sphinxAtStartPar
Pirmu brandos lygiu žymimi duomenys, kuriuose nėra aiškios struktūros,
pavyzdžiui \sphinxcode{\sphinxupquote{ikurta}} datos formatas nėra vienodas, kiekviena data užrašyta vis
kitokiu formatu.


\subsubsection{L102: Nėra vientisumo}
\label{\detokenize{branda:l102-nera-vientisumo}}\label{\detokenize{branda:l102}}
\sphinxAtStartPar
Pirmu brandos lygiu žymimi duomenys, kuruose nėra vientisumo, pavyzdžiui
\sphinxcode{\sphinxupquote{atstumas}} užrašytas laikantis tam tikros struktūros, tačiau skirtingais
vienetais.


\subsubsection{L103: Neįmanomas jungimas}
\label{\detokenize{branda:l103-neimanomas-jungimas}}\label{\detokenize{branda:l103}}
\sphinxAtStartPar
Pirmu brandos lygiu žymimi duomenys, kurių neįmanoma arba sudėtinga sujungti.
Pavyzdžiui \sphinxcode{\sphinxupquote{Filialas}} duomnų laukas \sphinxcode{\sphinxupquote{imone}} naudoja tam tikrą identifikatorių,
kuris nesutampa nei su vienu iš \sphinxcode{\sphinxupquote{Imone}} atributų, pagal kuriuose būtų galima
identifikuoti filialo įmonę.


\subsubsection{L104: Identifikatorius nėra unikalus}
\label{\detokenize{branda:l104-identifikatorius-nera-unikalus}}\label{\detokenize{branda:l104}}
\sphinxAtStartPar
Objekto identifikatorius nėra unikalus, turi pasikartojančių reikšmių.


\subsubsection{L105: Vienetų konvertavimo paklaida}
\label{\detokenize{branda:l105-vienetu-konvertavimo-paklaida}}\label{\detokenize{branda:l105}}
\sphinxAtStartPar
Tam tikrais atvejais, kai kiekybiniai duomenys pateikiami išskaidant į atskirus
duomenų laukus, gali būti prarandamas duomenų tikslumas.

\begin{sphinxadmonition}{note}{Pavyzdys}


\begin{savenotes}\sphinxattablestart
\sphinxthistablewithglobalstyle
\centering
\begin{tabulary}{\linewidth}[t]{TTTTT}
\sphinxtoprule
\sphinxstyletheadfamily 
\sphinxAtStartPar
model
&\sphinxstyletheadfamily 
\sphinxAtStartPar
property
&\sphinxstyletheadfamily 
\sphinxAtStartPar
type
&\sphinxstyletheadfamily 
\sphinxAtStartPar
ref
&\sphinxstyletheadfamily 
\sphinxAtStartPar
level
\\
\sphinxmidrule
\sphinxtableatstartofbodyhook\sphinxstartmulticolumn{2}%
\begin{varwidth}[t]{\sphinxcolwidth{2}{5}}
\sphinxAtStartPar
Assmuo
\par
\vskip-\baselineskip\vbox{\hbox{\strut}}\end{varwidth}%
\sphinxstopmulticolumn
&&
\sphinxAtStartPar
id
&
\sphinxAtStartPar
4
\\
\sphinxhline
\sphinxAtStartPar

&
\sphinxAtStartPar
id
&
\sphinxAtStartPar
integer
&&
\sphinxAtStartPar
4
\\
\sphinxhline
\sphinxAtStartPar

&
\sphinxAtStartPar
metai
&
\sphinxAtStartPar
integer
&
\sphinxAtStartPar
yr
&
\sphinxAtStartPar
4
\\
\sphinxhline
\sphinxAtStartPar

&
\sphinxAtStartPar
menesiai
&
\sphinxAtStartPar
integer
&
\sphinxAtStartPar
mo
&
\sphinxAtStartPar
1
\\
\sphinxhline
\sphinxAtStartPar

&
\sphinxAtStartPar
dienos
&
\sphinxAtStartPar
integer
&
\sphinxAtStartPar
d
&
\sphinxAtStartPar
1
\\
\sphinxbottomrule
\end{tabulary}
\sphinxtableafterendhook\par
\sphinxattableend\end{savenotes}

\sphinxAtStartPar
Šiame pavyzdyje nurodytas asmens amžius pateikiant atskirai metus, mėnesius
ir dienas, tarkim \sphinxstyleemphasis{25 metai, 5 mėnesiai, 29 dienos}.

\sphinxAtStartPar
Jei norėtume gauti amžių dienomis, rezultatas \(25*365 + 5*30 + 29 =
9304\) būtų netikslus, kadangi metai ir mėnesiai turi skirtingus dienų
skaičius, todėl konvertuojant rezultatą į dienas, gausime netikslų
rezultatą.

\sphinxAtStartPar
Kadangi nustatyti nėra galimybės nustatyti galutinio tikslaus mėnesio ir
dienų skaičiaus, nurodomas 1 brandos lygis.
\end{sphinxadmonition}


\subsection{L200: Nestandartinis pateikimas}
\label{\detokenize{branda:l200-nestandartinis-pateikimas}}\label{\detokenize{branda:l200}}
\sphinxAtStartPar
Duomenys kaupiami struktūruota, mašininiu būdu nuskaitoma forma, bet kokiu
formatu. Užpildytas {\hyperref[\detokenize{dimensijos:property.source}]{\sphinxcrossref{\sphinxcode{\sphinxupquote{property.source}}}}} stulpelis.

\sphinxAtStartPar
Plačiau apie brandos lygio kėlimą skaitykite skyriuje \DUrole{xref}{\DUrole{std}{\DUrole{std-ref}{to\sphinxhyphen{}level\sphinxhyphen{}2}}}.

\sphinxAtStartPar
Antru brandos lygiu žymimi duomenų laukai, kurie pateikti vieninga forma arba
pagal aiškų ir vienodą šabloną. Tačiau pateikimo būdas nėra standartinis.
Nestandartinis duomenų formatas yra toks, kuris neturi viešai skelbiamos ir
atviros formato specifikacijos arba, kuris nėra priimtas kaip standartas, kurį
prižiūri tam tikra standartizacijos organizacija.

\sphinxAtStartPar
\sphinxstylestrong{Pavyzdžiai}


\begin{savenotes}\sphinxattablestart
\sphinxthistablewithglobalstyle
\centering
\begin{tabulary}{\linewidth}[t]{TTT}
\sphinxtoprule
\sphinxstartmulticolumn{3}%
\begin{varwidth}[t]{\sphinxcolwidth{3}{3}}
\sphinxstyletheadfamily \sphinxAtStartPar
Imone
\par
\vskip-\baselineskip\vbox{\hbox{\strut}}\end{varwidth}%
\sphinxstopmulticolumn
\\
\sphinxhline\sphinxstyletheadfamily 
\sphinxAtStartPar
imones\_id
&\sphinxstyletheadfamily 
\sphinxAtStartPar
imones\_pavadinimas
&\sphinxstyletheadfamily 
\sphinxAtStartPar
rusis
\\
\sphinxmidrule
\sphinxtableatstartofbodyhook
\sphinxAtStartPar
42
&
\sphinxAtStartPar
UAB "Įmonė"
&
\sphinxAtStartPar
1
\\
\sphinxbottomrule
\end{tabulary}
\sphinxtableafterendhook\par
\sphinxattableend\end{savenotes}


\begin{savenotes}\sphinxattablestart
\sphinxthistablewithglobalstyle
\centering
\begin{tabulary}{\linewidth}[t]{TTTTT}
\sphinxtoprule
\sphinxstartmulticolumn{5}%
\begin{varwidth}[t]{\sphinxcolwidth{5}{5}}
\sphinxstyletheadfamily \sphinxAtStartPar
Filialas
\par
\vskip-\baselineskip\vbox{\hbox{\strut}}\end{varwidth}%
\sphinxstopmulticolumn
\\
\sphinxhline\sphinxstyletheadfamily 
\sphinxAtStartPar
ikurimo\_data
&\sphinxstyletheadfamily 
\sphinxAtStartPar
atstumas
&\sphinxstyletheadfamily 
\sphinxAtStartPar
imones\_id
&\sphinxstyletheadfamily 
\sphinxAtStartPar
imones\_pavadinimas.\_id
&\sphinxstyletheadfamily 
\sphinxAtStartPar
tel\_nr
\\
\sphinxmidrule
\sphinxtableatstartofbodyhook
\sphinxAtStartPar
1/9/21
&
\sphinxAtStartPar
1 m.
&
\sphinxAtStartPar
1
&
\sphinxAtStartPar
UAB "Įmonė"
&
\sphinxAtStartPar
(83) 111 11111
\\
\sphinxhline
\sphinxAtStartPar
2/9/21
&
\sphinxAtStartPar
2 m.
&
\sphinxAtStartPar
1
&
\sphinxAtStartPar
UAB "Įmonė"
&
\sphinxAtStartPar
(83) 222 22222
\\
\sphinxhline
\sphinxAtStartPar
3/9/21
&
\sphinxAtStartPar
3 m.
&
\sphinxAtStartPar
1
&
\sphinxAtStartPar
UAB "Įmonė"
&
\sphinxAtStartPar
(83) 333 33333
\\
\sphinxhline
\sphinxAtStartPar
4/9/21
&
\sphinxAtStartPar
4 m.
&
\sphinxAtStartPar
1
&
\sphinxAtStartPar
UAB "Įmonė"
&
\sphinxAtStartPar
(83) 444 44444
\\
\sphinxbottomrule
\end{tabulary}
\sphinxtableafterendhook\par
\sphinxattableend\end{savenotes}


\begin{savenotes}\sphinxattablestart
\sphinxthistablewithglobalstyle
\centering
\begin{tabulary}{\linewidth}[t]{TTTTTTTT}
\sphinxtoprule
\sphinxstartmulticolumn{8}%
\begin{varwidth}[t]{\sphinxcolwidth{8}{8}}
\sphinxstyletheadfamily \sphinxAtStartPar
Struktūros aprašas
\par
\vskip-\baselineskip\vbox{\hbox{\strut}}\end{varwidth}%
\sphinxstopmulticolumn
\\
\sphinxhline\sphinxstyletheadfamily 
\sphinxAtStartPar
d
&\sphinxstyletheadfamily 
\sphinxAtStartPar
r
&\sphinxstyletheadfamily 
\sphinxAtStartPar
b
&\sphinxstyletheadfamily 
\sphinxAtStartPar
m
&\sphinxstyletheadfamily 
\sphinxAtStartPar
property
&\sphinxstyletheadfamily 
\sphinxAtStartPar
type
&\sphinxstyletheadfamily 
\sphinxAtStartPar
ref
&\sphinxstyletheadfamily 
\sphinxAtStartPar
level
\\
\sphinxmidrule
\sphinxtableatstartofbodyhook\sphinxstartmulticolumn{5}%
\begin{varwidth}[t]{\sphinxcolwidth{5}{8}}
\sphinxAtStartPar
example
\par
\vskip-\baselineskip\vbox{\hbox{\strut}}\end{varwidth}%
\sphinxstopmulticolumn
&&&\\
\sphinxhline
\sphinxAtStartPar

&&&\sphinxstartmulticolumn{2}%
\begin{varwidth}[t]{\sphinxcolwidth{2}{8}}
\sphinxAtStartPar
JuridinisAsmuo
\par
\vskip-\baselineskip\vbox{\hbox{\strut}}\end{varwidth}%
\sphinxstopmulticolumn
&&
\sphinxAtStartPar
kodas
&
\sphinxAtStartPar
4
\\
\sphinxhline
\sphinxAtStartPar

&&&&
\sphinxAtStartPar
kodas
&
\sphinxAtStartPar
integer
&&
\sphinxAtStartPar
4
\\
\sphinxhline
\sphinxAtStartPar

&&&&
\sphinxAtStartPar
pavadinimas@lt
&
\sphinxAtStartPar
text
&&
\sphinxAtStartPar
4
\\
\sphinxhline
\sphinxAtStartPar

&&&\sphinxstartmulticolumn{2}%
\begin{varwidth}[t]{\sphinxcolwidth{2}{8}}
\sphinxAtStartPar
Imone
\par
\vskip-\baselineskip\vbox{\hbox{\strut}}\end{varwidth}%
\sphinxstopmulticolumn
&&
\sphinxAtStartPar
imones\_id
&
\sphinxAtStartPar
2
\\
\sphinxhline
\sphinxAtStartPar

&&&&
\sphinxAtStartPar
imones\_id
&
\sphinxAtStartPar
integer
&&
\sphinxAtStartPar
2
\\
\sphinxhline
\sphinxAtStartPar

&&&&
\sphinxAtStartPar
imones\_pavadinimas
&
\sphinxAtStartPar
string
&&
\sphinxAtStartPar
2
\\
\sphinxhline
\sphinxAtStartPar

&&&&
\sphinxAtStartPar
rusis
&
\sphinxAtStartPar
integer
&&
\sphinxAtStartPar
2
\\
\sphinxhline
\sphinxAtStartPar

&&&\sphinxstartmulticolumn{2}%
\begin{varwidth}[t]{\sphinxcolwidth{2}{8}}
\sphinxAtStartPar
Filialas
\par
\vskip-\baselineskip\vbox{\hbox{\strut}}\end{varwidth}%
\sphinxstopmulticolumn
&&&
\sphinxAtStartPar
3
\\
\sphinxhline
\sphinxAtStartPar

&&&&
\sphinxAtStartPar
ikurimo\_data
&
\sphinxAtStartPar
string
&&
\sphinxAtStartPar
2
\\
\sphinxhline
\sphinxAtStartPar

&&&&
\sphinxAtStartPar
atstumas
&
\sphinxAtStartPar
string
&&
\sphinxAtStartPar
2
\\
\sphinxhline
\sphinxAtStartPar

&&&&
\sphinxAtStartPar
imones\_id
&
\sphinxAtStartPar
integer
&&
\sphinxAtStartPar
2
\\
\sphinxhline
\sphinxAtStartPar

&&&&
\sphinxAtStartPar
imones\_pavadinimas
&
\sphinxAtStartPar
string
&&
\sphinxAtStartPar
2
\\
\sphinxhline
\sphinxAtStartPar

&&&&
\sphinxAtStartPar
tel\_nr
&
\sphinxAtStartPar
string
&&
\sphinxAtStartPar
2
\\
\sphinxbottomrule
\end{tabulary}
\sphinxtableafterendhook\par
\sphinxattableend\end{savenotes}


\subsubsection{L201: Nestandartiniai duomenų tipai}
\label{\detokenize{branda:l201-nestandartiniai-duomenu-tipai}}\label{\detokenize{branda:l201}}
\sphinxAtStartPar
Antru brandos lygiu žymimi duomenys, kurių nurodytas tipas neatitinka realaus
duomenų tipo. Pavyzdžiui:
\begin{itemize}
\item {} 
\sphinxAtStartPar
\sphinxcode{\sphinxupquote{ikurimo\_data}} \sphinxhyphen{} nurodytas \sphinxcode{\sphinxupquote{string}}, turėtu būti \sphinxcode{\sphinxupquote{date}}.

\item {} 
\sphinxAtStartPar
\sphinxcode{\sphinxupquote{imones\_pavadinimas}} \sphinxhyphen{} nurodytas \sphinxcode{\sphinxupquote{string}}, turėtu būti \sphinxcode{\sphinxupquote{text}}.

\item {} 
\sphinxAtStartPar
\sphinxcode{\sphinxupquote{atstumas}} \sphinxhyphen{} nurodytas \sphinxcode{\sphinxupquote{string}}, turėtu būti \sphinxcode{\sphinxupquote{integer}}.

\end{itemize}


\subsubsection{L202: Nestandartinis formatas}
\label{\detokenize{branda:l202-nestandartinis-formatas}}\label{\detokenize{branda:l202}}
\sphinxAtStartPar
Antru brandos lygiu žymimi duomenys, kurie pateikti nestandartiniu formatu.
Standartinis duomenų pateikimas nurodytas prie kiekvieno duomenų tipo skyriuje
{\hyperref[\detokenize{tipai:duomenu-tipai}]{\sphinxcrossref{\DUrole{std}{\DUrole{std-ref}{Duomenų tipai}}}}}. Payvzdžiui:
\begin{itemize}
\item {} 
\sphinxAtStartPar
\sphinxcode{\sphinxupquote{ikurimo\_data}} \sphinxhyphen{} nurodytas \sphinxcode{\sphinxupquote{DD/MM/YY}}, turėtu būti \sphinxcode{\sphinxupquote{YYYY\sphinxhyphen{}MM\sphinxhyphen{}DD}}.

\item {} 
\sphinxAtStartPar
\sphinxcode{\sphinxupquote{atstumas}} \sphinxhyphen{} nurodyta \sphinxcode{\sphinxupquote{X m.}}, turėtu būti \sphinxcode{\sphinxupquote{X}}.

\item {} 
\sphinxAtStartPar
\sphinxcode{\sphinxupquote{tel\_nr}} \sphinxhyphen{} nurodytas \sphinxcode{\sphinxupquote{(XX) XXX XXXXX}}, turėtu būti \sphinxcode{\sphinxupquote{+XXX\sphinxhyphen{}XXX\sphinxhyphen{}XXXXX}}.

\end{itemize}


\subsubsection{L203: Nestandartiniai kodiniai pavadinimai}
\label{\detokenize{branda:l203-nestandartiniai-kodiniai-pavadinimai}}\label{\detokenize{branda:l203}}
\sphinxAtStartPar
Antru brandos lygiu žymimi duomenys, kurių kodiniai pavadinimai, neatitinka
{\hyperref[\detokenize{pavadinimai:kodiniai-pavadinimai}]{\sphinxcrossref{\DUrole{std}{\DUrole{std-ref}{standartinių reikalavimų keliamų kodiniams pavadinimams}}}}}. Pavyzdžiui:
\begin{itemize}
\item {} 
\sphinxAtStartPar
\sphinxcode{\sphinxupquote{imones\_id}} \sphinxhyphen{} dubliuojamas modelio pavadinimas, turėtu būti \sphinxcode{\sphinxupquote{id}}.

\item {} 
\sphinxAtStartPar
\sphinxcode{\sphinxupquote{imones\_pavadinimas}} \sphinxhyphen{} dubliuojamas modelio pavadinimas, turėtu būti
\sphinxcode{\sphinxupquote{pavadinimas}}.

\item {} 
\sphinxAtStartPar
\sphinxcode{\sphinxupquote{ikurimo\_data}} \sphinxhyphen{} dubliuojamas tipo pavadinimas, turėtu būti \sphinxcode{\sphinxupquote{ikurta}}.

\end{itemize}


\begin{sphinxseealso}{Taip pat žiūrėkite:}

\begin{DUlineblock}{0em}
\item[] {\hyperref[\detokenize{pavadinimai:kodiniai-pavadinimai}]{\sphinxcrossref{\DUrole{std}{\DUrole{std-ref}{Kodiniai pavadinimai}}}}}
\end{DUlineblock}


\end{sphinxseealso}



\subsubsection{L204: Nepatikimi identifikatoriai}
\label{\detokenize{branda:l204-nepatikimi-identifikatoriai}}\label{\detokenize{branda:l204}}
\sphinxAtStartPar
Antru brandos lygiu žymimi duomenys, kurių \sphinxcode{\sphinxupquote{ref}} tipui naudojami nepatikimi
identifikatoriai, pavyzdžiui tokie, kaip pavadinimai, kurie gali keistis arba
kartotis. Pavyzdžiui:
\begin{itemize}
\item {} 
\sphinxAtStartPar
\sphinxcode{\sphinxupquote{imones\_pavadinimas}} \sphinxhyphen{} jungimas daromas per įmonės pavadinimą,
tačiau šiuo atveju kito varianto nėra, nes \sphinxcode{\sphinxupquote{Filialas.imones\_id}}
nesutampa su \sphinxcode{\sphinxupquote{Imone.imones\_id}}.

\end{itemize}


\subsubsection{L205: Denormalizuoti duomenys}
\label{\detokenize{branda:l205-denormalizuoti-duomenys}}\label{\detokenize{branda:l205}}
\sphinxAtStartPar
Antru brandos lygiu žymimi duomenys, kurie dubliuoja kito modelio duomenis ir
yra užrašyti nenurodant, kad tai yra duomenys dubliuojantys kito modelio
duomenis. Pavyzdžiui:
\begin{itemize}
\item {} 
\sphinxAtStartPar
\sphinxcode{\sphinxupquote{Filialas.imones\_id}} turėtu būti \sphinxcode{\sphinxupquote{Filialas.imone.imones\_id}}.

\item {} 
\sphinxAtStartPar
\sphinxcode{\sphinxupquote{Filialas.imones\_pavadinimas}} turėtu būti
\sphinxcode{\sphinxupquote{Filialas.imone.imones\_pavadinimas}}.

\end{itemize}

\sphinxAtStartPar
Plačiau apie denormalizuotus duomenis skaitykite skyriuje {\hyperref[\detokenize{identifikatoriai:ref-denorm}]{\sphinxcrossref{\DUrole{std}{\DUrole{std-ref}{Jungtinis modelis}}}}}.


\subsubsection{L206: Nenurodytas susiejimas}
\label{\detokenize{branda:l206-nenurodytas-susiejimas}}\label{\detokenize{branda:l206}}
\sphinxAtStartPar
Antru brandos lygiu žymimi duomenys, kurie siejasi su kitu modeliu, tačiau
tokia informacija nėra pateikta metaduomenyse. Pavyzdžiui:
\begin{itemize}
\item {} 
\sphinxAtStartPar
\sphinxcode{\sphinxupquote{Filialas.imone}} \sphinxhyphen{} \sphinxcode{\sphinxupquote{Filialas}} siejasi su \sphinxcode{\sphinxupquote{Imone}}, per
\sphinxcode{\sphinxupquote{Filialas.imones\_pavadiniams}}, todėl turėtu būti nurodytas \sphinxcode{\sphinxupquote{imone ref Imone}}
ryšys su \sphinxcode{\sphinxupquote{Imone}}.

\end{itemize}


\subsubsection{L207: Neatitinka modelio bazės}
\label{\detokenize{branda:l207-neatitinka-modelio-bazes}}\label{\detokenize{branda:l207}}
\sphinxAtStartPar
Antru brandos lygiu žymimi duomenys, kurie priklauso vienai semantinei klasei,
tačiau duomenų schema nesutampa su bazinio modelio schema. Pavyzdžiui:
\begin{itemize}
\item {} 
\sphinxAtStartPar
\sphinxcode{\sphinxupquote{Imone}} \sphinxhyphen{} priklauso semantinei klasei \sphinxcode{\sphinxupquote{JuridinisAsmuo}}, tačiau tai nėra
pažymėta metaduomenyse.

\item {} 
\sphinxAtStartPar
\sphinxcode{\sphinxupquote{Imone.imones\_id}} turėtu būti \sphinxcode{\sphinxupquote{Imone.kodas}}, kad sutaptu su baze
(\sphinxcode{\sphinxupquote{JuridinisAsmuo.kodas}}).

\item {} 
\sphinxAtStartPar
\sphinxcode{\sphinxupquote{Imone.imones\_pavadinimas}} turėtu būti \sphinxcode{\sphinxupquote{Imone.pavadinimas@lt}}, kad sutaptu su
baze (\sphinxcode{\sphinxupquote{JuridinisAsmuo.pavadinimas@lt}}).

\end{itemize}


\subsubsection{L208: Nenurodytas enum kodinėms reikšmėms}
\label{\detokenize{branda:l208-nenurodytas-enum-kodinems-reiksmems}}\label{\detokenize{branda:l208}}
\sphinxAtStartPar
Antru brandos lygiu žymimi kategoriniai duomenys, kurių reikšmės pateiktos
sutartiniais kodais, kurių prasmė nėra aiški. Pavyzdžiui:
\begin{itemize}
\item {} 
\sphinxAtStartPar
\sphinxcode{\sphinxupquote{Imone.rusis}} \sphinxhyphen{} įmonės rūšis žymima skaičiais, tačiau nėra aišku,
koks skaičius, ką reiškia, todėl reikia pateikti \sphinxcode{\sphinxupquote{enum}} sąrašą,
kuriame būtų nurodyta, ką koks skaičius reiškia. Plačiau skaityti
{\hyperref[\detokenize{dimensijos:enum}]{\sphinxcrossref{\DUrole{std}{\DUrole{std-ref}{enum}}}}}.

\end{itemize}


\subsubsection{L209: Nenurodyta modelio bazė}
\label{\detokenize{branda:l209-nenurodyta-modelio-baze}}\label{\detokenize{branda:l209}}
\sphinxAtStartPar
Modelis atitinka registre apibrėžtą esybę, tačiau nėra su ja susietas.


\subsubsection{L210: Išskaidyta atskirais komponentais}
\label{\detokenize{branda:l210-isskaidyta-atskirais-komponentais}}\label{\detokenize{branda:l210}}
\sphinxAtStartPar
{\hyperref[\detokenize{savokos:term-DSA}]{\sphinxtermref{\DUrole{xref}{\DUrole{std}{\DUrole{std-term}{DSA}}}}}} turi sudėtinius tipus, tokius kaip \sphinxcode{\sphinxupquote{date}}, \sphinxcode{\sphinxupquote{datetime}} ir
\sphinxcode{\sphinxupquote{geometry}}, kuri apima kelis atskirus duomenų komponentus, kurie pateikiami
kaip viena reikšmė, nustatytu formatu.

\sphinxAtStartPar
Jei komponentai yra išskaidyti į atskirus duomenų laukus, tuomet tai yra
nestandartinis duomenų pateikimas žymimas 2 brandos lygiu.


\subsection{L300: Nėra identifikatoriaus}
\label{\detokenize{branda:l300-nera-identifikatoriaus}}\label{\detokenize{branda:l300}}
\sphinxAtStartPar
Duomenys saugomi atviru, standartiniu formatu. Užpildytas
{\hyperref[\detokenize{dimensijos:property.type}]{\sphinxcrossref{\sphinxcode{\sphinxupquote{property.type}}}}} stulpelis ir duomenys atitinka nurodytą tipą.

\sphinxAtStartPar
Plačiau apie brandos lygio kėlimą skaitykite skyriuje \DUrole{xref}{\DUrole{std}{\DUrole{std-ref}{to\sphinxhyphen{}level\sphinxhyphen{}3}}}.

\sphinxAtStartPar
Trečias brandos lygis suteikiamas tada, kai duomenys pateikti vieninga
forma, vieningu masteliu, naudojamas formatas yra standartinis, tai
reiškia, kad yra viešai skelbiama ir atvira formato specifikacija arba
pats formatas yra patvirtintas ir prižiūrimas kokios nors
standartizacijos organizacijos.

\sphinxAtStartPar
\sphinxstylestrong{Pavyzdžiai}


\begin{savenotes}\sphinxattablestart
\sphinxthistablewithglobalstyle
\centering
\begin{tabulary}{\linewidth}[t]{TTT}
\sphinxtoprule
\sphinxstartmulticolumn{3}%
\begin{varwidth}[t]{\sphinxcolwidth{3}{3}}
\sphinxstyletheadfamily \sphinxAtStartPar
Imone
\par
\vskip-\baselineskip\vbox{\hbox{\strut}}\end{varwidth}%
\sphinxstopmulticolumn
\\
\sphinxhline\sphinxstyletheadfamily 
\sphinxAtStartPar
id
&\sphinxstyletheadfamily 
\sphinxAtStartPar
pavadinimas@lt
&\sphinxstyletheadfamily 
\sphinxAtStartPar
rusis
\\
\sphinxmidrule
\sphinxtableatstartofbodyhook
\sphinxAtStartPar
42
&
\sphinxAtStartPar
UAB "Įmonė"
&
\sphinxAtStartPar
juridinis
\\
\sphinxbottomrule
\end{tabulary}
\sphinxtableafterendhook\par
\sphinxattableend\end{savenotes}


\begin{savenotes}\sphinxattablestart
\sphinxthistablewithglobalstyle
\centering
\begin{tabulary}{\linewidth}[t]{TTTTT}
\sphinxtoprule
\sphinxstartmulticolumn{5}%
\begin{varwidth}[t]{\sphinxcolwidth{5}{5}}
\sphinxstyletheadfamily \sphinxAtStartPar
Filialas
\par
\vskip-\baselineskip\vbox{\hbox{\strut}}\end{varwidth}%
\sphinxstopmulticolumn
\\
\sphinxhline\sphinxstyletheadfamily 
\sphinxAtStartPar
ikurta
&\sphinxstyletheadfamily 
\sphinxAtStartPar
atstumas
&\sphinxstyletheadfamily 
\sphinxAtStartPar
imone.\_id
&\sphinxstyletheadfamily 
\sphinxAtStartPar
imone.pavadinimas@lt
&\sphinxstyletheadfamily 
\sphinxAtStartPar
tel\_nr
\\
\sphinxmidrule
\sphinxtableatstartofbodyhook
\sphinxAtStartPar
2021\sphinxhyphen{}09\sphinxhyphen{}01
&
\sphinxAtStartPar
1
&
\sphinxAtStartPar
42
&
\sphinxAtStartPar
UAB "Įmonė"
&
\sphinxAtStartPar
+37011111111
\\
\sphinxhline
\sphinxAtStartPar
2021\sphinxhyphen{}09\sphinxhyphen{}02
&
\sphinxAtStartPar
2
&
\sphinxAtStartPar
42
&
\sphinxAtStartPar
UAB "Įmonė"
&
\sphinxAtStartPar
+37022222222
\\
\sphinxhline
\sphinxAtStartPar
2021\sphinxhyphen{}09\sphinxhyphen{}03
&
\sphinxAtStartPar
3
&
\sphinxAtStartPar
42
&
\sphinxAtStartPar
UAB "Įmonė"
&
\sphinxAtStartPar
+37033333333
\\
\sphinxhline
\sphinxAtStartPar
2021\sphinxhyphen{}09\sphinxhyphen{}04
&
\sphinxAtStartPar
4
&
\sphinxAtStartPar
42
&
\sphinxAtStartPar
UAB "Įmonė"
&
\sphinxAtStartPar
+37044444444
\\
\sphinxbottomrule
\end{tabulary}
\sphinxtableafterendhook\par
\sphinxattableend\end{savenotes}


\begin{savenotes}\sphinxattablestart
\sphinxthistablewithglobalstyle
\centering
\begin{tabulary}{\linewidth}[t]{TTTTTTTTTT}
\sphinxtoprule
\sphinxstartmulticolumn{10}%
\begin{varwidth}[t]{\sphinxcolwidth{10}{10}}
\sphinxstyletheadfamily \sphinxAtStartPar
Struktūros aprašas
\par
\vskip-\baselineskip\vbox{\hbox{\strut}}\end{varwidth}%
\sphinxstopmulticolumn
\\
\sphinxhline\sphinxstyletheadfamily 
\sphinxAtStartPar
d
&\sphinxstyletheadfamily 
\sphinxAtStartPar
r
&\sphinxstyletheadfamily 
\sphinxAtStartPar
b
&\sphinxstyletheadfamily 
\sphinxAtStartPar
m
&\sphinxstyletheadfamily 
\sphinxAtStartPar
property
&\sphinxstyletheadfamily 
\sphinxAtStartPar
type
&\sphinxstyletheadfamily 
\sphinxAtStartPar
ref
&\sphinxstyletheadfamily 
\sphinxAtStartPar
level
&\sphinxstyletheadfamily 
\sphinxAtStartPar
prepare
&\sphinxstyletheadfamily 
\sphinxAtStartPar
title
\\
\sphinxmidrule
\sphinxtableatstartofbodyhook\sphinxstartmulticolumn{5}%
\begin{varwidth}[t]{\sphinxcolwidth{5}{10}}
\sphinxAtStartPar
example
\par
\vskip-\baselineskip\vbox{\hbox{\strut}}\end{varwidth}%
\sphinxstopmulticolumn
&&&&&\\
\sphinxhline
\sphinxAtStartPar

&&&\sphinxstartmulticolumn{2}%
\begin{varwidth}[t]{\sphinxcolwidth{2}{10}}
\sphinxAtStartPar
JuridinisAsmuo
\par
\vskip-\baselineskip\vbox{\hbox{\strut}}\end{varwidth}%
\sphinxstopmulticolumn
&&
\sphinxAtStartPar
kodas
&
\sphinxAtStartPar
4
&&\\
\sphinxhline
\sphinxAtStartPar

&&&&
\sphinxAtStartPar
kodas
&
\sphinxAtStartPar
integer
&&
\sphinxAtStartPar
4
&&\\
\sphinxhline
\sphinxAtStartPar

&&&&
\sphinxAtStartPar
pavadinimas@lt
&
\sphinxAtStartPar
text
&&
\sphinxAtStartPar
4
&&\\
\sphinxhline
\sphinxAtStartPar

&&\sphinxstartmulticolumn{3}%
\begin{varwidth}[t]{\sphinxcolwidth{3}{10}}
\sphinxAtStartPar
JuridinisAsmuo
\par
\vskip-\baselineskip\vbox{\hbox{\strut}}\end{varwidth}%
\sphinxstopmulticolumn
&&&
\sphinxAtStartPar
4
&&\\
\sphinxhline
\sphinxAtStartPar

&&&\sphinxstartmulticolumn{2}%
\begin{varwidth}[t]{\sphinxcolwidth{2}{10}}
\sphinxAtStartPar
Imone
\par
\vskip-\baselineskip\vbox{\hbox{\strut}}\end{varwidth}%
\sphinxstopmulticolumn
&&
\sphinxAtStartPar
kodas
&
\sphinxAtStartPar
4
&&\\
\sphinxhline
\sphinxAtStartPar

&&&&
\sphinxAtStartPar
kodas
&&&
\sphinxAtStartPar
4
&&\\
\sphinxhline
\sphinxAtStartPar

&&&&
\sphinxAtStartPar
pavadinimas@lt
&&&
\sphinxAtStartPar
4
&&\\
\sphinxhline
\sphinxAtStartPar

&&&&
\sphinxAtStartPar
rusis
&
\sphinxAtStartPar
string
&&
\sphinxAtStartPar
3
&&\\
\sphinxhline
\sphinxAtStartPar

&&\sphinxstartmulticolumn{3}%
\begin{varwidth}[t]{\sphinxcolwidth{3}{10}}
\sphinxAtStartPar
/
\par
\vskip-\baselineskip\vbox{\hbox{\strut}}\end{varwidth}%
\sphinxstopmulticolumn
&&&&&\\
\sphinxhline
\sphinxAtStartPar

&&&\sphinxstartmulticolumn{2}%
\begin{varwidth}[t]{\sphinxcolwidth{2}{10}}
\sphinxAtStartPar
Filialas
\par
\vskip-\baselineskip\vbox{\hbox{\strut}}\end{varwidth}%
\sphinxstopmulticolumn
&&&
\sphinxAtStartPar
3
&&\\
\sphinxhline
\sphinxAtStartPar

&&&&
\sphinxAtStartPar
ikurta
&
\sphinxAtStartPar
date
&&
\sphinxAtStartPar
3
&&\\
\sphinxhline
\sphinxAtStartPar

&&&&
\sphinxAtStartPar
atstumas
&
\sphinxAtStartPar
integer
&&
\sphinxAtStartPar
3
&&\\
\sphinxhline
\sphinxAtStartPar

&&&&
\sphinxAtStartPar
imone
&
\sphinxAtStartPar
ref
&
\sphinxAtStartPar
Imone
&
\sphinxAtStartPar
3
&&\\
\sphinxhline
\sphinxAtStartPar

&&&&
\sphinxAtStartPar
imone.kodas
&&&
\sphinxAtStartPar
4
&&\\
\sphinxhline
\sphinxAtStartPar

&&&&
\sphinxAtStartPar
imone.pavadinimas@lt
&&&
\sphinxAtStartPar
4
&&\\
\sphinxhline
\sphinxAtStartPar

&&&&
\sphinxAtStartPar
tel\_nr
&
\sphinxAtStartPar
string
&&
\sphinxAtStartPar
4
&&\\
\sphinxbottomrule
\end{tabulary}
\sphinxtableafterendhook\par
\sphinxattableend\end{savenotes}


\subsubsection{L301: Nėra globalaus objekto identifikatoriaus}
\label{\detokenize{branda:l301-nera-globalaus-objekto-identifikatoriaus}}\label{\detokenize{branda:l301}}
\sphinxAtStartPar
Nėra naudojamas globalus objekto identifikatorius, objektas identifikuojamas
naudojant tik lokalų identifikatorių. Tokiu atveju, objektas negali būti
nuskaitomas tiesiogiai, gali būti vykdoma tik atranka, nurodant filtrą, pagal
lokalų identifikatorių.
\begin{itemize}
\item {} 
\sphinxAtStartPar
\sphinxcode{\sphinxupquote{Filialas.imone}} \sphinxhyphen{} siejimas atliekamas per \sphinxcode{\sphinxupquote{Imone.kodas}}, o ne per
\sphinxcode{\sphinxupquote{Imone.\_id}}.

\end{itemize}


\subsubsection{L302: Nenurodyti matavimo vienetai}
\label{\detokenize{branda:l302-nenurodyti-matavimo-vienetai}}\label{\detokenize{branda:l302}}
\sphinxAtStartPar
Trečiu brandos lygiu žymimi kiekybiniai duomenys, kuriems nėra nurodyti
matavimo vienetai {\hyperref[\detokenize{dimensijos:property.ref}]{\sphinxcrossref{\sphinxcode{\sphinxupquote{property.ref}}}}} stulpelyje. Pavyzdžiui:
\begin{itemize}
\item {} 
\sphinxAtStartPar
\sphinxcode{\sphinxupquote{atstumas}} \sphinxhyphen{} nenurodyta, kokiais vienetais matuojamas atstumas.

\end{itemize}


\subsubsection{L303: Nenurodytas duomenų tikslumas}
\label{\detokenize{branda:l303-nenurodytas-duomenu-tikslumas}}\label{\detokenize{branda:l303}}
\sphinxAtStartPar
Trečiu brandos lygiu žymimi laiko ir erdviniai duomenys, kuriems nėra nurodytas
matavimo tikslumas. Matavimo tikslumas nurodomas \sphinxcode{\sphinxupquote{property.ref}} stulpelyje.
Pavyzdžiui:
\begin{itemize}
\item {} 
\sphinxAtStartPar
\sphinxcode{\sphinxupquote{ikurta}} \sphinxhyphen{} nenurodytas datos tikslumas, turėtu būti \sphinxcode{\sphinxupquote{D}} \sphinxhyphen{} vienos dienos
tiksumas.

\end{itemize}


\subsubsection{L304: Neaprašyti kategoriniai duomenys}
\label{\detokenize{branda:l304-neaprasyti-kategoriniai-duomenys}}\label{\detokenize{branda:l304}}
\sphinxAtStartPar
Trečiu brandos lygiu žymimi kategoriniai duomenys, kurių reikšmės pačios
savaime yra aiškios, tačiau neišvardintos struktūros apraše. Pavyzdžiui:
\begin{itemize}
\item {} 
\sphinxAtStartPar
\sphinxcode{\sphinxupquote{Imone.rusis}} \sphinxhyphen{} įmonės rūšies kategorijos duomenys yra pateikta
tekstine forma, tačiau, struktūros apraše nėra išvardintos visos
galimos kategorijos ir pats duomenų laukas nėra pažymėtas, kaip
kategorinis.

\end{itemize}


\subsection{L400: Nesusieata su žodynais}
\label{\detokenize{branda:l400-nesusieata-su-zodynais}}\label{\detokenize{branda:l400}}
\sphinxAtStartPar
Duomenų objektai turi aiškius, unikalius identifikatorius. Užpildyti
{\hyperref[\detokenize{dimensijos:model.ref}]{\sphinxcrossref{\sphinxcode{\sphinxupquote{model.ref}}}}} ir {\hyperref[\detokenize{dimensijos:property.ref}]{\sphinxcrossref{\sphinxcode{\sphinxupquote{property.ref}}}}} stulpeliai.

\begin{sphinxadmonition}{note}{Pastaba:}
\sphinxAtStartPar
{\hyperref[\detokenize{dimensijos:property.ref}]{\sphinxcrossref{\sphinxcode{\sphinxupquote{property.ref}}}}} stulpelis pildomas šiais atvejais:
\begin{itemize}
\item {} 
\sphinxAtStartPar
Jei duomenų laukas yra išorinis raktas (žiūrėti {\hyperref[\detokenize{identifikatoriai:ref-types}]{\sphinxcrossref{\DUrole{std}{\DUrole{std-ref}{Jungimo būdai}}}}}).

\item {} 
\sphinxAtStartPar
Jei duomenų laukas yra kiekybinis ir turi matavimo vienetus
(žiūrėti {\hyperref[\detokenize{vienetai:matavimo-vienetai}]{\sphinxcrossref{\DUrole{std}{\DUrole{std-ref}{Matavimo vienetai}}}}}).

\item {} 
\sphinxAtStartPar
Jei duomenų laukas žymi laiką ar vietą (žiūrėti
\DUrole{xref}{\DUrole{std}{\DUrole{std-ref}{temporal\sphinxhyphen{}types}}} ir \DUrole{xref}{\DUrole{std}{\DUrole{std-ref}{spatial\sphinxhyphen{}types}}}).

\end{itemize}
\end{sphinxadmonition}

\sphinxAtStartPar
Plačiau apie brandos lygio kėlimą skaitykite skyriuje \DUrole{xref}{\DUrole{std}{\DUrole{std-ref}{to\sphinxhyphen{}level\sphinxhyphen{}4}}}.

\sphinxAtStartPar
Ketvirtas duomenų brandos lygis labiau susijęs ne su pačių duomenų
formatu, bet su metaduomenimis, kurie lydi duomenis.

\sphinxAtStartPar
Duomenų struktūros apraše {\hyperref[\detokenize{dimensijos:model.ref}]{\sphinxcrossref{\sphinxcode{\sphinxupquote{model.ref}}}}} stulpelyje, pateikiamas
objektą unikaliai identifikuojančių laukų sąrašas, o
{\hyperref[\detokenize{dimensijos:property.type}]{\sphinxcrossref{\sphinxcode{\sphinxupquote{property.type}}}}} stulpelyje įrašomas \sphinxcode{\sphinxupquote{ref}} tipas, kuris nurodo
ryšį tarp dviejų objektų.

\sphinxAtStartPar
\sphinxstylestrong{Pavyzdžiai}


\begin{savenotes}\sphinxattablestart
\sphinxthistablewithglobalstyle
\centering
\begin{tabulary}{\linewidth}[t]{TTTT}
\sphinxtoprule
\sphinxstartmulticolumn{4}%
\begin{varwidth}[t]{\sphinxcolwidth{4}{4}}
\sphinxstyletheadfamily \sphinxAtStartPar
Imone
\par
\vskip-\baselineskip\vbox{\hbox{\strut}}\end{varwidth}%
\sphinxstopmulticolumn
\\
\sphinxhline\sphinxstyletheadfamily 
\sphinxAtStartPar
\_id
&\sphinxstyletheadfamily 
\sphinxAtStartPar
id
&\sphinxstyletheadfamily 
\sphinxAtStartPar
pavadinimas@lt
&\sphinxstyletheadfamily 
\sphinxAtStartPar
rusis
\\
\sphinxmidrule
\sphinxtableatstartofbodyhook
\sphinxAtStartPar
26510da5\sphinxhyphen{}f6a6\sphinxhyphen{}45b0\sphinxhyphen{}a9b9\sphinxhyphen{}27b3d0090a58
&
\sphinxAtStartPar
42
&
\sphinxAtStartPar
UAB "Įmonė"
&
\sphinxAtStartPar
1
\\
\sphinxbottomrule
\end{tabulary}
\sphinxtableafterendhook\par
\sphinxattableend\end{savenotes}


\begin{savenotes}\sphinxattablestart
\sphinxthistablewithglobalstyle
\centering
\begin{tabulary}{\linewidth}[t]{TTTTTTTT}
\sphinxtoprule
\sphinxstyletheadfamily 
\sphinxAtStartPar
Filialas
&\sphinxstyletheadfamily &\sphinxstartmulticolumn{6}%
\begin{varwidth}[t]{\sphinxcolwidth{6}{8}}
\sphinxstyletheadfamily \par
\vskip-\baselineskip\vbox{\hbox{\strut}}\end{varwidth}%
\sphinxstopmulticolumn
\\
\sphinxhline\sphinxstyletheadfamily 
\sphinxAtStartPar
\_id
&\sphinxstyletheadfamily 
\sphinxAtStartPar
id
&\sphinxstyletheadfamily 
\sphinxAtStartPar
ikurta
&\sphinxstyletheadfamily 
\sphinxAtStartPar
atstumas
&\sphinxstyletheadfamily 
\sphinxAtStartPar
imone.\_id
&\sphinxstyletheadfamily 
\sphinxAtStartPar
imone.id
&\sphinxstyletheadfamily 
\sphinxAtStartPar
imone.pavadinimas@lt
&\sphinxstyletheadfamily 
\sphinxAtStartPar
tel\_nr
\\
\sphinxmidrule
\sphinxtableatstartofbodyhook
\sphinxAtStartPar
63161bd2\sphinxhyphen{}158f\sphinxhyphen{}4d62\sphinxhyphen{}9804\sphinxhyphen{}636573abb9c7
&
\sphinxAtStartPar
1
&
\sphinxAtStartPar
2021\sphinxhyphen{}09\sphinxhyphen{}01
&
\sphinxAtStartPar
1
&
\sphinxAtStartPar
26510da5\sphinxhyphen{}f6a6\sphinxhyphen{}45b0\sphinxhyphen{}a9b9\sphinxhyphen{}27b3d0090a58
&
\sphinxAtStartPar
42
&
\sphinxAtStartPar
UAB "Įmonė"
&
\sphinxAtStartPar
+37011111111
\\
\sphinxhline
\sphinxAtStartPar
65ec7208\sphinxhyphen{}fb97\sphinxhyphen{}41a8\sphinxhyphen{}9cfc\sphinxhyphen{}dfedd197ced6
&
\sphinxAtStartPar
2
&
\sphinxAtStartPar
2021\sphinxhyphen{}09\sphinxhyphen{}02
&
\sphinxAtStartPar
2
&
\sphinxAtStartPar
26510da5\sphinxhyphen{}f6a6\sphinxhyphen{}45b0\sphinxhyphen{}a9b9\sphinxhyphen{}27b3d0090a58
&
\sphinxAtStartPar
42
&
\sphinxAtStartPar
UAB "Įmonė"
&
\sphinxAtStartPar
+37022222222
\\
\sphinxhline
\sphinxAtStartPar
2b8cdfa6\sphinxhyphen{}1396\sphinxhyphen{}431a\sphinxhyphen{}851c\sphinxhyphen{}c7c6eb7aa433
&
\sphinxAtStartPar
3
&
\sphinxAtStartPar
2021\sphinxhyphen{}09\sphinxhyphen{}03
&
\sphinxAtStartPar
3
&
\sphinxAtStartPar
26510da5\sphinxhyphen{}f6a6\sphinxhyphen{}45b0\sphinxhyphen{}a9b9\sphinxhyphen{}27b3d0090a58
&
\sphinxAtStartPar
42
&
\sphinxAtStartPar
UAB "Įmonė"
&
\sphinxAtStartPar
+37033333333
\\
\sphinxhline
\sphinxAtStartPar
1882bb9e\sphinxhyphen{}73ee\sphinxhyphen{}4057\sphinxhyphen{}b04d\sphinxhyphen{}d4af47f0aae8
&
\sphinxAtStartPar
4
&
\sphinxAtStartPar
2021\sphinxhyphen{}09\sphinxhyphen{}04
&
\sphinxAtStartPar
4
&
\sphinxAtStartPar
26510da5\sphinxhyphen{}f6a6\sphinxhyphen{}45b0\sphinxhyphen{}a9b9\sphinxhyphen{}27b3d0090a58
&
\sphinxAtStartPar
42
&
\sphinxAtStartPar
UAB "Įmonė"
&
\sphinxAtStartPar
+37044444444
\\
\sphinxbottomrule
\end{tabulary}
\sphinxtableafterendhook\par
\sphinxattableend\end{savenotes}


\begin{savenotes}\sphinxattablestart
\sphinxthistablewithglobalstyle
\centering
\begin{tabulary}{\linewidth}[t]{TTTTTTTTTT}
\sphinxtoprule
\sphinxstartmulticolumn{10}%
\begin{varwidth}[t]{\sphinxcolwidth{10}{10}}
\sphinxstyletheadfamily \sphinxAtStartPar
Struktūros aprašas
\par
\vskip-\baselineskip\vbox{\hbox{\strut}}\end{varwidth}%
\sphinxstopmulticolumn
\\
\sphinxhline\sphinxstyletheadfamily 
\sphinxAtStartPar
d
&\sphinxstyletheadfamily 
\sphinxAtStartPar
r
&\sphinxstyletheadfamily 
\sphinxAtStartPar
b
&\sphinxstyletheadfamily 
\sphinxAtStartPar
m
&\sphinxstyletheadfamily 
\sphinxAtStartPar
property
&\sphinxstyletheadfamily 
\sphinxAtStartPar
type
&\sphinxstyletheadfamily 
\sphinxAtStartPar
ref
&\sphinxstyletheadfamily 
\sphinxAtStartPar
level
&\sphinxstyletheadfamily 
\sphinxAtStartPar
prepare
&\sphinxstyletheadfamily 
\sphinxAtStartPar
title
\\
\sphinxmidrule
\sphinxtableatstartofbodyhook\sphinxstartmulticolumn{5}%
\begin{varwidth}[t]{\sphinxcolwidth{5}{10}}
\sphinxAtStartPar
example
\par
\vskip-\baselineskip\vbox{\hbox{\strut}}\end{varwidth}%
\sphinxstopmulticolumn
&&&&&\\
\sphinxhline
\sphinxAtStartPar

&&&\sphinxstartmulticolumn{2}%
\begin{varwidth}[t]{\sphinxcolwidth{2}{10}}
\sphinxAtStartPar
JuridinisAsmuo
\par
\vskip-\baselineskip\vbox{\hbox{\strut}}\end{varwidth}%
\sphinxstopmulticolumn
&&
\sphinxAtStartPar
kodas
&
\sphinxAtStartPar
4
&&\\
\sphinxhline
\sphinxAtStartPar

&&&&
\sphinxAtStartPar
kodas
&
\sphinxAtStartPar
integer
&&
\sphinxAtStartPar
4
&&\\
\sphinxhline
\sphinxAtStartPar

&&&&
\sphinxAtStartPar
pavadinimas@lt
&
\sphinxAtStartPar
text
&&
\sphinxAtStartPar
4
&&\\
\sphinxhline
\sphinxAtStartPar

&&\sphinxstartmulticolumn{3}%
\begin{varwidth}[t]{\sphinxcolwidth{3}{10}}
\sphinxAtStartPar
JuridinisAsmuo
\par
\vskip-\baselineskip\vbox{\hbox{\strut}}\end{varwidth}%
\sphinxstopmulticolumn
&&&
\sphinxAtStartPar
4
&&\\
\sphinxhline
\sphinxAtStartPar

&&&\sphinxstartmulticolumn{2}%
\begin{varwidth}[t]{\sphinxcolwidth{2}{10}}
\sphinxAtStartPar
Imone
\par
\vskip-\baselineskip\vbox{\hbox{\strut}}\end{varwidth}%
\sphinxstopmulticolumn
&&
\sphinxAtStartPar
kodas
&
\sphinxAtStartPar
4
&&\\
\sphinxhline
\sphinxAtStartPar

&&&&
\sphinxAtStartPar
id
&
\sphinxAtStartPar
integer
&&
\sphinxAtStartPar
4
&&\\
\sphinxhline
\sphinxAtStartPar

&&&&
\sphinxAtStartPar
pavadinimas@lt
&
\sphinxAtStartPar
text
&&
\sphinxAtStartPar
4
&&\\
\sphinxhline
\sphinxAtStartPar

&&&&
\sphinxAtStartPar
rusis
&
\sphinxAtStartPar
integer
&&
\sphinxAtStartPar
4
&&\\
\sphinxhline
\sphinxAtStartPar

&&&&&
\sphinxAtStartPar
enum
&&&
\sphinxAtStartPar
1
&
\sphinxAtStartPar
Juridinis
\\
\sphinxhline
\sphinxAtStartPar

&&&&&&&&
\sphinxAtStartPar
2
&
\sphinxAtStartPar
Fizinis
\\
\sphinxhline
\sphinxAtStartPar

&&\sphinxstartmulticolumn{3}%
\begin{varwidth}[t]{\sphinxcolwidth{3}{10}}
\sphinxAtStartPar
/
\par
\vskip-\baselineskip\vbox{\hbox{\strut}}\end{varwidth}%
\sphinxstopmulticolumn
&&&&&\\
\sphinxhline
\sphinxAtStartPar

&&&\sphinxstartmulticolumn{2}%
\begin{varwidth}[t]{\sphinxcolwidth{2}{10}}
\sphinxAtStartPar
Filialas
\par
\vskip-\baselineskip\vbox{\hbox{\strut}}\end{varwidth}%
\sphinxstopmulticolumn
&&
\sphinxAtStartPar
id
&
\sphinxAtStartPar
4
&&\\
\sphinxhline
\sphinxAtStartPar

&&&&
\sphinxAtStartPar
id
&
\sphinxAtStartPar
integer
&&
\sphinxAtStartPar
4
&&\\
\sphinxhline
\sphinxAtStartPar

&&&&
\sphinxAtStartPar
ikurta
&
\sphinxAtStartPar
date
&
\sphinxAtStartPar
D
&
\sphinxAtStartPar
4
&&\\
\sphinxhline
\sphinxAtStartPar

&&&&
\sphinxAtStartPar
atstumas
&
\sphinxAtStartPar
integer
&
\sphinxAtStartPar
km
&
\sphinxAtStartPar
4
&&\\
\sphinxhline
\sphinxAtStartPar

&&&&
\sphinxAtStartPar
imone
&
\sphinxAtStartPar
ref
&
\sphinxAtStartPar
Imone
&
\sphinxAtStartPar
4
&&\\
\sphinxhline
\sphinxAtStartPar

&&&&
\sphinxAtStartPar
imone.id
&&&
\sphinxAtStartPar
4
&&\\
\sphinxhline
\sphinxAtStartPar

&&&&
\sphinxAtStartPar
imone.pavadinimas@lt
&&&
\sphinxAtStartPar
4
&&\\
\sphinxhline
\sphinxAtStartPar

&&&&
\sphinxAtStartPar
tel\_nr
&
\sphinxAtStartPar
string
&&
\sphinxAtStartPar
4
&&\\
\sphinxbottomrule
\end{tabulary}
\sphinxtableafterendhook\par
\sphinxattableend\end{savenotes}


\subsubsection{L401: Nesusieta su standartiniu žodynu}
\label{\detokenize{branda:l401-nesusieta-su-standartiniu-zodynu}}\label{\detokenize{branda:l401}}
\sphinxAtStartPar
Ketvirtu brandos lygiu žimimi duomenys, kurie nėra susieti su standartiniais
žodynais ar ontologijomis. Siejimas su žodynais atliekamas \sphinxcode{\sphinxupquote{model.uri}} ir
\sphinxcode{\sphinxupquote{property.uri}} stulpeluose.


\subsection{L500: Trūkumų nėra}
\label{\detokenize{branda:l500-trukumu-nera}}\label{\detokenize{branda:l500}}
\sphinxAtStartPar
Modeliai iš įstaigų duomenų rinkinių vardų erdvės susieti su modeliais
iš standartų vardų erdvės, užpildyta {\hyperref[\detokenize{formatas:base}]{\sphinxcrossref{\sphinxcode{\sphinxupquote{base}}}}} eilutė. Standartų
vardų erdvėje esantiems {\hyperref[\detokenize{savokos:term-modelis}]{\sphinxtermref{\DUrole{xref}{\DUrole{std}{\DUrole{std-term}{modeliams}}}}}} ir jų
{\hyperref[\detokenize{savokos:term-savybe}]{\sphinxtermref{\DUrole{xref}{\DUrole{std}{\DUrole{std-term}{savybėms}}}}}} užpildytas {\hyperref[\detokenize{formatas:uri}]{\sphinxcrossref{\sphinxcode{\sphinxupquote{uri}}}}} stulpelis.

\sphinxAtStartPar
Daugiau apie vardų erdves skaitykite skyrelyje: \DUrole{xref}{\DUrole{std}{\DUrole{std-ref}{vardu\sphinxhyphen{}erdves}}}.

\sphinxAtStartPar
Plačiau apie brandos lygio kėlimą skaitykite skyriuje \DUrole{xref}{\DUrole{std}{\DUrole{std-ref}{to\sphinxhyphen{}level\sphinxhyphen{}5}}}.

\sphinxAtStartPar
Penkto brandos lygio duomenys yra lygiai tokie patys, kaip ir ketvirto
brandos lygio, tačiau penktame brandos lygyje, duomenys yra praturtinami
metaduomenimis, pateikiant nuorodas į išorinius žodynus arba bent jau
pateikiant aiškius pavadinimus ir aprašymus, užpildant \sphinxcode{\sphinxupquote{title}} ir
\sphinxcode{\sphinxupquote{description}} stulpelius.

\sphinxAtStartPar
Penktame brandos lygyje visas dėmesys yra sutelkiamas į semantinę
duomenų prasmę.

\sphinxAtStartPar
\sphinxstylestrong{Pavyzdžiai}


\begin{savenotes}\sphinxattablestart
\sphinxthistablewithglobalstyle
\centering
\begin{tabulary}{\linewidth}[t]{TTTT}
\sphinxtoprule
\sphinxstartmulticolumn{4}%
\begin{varwidth}[t]{\sphinxcolwidth{4}{4}}
\sphinxstyletheadfamily \sphinxAtStartPar
Imone
\par
\vskip-\baselineskip\vbox{\hbox{\strut}}\end{varwidth}%
\sphinxstopmulticolumn
\\
\sphinxhline\sphinxstyletheadfamily 
\sphinxAtStartPar
\_id
&\sphinxstyletheadfamily 
\sphinxAtStartPar
id
&\sphinxstyletheadfamily 
\sphinxAtStartPar
pavadinimas@lt
&\sphinxstyletheadfamily 
\sphinxAtStartPar
rusis
\\
\sphinxmidrule
\sphinxtableatstartofbodyhook
\sphinxAtStartPar
26510da5\sphinxhyphen{}f6a6\sphinxhyphen{}45b0\sphinxhyphen{}a9b9\sphinxhyphen{}27b3d0090a58
&
\sphinxAtStartPar
42
&
\sphinxAtStartPar
UAB "Įmonė"
&
\sphinxAtStartPar
1
\\
\sphinxbottomrule
\end{tabulary}
\sphinxtableafterendhook\par
\sphinxattableend\end{savenotes}


\begin{savenotes}\sphinxattablestart
\sphinxthistablewithglobalstyle
\centering
\begin{tabulary}{\linewidth}[t]{TTTTTTTT}
\sphinxtoprule
\sphinxstyletheadfamily 
\sphinxAtStartPar
Filialas
&\sphinxstartmulticolumn{7}%
\begin{varwidth}[t]{\sphinxcolwidth{7}{8}}
\sphinxstyletheadfamily \par
\vskip-\baselineskip\vbox{\hbox{\strut}}\end{varwidth}%
\sphinxstopmulticolumn
\\
\sphinxhline\sphinxstyletheadfamily 
\sphinxAtStartPar
\_id
&\sphinxstyletheadfamily 
\sphinxAtStartPar
id
&\sphinxstyletheadfamily 
\sphinxAtStartPar
ikurta
&\sphinxstyletheadfamily 
\sphinxAtStartPar
atstumas
&\sphinxstyletheadfamily 
\sphinxAtStartPar
imone.\_id
&\sphinxstyletheadfamily 
\sphinxAtStartPar
imone.id
&\sphinxstyletheadfamily 
\sphinxAtStartPar
imone.pavadinimas@lt
&\sphinxstyletheadfamily 
\sphinxAtStartPar
tel\_nr
\\
\sphinxmidrule
\sphinxtableatstartofbodyhook
\sphinxAtStartPar
63161bd2\sphinxhyphen{}158f\sphinxhyphen{}4d62\sphinxhyphen{}9804\sphinxhyphen{}636573abb9c7
&
\sphinxAtStartPar
1
&
\sphinxAtStartPar
2021\sphinxhyphen{}09\sphinxhyphen{}01
&
\sphinxAtStartPar
1
&
\sphinxAtStartPar
26510da5\sphinxhyphen{}f6a6\sphinxhyphen{}45b0\sphinxhyphen{}a9b9\sphinxhyphen{}27b3d0090a58
&
\sphinxAtStartPar
42
&
\sphinxAtStartPar
UAB "Įmonė"
&
\sphinxAtStartPar
tel:+37011111111
\\
\sphinxhline
\sphinxAtStartPar
65ec7208\sphinxhyphen{}fb97\sphinxhyphen{}41a8\sphinxhyphen{}9cfc\sphinxhyphen{}dfedd197ced6
&
\sphinxAtStartPar
2
&
\sphinxAtStartPar
2021\sphinxhyphen{}09\sphinxhyphen{}02
&
\sphinxAtStartPar
2
&
\sphinxAtStartPar
26510da5\sphinxhyphen{}f6a6\sphinxhyphen{}45b0\sphinxhyphen{}a9b9\sphinxhyphen{}27b3d0090a58
&
\sphinxAtStartPar
42
&
\sphinxAtStartPar
UAB "Įmonė"
&
\sphinxAtStartPar
tel:+37022222222
\\
\sphinxhline
\sphinxAtStartPar
2b8cdfa6\sphinxhyphen{}1396\sphinxhyphen{}431a\sphinxhyphen{}851c\sphinxhyphen{}c7c6eb7aa433
&
\sphinxAtStartPar
3
&
\sphinxAtStartPar
2021\sphinxhyphen{}09\sphinxhyphen{}03
&
\sphinxAtStartPar
3
&
\sphinxAtStartPar
26510da5\sphinxhyphen{}f6a6\sphinxhyphen{}45b0\sphinxhyphen{}a9b9\sphinxhyphen{}27b3d0090a58
&
\sphinxAtStartPar
42
&
\sphinxAtStartPar
UAB "Įmonė"
&
\sphinxAtStartPar
tel:+37033333333
\\
\sphinxhline
\sphinxAtStartPar
1882bb9e\sphinxhyphen{}73ee\sphinxhyphen{}4057\sphinxhyphen{}b04d\sphinxhyphen{}d4af47f0aae8
&
\sphinxAtStartPar
4
&
\sphinxAtStartPar
2021\sphinxhyphen{}09\sphinxhyphen{}04
&
\sphinxAtStartPar
4
&
\sphinxAtStartPar
26510da5\sphinxhyphen{}f6a6\sphinxhyphen{}45b0\sphinxhyphen{}a9b9\sphinxhyphen{}27b3d0090a58
&
\sphinxAtStartPar
42
&
\sphinxAtStartPar
UAB "Įmonė"
&
\sphinxAtStartPar
tel:+37044444444
\\
\sphinxbottomrule
\end{tabulary}
\sphinxtableafterendhook\par
\sphinxattableend\end{savenotes}


\begin{savenotes}\sphinxattablestart
\sphinxthistablewithglobalstyle
\centering
\begin{tabulary}{\linewidth}[t]{TTTTTTTTTTT}
\sphinxtoprule
\sphinxstartmulticolumn{11}%
\begin{varwidth}[t]{\sphinxcolwidth{11}{11}}
\sphinxstyletheadfamily \sphinxAtStartPar
Struktūros aprašas
\par
\vskip-\baselineskip\vbox{\hbox{\strut}}\end{varwidth}%
\sphinxstopmulticolumn
\\
\sphinxhline\sphinxstyletheadfamily 
\sphinxAtStartPar
d
&\sphinxstyletheadfamily 
\sphinxAtStartPar
r
&\sphinxstyletheadfamily 
\sphinxAtStartPar
b
&\sphinxstyletheadfamily 
\sphinxAtStartPar
m
&\sphinxstyletheadfamily 
\sphinxAtStartPar
property
&\sphinxstyletheadfamily 
\sphinxAtStartPar
type
&\sphinxstyletheadfamily 
\sphinxAtStartPar
ref
&\sphinxstyletheadfamily 
\sphinxAtStartPar
level
&\sphinxstyletheadfamily 
\sphinxAtStartPar
uri
&\sphinxstyletheadfamily 
\sphinxAtStartPar
prepare
&\sphinxstyletheadfamily 
\sphinxAtStartPar
title
\\
\sphinxmidrule
\sphinxtableatstartofbodyhook\sphinxstartmulticolumn{5}%
\begin{varwidth}[t]{\sphinxcolwidth{5}{11}}
\sphinxAtStartPar
example
\par
\vskip-\baselineskip\vbox{\hbox{\strut}}\end{varwidth}%
\sphinxstopmulticolumn
&&&&&&\\
\sphinxhline
\sphinxAtStartPar

&&&&&
\sphinxAtStartPar
prefix
&
\sphinxAtStartPar
foaf
&&
\sphinxAtStartPar
http://xmlns.com/foaf/0.1/
&&\\
\sphinxhline
\sphinxAtStartPar

&&&&&&
\sphinxAtStartPar
dct
&&
\sphinxAtStartPar
http://purl.org/dc/terms/
&&\\
\sphinxhline
\sphinxAtStartPar

&&&&&&
\sphinxAtStartPar
schema
&&
\sphinxAtStartPar
http://schema.org/
&&\\
\sphinxhline
\sphinxAtStartPar

&&&\sphinxstartmulticolumn{2}%
\begin{varwidth}[t]{\sphinxcolwidth{2}{11}}
\sphinxAtStartPar
JuridinisAsmuo
\par
\vskip-\baselineskip\vbox{\hbox{\strut}}\end{varwidth}%
\sphinxstopmulticolumn
&&
\sphinxAtStartPar
kodas
&
\sphinxAtStartPar
4
&&&\\
\sphinxhline
\sphinxAtStartPar

&&&&
\sphinxAtStartPar
kodas
&
\sphinxAtStartPar
integer
&&
\sphinxAtStartPar
4
&&&\\
\sphinxhline
\sphinxAtStartPar

&&&&
\sphinxAtStartPar
pavadinimas@lt
&
\sphinxAtStartPar
text
&&
\sphinxAtStartPar
4
&&&\\
\sphinxhline
\sphinxAtStartPar

&&\sphinxstartmulticolumn{3}%
\begin{varwidth}[t]{\sphinxcolwidth{3}{11}}
\sphinxAtStartPar
JuridinisAsmuo
\par
\vskip-\baselineskip\vbox{\hbox{\strut}}\end{varwidth}%
\sphinxstopmulticolumn
&&&
\sphinxAtStartPar
4
&&&\\
\sphinxhline
\sphinxAtStartPar

&&&\sphinxstartmulticolumn{2}%
\begin{varwidth}[t]{\sphinxcolwidth{2}{11}}
\sphinxAtStartPar
Imone
\par
\vskip-\baselineskip\vbox{\hbox{\strut}}\end{varwidth}%
\sphinxstopmulticolumn
&&
\sphinxAtStartPar
id
&
\sphinxAtStartPar
5
&
\sphinxAtStartPar
foaf:Organization
&&\\
\sphinxhline
\sphinxAtStartPar

&&&&
\sphinxAtStartPar
id
&&&
\sphinxAtStartPar
5
&
\sphinxAtStartPar
dct:identifier
&&\\
\sphinxhline
\sphinxAtStartPar

&&&&
\sphinxAtStartPar
pavadinimas@lt
&&&
\sphinxAtStartPar
5
&
\sphinxAtStartPar
dct:title
&&\\
\sphinxhline
\sphinxAtStartPar

&&&&
\sphinxAtStartPar
rusis
&
\sphinxAtStartPar
integer
&&
\sphinxAtStartPar
4
&&&\\
\sphinxhline
\sphinxAtStartPar

&&&&&
\sphinxAtStartPar
enum
&&&&
\sphinxAtStartPar
1
&
\sphinxAtStartPar
Juridinis
\\
\sphinxhline
\sphinxAtStartPar

&&&&&&&&&
\sphinxAtStartPar
2
&
\sphinxAtStartPar
Fizinis
\\
\sphinxhline
\sphinxAtStartPar

&&\sphinxstartmulticolumn{3}%
\begin{varwidth}[t]{\sphinxcolwidth{3}{11}}
\sphinxAtStartPar
/
\par
\vskip-\baselineskip\vbox{\hbox{\strut}}\end{varwidth}%
\sphinxstopmulticolumn
&&&&&&\\
\sphinxhline
\sphinxAtStartPar

&&&\sphinxstartmulticolumn{2}%
\begin{varwidth}[t]{\sphinxcolwidth{2}{11}}
\sphinxAtStartPar
Filialas
\par
\vskip-\baselineskip\vbox{\hbox{\strut}}\end{varwidth}%
\sphinxstopmulticolumn
&&
\sphinxAtStartPar
id
&
\sphinxAtStartPar
5
&
\sphinxAtStartPar
schema:LocalBusiness
&&\\
\sphinxhline
\sphinxAtStartPar

&&&&
\sphinxAtStartPar
id
&
\sphinxAtStartPar
date
&
\sphinxAtStartPar
1D
&
\sphinxAtStartPar
5
&
\sphinxAtStartPar
dct:identifier
&&\\
\sphinxhline
\sphinxAtStartPar

&&&&
\sphinxAtStartPar
ikurta
&
\sphinxAtStartPar
date
&
\sphinxAtStartPar
1D
&
\sphinxAtStartPar
5
&
\sphinxAtStartPar
dct:created
&&\\
\sphinxhline
\sphinxAtStartPar

&&&&
\sphinxAtStartPar
atstumas
&
\sphinxAtStartPar
integer
&
\sphinxAtStartPar
km
&
\sphinxAtStartPar
5
&
\sphinxAtStartPar
schema:distance
&&\\
\sphinxhline
\sphinxAtStartPar

&&&&
\sphinxAtStartPar
imone
&
\sphinxAtStartPar
ref
&
\sphinxAtStartPar
Imone
&
\sphinxAtStartPar
5
&
\sphinxAtStartPar
foaf:Organization
&&\\
\sphinxhline
\sphinxAtStartPar

&&&&
\sphinxAtStartPar
imone.id
&
\sphinxAtStartPar
integer
&&
\sphinxAtStartPar
5
&
\sphinxAtStartPar
dct:identifier
&&\\
\sphinxhline
\sphinxAtStartPar

&&&&
\sphinxAtStartPar
imone.pavadinimas@lt
&
\sphinxAtStartPar
text
&&
\sphinxAtStartPar
5
&
\sphinxAtStartPar
dct:title
&&\\
\sphinxhline
\sphinxAtStartPar

&&&&
\sphinxAtStartPar
tel\_nr
&
\sphinxAtStartPar
string
&&
\sphinxAtStartPar
5
&
\sphinxAtStartPar
foaf:phone
&&\\
\sphinxbottomrule
\end{tabulary}
\sphinxtableafterendhook\par
\sphinxattableend\end{savenotes}

\sphinxstepscope


\section{Prieigos lygiai}
\label{\detokenize{prieiga:prieigos-lygiai}}\label{\detokenize{prieiga:access}}\label{\detokenize{prieiga::doc}}
\sphinxAtStartPar
Duomenų prieigos lygis nurodomas {\hyperref[\detokenize{prieiga:id0}]{\sphinxcrossref{\sphinxcode{\sphinxupquote{access}}}}} stulpelyje.
\index{access (įtaisytasis kintamasis)@\spxentry{access}\spxextra{įtaisytasis kintamasis}}

\begin{fulllineitems}
\phantomsection\label{\detokenize{prieiga:id0}}
\pysigstartsignatures
\pysigline
{\sphinxbfcode{\sphinxupquote{access}}}
\pysigstopsignatures\index{open (įtaisytasis kintamasis)@\spxentry{open}\spxextra{įtaisytasis kintamasis}}

\begin{fulllineitems}
\phantomsection\label{\detokenize{prieiga:open}}
\pysigstartsignatures
\pysigline
{\sphinxbfcode{\sphinxupquote{open}}}
\pysigstopsignatures
\sphinxAtStartPar
\sphinxstylestrong{Atviri duomenys}

\sphinxAtStartPar
Duomenys skirti viešam naudojimui, neribojant panaudojimo tikslo, pagal
vieną iš atvirų duomenų licencijų.

\end{fulllineitems}

\index{public (įtaisytasis kintamasis)@\spxentry{public}\spxextra{įtaisytasis kintamasis}}

\begin{fulllineitems}
\phantomsection\label{\detokenize{prieiga:public}}
\pysigstartsignatures
\pysigline
{\sphinxbfcode{\sphinxupquote{public}}}
\pysigstopsignatures
\sphinxAtStartPar
\sphinxstylestrong{Ribotos prieigos duomenys}

\sphinxAtStartPar
Duomenys skirti viešam naudojimui, tiek privačiame tiek viešąjame
sektoriuje, pagal duomenų teikimo sutartį.

\end{fulllineitems}

\index{protected (įtaisytasis kintamasis)@\spxentry{protected}\spxextra{įtaisytasis kintamasis}}

\begin{fulllineitems}
\phantomsection\label{\detokenize{prieiga:protected}}
\pysigstartsignatures
\pysigline
{\sphinxbfcode{\sphinxupquote{protected}}}
\pysigstopsignatures
\sphinxAtStartPar
\sphinxstylestrong{Valstybinio sektoriaus duomenys}

\sphinxAtStartPar
Duomenys skirtin naudoti tik tarp valstybinio sektoriaus institucijų,
neteikiami privačiam sektoriui.

\end{fulllineitems}

\index{private (įtaisytasis kintamasis)@\spxentry{private}\spxextra{įtaisytasis kintamasis}}

\begin{fulllineitems}
\phantomsection\label{\detokenize{prieiga:private}}
\pysigstartsignatures
\pysigline
{\sphinxbfcode{\sphinxupquote{private}}}
\pysigstopsignatures
\sphinxAtStartPar
\sphinxstylestrong{Vidiniai duomenys}

\sphinxAtStartPar
Duomenys skirti tik vidiniam konkrečios sistemos naudojimui, neteikiami
už vienos sistemos ribų.

\end{fulllineitems}


\end{fulllineitems}


\sphinxAtStartPar
Viešam pakartotiniam naudojimui gali būti teikiami tik \sphinxcode{\sphinxupquote{public}} ir \sphinxcode{\sphinxupquote{open}}
prieigos lygio duomenys.

\sphinxAtStartPar
\sphinxcode{\sphinxupquote{public}} duomenys gali būti teikiami tik autorizuotiems duomenų valdytojams,
kurie yra susipažinę ir sutinka su duomenų naudojimo taisyklėmis ir naudoja
duomenis tik \sphinxhref{https://gdpr-info.eu/art-5-gdpr/}{nurodytu tikslu} (\sphinxstyleemphasis{purpose limitation}), laikosi \sphinxhref{https://gdpr-info.eu/}{BDAR}
reikalavimų.
Asmens duomenys gali būti viešinami tik public ar žemesniu prieigos lygiu.

\sphinxAtStartPar
\sphinxcode{\sphinxupquote{open}} duomenys turėtu būti teikiami atvirai be jokios autorizacijos ir
neribojant duomenų naudojimo tikslo. Asmens duomenys negali būti teikiami \sphinxcode{\sphinxupquote{open}}
prieigos lygiu.

\sphinxAtStartPar
Prieigos lygiai gali būti paveldimi iš aukštesnės dimensijos. Tačiau žemesnė
dimensija apsprendžia realų prieigos lygį. Pavyzdžiui jei {\hyperref[\detokenize{dimensijos:dataset.access}]{\sphinxcrossref{\sphinxcode{\sphinxupquote{dataset.access}}}}}
yra \sphinxcode{\sphinxupquote{private}}, o toje {\hyperref[\detokenize{formatas:dataset}]{\sphinxcrossref{\sphinxcode{\sphinxupquote{dataset}}}}} dimensijoje esantis {\hyperref[\detokenize{formatas:property}]{\sphinxcrossref{\sphinxcode{\sphinxupquote{property}}}}} yra
\sphinxcode{\sphinxupquote{open}}, tada visos to {\hyperref[\detokenize{formatas:property}]{\sphinxcrossref{\sphinxcode{\sphinxupquote{property}}}}} aukštesnės dimensijos taip pat tampa
\sphinxcode{\sphinxupquote{open}}, nors visos kitos dimensijos yra \sphinxcode{\sphinxupquote{private}}, nes paveldi
{\hyperref[\detokenize{dimensijos:dataset.access}]{\sphinxcrossref{\sphinxcode{\sphinxupquote{dataset.access}}}}} reikšmę.

\sphinxstepscope


\section{Duomenų schemos}
\label{\detokenize{schemos/index:duomenu-schemos}}\label{\detokenize{schemos/index::doc}}
\sphinxAtStartPar
Duomenų struktūra aprašoma schemos pagalba.

\sphinxAtStartPar
Žemiau rasite sąrašą schemų kalbų su aprašymais, kaip kiekviena schemų kalba
siejama su duomenų struktūros aprašu.


\subsection{Schemų sąrašas}
\label{\detokenize{schemos/index:schemu-sarasas}}
\sphinxstepscope


\subsubsection{XSD}
\label{\detokenize{schemos/xsd:xsd}}\label{\detokenize{schemos/xsd::doc}}
\sphinxAtStartPar
\sphinxhref{https://www.w3.org/TR/xmlschema11-1/}{XML Schema Definition (XSD)} yra schemų kalba skirta XML duomenims aprašyti.

\sphinxAtStartPar
XSD specifikacija susideda iš šių dalių:
\begin{itemize}
\item {} 
\sphinxAtStartPar
\sphinxhref{https://www.w3.org/TR/xmlschema11-1/}{W3C XML Schema Definition Language (XSD) 1.1 Part 1: Structures}

\item {} 
\sphinxAtStartPar
\sphinxhref{https://www.w3.org/TR/xmlschema11-2/}{W3C XML Schema Definition Language (XSD) 1.1 Part 2: Datatypes}

\end{itemize}


\paragraph{Jungtinis modelis}
\label{\detokenize{schemos/xsd:jungtinis-modelis}}\label{\detokenize{schemos/xsd:xsd-aggregate-model}}
\sphinxAtStartPar
Jungtiniai modeliai (angl. \sphinxstyleemphasis{Aggregate Model}) yra modelis, kuris yra sujungtas
iš vieno ar kelių papildomų modelių.


\subparagraph{Šakninis modelis}
\label{\detokenize{schemos/xsd:sakninis-modelis}}\label{\detokenize{schemos/xsd:xsd-aggregate-root}}
\sphinxAtStartPar
Šakninis modelis (angl. \sphinxstyleemphasis{Aggregate Root}) yra jungtinio modelio pradžios
taškas, per kurį pasiekiami kitų jungtinio modelio sudėtyje esančių modelių
duomenys.


\subparagraph{Dalinis modelis}
\label{\detokenize{schemos/xsd:dalinis-modelis}}\label{\detokenize{schemos/xsd:xsd-aggregate-part}}
\sphinxAtStartPar
Dalinis modelis (angl. \sphinxstyleemphasis{Aggregate Part}) yra Jungtinio modelio sudedamoji dalis
ir atskirai nenaudojamas duomenims gauti. Dalinio modelio duomenys yra teikiami
tik kaip jungtinio modelio dalis.

\sphinxAtStartPar
Dalinio modelio atveju, nėra pildomas {\hyperref[\detokenize{dimensijos:model.source}]{\sphinxcrossref{\sphinxcode{\sphinxupquote{model.source}}}}}, kadangi dalinio
modelio duomenys gali būti pasiekiami tik per jungtinį modelį.

\begin{sphinxadmonition}{note}{Pavyzdys}

\sphinxAtStartPar
\sphinxstylestrong{Duomenys}

\begin{sphinxVerbatim}[commandchars=\\\{\}]
\PYG{n+nt}{\PYGZlt{}country}\PYG{+w}{ }\PYG{n+na}{name=}\PYG{l+s}{\PYGZdq{}Lietuva\PYGZdq{}}\PYG{n+nt}{\PYGZgt{}}
\PYG{+w}{    }\PYG{n+nt}{\PYGZlt{}city}\PYG{+w}{ }\PYG{n+na}{name=}\PYG{l+s}{\PYGZdq{}Vilnius\PYGZdq{}}\PYG{+w}{ }\PYG{n+nt}{/\PYGZgt{}}
\PYG{n+nt}{\PYGZlt{}/country\PYGZgt{}}
\end{sphinxVerbatim}

\sphinxAtStartPar
\sphinxstylestrong{Schema}

\begin{sphinxVerbatim}[commandchars=\\\{\}]
\PYG{n+nt}{\PYGZlt{}xs:schema}\PYG{+w}{ }\PYG{n+na}{xmlns:xs=}\PYG{l+s}{\PYGZdq{}http://www.w3.org/2001/XMLSchema\PYGZdq{}}\PYG{n+nt}{\PYGZgt{}}

\PYG{+w}{    }\PYG{n+nt}{\PYGZlt{}xs:complexType}\PYG{+w}{ }\PYG{n+na}{name=}\PYG{l+s}{\PYGZdq{}city\PYGZdq{}}\PYG{n+nt}{\PYGZgt{}}
\PYG{+w}{        }\PYG{n+nt}{\PYGZlt{}xs:attribute}\PYG{+w}{ }\PYG{n+na}{name=}\PYG{l+s}{\PYGZdq{}name\PYGZdq{}}\PYG{+w}{ }\PYG{n+na}{type=}\PYG{l+s}{\PYGZdq{}xs:string\PYGZdq{}}\PYG{+w}{ }\PYG{n+nt}{/\PYGZgt{}}
\PYG{+w}{    }\PYG{n+nt}{\PYGZlt{}/xs:complexType\PYGZgt{}}

\PYG{+w}{    }\PYG{n+nt}{\PYGZlt{}xs:element}\PYG{+w}{ }\PYG{n+na}{name=}\PYG{l+s}{\PYGZdq{}country\PYGZdq{}}\PYG{n+nt}{\PYGZgt{}}
\PYG{+w}{        }\PYG{n+nt}{\PYGZlt{}xs:complexType}\PYG{n+nt}{\PYGZgt{}}
\PYG{+w}{            }\PYG{n+nt}{\PYGZlt{}xs:sequence}\PYG{n+nt}{\PYGZgt{}}
\PYG{+w}{                }\PYG{n+nt}{\PYGZlt{}xs:element}\PYG{+w}{ }\PYG{n+na}{type=}\PYG{l+s}{\PYGZdq{}city\PYGZdq{}}\PYG{+w}{ }\PYG{n+na}{maxOccurs=}\PYG{l+s}{\PYGZdq{}unbounded\PYGZdq{}}\PYG{+w}{ }\PYG{n+nt}{/\PYGZgt{}}
\PYG{+w}{                }\PYG{n+nt}{\PYGZlt{}xs:attribute}\PYG{+w}{ }\PYG{n+na}{name=}\PYG{l+s}{\PYGZdq{}name\PYGZdq{}}\PYG{+w}{ }\PYG{n+na}{type=}\PYG{l+s}{\PYGZdq{}xs:string\PYGZdq{}}\PYG{+w}{ }\PYG{n+nt}{/\PYGZgt{}}
\PYG{+w}{            }\PYG{n+nt}{\PYGZlt{}/xs:sequence\PYGZgt{}}
\PYG{+w}{        }\PYG{n+nt}{\PYGZlt{}/xs:complexType\PYGZgt{}}
\PYG{+w}{    }\PYG{n+nt}{\PYGZlt{}/xs:element\PYGZgt{}}

\PYG{n+nt}{\PYGZlt{}/xs:schema\PYGZgt{}}
\end{sphinxVerbatim}

\sphinxAtStartPar
\sphinxstylestrong{Struktūros aprašas}


\begin{savenotes}\sphinxattablestart
\sphinxthistablewithglobalstyle
\centering
\begin{tabulary}{\linewidth}[t]{TTTTTTTT}
\sphinxtoprule
\sphinxstyletheadfamily 
\sphinxAtStartPar
dataset
&\sphinxstyletheadfamily 
\sphinxAtStartPar
model
&\sphinxstyletheadfamily 
\sphinxAtStartPar
property
&\sphinxstyletheadfamily 
\sphinxAtStartPar
type
&\sphinxstyletheadfamily 
\sphinxAtStartPar
ref
&\sphinxstyletheadfamily 
\sphinxAtStartPar
source
&\sphinxstyletheadfamily 
\sphinxAtStartPar
prepare
&\sphinxstyletheadfamily 
\sphinxAtStartPar
level
\\
\sphinxmidrule
\sphinxtableatstartofbodyhook
\sphinxAtStartPar

&&&
\sphinxAtStartPar
schema
&
\sphinxAtStartPar
xsd
&
\sphinxAtStartPar
country.xsd
&&\\
\sphinxhline\sphinxstartmulticolumn{3}%
\begin{varwidth}[t]{\sphinxcolwidth{3}{8}}
\sphinxAtStartPar
xsd
\par
\vskip-\baselineskip\vbox{\hbox{\strut}}\end{varwidth}%
\sphinxstopmulticolumn
&&&&&\\
\sphinxhline
\sphinxAtStartPar

&\sphinxstartmulticolumn{2}%
\begin{varwidth}[t]{\sphinxcolwidth{2}{8}}
\sphinxAtStartPar
\sphinxstylestrong{Country}
\par
\vskip-\baselineskip\vbox{\hbox{\strut}}\end{varwidth}%
\sphinxstopmulticolumn
&&&
\sphinxAtStartPar
/country
&&
\sphinxAtStartPar
0
\\
\sphinxhline
\sphinxAtStartPar

&&
\sphinxAtStartPar
name
&
\sphinxAtStartPar
string
&&
\sphinxAtStartPar
@name
&&\\
\sphinxhline
\sphinxAtStartPar

&&
\sphinxAtStartPar
cities{[}{]}
&
\sphinxAtStartPar
backref
&
\sphinxAtStartPar
City
&
\sphinxAtStartPar
city
&
\sphinxAtStartPar
expand()
&\\
\sphinxhline
\sphinxAtStartPar

&\sphinxstartmulticolumn{2}%
\begin{varwidth}[t]{\sphinxcolwidth{2}{8}}
\sphinxAtStartPar
\sphinxstylestrong{City/:part}
\par
\vskip-\baselineskip\vbox{\hbox{\strut}}\end{varwidth}%
\sphinxstopmulticolumn
&&&&
\sphinxAtStartPar
0
&\\
\sphinxhline
\sphinxAtStartPar

&&
\sphinxAtStartPar
name
&
\sphinxAtStartPar
string
&&
\sphinxAtStartPar
@name
&&
\sphinxAtStartPar
0
\\
\sphinxbottomrule
\end{tabulary}
\sphinxtableafterendhook\par
\sphinxattableend\end{savenotes}
\end{sphinxadmonition}

\sphinxAtStartPar
Pavyzdyje:
\begin{itemize}
\item {} 
\sphinxAtStartPar
\sphinxcode{\sphinxupquote{Country}} modelis yra {\hyperref[\detokenize{schemos/xsd:xsd-aggregate-model}]{\sphinxcrossref{\DUrole{std}{\DUrole{std-ref}{Jungtinis modelis}}}}} ir jungtinio modelio
{\hyperref[\detokenize{schemos/xsd:xsd-aggregate-root}]{\sphinxcrossref{\DUrole{std}{\DUrole{std-ref}{Šakninis modelis}}}}}.

\item {} 
\sphinxAtStartPar
\sphinxcode{\sphinxupquote{City}} yra {\hyperref[\detokenize{schemos/xsd:xsd-aggregate-part}]{\sphinxcrossref{\DUrole{std}{\DUrole{std-ref}{Dalinis modelis}}}}}, kadangi tai žymi \sphinxcode{\sphinxupquote{/:part}} žymė, taip
pat \sphinxcode{\sphinxupquote{City}} neturi užpildyto {\hyperref[\detokenize{dimensijos:model.source}]{\sphinxcrossref{\sphinxcode{\sphinxupquote{model.source}}}}} stulpelio, tai reiškia, kad
tiesiogiai \sphinxcode{\sphinxupquote{City}} duomenų gauti galimybės nėra, juos galima gauti tik per
\sphinxcode{\sphinxupquote{Country}} jungtinį modelį, kurio sudėtyje yra ir \sphinxcode{\sphinxupquote{City}}, prieinamas per
\sphinxcode{\sphinxupquote{Country/cities}} savybę.

\item {} 
\sphinxAtStartPar
\sphinxcode{\sphinxupquote{Country/cities}} savybė turi {\hyperref[\detokenize{formules:expand}]{\sphinxcrossref{\sphinxcode{\sphinxupquote{expand()}}}}} funkciją įrašytą į
{\hyperref[\detokenize{dimensijos:property.prepare}]{\sphinxcrossref{\sphinxcode{\sphinxupquote{property.prepare}}}}}, kuri įtraukia visas tiesiogines \sphinxcode{\sphinxupquote{City}} savybes į
jungtinį \sphinxcode{\sphinxupquote{Country}} modelį.

\end{itemize}


\paragraph{Elementai}
\label{\detokenize{schemos/xsd:elementai}}

\subparagraph{element}
\label{\detokenize{schemos/xsd:element}}\label{\detokenize{schemos/xsd:xsd-element}}
\sphinxAtStartPar
XSD {\hyperref[\detokenize{schemos/xsd:xsd-element}]{\sphinxcrossref{\DUrole{std}{\DUrole{std-ref}{element}}}}} atitinka DSA loginio modelio:
\begin{itemize}
\item {} 
\sphinxAtStartPar
{\hyperref[\detokenize{formatas:model}]{\sphinxcrossref{\sphinxcode{\sphinxupquote{model}}}}} \sphinxhyphen{} jei elemento tipas yra {\hyperref[\detokenize{schemos/xsd:xsd-complextype}]{\sphinxcrossref{\DUrole{std}{\DUrole{std-ref}{complexType}}}}},

\item {} 
\sphinxAtStartPar
{\hyperref[\detokenize{formatas:property}]{\sphinxcrossref{\sphinxcode{\sphinxupquote{property}}}}} \sphinxhyphen{} jei elemento tipas yra {\hyperref[\detokenize{schemos/xsd:xsd-simpletype}]{\sphinxcrossref{\DUrole{std}{\DUrole{std-ref}{simpleType}}}}}.

\end{itemize}

\sphinxAtStartPar
Jei \sphinxcode{\sphinxupquote{xsd\_element}} tipas nėra nurodytas, tada pagal XSD specifikaciją elemento
tipas yra \sphinxcode{\sphinxupquote{xs:anyType}}. DSA neturi \sphinxcode{\sphinxupquote{xs:anyType}} analogo, todėl tokiu atveju
turėtu būti naudojamas DSA \sphinxcode{\sphinxupquote{string}} tipas, kur \sphinxcode{\sphinxupquote{xs:anyType}} reikšmė yra
pateikiama, kaip tekstinė reprezentacija.

\sphinxAtStartPar
Priklausomai nuo to, kur {\hyperref[\detokenize{schemos/xsd:xsd-element}]{\sphinxcrossref{\DUrole{std}{\DUrole{std-ref}{element}}}}} yra deklaruotas
{\hyperref[\detokenize{schemos/xsd:xsd-complextype}]{\sphinxcrossref{\DUrole{std}{\DUrole{std-ref}{complexType}}}}} atžvilgiu, pagal nutylėjimą atliekama sekanti XSD
interpretacija:
\begin{itemize}
\item {} 
\sphinxAtStartPar
Jei {\hyperref[\detokenize{schemos/xsd:xsd-element}]{\sphinxcrossref{\DUrole{std}{\DUrole{std-ref}{element}}}}} ({\hyperref[\detokenize{schemos/xsd:xsd-complextype}]{\sphinxcrossref{\DUrole{std}{\DUrole{std-ref}{complexType}}}}} tipo) yra {\hyperref[\detokenize{schemos/xsd:xsd-complextype}]{\sphinxcrossref{\DUrole{std}{\DUrole{std-ref}{complexType}}}}} sudėtyje, laikoma, kad
modelis yra kito {\hyperref[\detokenize{identifikatoriai:ref-denorm}]{\sphinxcrossref{\DUrole{std}{\DUrole{std-ref}{Jungtinis modelis}}}}} dalis, todėl pagal nutylėjimą nenurodomas
\sphinxcode{\sphinxupquote{model.source`ir šis modelis žymimas kaip dalinis, prie jo pavadinimo pridedant žymę `/:part}}.

\item {} 
\sphinxAtStartPar
Jei {\hyperref[\detokenize{schemos/xsd:xsd-element}]{\sphinxcrossref{\DUrole{std}{\DUrole{std-ref}{element}}}}} ({\hyperref[\detokenize{schemos/xsd:xsd-complextype}]{\sphinxcrossref{\DUrole{std}{\DUrole{std-ref}{complexType}}}}} tipo) nėra {\hyperref[\detokenize{schemos/xsd:xsd-complextype}]{\sphinxcrossref{\DUrole{std}{\DUrole{std-ref}{complexType}}}}} sudėtyje ir deklaruotas
atskirai, bet XSD schemoje yra bent vienas \sphinxcode{\sphinxupquote{complexType}}, kurio viduje yra elementas, turintis
atributą \sphinxcode{\sphinxupquote{ref}}, kurio reikšmė yra šio elemento pavadinimas, tada laikoma, kad modelis
yra kito {\hyperref[\detokenize{identifikatoriai:ref-denorm}]{\sphinxcrossref{\DUrole{std}{\DUrole{std-ref}{Jungtinis modelis}}}}} dalis, todėl pagal nutylėjimą nenurodomas {\hyperref[\detokenize{dimensijos:model.source}]{\sphinxcrossref{\sphinxcode{\sphinxupquote{model.source}}}}} ir
šis modelis žymimas kaip dalinis, prie jo pavadinimo pridedant žymę \sphinxcode{\sphinxupquote{/:part}}.

\item {} 
\sphinxAtStartPar
Jei {\hyperref[\detokenize{schemos/xsd:xsd-element}]{\sphinxcrossref{\DUrole{std}{\DUrole{std-ref}{element}}}}} ({\hyperref[\detokenize{schemos/xsd:xsd-complextype}]{\sphinxcrossref{\DUrole{std}{\DUrole{std-ref}{complexType}}}}} tipo) nėra {\hyperref[\detokenize{schemos/xsd:xsd-complextype}]{\sphinxcrossref{\DUrole{std}{\DUrole{std-ref}{complexType}}}}} sudėtyje ir yra deklaruotas
atskirai, tada laikoma, kad modelis gali būti pasiekiamas tiesiogiai ir tokio
modelio {\hyperref[\detokenize{dimensijos:model.source}]{\sphinxcrossref{\sphinxcode{\sphinxupquote{model.source}}}}} yra pildomas bei šis modelis nežymimas kaip dalinis.

\item {} 
\sphinxAtStartPar
Jei {\hyperref[\detokenize{schemos/xsd:xsd-simpletype}]{\sphinxcrossref{\DUrole{std}{\DUrole{std-ref}{simpleType}}}}} tipo {\hyperref[\detokenize{schemos/xsd:xsd-element}]{\sphinxcrossref{\DUrole{std}{\DUrole{std-ref}{element}}}}} yra {\hyperref[\detokenize{schemos/xsd:xsd-complextype}]{\sphinxcrossref{\DUrole{std}{\DUrole{std-ref}{complexType}}}}} sudėtyje,
iš jo sukuriama {\hyperref[\detokenize{formatas:property}]{\sphinxcrossref{\sphinxcode{\sphinxupquote{property}}}}}, ir pridedama prie {\hyperref[\detokenize{formatas:model}]{\sphinxcrossref{\sphinxcode{\sphinxupquote{model}}}}}, kuris kuriamas iš elemento, kurį
aprašo šis {\hyperref[\detokenize{schemos/xsd:xsd-complextype}]{\sphinxcrossref{\DUrole{std}{\DUrole{std-ref}{complexType}}}}}. Šios {\hyperref[\detokenize{dimensijos:property.source}]{\sphinxcrossref{\sphinxcode{\sphinxupquote{property.source}}}}} formuojamas iš elemento pavadinimo,
prie jo pridedant \sphinxcode{\sphinxupquote{/text()}}

\item {} 
\sphinxAtStartPar
Jei {\hyperref[\detokenize{schemos/xsd:xsd-simpletype}]{\sphinxcrossref{\DUrole{std}{\DUrole{std-ref}{simpleType}}}}} tipo {\hyperref[\detokenize{schemos/xsd:xsd-element}]{\sphinxcrossref{\DUrole{std}{\DUrole{std-ref}{element}}}}} nėra {\hyperref[\detokenize{schemos/xsd:xsd-complextype}]{\sphinxcrossref{\DUrole{std}{\DUrole{std-ref}{complexType}}}}} sudėtyje, bet yra
deklaruotas atskirai, bei nėra nei vieno kito elemento, kuris per \sphinxcode{\sphinxupquote{ref}} ar \sphinxcode{\sphinxupquote{type}} referuotų
į šį elementą, iš jo sukuriama {\hyperref[\detokenize{formatas:property}]{\sphinxcrossref{\sphinxcode{\sphinxupquote{property}}}}}, ir pridedama prie specialaus
\sphinxcode{\sphinxupquote{Resource}} {\hyperref[\detokenize{formatas:model}]{\sphinxcrossref{\sphinxcode{\sphinxupquote{model}}}}}.

\end{itemize}

\begin{sphinxadmonition}{note}{Pavyzdys}

\sphinxAtStartPar
\sphinxstylestrong{Duomenys}

\begin{sphinxVerbatim}[commandchars=\\\{\}]
\PYG{n+nt}{\PYGZlt{}Country}\PYG{n+nt}{\PYGZgt{}}
\PYG{+w}{  }\PYG{n+nt}{\PYGZlt{}name}\PYG{n+nt}{\PYGZgt{}}France\PYG{n+nt}{\PYGZlt{}/name\PYGZgt{}}
\PYG{+w}{  }\PYG{n+nt}{\PYGZlt{}numberOfMunicipalities}\PYG{n+nt}{\PYGZgt{}}35\PYG{n+nt}{\PYGZlt{}/numberOfMunicipalities\PYGZgt{}}

\PYG{+w}{  }\PYG{n+nt}{\PYGZlt{}City}\PYG{n+nt}{\PYGZgt{}}
\PYG{+w}{    }\PYG{n+nt}{\PYGZlt{}name}\PYG{n+nt}{\PYGZgt{}}Paris\PYG{n+nt}{\PYGZlt{}/name\PYGZgt{}}
\PYG{+w}{  }\PYG{n+nt}{\PYGZlt{}/City\PYGZgt{}}

\PYG{+w}{  }\PYG{n+nt}{\PYGZlt{}City}\PYG{n+nt}{\PYGZgt{}}
\PYG{+w}{    }\PYG{n+nt}{\PYGZlt{}name}\PYG{n+nt}{\PYGZgt{}}Lyon\PYG{n+nt}{\PYGZlt{}/name\PYGZgt{}}
\PYG{+w}{  }\PYG{n+nt}{\PYGZlt{}/City\PYGZgt{}}

\PYG{n+nt}{\PYGZlt{}/Country\PYGZgt{}}
\end{sphinxVerbatim}

\sphinxAtStartPar
\sphinxstylestrong{Schema}

\begin{sphinxVerbatim}[commandchars=\\\{\}]
\PYG{n+nt}{\PYGZlt{}xs:schema}\PYG{+w}{ }\PYG{n+na}{xmlns:xs=}\PYG{l+s}{\PYGZdq{}http://www.w3.org/2001/XMLSchema\PYGZdq{}}\PYG{n+nt}{\PYGZgt{}}

\PYG{+w}{  }\PYG{n+nt}{\PYGZlt{}xs:element}\PYG{+w}{ }\PYG{n+na}{name=}\PYG{l+s}{\PYGZdq{}Country\PYGZdq{}}\PYG{n+nt}{\PYGZgt{}}
\PYG{+w}{    }\PYG{n+nt}{\PYGZlt{}xs:complexType}\PYG{n+nt}{\PYGZgt{}}
\PYG{+w}{      }\PYG{n+nt}{\PYGZlt{}xs:sequence}\PYG{n+nt}{\PYGZgt{}}

\PYG{+w}{        }\PYG{n+nt}{\PYGZlt{}xs:element}\PYG{+w}{ }\PYG{n+na}{name=}\PYG{l+s}{\PYGZdq{}name\PYGZdq{}}\PYG{+w}{ }\PYG{n+na}{type=}\PYG{l+s}{\PYGZdq{}xs:string\PYGZdq{}}\PYG{+w}{ }\PYG{n+nt}{/\PYGZgt{}}

\PYG{+w}{        }\PYG{n+nt}{\PYGZlt{}xs:element}\PYG{+w}{ }\PYG{n+na}{name=}\PYG{l+s}{\PYGZdq{}numberOfMunicipalities\PYGZdq{}}\PYG{n+nt}{\PYGZgt{}}
\PYG{+w}{          }\PYG{n+nt}{\PYGZlt{}xs:simpleType}\PYG{n+nt}{\PYGZgt{}}
\PYG{+w}{            }\PYG{n+nt}{\PYGZlt{}xs:restriction}\PYG{+w}{ }\PYG{n+na}{base=}\PYG{l+s}{\PYGZdq{}xs:integer\PYGZdq{}}\PYG{n+nt}{\PYGZgt{}}
\PYG{+w}{              }\PYG{n+nt}{\PYGZlt{}xs:minInclusive}\PYG{+w}{ }\PYG{n+na}{value=}\PYG{l+s}{\PYGZdq{}1\PYGZdq{}}\PYG{+w}{ }\PYG{n+nt}{/\PYGZgt{}}
\PYG{+w}{            }\PYG{n+nt}{\PYGZlt{}/xs:restriction\PYGZgt{}}
\PYG{+w}{          }\PYG{n+nt}{\PYGZlt{}/xs:simpleType\PYGZgt{}}
\PYG{+w}{        }\PYG{n+nt}{\PYGZlt{}/xs:element\PYGZgt{}}

\PYG{+w}{        }\PYG{n+nt}{\PYGZlt{}xs:element}\PYG{+w}{ }\PYG{n+na}{ref=}\PYG{l+s}{\PYGZdq{}governance\PYGZdq{}}\PYG{n+nt}{/\PYGZgt{}}

\PYG{+w}{        }\PYG{n+nt}{\PYGZlt{}xs:element}\PYG{+w}{ }\PYG{n+na}{name=}\PYG{l+s}{\PYGZdq{}City\PYGZdq{}}\PYG{+w}{ }\PYG{n+na}{maxOccurs=}\PYG{l+s}{\PYGZdq{}unbounded\PYGZdq{}}\PYG{n+nt}{\PYGZgt{}}
\PYG{+w}{          }\PYG{n+nt}{\PYGZlt{}xs:complexType}\PYG{n+nt}{\PYGZgt{}}
\PYG{+w}{            }\PYG{n+nt}{\PYGZlt{}xs:sequence}\PYG{n+nt}{\PYGZgt{}}
\PYG{+w}{              }\PYG{n+nt}{\PYGZlt{}xs:element}\PYG{+w}{ }\PYG{n+na}{name=}\PYG{l+s}{\PYGZdq{}name\PYGZdq{}}\PYG{+w}{ }\PYG{n+na}{type=}\PYG{l+s}{\PYGZdq{}xs:string\PYGZdq{}}\PYG{+w}{ }\PYG{n+nt}{/\PYGZgt{}}
\PYG{+w}{            }\PYG{n+nt}{\PYGZlt{}/xs:sequence\PYGZgt{}}
\PYG{+w}{          }\PYG{n+nt}{\PYGZlt{}/xs:complexType\PYGZgt{}}
\PYG{+w}{        }\PYG{n+nt}{\PYGZlt{}/xs:element\PYGZgt{}}

\PYG{+w}{      }\PYG{n+nt}{\PYGZlt{}/xs:sequence\PYGZgt{}}
\PYG{+w}{    }\PYG{n+nt}{\PYGZlt{}/xs:complexType\PYGZgt{}}
\PYG{+w}{  }\PYG{n+nt}{\PYGZlt{}/xs:element\PYGZgt{}}

\PYG{+w}{  }\PYG{n+nt}{\PYGZlt{}xs:element}\PYG{+w}{ }\PYG{n+na}{name=}\PYG{l+s}{\PYGZdq{}governance\PYGZdq{}}\PYG{n+nt}{\PYGZgt{}}
\PYG{+w}{    }\PYG{n+nt}{\PYGZlt{}xs:complexType}\PYG{n+nt}{\PYGZgt{}}
\PYG{+w}{      }\PYG{n+nt}{\PYGZlt{}xs:sequence}\PYG{n+nt}{\PYGZgt{}}
\PYG{+w}{        }\PYG{n+nt}{\PYGZlt{}element}\PYG{+w}{ }\PYG{n+na}{name=}\PYG{l+s}{\PYGZdq{}president\PYGZdq{}}\PYG{+w}{ }\PYG{n+na}{type=}\PYG{l+s}{\PYGZdq{}string\PYGZdq{}}\PYG{+w}{ }\PYG{n+nt}{/\PYGZgt{}}
\PYG{+w}{      }\PYG{n+nt}{\PYGZlt{}/xs:sequence\PYGZgt{}}
\PYG{+w}{    }\PYG{n+nt}{\PYGZlt{}/xs:complexType\PYGZgt{}}
\PYG{+w}{  }\PYG{n+nt}{\PYGZlt{}/xs:element\PYGZgt{}}

\PYG{n+nt}{\PYGZlt{}/xs:schema\PYGZgt{}}
\end{sphinxVerbatim}

\sphinxAtStartPar
\sphinxstylestrong{Struktūros aprašas}
\end{sphinxadmonition}

\sphinxAtStartPar
Pavyzdyje:
\begin{itemize}
\item {} 
\sphinxAtStartPar
\sphinxcode{\sphinxupquote{Country}} \sphinxcode{\sphinxupquote{element}} tampa modeliu, nes jis yra pirminio lygio, ir jo tipas yra {\hyperref[\detokenize{schemos/xsd:xsd-complextype}]{\sphinxcrossref{\DUrole{std}{\DUrole{std-ref}{complexType}}}}}.

\end{itemize}
\begin{quote}

\sphinxAtStartPar
XML struktūroje jis tampa šakniniu elementu, todėl iš jo kilęs modelis irgi nurodomas kaip šakninis
modelis, galintis eiti atskirai, ir nėra žymimas \sphinxcode{\sphinxupquote{/:part}}.
\end{quote}
\begin{itemize}
\item {} 
\sphinxAtStartPar
\sphinxcode{\sphinxupquote{name}} \sphinxcode{\sphinxupquote{element}} tampa {\hyperref[\detokenize{formatas:model}]{\sphinxcrossref{\sphinxcode{\sphinxupquote{model}}}}} \sphinxcode{\sphinxupquote{Country}} {\hyperref[\detokenize{formatas:property}]{\sphinxcrossref{\sphinxcode{\sphinxupquote{property}}}}}, nes jis yra viduje {\hyperref[\detokenize{schemos/xsd:xsd-complextype}]{\sphinxcrossref{\DUrole{std}{\DUrole{std-ref}{complexType}}}}},
kuris yra viduje

\end{itemize}
\begin{description}
\sphinxlineitem{\sphinxcode{\sphinxupquote{Country}} \sphinxcode{\sphinxupquote{element}} ir pats yra {\hyperref[\detokenize{schemos/xsd:xsd-simpletype}]{\sphinxcrossref{\DUrole{std}{\DUrole{std-ref}{simpleType}}}}}. Jo tipas šiuo atveju nurodomas paties elemento aprašyme}
\sphinxAtStartPar
ir yra \sphinxcode{\sphinxupquote{string}}. Šis \sphinxcode{\sphinxupquote{string}} tipas DSA taip pat tampa \sphinxcode{\sphinxupquote{string}} tipu.

\end{description}
\begin{itemize}
\item {} 
\sphinxAtStartPar
\sphinxcode{\sphinxupquote{numberOfMunicipalities}} \sphinxcode{\sphinxupquote{element}} taip pat tampa \sphinxcode{\sphinxupquote{Country}} modelio {\hyperref[\detokenize{formatas:property}]{\sphinxcrossref{\sphinxcode{\sphinxupquote{property}}}}}. Jam tipas nurodytas
atskirame {\hyperref[\detokenize{schemos/xsd:xsd-simpletype}]{\sphinxcrossref{\DUrole{std}{\DUrole{std-ref}{simpleType}}}}}, kuriame nurodoma, kad jo pagrindas (\sphinxcode{\sphinxupquote{base}}) yra \sphinxcode{\sphinxupquote{integer}},
ir nurodyti apribojimai (\sphinxcode{\sphinxupquote{restriction}}). Šis \sphinxcode{\sphinxupquote{base}} tipas ir yra konvertuojamas į DSA tipą,
šiuo konkrečiu atveju \sphinxhyphen{} į \sphinxcode{\sphinxupquote{integer}} tipą. Kadangi DSA netaiko apribojimų reikšmėms, tai visi apribojimai,
kurie yra nurodyti \sphinxcode{\sphinxupquote{restriction}} (šiame pavyzdyje, \sphinxcode{\sphinxupquote{minInclusive}}) ignoruojami. Kadangi {\hyperref[\detokenize{formatas:property}]{\sphinxcrossref{\sphinxcode{\sphinxupquote{property}}}}}
pavadinimas turi būti sudarytas iš mažųjų raidžių, o tarpai tarp žodžių atskiriami pabraukimais (\_), tai
{\hyperref[\detokenize{formatas:property}]{\sphinxcrossref{\sphinxcode{\sphinxupquote{property}}}}} pavadinimas tampa \sphinxcode{\sphinxupquote{number\_of\_municipalities}}.

\item {} 
\sphinxAtStartPar
Sekantis elementas, \sphinxcode{\sphinxupquote{<xs:element ref="governance"/>}}, neturi pavadinimo, bet jame yra atributas \sphinxcode{\sphinxupquote{ref}},
kas nurodo, kad jo aprašymas referuojamas kitam, globaliam elementui, pavadinimu \sphinxcode{\sphinxupquote{governance}}.
Šiuo atveju iš šio \DUrole{xref}{\DUrole{std}{\DUrole{std-ref}{element}}} kuriamai {\hyperref[\detokenize{formatas:property}]{\sphinxcrossref{\sphinxcode{\sphinxupquote{property}}}}} suteikiamas pavadinimas pagal
\sphinxcode{\sphinxupquote{ref}} atributą, o į jo \sphinxcode{\sphinxupquote{ref}} stulpelį įrašomas modelio, sukurto iš referuojamo \DUrole{xref}{\DUrole{std}{\DUrole{std-ref}{element}}}, pavadinimas.

\item {} 
\sphinxAtStartPar
iš atskirai apibrėžto elemento \sphinxcode{\sphinxupquote{<xs:element ref="governance"/>}} sukuriamas {\hyperref[\detokenize{formatas:model}]{\sphinxcrossref{\sphinxcode{\sphinxupquote{model}}}}} Governance.

\end{itemize}


\subparagraph{Santrauka: XSD elementų ir DSA Atitikimas}
\label{\detokenize{schemos/xsd:santrauka-xsd-elementu-ir-dsa-atitikimas}}

\subparagraph{attribute}
\label{\detokenize{schemos/xsd:attribute}}\label{\detokenize{schemos/xsd:xsd-attribute}}
\sphinxAtStartPar
XSD {\hyperref[\detokenize{schemos/xsd:xsd-attribute}]{\sphinxcrossref{\DUrole{std}{\DUrole{std-ref}{attribute}}}}} atitinka DSA loginio modelio {\hyperref[\detokenize{formatas:property}]{\sphinxcrossref{\sphinxcode{\sphinxupquote{property}}}}}.

\sphinxAtStartPar
Iš {\hyperref[\detokenize{schemos/xsd:xsd-attribute}]{\sphinxcrossref{\DUrole{std}{\DUrole{std-ref}{attribute}}}}} atributo \sphinxcode{\sphinxupquote{name}} formuojamas {\hyperref[\detokenize{formatas:property}]{\sphinxcrossref{\sphinxcode{\sphinxupquote{property}}}}} pavadinimas. Jei \sphinxcode{\sphinxupquote{name}}
susideda iš kelių žodžių, {\hyperref[\detokenize{formatas:property}]{\sphinxcrossref{\sphinxcode{\sphinxupquote{property}}}}} pavadinimas taip pat susidės iš kelių žodžių,
tačiau jie bus mažosiomis raidėmis ir atskirti pabraukimo ženklu (\_).

\sphinxAtStartPar
{\hyperref[\detokenize{dimensijos:property.source}]{\sphinxcrossref{\sphinxcode{\sphinxupquote{property.source}}}}} yra formuojamas iš {\hyperref[\detokenize{schemos/xsd:xsd-attribute}]{\sphinxcrossref{\DUrole{std}{\DUrole{std-ref}{attribute}}}}} atributo \sphinxcode{\sphinxupquote{name}}, priekyje pridedant \sphinxcode{\sphinxupquote{@}}.

\sphinxAtStartPar
Jei \sphinxcode{\sphinxupquote{attribute}} tipas yra nurodytas \sphinxcode{\sphinxupquote{attribute}} elemente esančiu {\hyperref[\detokenize{schemos/xsd:xsd-type}]{\sphinxcrossref{\DUrole{std}{\DUrole{std-ref}{type}}}}} atributu,
tai {\hyperref[\detokenize{formatas:property}]{\sphinxcrossref{\sphinxcode{\sphinxupquote{property}}}}} tipas formuojamas iš {\hyperref[\detokenize{schemos/xsd:xsd-attribute}]{\sphinxcrossref{\DUrole{std}{\DUrole{std-ref}{attribute}}}}},  naudojantis \DUrole{xref}{\DUrole{std}{\DUrole{std-ref}{xd\_type\_conversion}}}.

\sphinxAtStartPar
Jei \sphinxcode{\sphinxupquote{attribute}} tipas aprašytas {\hyperref[\detokenize{schemos/xsd:xsd-simpletype}]{\sphinxcrossref{\DUrole{std}{\DUrole{std-ref}{simpleType}}}}}, tai {\hyperref[\detokenize{dimensijos:property.type}]{\sphinxcrossref{\sphinxcode{\sphinxupquote{property.type}}}}} formuojamas iš šio
\DUrole{xref}{\DUrole{std}{\DUrole{std-ref}{simpleType}}} viduje esančio {\hyperref[\detokenize{schemos/xsd:xsd-restriction}]{\sphinxcrossref{\DUrole{std}{\DUrole{std-ref}{restriction}}}}} {\hyperref[\detokenize{schemos/xsd:xsd-base}]{\sphinxcrossref{\DUrole{std}{\DUrole{std-ref}{base}}}}} nurodyto {\hyperref[\detokenize{schemos/xsd:xsd-type}]{\sphinxcrossref{\DUrole{std}{\DUrole{std-ref}{type}}}}}.

\sphinxAtStartPar
Jei \sphinxcode{\sphinxupquote{attribute}} elemento sudėtyje yra \DUrole{xref}{\DUrole{std}{\DUrole{std-ref}{annotation}}}, iš jo formuojamas aprašymas \sphinxhyphen{}
{\hyperref[\detokenize{dimensijos:property.description}]{\sphinxcrossref{\sphinxcode{\sphinxupquote{property.description}}}}}.

\sphinxAtStartPar
Jei \sphinxcode{\sphinxupquote{attribute}} turi atributą {\hyperref[\detokenize{schemos/xsd:xsd-use}]{\sphinxcrossref{\DUrole{std}{\DUrole{std-ref}{use}}}}} su reikšme \sphinxcode{\sphinxupquote{required}}, tai prie {\hyperref[\detokenize{formatas:property}]{\sphinxcrossref{\sphinxcode{\sphinxupquote{property}}}}}
pavadinimo pridedama \sphinxcode{\sphinxupquote{required}} žymė, reiškianti, kad ši {\hyperref[\detokenize{formatas:property}]{\sphinxcrossref{\sphinxcode{\sphinxupquote{property}}}}} yra privaloma.

\begin{sphinxadmonition}{note}{Pavyzdys}

\sphinxAtStartPar
\sphinxstylestrong{Duomenys}

\begin{sphinxVerbatim}[commandchars=\\\{\}]
\PYG{n+nt}{\PYGZlt{}country}\PYG{+w}{ }\PYG{n+na}{name=}\PYG{l+s}{\PYGZdq{}France\PYGZdq{}}\PYG{+w}{ }\PYG{n+na}{capital=}\PYG{l+s}{\PYGZdq{}Paris\PYGZdq{}}\PYG{+w}{ }\PYG{n+nt}{/\PYGZgt{}}
\end{sphinxVerbatim}

\sphinxAtStartPar
\sphinxstylestrong{Schema}

\begin{sphinxVerbatim}[commandchars=\\\{\}]
\PYG{n+nt}{\PYGZlt{}xs:schema}\PYG{+w}{ }\PYG{n+na}{xmlns:xs=}\PYG{l+s}{\PYGZdq{}http://www.w3.org/2001/XMLSchema\PYGZdq{}}\PYG{+w}{ }\PYG{n+na}{elementFormDefault=}\PYG{l+s}{\PYGZdq{}qualified\PYGZdq{}}\PYG{n+nt}{\PYGZgt{}}

\PYG{+w}{    }\PYG{n+nt}{\PYGZlt{}xs:element}\PYG{+w}{ }\PYG{n+na}{name=}\PYG{l+s}{\PYGZdq{}country\PYGZdq{}}\PYG{n+nt}{\PYGZgt{}}
\PYG{+w}{        }\PYG{n+nt}{\PYGZlt{}xs:complexType}\PYG{n+nt}{\PYGZgt{}}
\PYG{+w}{            }\PYG{n+nt}{\PYGZlt{}xs:attribute}\PYG{+w}{ }\PYG{n+na}{name=}\PYG{l+s}{\PYGZdq{}name\PYGZdq{}}\PYG{+w}{ }\PYG{n+na}{type=}\PYG{l+s}{\PYGZdq{}xs:string\PYGZdq{}}\PYG{+w}{ }\PYG{n+na}{use=}\PYG{l+s}{\PYGZdq{}required\PYGZdq{}}\PYG{n+nt}{/\PYGZgt{}}
\PYG{+w}{            }\PYG{n+nt}{\PYGZlt{}xs:attribute}\PYG{+w}{ }\PYG{n+na}{name=}\PYG{l+s}{\PYGZdq{}capital\PYGZdq{}}\PYG{+w}{ }\PYG{n+na}{type=}\PYG{l+s}{\PYGZdq{}xs:string\PYGZdq{}}\PYG{+w}{ }\PYG{n+na}{use=}\PYG{l+s}{\PYGZdq{}required\PYGZdq{}}\PYG{n+nt}{/\PYGZgt{}}
\PYG{+w}{        }\PYG{n+nt}{\PYGZlt{}/xs:complexType\PYGZgt{}}
\PYG{+w}{    }\PYG{n+nt}{\PYGZlt{}/xs:element\PYGZgt{}}

\PYG{n+nt}{\PYGZlt{}/xs:schema\PYGZgt{}}
\end{sphinxVerbatim}

\sphinxAtStartPar
\sphinxstylestrong{Struktūros aprašas}


\begin{savenotes}\sphinxattablestart
\sphinxthistablewithglobalstyle
\centering
\begin{tabulary}{\linewidth}[t]{TTTTTTTT}
\sphinxtoprule
\sphinxstyletheadfamily 
\sphinxAtStartPar
dataset
&\sphinxstyletheadfamily 
\sphinxAtStartPar
model
&\sphinxstyletheadfamily 
\sphinxAtStartPar
property
&\sphinxstyletheadfamily 
\sphinxAtStartPar
type
&\sphinxstyletheadfamily 
\sphinxAtStartPar
ref
&\sphinxstyletheadfamily 
\sphinxAtStartPar
source
&\sphinxstyletheadfamily 
\sphinxAtStartPar
prepare
&\sphinxstyletheadfamily 
\sphinxAtStartPar
level
\\
\sphinxmidrule
\sphinxtableatstartofbodyhook
\sphinxAtStartPar

&&&
\sphinxAtStartPar
schema
&
\sphinxAtStartPar
xsd
&
\sphinxAtStartPar
country.xsd
&&\\
\sphinxhline\sphinxstartmulticolumn{3}%
\begin{varwidth}[t]{\sphinxcolwidth{3}{8}}
\sphinxAtStartPar
xsd
\par
\vskip-\baselineskip\vbox{\hbox{\strut}}\end{varwidth}%
\sphinxstopmulticolumn
&&&&&\\
\sphinxhline
\sphinxAtStartPar

&\sphinxstartmulticolumn{2}%
\begin{varwidth}[t]{\sphinxcolwidth{2}{8}}
\sphinxAtStartPar
\sphinxstylestrong{Country}
\par
\vskip-\baselineskip\vbox{\hbox{\strut}}\end{varwidth}%
\sphinxstopmulticolumn
&&&
\sphinxAtStartPar
/country
&&
\sphinxAtStartPar
0
\\
\sphinxhline
\sphinxAtStartPar

&&
\sphinxAtStartPar
name
&
\sphinxAtStartPar
string
&&
\sphinxAtStartPar
@name
&&\\
\sphinxhline
\sphinxAtStartPar

&&
\sphinxAtStartPar
capital
&
\sphinxAtStartPar
string
&&
\sphinxAtStartPar
@capital
&&\\
\sphinxbottomrule
\end{tabulary}
\sphinxtableafterendhook\par
\sphinxattableend\end{savenotes}
\end{sphinxadmonition}

\sphinxAtStartPar
Pavyzdyje:
\begin{itemize}
\item {} 
\sphinxAtStartPar
XSD \sphinxcode{\sphinxupquote{attribute}} kurio \sphinxcode{\sphinxupquote{name}} reikšmė yra \sphinxcode{\sphinxupquote{name}} tampa {\hyperref[\detokenize{formatas:property}]{\sphinxcrossref{\sphinxcode{\sphinxupquote{property}}}}} su pavadinimu \sphinxcode{\sphinxupquote{name}}.
Jo tipas yra \sphinxcode{\sphinxupquote{string}}, tai konvertuojasi į DSA {\hyperref[\detokenize{dimensijos:property.type}]{\sphinxcrossref{\sphinxcode{\sphinxupquote{property.type}}}}} \sphinxcode{\sphinxupquote{string}}. {\hyperref[\detokenize{dimensijos:property.source}]{\sphinxcrossref{\sphinxcode{\sphinxupquote{property.source}}}}}
padaromas iš \sphinxcode{\sphinxupquote{attribute}} \sphinxcode{\sphinxupquote{name}} \sphinxcode{\sphinxupquote{name}}, prie jo pridedant \sphinxcode{\sphinxupquote{@}} ir tampa \sphinxcode{\sphinxupquote{@name}}.

\item {} 
\sphinxAtStartPar
XSD \sphinxcode{\sphinxupquote{attribute}} kurio \sphinxcode{\sphinxupquote{name}} reikšmė yra \sphinxcode{\sphinxupquote{capital}} tampa {\hyperref[\detokenize{formatas:property}]{\sphinxcrossref{\sphinxcode{\sphinxupquote{property}}}}} su pavadinimu \sphinxcode{\sphinxupquote{capital}}.
Jo tipas yra \sphinxcode{\sphinxupquote{string}}, todėl konvertuojasi į DSA {\hyperref[\detokenize{dimensijos:property.type}]{\sphinxcrossref{\sphinxcode{\sphinxupquote{property.type}}}}} \sphinxcode{\sphinxupquote{string}}. {\hyperref[\detokenize{dimensijos:property.source}]{\sphinxcrossref{\sphinxcode{\sphinxupquote{property.source}}}}}
padaromas iš \sphinxcode{\sphinxupquote{attribute}} \sphinxcode{\sphinxupquote{name}} \sphinxcode{\sphinxupquote{capital}}, prie jo pridedant \sphinxcode{\sphinxupquote{@}} ir tampa \sphinxcode{\sphinxupquote{@capital}}.

\end{itemize}


\subparagraph{simpleType}
\label{\detokenize{schemos/xsd:simpletype}}\label{\detokenize{schemos/xsd:xsd-simpletype}}
\sphinxAtStartPar
Jei elemento ar atributo tipas aprašytas naudojant \sphinxcode{\sphinxupquote{simpleType}}, į DSA tipą jis
konvertuojamas naudojant konvertavimo lentelę \DUrole{xref}{\DUrole{std}{\DUrole{std-ref}{xsd\_type\_conversion}}}.

\sphinxAtStartPar
\sphinxcode{\sphinxupquote{simpleType}} viduje gali būti \DUrole{xref}{\DUrole{std}{\DUrole{std-ref}{restriction}}} arba \DUrole{xref}{\DUrole{std}{\DUrole{std-ref}{extension}}}. Jie abu naudojami smulkesniam
\sphinxcode{\sphinxupquote{simpleType}} aprašymui. Dauguma jų naudojami duomenų validavimui, o DSA duomenų validavimo
taisyklės netaikomos, tai šie apribojimai dažniausiai yra ignoruojami. Jų aprašymus rasite žemiau.

\sphinxAtStartPar
Jei \sphinxcode{\sphinxupquote{simpleType}} elementas turi {\hyperref[\detokenize{schemos/xsd:xsd-annotation}]{\sphinxcrossref{\DUrole{std}{\DUrole{std-ref}{annotation}}}}}, jo turinys pridedamas prie iš šį \sphinxcode{\sphinxupquote{simpleType}}
naudojančio elemento sukurtos \sphinxcode{\sphinxupquote{property}} aprašymo: {\hyperref[\detokenize{dimensijos:property.description}]{\sphinxcrossref{\sphinxcode{\sphinxupquote{property.description}}}}}.

\sphinxAtStartPar
\sphinxcode{\sphinxupquote{simpleType}} gali būti aprašomas ir atskirai. Tokiu atveju, iš jo nustatytas {\hyperref[\detokenize{dimensijos:property.type}]{\sphinxcrossref{\sphinxcode{\sphinxupquote{property.type}}}}} bus
pridėtas toms {\hyperref[\detokenize{formatas:property}]{\sphinxcrossref{\sphinxcode{\sphinxupquote{property}}}}}, kurios sukurtos iš į šį tipą referuojančių elementų arba atributų.

\begin{sphinxadmonition}{note}{Pavyzdys}

\sphinxAtStartPar
\sphinxstylestrong{Duomenys}

\begin{sphinxVerbatim}[commandchars=\\\{\}]
\PYG{n+nt}{\PYGZlt{}numberOfMunicipalities}\PYG{n+nt}{\PYGZgt{}}5\PYG{n+nt}{\PYGZlt{}/numberOfMunicipalities\PYGZgt{}}
\end{sphinxVerbatim}

\sphinxAtStartPar
\sphinxstylestrong{Schema}

\begin{sphinxVerbatim}[commandchars=\\\{\}]
\PYG{n+nt}{\PYGZlt{}xs:schema}\PYG{+w}{ }\PYG{n+na}{xmlns:xs=}\PYG{l+s}{\PYGZdq{}http://www.w3.org/2001/XMLSchema\PYGZdq{}}\PYG{n+nt}{\PYGZgt{}}

\PYG{+w}{    }\PYG{n+nt}{\PYGZlt{}xs:element}\PYG{+w}{ }\PYG{n+na}{name=}\PYG{l+s}{\PYGZdq{}country\PYGZdq{}}\PYG{n+nt}{\PYGZgt{}}
\PYG{+w}{        }\PYG{n+nt}{\PYGZlt{}xs:complexType}\PYG{n+nt}{\PYGZgt{}}
\PYG{+w}{            }\PYG{n+nt}{\PYGZlt{}xs:element}\PYG{+w}{ }\PYG{n+na}{name=}\PYG{l+s}{\PYGZdq{}population\PYGZdq{}}\PYG{n+nt}{\PYGZgt{}}
\PYG{+w}{          }\PYG{n+nt}{\PYGZlt{}xs:simpleType}\PYG{n+nt}{\PYGZgt{}}
\PYG{+w}{            }\PYG{n+nt}{\PYGZlt{}xs:restriction}\PYG{+w}{ }\PYG{n+na}{base=}\PYG{l+s}{\PYGZdq{}xs:integer\PYGZdq{}}\PYG{n+nt}{\PYGZgt{}}
\PYG{+w}{              }\PYG{n+nt}{\PYGZlt{}xs:minInclusive}\PYG{+w}{ }\PYG{n+na}{value=}\PYG{l+s}{\PYGZdq{}1\PYGZdq{}}\PYG{+w}{ }\PYG{n+nt}{/\PYGZgt{}}
\PYG{+w}{            }\PYG{n+nt}{\PYGZlt{}/xs:restriction\PYGZgt{}}
\PYG{+w}{          }\PYG{n+nt}{\PYGZlt{}/xs:simpleType\PYGZgt{}}
\PYG{+w}{        }\PYG{n+nt}{\PYGZlt{}/xs:element\PYGZgt{}}
\PYG{+w}{        }\PYG{n+nt}{\PYGZlt{}/xs:complexType\PYGZgt{}}
\PYG{+w}{    }\PYG{n+nt}{\PYGZlt{}/xs:element\PYGZgt{}}

\PYG{n+nt}{\PYGZlt{}/xs:schema\PYGZgt{}}
\end{sphinxVerbatim}

\sphinxAtStartPar
\sphinxstylestrong{Struktūros aprašas}
\end{sphinxadmonition}

\sphinxAtStartPar
Pavyzdyje:
\begin{itemize}
\item {} 
\sphinxAtStartPar
iš \sphinxcode{\sphinxupquote{simpleType}}, kurio viduje nurodytas {\hyperref[\detokenize{schemos/xsd:xsd-restriction}]{\sphinxcrossref{\DUrole{std}{\DUrole{std-ref}{restriction}}}}}, kurio {\hyperref[\detokenize{schemos/xsd:xsd-base}]{\sphinxcrossref{\DUrole{std}{\DUrole{std-ref}{base}}}}} yra \sphinxcode{\sphinxupquote{string}},
gaunamas DSA {\hyperref[\detokenize{dimensijos:property.type}]{\sphinxcrossref{\sphinxcode{\sphinxupquote{property.type}}}}} \sphinxcode{\sphinxupquote{string}}.

\end{itemize}


\subparagraph{complexType}
\label{\detokenize{schemos/xsd:complextype}}\label{\detokenize{schemos/xsd:xsd-complextype}}
\sphinxAtStartPar
{\hyperref[\detokenize{schemos/xsd:xsd-complextype}]{\sphinxcrossref{\DUrole{std}{\DUrole{std-ref}{complexType}}}}} gali būti arba {\hyperref[\detokenize{schemos/xsd:xsd-element}]{\sphinxcrossref{\DUrole{std}{\DUrole{std-ref}{element}}}}} sudėtyje, arba atskirai.
Jei {\hyperref[\detokenize{schemos/xsd:xsd-complextype}]{\sphinxcrossref{\DUrole{std}{\DUrole{std-ref}{complexType}}}}} yra {\hyperref[\detokenize{schemos/xsd:xsd-element}]{\sphinxcrossref{\DUrole{std}{\DUrole{std-ref}{element}}}}} sudėtyje, iš jų abiejų
kartu kuriamas {\hyperref[\detokenize{formatas:model}]{\sphinxcrossref{\sphinxcode{\sphinxupquote{model}}}}}. {\hyperref[\detokenize{formatas:model}]{\sphinxcrossref{\sphinxcode{\sphinxupquote{model}}}}} struktūra nustatoma iš {\hyperref[\detokenize{schemos/xsd:xsd-complextype}]{\sphinxcrossref{\DUrole{std}{\DUrole{std-ref}{complexType}}}}}, o
pavadinimas \sphinxhyphen{} iš {\hyperref[\detokenize{schemos/xsd:xsd-element}]{\sphinxcrossref{\DUrole{std}{\DUrole{std-ref}{element}}}}} atributo \sphinxcode{\sphinxupquote{name}}.

\sphinxAtStartPar
Jei {\hyperref[\detokenize{schemos/xsd:xsd-complextype}]{\sphinxcrossref{\DUrole{std}{\DUrole{std-ref}{complexType}}}}} yra aprašytas atskirai, iš jo kuriamas {\hyperref[\detokenize{formatas:model}]{\sphinxcrossref{\sphinxcode{\sphinxupquote{model}}}}},
kurio pavadinimas nustatomas iš {\hyperref[\detokenize{schemos/xsd:xsd-complextype}]{\sphinxcrossref{\DUrole{std}{\DUrole{std-ref}{complexType}}}}} pavadinimo.

\sphinxAtStartPar
{\hyperref[\detokenize{schemos/xsd:xsd-complextype}]{\sphinxcrossref{\DUrole{std}{\DUrole{std-ref}{complexType}}}}} gali turėti atributą \sphinxcode{\sphinxupquote{mixed}}. Jis reiškia, kad šiuo {\hyperref[\detokenize{schemos/xsd:xsd-complextype}]{\sphinxcrossref{\DUrole{std}{\DUrole{std-ref}{complexType}}}}}
aprašytas {\hyperref[\detokenize{schemos/xsd:xsd-element}]{\sphinxcrossref{\DUrole{std}{\DUrole{std-ref}{element}}}}} turės galimybę viduje turėti teksto. Tokiu atveju, prie {\hyperref[\detokenize{formatas:model}]{\sphinxcrossref{\sphinxcode{\sphinxupquote{model}}}}}
pridedama {\hyperref[\detokenize{formatas:property}]{\sphinxcrossref{\sphinxcode{\sphinxupquote{property}}}}} su pavadinimu \sphinxcode{\sphinxupquote{text}} ir tipu \sphinxcode{\sphinxupquote{string}}. Jos {\hyperref[\detokenize{dimensijos:property.source}]{\sphinxcrossref{\sphinxcode{\sphinxupquote{property.source}}}}} yra \sphinxcode{\sphinxupquote{text()}}.

\sphinxAtStartPar
Jei \sphinxcode{\sphinxupquote{complexType}} sudėtyje yra {\hyperref[\detokenize{schemos/xsd:xsd-choice}]{\sphinxcrossref{\DUrole{std}{\DUrole{std-ref}{choice}}}}} elementas ir šio elemento atributas {\hyperref[\detokenize{schemos/xsd:xsd-maxoccurs}]{\sphinxcrossref{\DUrole{std}{\DUrole{std-ref}{maxOccurs}}}}}
yra daugiau, nei \sphinxcode{\sphinxupquote{1}} arba yra \sphinxcode{\sphinxupquote{unbounded}}, iš šio complexType kuriama po vieną {\hyperref[\detokenize{formatas:model}]{\sphinxcrossref{\sphinxcode{\sphinxupquote{model}}}}} kiekvienam
{\hyperref[\detokenize{schemos/xsd:xsd-choice}]{\sphinxcrossref{\DUrole{std}{\DUrole{std-ref}{choice}}}}} pasirinkimui, kai šis pasirinkimas pridedamas prie kitų {\hyperref[\detokenize{dimensijos:property}]{\sphinxcrossref{\DUrole{std}{\DUrole{std-ref}{property}}}}}.

\sphinxAtStartPar
Jei \sphinxcode{\sphinxupquote{complexType}} viduje yra \DUrole{xref}{\DUrole{std}{\DUrole{std-ref}{complexContent}}}, kurio viduje yra \DUrole{xref}{\DUrole{std}{\DUrole{std-ref}{extension}}}, kurio {\hyperref[\detokenize{dimensijos:base}]{\sphinxcrossref{\DUrole{std}{\DUrole{std-ref}{base}}}}}
rodo į kitą, atskirai apibrėžtą elementą, tai prie {\hyperref[\detokenize{formatas:model}]{\sphinxcrossref{\sphinxcode{\sphinxupquote{model}}}}}, stulpelyje \sphinxcode{\sphinxupquote{prepare}}, nurodoma
funkcija {\color{red}\bfseries{}:function:`extends()`}, jos parametru nurodžius {\hyperref[\detokenize{formatas:model}]{\sphinxcrossref{\sphinxcode{\sphinxupquote{model}}}}}, kuris buvo sukurtas iš to tipo.

\sphinxAtStartPar
\sphinxcode{\sphinxupquote{complexType}} sudėtyje gali būti įvairios konstrukcijos, aprašančios atributus ir elementus, iš kurių šiam
{\hyperref[\detokenize{formatas:model}]{\sphinxcrossref{\sphinxcode{\sphinxupquote{model}}}}} formuojamos {\hyperref[\detokenize{formatas:property}]{\sphinxcrossref{\sphinxcode{\sphinxupquote{property}}}}}:
\begin{itemize}
\item {} 
\sphinxAtStartPar
\DUrole{xref}{\DUrole{std}{\DUrole{std-ref}{attribute}}}

\item {} 
\sphinxAtStartPar
\DUrole{xref}{\DUrole{std}{\DUrole{std-ref}{sequence}}}

\item {} 
\sphinxAtStartPar
\DUrole{xref}{\DUrole{std}{\DUrole{std-ref}{choice}}}

\item {} 
\sphinxAtStartPar
\DUrole{xref}{\DUrole{std}{\DUrole{std-ref}{all}}}

\item {} 
\sphinxAtStartPar
\DUrole{xref}{\DUrole{std}{\DUrole{std-ref}{simpleContent}}}

\item {} 
\sphinxAtStartPar
\DUrole{xref}{\DUrole{std}{\DUrole{std-ref}{complexContent}}}

\end{itemize}


\subparagraph{sequence}
\label{\detokenize{schemos/xsd:sequence}}\label{\detokenize{schemos/xsd:xsd-sequence}}
\sphinxAtStartPar
\sphinxcode{\sphinxupquote{sequence}} elementas būna \DUrole{xref}{\DUrole{std}{\DUrole{std-ref}{complexType}}} sudėtyje. Jis nurodo \DUrole{xref}{\DUrole{std}{\DUrole{std-ref}{element}}} elementų seką.
Kiekvienas \sphinxcode{\sphinxupquote{sequence}} viduje esantis \DUrole{xref}{\DUrole{std}{\DUrole{std-ref}{element}}} yra apdorojamas, ir iš jo sukurta savybė ar savybės
pridedamos prie iš \DUrole{xref}{\DUrole{std}{\DUrole{std-ref}{complexType}}} sukurto modelio.

\sphinxAtStartPar
Jei \sphinxcode{\sphinxupquote{sequence}} viduje yra {\hyperref[\detokenize{schemos/xsd:xsd-choice}]{\sphinxcrossref{\DUrole{std}{\DUrole{std-ref}{choice}}}}}, kurio {\hyperref[\detokenize{schemos/xsd:xsd-maxoccurs}]{\sphinxcrossref{\DUrole{std}{\DUrole{std-ref}{maxOccurs}}}}} yra "1", tai kiekvienam šio
{\hyperref[\detokenize{schemos/xsd:xsd-choice}]{\sphinxcrossref{\DUrole{std}{\DUrole{std-ref}{choice}}}}} pasirinkimui iš jo ir likusių savybių kuriamas atskiras modelis.

\sphinxAtStartPar
Jei \sphinxcode{\sphinxupquote{sequence}} turi atributą {\hyperref[\detokenize{schemos/xsd:xsd-maxoccurs}]{\sphinxcrossref{\DUrole{std}{\DUrole{std-ref}{maxOccurs}}}}} ir jo reikšmė yra daugiau nei 1 ar yra "unbounded",
tai kiekviena {\hyperref[\detokenize{formatas:property}]{\sphinxcrossref{\sphinxcode{\sphinxupquote{property}}}}}, sukurto iš \sphinxcode{\sphinxupquote{sequence viduje esančių elementų, tampa masyvu, kas reiškia,
kad prie jos pavadinimo prisideda `{[}{]}}}, o jei jos tipas būtų buvęs \sphinxcode{\sphinxupquote{ref}}, jis pasikeičia į \sphinxcode{\sphinxupquote{backref}}.


\subparagraph{choice}
\label{\detokenize{schemos/xsd:choice}}\label{\detokenize{schemos/xsd:xsd-choice}}
\sphinxAtStartPar
Jei \sphinxcode{\sphinxupquote{choice}} elemento atributas {\hyperref[\detokenize{schemos/xsd:xsd-maxoccurs}]{\sphinxcrossref{\DUrole{std}{\DUrole{std-ref}{maxOccurs}}}}} yra lygus "1", tai \sphinxcode{\sphinxupquote{choice}} verčiamas į DSA lygiai
taip pat, kaip ir {\hyperref[\detokenize{schemos/xsd:xsd-sequence}]{\sphinxcrossref{\DUrole{std}{\DUrole{std-ref}{sequence}}}}}.

\sphinxAtStartPar
Jei \sphinxcode{\sphinxupquote{choice}} elemento atributas {\hyperref[\detokenize{schemos/xsd:xsd-maxoccurs}]{\sphinxcrossref{\DUrole{std}{\DUrole{std-ref}{maxOccurs}}}}} yra daugiau nei "1" arba yra "unbounded", tai su
kiekvienu šio \sphinxcode{\sphinxupquote{choice}} viduje esančiu pasirinkimu (tai gali būti {\hyperref[\detokenize{schemos/xsd:xsd-element}]{\sphinxcrossref{\DUrole{std}{\DUrole{std-ref}{element}}}}}, {\hyperref[\detokenize{schemos/xsd:xsd-sequence}]{\sphinxcrossref{\DUrole{std}{\DUrole{std-ref}{sequence}}}}}
ar kitas \sphinxcode{\sphinxupquote{choice}}) bus kuriamas atskiras {\hyperref[\detokenize{formatas:model}]{\sphinxcrossref{\sphinxcode{\sphinxupquote{model}}}}} iš {\hyperref[\detokenize{schemos/xsd:xsd-complextype}]{\sphinxcrossref{\DUrole{std}{\DUrole{std-ref}{complexType}}}}}, kurio sudėtyje yra šis
\sphinxcode{\sphinxupquote{choice}} (tiesiogiai, ar kito {\hyperref[\detokenize{schemos/xsd:xsd-sequence}]{\sphinxcrossref{\DUrole{std}{\DUrole{std-ref}{sequence}}}}} ar \sphinxcode{\sphinxupquote{choice}} viduje).

\begin{sphinxadmonition}{note}{Pavyzdys}

\sphinxAtStartPar
\sphinxstylestrong{Duomenys}

\sphinxAtStartPar
Pirmas variantas:

\begin{sphinxVerbatim}[commandchars=\\\{\}]
\PYG{n+nt}{\PYGZlt{}country}\PYG{n+nt}{\PYGZgt{}}
\PYG{+w}{    }\PYG{n+nt}{\PYGZlt{}population}\PYG{n+nt}{\PYGZgt{}}800000\PYG{n+nt}{\PYGZlt{}/population\PYGZgt{}}
\PYG{+w}{    }\PYG{n+nt}{\PYGZlt{}area}\PYG{n+nt}{\PYGZgt{}}700.5\PYG{n+nt}{\PYGZlt{}/area\PYGZgt{}}
\PYG{+w}{    }\PYG{n+nt}{\PYGZlt{}king\PYGZus{}or\PYGZus{}queen}\PYG{n+nt}{\PYGZgt{}}Elžbieta\PYG{+w}{ }II\PYG{n+nt}{\PYGZlt{}/king\PYGZus{}or\PYGZus{}queen\PYGZgt{}}
\PYG{n+nt}{\PYGZlt{}/country\PYGZgt{}}
\end{sphinxVerbatim}

\sphinxAtStartPar
Antras variantas:

\begin{sphinxVerbatim}[commandchars=\\\{\}]
\PYG{n+nt}{\PYGZlt{}country}\PYG{n+nt}{\PYGZgt{}}
\PYG{+w}{    }\PYG{n+nt}{\PYGZlt{}population}\PYG{n+nt}{\PYGZgt{}}1000000\PYG{n+nt}{\PYGZlt{}/population\PYGZgt{}}
\PYG{+w}{    }\PYG{n+nt}{\PYGZlt{}area}\PYG{n+nt}{\PYGZgt{}}500.0\PYG{n+nt}{\PYGZlt{}/area\PYGZgt{}}
\PYG{+w}{    }\PYG{n+nt}{\PYGZlt{}president}\PYG{n+nt}{\PYGZgt{}}Ona\PYG{+w}{ }Grybauskaitė\PYG{n+nt}{\PYGZlt{}/president\PYGZgt{}}
\PYG{n+nt}{\PYGZlt{}/country\PYGZgt{}}
\end{sphinxVerbatim}

\sphinxAtStartPar
\sphinxstylestrong{Schema}

\begin{sphinxVerbatim}[commandchars=\\\{\}]
\PYG{n+nt}{\PYGZlt{}xs:schema}\PYG{+w}{ }\PYG{n+na}{xmlns:xs=}\PYG{l+s}{\PYGZdq{}http://www.w3.org/2001/XMLSchema\PYGZdq{}}\PYG{n+nt}{\PYGZgt{}}

\PYG{+w}{    }\PYG{n+nt}{\PYGZlt{}xs:element}\PYG{+w}{ }\PYG{n+na}{name=}\PYG{l+s}{\PYGZdq{}country\PYGZdq{}}\PYG{n+nt}{\PYGZgt{}}
\PYG{+w}{        }\PYG{n+nt}{\PYGZlt{}xs:complexType}\PYG{n+nt}{\PYGZgt{}}
\PYG{+w}{            }\PYG{n+nt}{\PYGZlt{}xs:element}\PYG{+w}{ }\PYG{n+na}{name=}\PYG{l+s}{\PYGZdq{}population\PYGZdq{}}\PYG{+w}{ }\PYG{n+na}{type=}\PYG{l+s}{\PYGZdq{}xs:integer\PYGZdq{}}\PYG{+w}{ }\PYG{n+nt}{/\PYGZgt{}}
\PYG{+w}{            }\PYG{n+nt}{\PYGZlt{}xs:element}\PYG{+w}{ }\PYG{n+na}{name=}\PYG{l+s}{\PYGZdq{}area\PYGZdq{}}\PYG{+w}{ }\PYG{n+na}{type=}\PYG{l+s}{\PYGZdq{}xs:decimal\PYGZdq{}}\PYG{+w}{ }\PYG{n+nt}{/\PYGZgt{}}
\PYG{+w}{            }\PYG{n+nt}{\PYGZlt{}xs:choice}\PYG{+w}{ }\PYG{n+na}{maxOccurs=}\PYG{l+s}{\PYGZdq{}1\PYGZdq{}}\PYG{n+nt}{\PYGZgt{}}
\PYG{+w}{                }\PYG{n+nt}{\PYGZlt{}xs:element}\PYG{+w}{ }\PYG{n+na}{name=}\PYG{l+s}{\PYGZdq{}president\PYGZdq{}}\PYG{+w}{ }\PYG{n+na}{type=}\PYG{l+s}{\PYGZdq{}xs:string\PYGZdq{}}\PYG{+w}{ }\PYG{n+nt}{/\PYGZgt{}}
\PYG{+w}{                }\PYG{n+nt}{\PYGZlt{}xs:element}\PYG{+w}{ }\PYG{n+na}{name=}\PYG{l+s}{\PYGZdq{}king\PYGZus{}or\PYGZus{}queen\PYGZdq{}}\PYG{+w}{ }\PYG{n+na}{type=}\PYG{l+s}{\PYGZdq{}xs:string\PYGZdq{}}\PYG{+w}{ }\PYG{n+nt}{/\PYGZgt{}}
\PYG{+w}{            }\PYG{n+nt}{\PYGZlt{}/xs:choice\PYGZgt{}}

\PYG{+w}{        }\PYG{n+nt}{\PYGZlt{}/xs:complexType\PYGZgt{}}
\PYG{+w}{    }\PYG{n+nt}{\PYGZlt{}/xs:element\PYGZgt{}}

\PYG{n+nt}{\PYGZlt{}/xs:schema\PYGZgt{}}
\end{sphinxVerbatim}

\sphinxAtStartPar
{\color{red}\bfseries{}**}Struktūros aprašas \sphinxhyphen{} **
\end{sphinxadmonition}


\subparagraph{all}
\label{\detokenize{schemos/xsd:all}}\label{\detokenize{schemos/xsd:xsd-all}}
\sphinxAtStartPar
Elementas \sphinxcode{\sphinxupquote{all}} reiškia, kad jo viduje aprašyti elementai turi eiti nurodyta tvarka, ir maksimalliai
gali būti po 1 kartą. Minimaliai gali būti taip, kaip nurodyta prie kiekvieno elemento naudojant
{\hyperref[\detokenize{schemos/xsd:xsd-minoccurs}]{\sphinxcrossref{\DUrole{std}{\DUrole{std-ref}{minOccurs}}}}}. Taigi, iš kiekvieno \sphinxcode{\sphinxupquote{all}} viduje esamų elementų bus kuriama savybė,
ir galbūt modelis, kaip nurodyta {\hyperref[\detokenize{schemos/xsd:xsd-element}]{\sphinxcrossref{\DUrole{std}{\DUrole{std-ref}{element}}}}}.


\subparagraph{complexContent}
\label{\detokenize{schemos/xsd:complexcontent}}\label{\detokenize{schemos/xsd:xsd-complex-content}}
\sphinxAtStartPar
\sphinxcode{\sphinxupquote{complexContent}} būna {\hyperref[\detokenize{schemos/xsd:xsd-complextype}]{\sphinxcrossref{\DUrole{std}{\DUrole{std-ref}{complexType}}}}} viduje ir aprašo sudėtinį turinį.

\sphinxAtStartPar
\sphinxcode{\sphinxupquote{complexContent}} viduje būna {\hyperref[\detokenize{schemos/xsd:xsd-extension}]{\sphinxcrossref{\DUrole{std}{\DUrole{std-ref}{extension}}}}}, kuris turi atributą {\hyperref[\detokenize{schemos/xsd:xsd-base}]{\sphinxcrossref{\DUrole{std}{\DUrole{std-ref}{base}}}}}.
Šis atributas nurodo, kokio kito tipo pagrindu kuriamas šis tipas. Iš {\hyperref[\detokenize{schemos/xsd:xsd-base}]{\sphinxcrossref{\DUrole{std}{\DUrole{std-ref}{base}}}}} nurodomo
tipo sukurtas modelis įdedamas į iš šio \sphinxcode{\sphinxupquote{complexContent}} tėvinio \DUrole{xref}{\DUrole{std}{\DUrole{std-ref}{complexType}}} prepare
stulpelyje nurodomą funkciją {\color{red}\bfseries{}:function:`extends()`}.

\sphinxAtStartPar
{\hyperref[\detokenize{schemos/xsd:xsd-extension}]{\sphinxcrossref{\DUrole{std}{\DUrole{std-ref}{extension}}}}} viduje
gali būti {\hyperref[\detokenize{schemos/xsd:xsd-sequence}]{\sphinxcrossref{\DUrole{std}{\DUrole{std-ref}{sequence}}}}}, {\hyperref[\detokenize{schemos/xsd:xsd-choice}]{\sphinxcrossref{\DUrole{std}{\DUrole{std-ref}{choice}}}}} ir {\hyperref[\detokenize{schemos/xsd:xsd-all}]{\sphinxcrossref{\DUrole{std}{\DUrole{std-ref}{all}}}}}, o taip pat {\hyperref[\detokenize{schemos/xsd:xsd-attribute}]{\sphinxcrossref{\DUrole{std}{\DUrole{std-ref}{attribute}}}}}.

\sphinxAtStartPar
Iš šių {\hyperref[\detokenize{schemos/xsd:xsd-attribute}]{\sphinxcrossref{\DUrole{std}{\DUrole{std-ref}{attribute}}}}} bei iš {\hyperref[\detokenize{schemos/xsd:xsd-sequence}]{\sphinxcrossref{\DUrole{std}{\DUrole{std-ref}{sequence}}}}}, {\hyperref[\detokenize{schemos/xsd:xsd-choice}]{\sphinxcrossref{\DUrole{std}{\DUrole{std-ref}{choice}}}}} ir {\hyperref[\detokenize{schemos/xsd:xsd-all}]{\sphinxcrossref{\DUrole{std}{\DUrole{std-ref}{all}}}}} viduje
esančių {\hyperref[\detokenize{schemos/xsd:xsd-element}]{\sphinxcrossref{\DUrole{std}{\DUrole{std-ref}{element}}}}} kuriamos savybės, ir pridedamos prie iš {\hyperref[\detokenize{schemos/xsd:xsd-complextype}]{\sphinxcrossref{\DUrole{std}{\DUrole{std-ref}{complexType}}}}} sukurto
modelio, pagal tas pačias taisykles, kaip ir iš tiesiogiai {\hyperref[\detokenize{schemos/xsd:xsd-complextype}]{\sphinxcrossref{\DUrole{std}{\DUrole{std-ref}{complexType}}}}} esančių tokių
pačių elementų.

\begin{sphinxadmonition}{note}{Pavyzdys}

\sphinxAtStartPar
\sphinxstylestrong{Duomenys}

\begin{sphinxVerbatim}[commandchars=\\\{\}]
\PYG{n+nt}{\PYGZlt{}example}\PYG{n+nt}{\PYGZgt{}}
\PYG{+w}{    }\PYG{n+nt}{\PYGZlt{}country}\PYG{+w}{ }\PYG{n+na}{id=}\PYG{l+s}{\PYGZdq{}C1\PYGZdq{}}\PYG{n+nt}{\PYGZgt{}}
\PYG{+w}{        }\PYG{n+nt}{\PYGZlt{}name}\PYG{n+nt}{\PYGZgt{}}Lithuania\PYG{n+nt}{\PYGZlt{}/name\PYGZgt{}}
\PYG{+w}{        }\PYG{n+nt}{\PYGZlt{}capital}\PYG{n+nt}{\PYGZgt{}}Vilnius\PYG{n+nt}{\PYGZlt{}/capital\PYGZgt{}}
\PYG{+w}{    }\PYG{n+nt}{\PYGZlt{}/country\PYGZgt{}}

\PYG{+w}{    }\PYG{n+nt}{\PYGZlt{}city}\PYG{+w}{ }\PYG{n+na}{id=}\PYG{l+s}{\PYGZdq{}CT1\PYGZdq{}}\PYG{n+nt}{\PYGZgt{}}
\PYG{+w}{        }\PYG{n+nt}{\PYGZlt{}name}\PYG{n+nt}{\PYGZgt{}}Kaunas\PYG{n+nt}{\PYGZlt{}/name\PYGZgt{}}
\PYG{+w}{        }\PYG{n+nt}{\PYGZlt{}country}\PYG{n+nt}{\PYGZgt{}}Lithuania\PYG{n+nt}{\PYGZlt{}/country\PYGZgt{}}
\PYG{+w}{    }\PYG{n+nt}{\PYGZlt{}/city\PYGZgt{}}
\PYG{n+nt}{\PYGZlt{}/example\PYGZgt{}}
\end{sphinxVerbatim}

\sphinxAtStartPar
\sphinxstylestrong{Schema}

\begin{sphinxVerbatim}[commandchars=\\\{\}]
\PYG{n+nt}{\PYGZlt{}xs:schema}\PYG{+w}{ }\PYG{n+na}{xmlns:xs=}\PYG{l+s}{\PYGZdq{}http://www.w3.org/2001/XMLSchema\PYGZdq{}}\PYG{+w}{ }\PYG{n+na}{elementFormDefault=}\PYG{l+s}{\PYGZdq{}qualified\PYGZdq{}}\PYG{n+nt}{\PYGZgt{}}

\PYG{+w}{    }\PYG{n+nt}{\PYGZlt{}xs:complexType}\PYG{+w}{ }\PYG{n+na}{name=}\PYG{l+s}{\PYGZdq{}Place\PYGZdq{}}\PYG{n+nt}{\PYGZgt{}}
\PYG{+w}{        }\PYG{n+nt}{\PYGZlt{}xs:sequence}\PYG{n+nt}{\PYGZgt{}}
\PYG{+w}{            }\PYG{n+nt}{\PYGZlt{}xs:element}\PYG{+w}{ }\PYG{n+na}{name=}\PYG{l+s}{\PYGZdq{}name\PYGZdq{}}\PYG{+w}{ }\PYG{n+na}{type=}\PYG{l+s}{\PYGZdq{}xs:string\PYGZdq{}}\PYG{n+nt}{/\PYGZgt{}}
\PYG{+w}{        }\PYG{n+nt}{\PYGZlt{}/xs:sequence\PYGZgt{}}
\PYG{+w}{        }\PYG{n+nt}{\PYGZlt{}xs:attribute}\PYG{+w}{ }\PYG{n+na}{name=}\PYG{l+s}{\PYGZdq{}id\PYGZdq{}}\PYG{+w}{ }\PYG{n+na}{type=}\PYG{l+s}{\PYGZdq{}xs:string\PYGZdq{}}\PYG{+w}{ }\PYG{n+na}{use=}\PYG{l+s}{\PYGZdq{}required\PYGZdq{}}\PYG{n+nt}{/\PYGZgt{}}
\PYG{+w}{    }\PYG{n+nt}{\PYGZlt{}/xs:complexType\PYGZgt{}}

\PYG{+w}{    }\PYG{n+nt}{\PYGZlt{}xs:complexType}\PYG{+w}{ }\PYG{n+na}{name=}\PYG{l+s}{\PYGZdq{}Country\PYGZdq{}}\PYG{n+nt}{\PYGZgt{}}
\PYG{+w}{        }\PYG{n+nt}{\PYGZlt{}xs:complexContent}\PYG{n+nt}{\PYGZgt{}}
\PYG{+w}{            }\PYG{n+nt}{\PYGZlt{}xs:extension}\PYG{+w}{ }\PYG{n+na}{base=}\PYG{l+s}{\PYGZdq{}Place\PYGZdq{}}\PYG{n+nt}{\PYGZgt{}}
\PYG{+w}{                }\PYG{n+nt}{\PYGZlt{}xs:sequence}\PYG{n+nt}{\PYGZgt{}}
\PYG{+w}{                    }\PYG{n+nt}{\PYGZlt{}xs:element}\PYG{+w}{ }\PYG{n+na}{name=}\PYG{l+s}{\PYGZdq{}capital\PYGZdq{}}\PYG{+w}{ }\PYG{n+na}{type=}\PYG{l+s}{\PYGZdq{}xs:string\PYGZdq{}}\PYG{n+nt}{/\PYGZgt{}}
\PYG{+w}{                }\PYG{n+nt}{\PYGZlt{}/xs:sequence\PYGZgt{}}
\PYG{+w}{            }\PYG{n+nt}{\PYGZlt{}/xs:extension\PYGZgt{}}
\PYG{+w}{        }\PYG{n+nt}{\PYGZlt{}/xs:complexContent\PYGZgt{}}
\PYG{+w}{    }\PYG{n+nt}{\PYGZlt{}/xs:complexType\PYGZgt{}}

\PYG{+w}{    }\PYG{n+nt}{\PYGZlt{}xs:complexType}\PYG{+w}{ }\PYG{n+na}{name=}\PYG{l+s}{\PYGZdq{}City\PYGZdq{}}\PYG{n+nt}{\PYGZgt{}}
\PYG{+w}{        }\PYG{n+nt}{\PYGZlt{}xs:complexContent}\PYG{n+nt}{\PYGZgt{}}
\PYG{+w}{            }\PYG{n+nt}{\PYGZlt{}xs:extension}\PYG{+w}{ }\PYG{n+na}{base=}\PYG{l+s}{\PYGZdq{}Place\PYGZdq{}}\PYG{n+nt}{\PYGZgt{}}
\PYG{+w}{                }\PYG{n+nt}{\PYGZlt{}xs:sequence}\PYG{n+nt}{\PYGZgt{}}
\PYG{+w}{                    }\PYG{n+nt}{\PYGZlt{}xs:element}\PYG{+w}{ }\PYG{n+na}{name=}\PYG{l+s}{\PYGZdq{}country\PYGZdq{}}\PYG{+w}{ }\PYG{n+na}{type=}\PYG{l+s}{\PYGZdq{}xs:string\PYGZdq{}}\PYG{n+nt}{/\PYGZgt{}}
\PYG{+w}{                }\PYG{n+nt}{\PYGZlt{}/xs:sequence\PYGZgt{}}
\PYG{+w}{            }\PYG{n+nt}{\PYGZlt{}/xs:extension\PYGZgt{}}
\PYG{+w}{        }\PYG{n+nt}{\PYGZlt{}/xs:complexContent\PYGZgt{}}
\PYG{+w}{    }\PYG{n+nt}{\PYGZlt{}/xs:complexType\PYGZgt{}}

\PYG{+w}{    }\PYG{n+nt}{\PYGZlt{}xs:element}\PYG{+w}{ }\PYG{n+na}{name=}\PYG{l+s}{\PYGZdq{}country\PYGZdq{}}\PYG{+w}{ }\PYG{n+na}{type=}\PYG{l+s}{\PYGZdq{}Country\PYGZdq{}}\PYG{n+nt}{/\PYGZgt{}}
\PYG{+w}{    }\PYG{n+nt}{\PYGZlt{}xs:element}\PYG{+w}{ }\PYG{n+na}{name=}\PYG{l+s}{\PYGZdq{}city\PYGZdq{}}\PYG{+w}{ }\PYG{n+na}{type=}\PYG{l+s}{\PYGZdq{}City\PYGZdq{}}\PYG{n+nt}{/\PYGZgt{}}

\PYG{n+nt}{\PYGZlt{}/xs:schema\PYGZgt{}}
\end{sphinxVerbatim}

\sphinxAtStartPar
{\color{red}\bfseries{}**}Struktūros aprašas \sphinxhyphen{} **


\begin{savenotes}\sphinxattablestart
\sphinxthistablewithglobalstyle
\centering
\begin{tabulary}{\linewidth}[t]{TTTTTTTT}
\sphinxtoprule
\sphinxstyletheadfamily 
\sphinxAtStartPar
dataset
&\sphinxstyletheadfamily 
\sphinxAtStartPar
model
&\sphinxstyletheadfamily 
\sphinxAtStartPar
property
&\sphinxstyletheadfamily 
\sphinxAtStartPar
type
&\sphinxstyletheadfamily 
\sphinxAtStartPar
ref
&\sphinxstyletheadfamily 
\sphinxAtStartPar
source
&\sphinxstyletheadfamily 
\sphinxAtStartPar
prepare
&\sphinxstyletheadfamily 
\sphinxAtStartPar
level
\\
\sphinxmidrule
\sphinxtableatstartofbodyhook
\sphinxAtStartPar
xsd
&&&&&&&\\
\sphinxhline&&&&&&&\\
\sphinxhline&&&&&&&\\
\sphinxhline&&&&&&&\\
\sphinxhline&&&&&&&\\
\sphinxhline&&&&&&&\\
\sphinxhline&&&&&&&\\
\sphinxbottomrule
\end{tabulary}
\sphinxtableafterendhook\par
\sphinxattableend\end{savenotes}
\end{sphinxadmonition}

\sphinxAtStartPar
Pavyzdyje:
\begin{itemize}
\item {} 
\sphinxAtStartPar
\DUrole{xref}{\DUrole{std}{\DUrole{std-ref}{complexType}}} Place tampa {\hyperref[\detokenize{formatas:model}]{\sphinxcrossref{\sphinxcode{\sphinxupquote{model}}}}} Place, o \DUrole{xref}{\DUrole{std}{\DUrole{std-ref}{complexType}}} Place viduje esantis
{\hyperref[\detokenize{schemos/xsd:xsd-element}]{\sphinxcrossref{\DUrole{std}{\DUrole{std-ref}{element}}}}} \sphinxcode{\sphinxupquote{name}} ir {\hyperref[\detokenize{schemos/xsd:xsd-attribute}]{\sphinxcrossref{\DUrole{std}{\DUrole{std-ref}{attribute}}}}} \sphinxcode{\sphinxupquote{id}} tampa jo savybėmis ({\hyperref[\detokenize{formatas:property}]{\sphinxcrossref{\sphinxcode{\sphinxupquote{property}}}}}).

\item {} 
\sphinxAtStartPar
\DUrole{xref}{\DUrole{std}{\DUrole{std-ref}{complexType}}} City tampa {\hyperref[\detokenize{formatas:model}]{\sphinxcrossref{\sphinxcode{\sphinxupquote{model}}}}} City, o \DUrole{xref}{\DUrole{std}{\DUrole{std-ref}{complexType}}} City viduje esančiame
{\color{red}\bfseries{}:rex:`xsd\_extension`} esantis {\hyperref[\detokenize{schemos/xsd:xsd-element}]{\sphinxcrossref{\DUrole{std}{\DUrole{std-ref}{element}}}}} \sphinxcode{\sphinxupquote{country}} tampa {\hyperref[\detokenize{formatas:property}]{\sphinxcrossref{\sphinxcode{\sphinxupquote{property}}}}} \sphinxcode{\sphinxupquote{country}}.

\sphinxAtStartPar
{\hyperref[\detokenize{schemos/xsd:xsd-extension}]{\sphinxcrossref{\DUrole{std}{\DUrole{std-ref}{extension}}}}} {\hyperref[\detokenize{schemos/xsd:xsd-base}]{\sphinxcrossref{\DUrole{std}{\DUrole{std-ref}{base}}}}} atributas nurodo į \DUrole{xref}{\DUrole{std}{\DUrole{std-ref}{complexType}}} \sphinxcode{\sphinxupquote{Place}}, todėl
iš jo sukurtas {\hyperref[\detokenize{formatas:model}]{\sphinxcrossref{\sphinxcode{\sphinxupquote{model}}}}} \sphinxcode{\sphinxupquote{Place}} nurodomas {\hyperref[\detokenize{formatas:model}]{\sphinxcrossref{\sphinxcode{\sphinxupquote{model}}}}} \sphinxcode{\sphinxupquote{City}} \sphinxcode{\sphinxupquote{prepare}} stulpelyje
esančioje {\color{red}\bfseries{}:function:`extends`} funkcijoje. Tai reiškia, kad vėliau, interpretuojant šį DSA,
visos {\hyperref[\detokenize{formatas:model}]{\sphinxcrossref{\sphinxcode{\sphinxupquote{model}}}}} \sphinxcode{\sphinxupquote{Place}} esančios {\hyperref[\detokenize{formatas:property}]{\sphinxcrossref{\sphinxcode{\sphinxupquote{property}}}}} įtraukiamos į {\hyperref[\detokenize{formatas:model}]{\sphinxcrossref{\sphinxcode{\sphinxupquote{model}}}}} City.

\item {} 
\sphinxAtStartPar
analogiškai su \sphinxcode{\sphinxupquote{Country}}.

\end{itemize}


\subparagraph{simpleContent}
\label{\detokenize{schemos/xsd:simplecontent}}\label{\detokenize{schemos/xsd:xsd-simple-content}}
\sphinxAtStartPar
\sphinxcode{\sphinxupquote{simpleContent}} elementas būna viduje \sphinxcode{\sphinxupquote{complexType}} elemento. Viduje \sphinxcode{\sphinxupquote{simpleContent}} elemento gali
būti arba {\hyperref[\detokenize{schemos/xsd:xsd-restriction}]{\sphinxcrossref{\DUrole{std}{\DUrole{std-ref}{restriction}}}}} arba {\hyperref[\detokenize{schemos/xsd:xsd-extension}]{\sphinxcrossref{\DUrole{std}{\DUrole{std-ref}{extension}}}}} elementas.

\sphinxAtStartPar
Jei \sphinxcode{\sphinxupquote{simpleContent}} viduje naudojamas {\hyperref[\detokenize{schemos/xsd:xsd-extension}]{\sphinxcrossref{\DUrole{std}{\DUrole{std-ref}{extension}}}}}, tai {\hyperref[\detokenize{schemos/xsd:xsd-extension}]{\sphinxcrossref{\DUrole{std}{\DUrole{std-ref}{extension}}}}} viduje nurodomi
{\hyperref[\detokenize{schemos/xsd:xsd-attribute}]{\sphinxcrossref{\DUrole{std}{\DUrole{std-ref}{attribute}}}}}. Iš kiekvieno jų kuriama {\hyperref[\detokenize{formatas:property}]{\sphinxcrossref{\sphinxcode{\sphinxupquote{property}}}}} ir pridedama prie {\hyperref[\detokenize{formatas:model}]{\sphinxcrossref{\sphinxcode{\sphinxupquote{model}}}}}, sukurto
iš \DUrole{xref}{\DUrole{std}{\DUrole{std-ref}{xsd\_complex\_type}}}, kurio viduje yra. Taip pat, prie modelio, sukurto iš \DUrole{xref}{\DUrole{std}{\DUrole{std-ref}{xsd\_complex\_type}}} pridedama {\hyperref[\detokenize{formatas:property}]{\sphinxcrossref{\sphinxcode{\sphinxupquote{property}}}}} pavadinimu
\sphinxcode{\sphinxupquote{text}} ir jai priskiriamas tipas, kuris gaunamas iš {\hyperref[\detokenize{schemos/xsd:xsd-base}]{\sphinxcrossref{\DUrole{std}{\DUrole{std-ref}{base}}}}}, pagal tipų siejimo lentelę
\DUrole{xref}{\DUrole{std}{\DUrole{std-ref}{xsd\_type\_conversion}}}.

\sphinxAtStartPar
Jei \sphinxcode{\sphinxupquote{simpleContent}} viduje naudojamas {\hyperref[\detokenize{schemos/xsd:xsd-restriction}]{\sphinxcrossref{\DUrole{std}{\DUrole{std-ref}{restriction}}}}}, tai reiškia, kad tipas, kurio viduje
yra šis mazgas, yra apribojamas. Apribojimai gali būti tokie, kaip minimalios ar maximalios reikšmės,
ilgis ar kitos duomenų validacijos taisyklės. Dauguma jų yra ignoruojami, nes DSA duomenų reikšmių
apribojimui įrankių neturi. Tačiau, jei {\hyperref[\detokenize{schemos/xsd:xsd-restriction}]{\sphinxcrossref{\DUrole{std}{\DUrole{std-ref}{restriction}}}}} viduje yra \DUrole{xref}{\DUrole{std}{\DUrole{std-ref}{enumeration}}},
tai išvardintos reikšmės perkeliamos į {\hyperref[\detokenize{dimensijos:enum}]{\sphinxcrossref{\DUrole{std}{\DUrole{std-ref}{enum}}}}}. Išsamiau paaiškinta prie {\hyperref[\detokenize{schemos/xsd:xsd-enumeration}]{\sphinxcrossref{\DUrole{std}{\DUrole{std-ref}{enumeration}}}}}.


\subparagraph{enumeration}
\label{\detokenize{schemos/xsd:enumeration}}\label{\detokenize{schemos/xsd:xsd-enumeration}}
\sphinxAtStartPar
\sphinxcode{\sphinxupquote{enumeration}} išvardija reikšmes, iš kurių gali būti pasirenkama {\hyperref[\detokenize{schemos/xsd:xsd-element}]{\sphinxcrossref{\DUrole{std}{\DUrole{std-ref}{element}}}}} arba \sphinxcode{\sphinxupquote{xsd\_attribute}}
reikšmė. DSA jo atitiktis yra {\hyperref[\detokenize{dimensijos:enum}]{\sphinxcrossref{\DUrole{std}{\DUrole{std-ref}{enum}}}}}. \sphinxcode{\sphinxupquote{enumeration}} būna \DUrole{xref}{\DUrole{std}{\DUrole{std-ref}{xsd\_simple\_type}}} sudėtyje esančio
{\hyperref[\detokenize{schemos/xsd:xsd-restriction}]{\sphinxcrossref{\DUrole{std}{\DUrole{std-ref}{restriction}}}}} viduje, o šis \DUrole{xref}{\DUrole{std}{\DUrole{std-ref}{xsd\_simple\_type}}} aprašo {\hyperref[\detokenize{schemos/xsd:xsd-element}]{\sphinxcrossref{\DUrole{std}{\DUrole{std-ref}{element}}}}} arba
{\hyperref[\detokenize{schemos/xsd:xsd-attribute}]{\sphinxcrossref{\DUrole{std}{\DUrole{std-ref}{attribute}}}}} tipą. Taigi, iš \sphinxcode{\sphinxupquote{enumeration}} gautas reikšmių sąrašas perkeliamas
į DSA savybės, suformuotos iš {\hyperref[\detokenize{schemos/xsd:xsd-element}]{\sphinxcrossref{\DUrole{std}{\DUrole{std-ref}{element}}}}} arba {\hyperref[\detokenize{schemos/xsd:xsd-attribute}]{\sphinxcrossref{\DUrole{std}{\DUrole{std-ref}{attribute}}}}} {\hyperref[\detokenize{dimensijos:enum}]{\sphinxcrossref{\DUrole{std}{\DUrole{std-ref}{enum}}}}} reikšmes.

\begin{sphinxadmonition}{note}{Pavyzdys}

\sphinxAtStartPar
\sphinxstylestrong{Duomenys}

\sphinxAtStartPar
Pirmas variantas:

\begin{sphinxVerbatim}[commandchars=\\\{\}]
\PYG{n+nt}{\PYGZlt{}country}\PYG{n+nt}{\PYGZgt{}}
\PYG{+w}{    }\PYG{n+nt}{\PYGZlt{}population}\PYG{n+nt}{\PYGZgt{}}800000\PYG{n+nt}{\PYGZlt{}/population\PYGZgt{}}
\PYG{+w}{    }\PYG{n+nt}{\PYGZlt{}area}\PYG{n+nt}{\PYGZgt{}}700.5\PYG{n+nt}{\PYGZlt{}/area\PYGZgt{}}
\PYG{+w}{    }\PYG{n+nt}{\PYGZlt{}king\PYGZus{}or\PYGZus{}queen}\PYG{n+nt}{\PYGZgt{}}Elžbieta\PYG{+w}{ }II\PYG{n+nt}{\PYGZlt{}/king\PYGZus{}or\PYGZus{}queen\PYGZgt{}}
\PYG{n+nt}{\PYGZlt{}/country\PYGZgt{}}
\end{sphinxVerbatim}

\sphinxAtStartPar
Antras variantas:

\begin{sphinxVerbatim}[commandchars=\\\{\}]
\PYG{n+nt}{\PYGZlt{}country}\PYG{n+nt}{\PYGZgt{}}
\PYG{+w}{    }\PYG{n+nt}{\PYGZlt{}population}\PYG{n+nt}{\PYGZgt{}}1000000\PYG{n+nt}{\PYGZlt{}/population\PYGZgt{}}
\PYG{+w}{    }\PYG{n+nt}{\PYGZlt{}area}\PYG{n+nt}{\PYGZgt{}}500.0\PYG{n+nt}{\PYGZlt{}/area\PYGZgt{}}
\PYG{+w}{    }\PYG{n+nt}{\PYGZlt{}president}\PYG{n+nt}{\PYGZgt{}}Ona\PYG{+w}{ }Grybauskaitė\PYG{n+nt}{\PYGZlt{}/president\PYGZgt{}}
\PYG{n+nt}{\PYGZlt{}/country\PYGZgt{}}
\end{sphinxVerbatim}

\sphinxAtStartPar
\sphinxstylestrong{Schema}

\begin{sphinxVerbatim}[commandchars=\\\{\}]

\end{sphinxVerbatim}
\end{sphinxadmonition}

\sphinxAtStartPar
<xs:schema xmlns:xs="http://www.w3.org/2001/XMLSchema" elementFormDefault="qualified">
\begin{quote}
\begin{description}
\sphinxlineitem{<xs:element name="Country">}\begin{description}
\sphinxlineitem{<xs:complexType>}\begin{description}
\sphinxlineitem{<xs:sequence>}
\sphinxAtStartPar
<xs:element name="name" type="xs:string" />
<xs:element name="head\_of\_state">
\begin{quote}
\begin{description}
\sphinxlineitem{<xs:simpleType>}\begin{description}
\sphinxlineitem{<xs:restriction base="xs:string">}
\sphinxAtStartPar
<xs:enumeration value="President" />
<xs:enumeration value="Monarch" />
<xs:enumeration value="PrimeMinister" />

\end{description}

\sphinxAtStartPar
</xs:restriction>

\end{description}

\sphinxAtStartPar
</xs:simpleType>
\end{quote}

\sphinxAtStartPar
</xs:element>

\end{description}

\sphinxAtStartPar
</xs:sequence>

\end{description}

\sphinxAtStartPar
</xs:complexType>

\end{description}

\sphinxAtStartPar
</xs:element>
\end{quote}

\sphinxAtStartPar
</xs:schema>
\begin{quote}

\sphinxAtStartPar
{\color{red}\bfseries{}**}Struktūros aprašas \sphinxhyphen{} **


\begin{savenotes}\sphinxattablestart
\sphinxthistablewithglobalstyle
\centering
\begin{tabulary}{\linewidth}[t]{TTTTTTTT}
\sphinxtoprule
\sphinxstyletheadfamily 
\sphinxAtStartPar
dataset
&\sphinxstyletheadfamily 
\sphinxAtStartPar
model
&\sphinxstyletheadfamily 
\sphinxAtStartPar
property
&\sphinxstyletheadfamily 
\sphinxAtStartPar
type
&\sphinxstyletheadfamily 
\sphinxAtStartPar
ref
&\sphinxstyletheadfamily 
\sphinxAtStartPar
source       prepare
&\sphinxstyletheadfamily 
\sphinxAtStartPar
level
&\sphinxstyletheadfamily \\
\sphinxmidrule
\sphinxtableatstartofbodyhook
\sphinxAtStartPar

&&&
\sphinxAtStartPar
schema
&
\sphinxAtStartPar
xsd
&
\sphinxAtStartPar
country.xsd
&&\\
\sphinxhline\sphinxstartmulticolumn{3}%
\begin{varwidth}[t]{\sphinxcolwidth{3}{8}}
\sphinxAtStartPar
xsd
\par
\vskip-\baselineskip\vbox{\hbox{\strut}}\end{varwidth}%
\sphinxstopmulticolumn
&&&&&\\
\sphinxhline
\sphinxAtStartPar

&\sphinxstartmulticolumn{2}%
\begin{varwidth}[t]{\sphinxcolwidth{2}{8}}
\sphinxAtStartPar
\sphinxstylestrong{Country}
\par
\vskip-\baselineskip\vbox{\hbox{\strut}}\end{varwidth}%
\sphinxstopmulticolumn
&&&
\sphinxAtStartPar
/country
&
\sphinxAtStartPar
0
&\\
\sphinxhline
\sphinxAtStartPar

&&
\sphinxAtStartPar
name
&
\sphinxAtStartPar
string
&&
\sphinxAtStartPar
name/text()
&&\\
\sphinxhline
\sphinxAtStartPar

&&
\sphinxAtStartPar
head\_of\_state
&
\sphinxAtStartPar
string
&&
\sphinxAtStartPar
head\_of\_state/text()
&&\\
\sphinxhline
\sphinxAtStartPar

&&&
\sphinxAtStartPar
enum
&&
\sphinxAtStartPar
President
&&\\
\sphinxhline
\sphinxAtStartPar

&&&&&
\sphinxAtStartPar
Monarch
&&\\
\sphinxhline
\sphinxAtStartPar

&&&&&
\sphinxAtStartPar
PrimeMinister
&&\\
\sphinxbottomrule
\end{tabulary}
\sphinxtableafterendhook\par
\sphinxattableend\end{savenotes}
\end{quote}


\subparagraph{annotation}
\label{\detokenize{schemos/xsd:annotation}}\label{\detokenize{schemos/xsd:xsd-annotation}}
\sphinxAtStartPar
\sphinxcode{\sphinxupquote{annotation}} viduje būna informacija apie elementą, kurio viduje jis yra. Jo viduje gali būti
elementai {\hyperref[\detokenize{schemos/xsd:xsd-documentation}]{\sphinxcrossref{\DUrole{std}{\DUrole{std-ref}{documentation}}}}} ir {\hyperref[\detokenize{schemos/xsd:xsd-appinfo}]{\sphinxcrossref{\DUrole{std}{\DUrole{std-ref}{union}}}}}. {\hyperref[\detokenize{schemos/xsd:xsd-appinfo}]{\sphinxcrossref{\DUrole{std}{\DUrole{std-ref}{union}}}}} elementas ignoruojamas,
o {\hyperref[\detokenize{schemos/xsd:xsd-documentation}]{\sphinxcrossref{\DUrole{std}{\DUrole{std-ref}{documentation}}}}} viduje esantis tekstas perkeliamas į lauką {\hyperref[\detokenize{dimensijos:property.description}]{\sphinxcrossref{\sphinxcode{\sphinxupquote{property.description}}}}}
arba į {\hyperref[\detokenize{dimensijos:model.description}]{\sphinxcrossref{\sphinxcode{\sphinxupquote{model.description}}}}}, kuris kuriamas iš \DUrole{xref}{\DUrole{std}{\DUrole{std-ref}{element}}} ar \DUrole{xref}{\DUrole{std}{\DUrole{std-ref}{attribute}}}, kurio
viduje \sphinxcode{\sphinxupquote{annotation}} yra.

\begin{sphinxadmonition}{note}{Pavyzdys}

\sphinxAtStartPar
\sphinxstylestrong{Duomenys}

\begin{sphinxVerbatim}[commandchars=\\\{\}]
\PYG{n+nt}{\PYGZlt{}Country}\PYG{+w}{ }\PYG{n+na}{name=}\PYG{l+s}{\PYGZdq{}Lithuania\PYGZdq{}}\PYG{n+nt}{\PYGZgt{}}
\PYG{+w}{    }\PYG{n+nt}{\PYGZlt{}Capital}\PYG{n+nt}{\PYGZgt{}}Vilnius\PYG{n+nt}{\PYGZlt{}/Capital\PYGZgt{}}
\PYG{n+nt}{\PYGZlt{}/Country\PYGZgt{}}
\end{sphinxVerbatim}

\sphinxAtStartPar
\sphinxstylestrong{Schema}

\begin{sphinxVerbatim}[commandchars=\\\{\}]
\PYG{n+nt}{\PYGZlt{}xs:schema}\PYG{+w}{ }\PYG{n+na}{xmlns:xs=}\PYG{l+s}{\PYGZdq{}http://www.w3.org/2001/XMLSchema\PYGZdq{}}\PYG{n+nt}{\PYGZgt{}}

\PYG{+w}{    }\PYG{n+nt}{\PYGZlt{}xs:element}\PYG{+w}{ }\PYG{n+na}{name=}\PYG{l+s}{\PYGZdq{}Country\PYGZdq{}}\PYG{n+nt}{\PYGZgt{}}
\PYG{+w}{        }\PYG{n+nt}{\PYGZlt{}xs:annotation}\PYG{n+nt}{\PYGZgt{}}
\PYG{+w}{            }\PYG{n+nt}{\PYGZlt{}xs:documentation}\PYG{n+nt}{\PYGZgt{}}
\PYG{+w}{                }Represents\PYG{+w}{ }a\PYG{+w}{ }country,\PYG{+w}{ }with\PYG{+w}{ }its\PYG{+w}{ }name\PYG{+w}{ }as\PYG{+w}{ }and\PYG{+w}{ }its\PYG{+w}{ }capital.
\PYG{+w}{            }\PYG{n+nt}{\PYGZlt{}/xs:documentation\PYGZgt{}}
\PYG{+w}{        }\PYG{n+nt}{\PYGZlt{}/xs:annotation\PYGZgt{}}
\PYG{+w}{        }\PYG{n+nt}{\PYGZlt{}xs:complexType}\PYG{n+nt}{\PYGZgt{}}
\PYG{+w}{            }\PYG{n+nt}{\PYGZlt{}xs:sequence}\PYG{n+nt}{\PYGZgt{}}
\PYG{+w}{                }\PYG{n+nt}{\PYGZlt{}xs:element}\PYG{+w}{ }\PYG{n+na}{name=}\PYG{l+s}{\PYGZdq{}Capital\PYGZdq{}}\PYG{+w}{ }\PYG{n+na}{type=}\PYG{l+s}{\PYGZdq{}xs:string\PYGZdq{}}\PYG{n+nt}{\PYGZgt{}}
\PYG{+w}{                    }\PYG{n+nt}{\PYGZlt{}xs:annotation}\PYG{n+nt}{\PYGZgt{}}
\PYG{+w}{                        }\PYG{n+nt}{\PYGZlt{}xs:documentation}\PYG{n+nt}{\PYGZgt{}}
\PYG{+w}{                            }Represents\PYG{+w}{ }the\PYG{+w}{ }capital\PYG{+w}{ }city\PYG{+w}{ }of\PYG{+w}{ }the\PYG{+w}{ }country.
\PYG{+w}{                        }\PYG{n+nt}{\PYGZlt{}/xs:documentation\PYGZgt{}}
\PYG{+w}{                    }\PYG{n+nt}{\PYGZlt{}/xs:annotation\PYGZgt{}}
\PYG{+w}{                }\PYG{n+nt}{\PYGZlt{}/xs:element\PYGZgt{}}
\PYG{+w}{            }\PYG{n+nt}{\PYGZlt{}/xs:sequence\PYGZgt{}}
\PYG{+w}{            }\PYG{n+nt}{\PYGZlt{}xs:attribute}\PYG{+w}{ }\PYG{n+na}{name=}\PYG{l+s}{\PYGZdq{}name\PYGZdq{}}\PYG{+w}{ }\PYG{n+na}{type=}\PYG{l+s}{\PYGZdq{}xs:string\PYGZdq{}}\PYG{+w}{ }\PYG{n+na}{use=}\PYG{l+s}{\PYGZdq{}required\PYGZdq{}}\PYG{n+nt}{\PYGZgt{}}
\PYG{+w}{                }\PYG{n+nt}{\PYGZlt{}xs:annotation}\PYG{n+nt}{\PYGZgt{}}
\PYG{+w}{                    }\PYG{n+nt}{\PYGZlt{}xs:documentation}\PYG{n+nt}{\PYGZgt{}}
\PYG{+w}{                        }Specifies\PYG{+w}{ }the\PYG{+w}{ }name\PYG{+w}{ }of\PYG{+w}{ }the\PYG{+w}{ }country.
\PYG{+w}{                    }\PYG{n+nt}{\PYGZlt{}/xs:documentation\PYGZgt{}}
\PYG{+w}{                }\PYG{n+nt}{\PYGZlt{}/xs:annotation\PYGZgt{}}
\PYG{+w}{            }\PYG{n+nt}{\PYGZlt{}/xs:attribute\PYGZgt{}}
\PYG{+w}{        }\PYG{n+nt}{\PYGZlt{}/xs:complexType\PYGZgt{}}
\PYG{+w}{    }\PYG{n+nt}{\PYGZlt{}/xs:element\PYGZgt{}}

\PYG{n+nt}{\PYGZlt{}/xs:schema\PYGZgt{}}
\end{sphinxVerbatim}

\sphinxAtStartPar
\sphinxstylestrong{Struktūros aprašas}
\end{sphinxadmonition}


\subparagraph{documentation}
\label{\detokenize{schemos/xsd:documentation}}\label{\detokenize{schemos/xsd:xsd-documentation}}
\sphinxAtStartPar
\sphinxcode{\sphinxupquote{documentation}} elementas visada būna viduje {\hyperref[\detokenize{schemos/xsd:xsd-annotation}]{\sphinxcrossref{\DUrole{std}{\DUrole{std-ref}{annotation}}}}} elemento ir iš jų abiejų kartu
formuojamas aprašymas \sphinxhyphen{} {\hyperref[\detokenize{dimensijos:model.description}]{\sphinxcrossref{\sphinxcode{\sphinxupquote{model.description}}}}} arba {\hyperref[\detokenize{dimensijos:model.description}]{\sphinxcrossref{\sphinxcode{\sphinxupquote{model.description}}}}}. Daugiau
informacijos prie {\hyperref[\detokenize{schemos/xsd:xsd-annotation}]{\sphinxcrossref{\DUrole{std}{\DUrole{std-ref}{annotation}}}}}


\subparagraph{restriction}
\label{\detokenize{schemos/xsd:restriction}}\label{\detokenize{schemos/xsd:xsd-restriction}}
\sphinxAtStartPar
\sphinxcode{\sphinxupquote{restriction}} yra taikomas galimų duomenų reikšmių ribojimui, kaip pavyzdžiui, minimalioms ar
maksimalioms reikšmėms, teksto ilgio ribojimams. DSA šie ribojimai neaprašomi, taigi dauguma šių
žymių ignoruojama.

\sphinxAtStartPar
Vienintelis atvejis, kai \sphinxcode{\sphinxupquote{restriction}} perkeliamas į DSA, yra, kai jis naudojamas kartu
su {\hyperref[\detokenize{schemos/xsd:xsd-enumeration}]{\sphinxcrossref{\DUrole{std}{\DUrole{std-ref}{enumeration}}}}}. Tai naudojama aprašyti išvardijamoms galimoms reikšmėms. Daugiau
aprašyta prie {\hyperref[\detokenize{schemos/xsd:xsd-enumeration}]{\sphinxcrossref{\DUrole{std}{\DUrole{std-ref}{enumeration}}}}}.


\subparagraph{extension}
\label{\detokenize{schemos/xsd:extension}}\label{\detokenize{schemos/xsd:xsd-extension}}
\sphinxAtStartPar
\sphinxcode{\sphinxupquote{extension}} mazgas visada eina viduje \DUrole{xref}{\DUrole{std}{\DUrole{std-ref}{xsd\_simple\_type}}} arba \DUrole{xref}{\DUrole{std}{\DUrole{std-ref}{xsd\_complex\_type}}}.
Kaip jis veikia šiuose mazguose, aprašyta prie jų.


\subparagraph{union}
\label{\detokenize{schemos/xsd:union}}\label{\detokenize{schemos/xsd:xsd-union}}

\subparagraph{union}
\label{\detokenize{schemos/xsd:xsd-appinfo}}\label{\detokenize{schemos/xsd:id15}}

\paragraph{Atributai}
\label{\detokenize{schemos/xsd:atributai}}

\subparagraph{base}
\label{\detokenize{schemos/xsd:base}}\label{\detokenize{schemos/xsd:xsd-base}}
\sphinxAtStartPar
\sphinxcode{\sphinxupquote{base}} naudojamas viduje {\hyperref[\detokenize{schemos/xsd:xsd-extension}]{\sphinxcrossref{\DUrole{std}{\DUrole{std-ref}{extension}}}}} arba {\hyperref[\detokenize{schemos/xsd:xsd-restriction}]{\sphinxcrossref{\DUrole{std}{\DUrole{std-ref}{restriction}}}}}, kai norima išplėsti arba
susiaurinti tam tikro tipo reikšmes. Daugiau apie \sphinxcode{\sphinxupquote{base}} naudojimą aprašyta prie
{\hyperref[\detokenize{schemos/xsd:xsd-complex-content}]{\sphinxcrossref{\DUrole{std}{\DUrole{std-ref}{complexContent}}}}} ir prie {\hyperref[\detokenize{schemos/xsd:xsd-simple-content}]{\sphinxcrossref{\DUrole{std}{\DUrole{std-ref}{simpleContent}}}}}.


\subparagraph{unique}
\label{\detokenize{schemos/xsd:unique}}\label{\detokenize{schemos/xsd:xsd-unique}}

\subparagraph{minOccurs}
\label{\detokenize{schemos/xsd:minoccurs}}\label{\detokenize{schemos/xsd:xsd-minoccurs}}
\sphinxAtStartPar
\sphinxcode{\sphinxupquote{minOccurs}} naudojamas nurodyti elemento minimalų pasikartojimų skaičių. Pagal šį atributą taip pat
galim nustatyti, ar elementas privalomas, ar ne. Jei \sphinxcode{\sphinxupquote{minOccurs}} yra lygus \sphinxcode{\sphinxupquote{0}}, tai elementas
neprivalomas, o jei didesnis nei 0, elementas privalomas. DSA privalomumas nurodomas prie
\sphinxcode{\sphinxupquote{property.name}} pridedant \sphinxcode{\sphinxupquote{required}} jei ji yra privaloma, arbe nepridedant nieko, jei neprivaloma.

\sphinxAtStartPar
Jei {\hyperref[\detokenize{schemos/xsd:xsd-element}]{\sphinxcrossref{\DUrole{std}{\DUrole{std-ref}{element}}}}} turi atributą \sphinxcode{\sphinxupquote{minOccurs}}, kurio reikšmė lugi '0', reiškia iš šio elemento
sukurta savybė yra neprivaloma, ir žymė \sphinxcode{\sphinxupquote{required}} nepridedama, o jei \sphinxcode{\sphinxupquote{minOccurs}} atributo reikšmė
yra \sphinxcode{\sphinxupquote{1}} arba didesnė, arba jei šis atributas visai nenurodytas (pagal nutylėjimą jo reikšmė lygi \sphinxcode{\sphinxupquote{1}}),
reiškia, kad iš jo sukurta savybė yra privaloma ir prie jos pavadinimo pridedama žymė \sphinxcode{\sphinxupquote{required}}.


\subparagraph{maxOccurs}
\label{\detokenize{schemos/xsd:maxoccurs}}\label{\detokenize{schemos/xsd:xsd-maxoccurs}}
\sphinxAtStartPar
\sphinxcode{\sphinxupquote{maxOccurs}} žymi, kiek daugiausiai kartų elementas gali pasikartoti. Jei ši reikšmė yra 1, arba jei
šis atributas iš viso nenurodytas (jo numatytoji reikšmė yra 1), reiškia elementas gali būti tik
vieną kartą. Tokiu atveju, DSA tai yra įprasta, vieną objektą ar savybę žyminti {\hyperref[\detokenize{formatas:property}]{\sphinxcrossref{\sphinxcode{\sphinxupquote{property}}}}}.
Jei \sphinxcode{\sphinxupquote{maxOccurs}} reikšmė yra daugiau nei \sphinxcode{\sphinxupquote{1}} arba \sphinxcode{\sphinxupquote{unbounded}}, tai reiškia, kad elementas gali
pasikartoti daug kartų, tai iš jo padaryta {\hyperref[\detokenize{formatas:property}]{\sphinxcrossref{\sphinxcode{\sphinxupquote{property}}}}} bus masyvas, ir prie jos pavadinimo
bus pridėti laužtiniai skliaustai (\sphinxcode{\sphinxupquote{{[}{]}}}), o jei tai yra į kitą {\hyperref[\detokenize{formatas:model}]{\sphinxcrossref{\sphinxcode{\sphinxupquote{model}}}}} rodanti savybė, tai
jos tipas bus ne {\hyperref[\detokenize{formatas:ref}]{\sphinxcrossref{\sphinxcode{\sphinxupquote{ref}}}}}, bet \sphinxcode{\sphinxupquote{backref}}.


\subparagraph{nillable}
\label{\detokenize{schemos/xsd:nillable}}\label{\detokenize{schemos/xsd:xsd-nillable}}

\subparagraph{type}
\label{\detokenize{schemos/xsd:type}}\label{\detokenize{schemos/xsd:xsd-type}}
\sphinxAtStartPar
XSD tipas gali būti nurodytas pačiame elemente, nurodant atributą \sphinxcode{\sphinxupquote{type}}, arba aprašytas po jo
einančiame \DUrole{xref}{\DUrole{std}{\DUrole{std-ref}{simpleType}}} arba \DUrole{xref}{\DUrole{std}{\DUrole{std-ref}{complexType}}}.

\sphinxAtStartPar
Jei tipas aprašytas pačiame elemente ar atribute, į DSA {\hyperref[\detokenize{dimensijos:property.type}]{\sphinxcrossref{\sphinxcode{\sphinxupquote{property.type}}}}} jis konvertuojamas naudojant
konvertavimo lentelę \DUrole{xref}{\DUrole{std}{\DUrole{std-ref}{xsd\_type\_conversion}}}.

\sphinxAtStartPar
Taip pat elemento tipas gali būti aprašytas naudojant \DUrole{xref}{\DUrole{std}{\DUrole{std-ref}{simpleType}}} ir \DUrole{xref}{\DUrole{std}{\DUrole{std-ref}{complexType}}}.


\subparagraph{use}
\label{\detokenize{schemos/xsd:use}}\label{\detokenize{schemos/xsd:xsd-use}}
\sphinxAtStartPar
\sphinxcode{\sphinxupquote{use}} naudojamas aprašant {\hyperref[\detokenize{schemos/xsd:xsd-attribute}]{\sphinxcrossref{\DUrole{std}{\DUrole{std-ref}{attribute}}}}}, ir nurodo, ar elementas yra privalomas,
ar ne. Jei \sphinxcode{\sphinxupquote{use}} nenurodytas, naudojama jo numatytoji reikšmė, kuri yra \sphinxcode{\sphinxupquote{optional}},
ir tai reiškia, kad {\hyperref[\detokenize{schemos/xsd:xsd-attribute}]{\sphinxcrossref{\DUrole{std}{\DUrole{std-ref}{attribute}}}}} nėra privalomas, taigi DSA jis taip pat nežymimas
kaip privalomas. Jei \sphinxcode{\sphinxupquote{use}} reikšmė yra "required", reiškia, kad šis {\hyperref[\detokenize{schemos/xsd:xsd-attribute}]{\sphinxcrossref{\DUrole{std}{\DUrole{std-ref}{attribute}}}}} yra
privalomas, ir DSA prie jo pavadinimo pridedama žymė \sphinxcode{\sphinxupquote{required}}.


\paragraph{Duomenų tipai}
\label{\detokenize{schemos/xsd:duomenu-tipai}}

\begin{savenotes}
\sphinxatlongtablestart
\sphinxthistablewithglobalstyle
\makeatletter
  \LTleft \@totalleftmargin plus1fill
  \LTright\dimexpr\columnwidth-\@totalleftmargin-\linewidth\relax plus1fill
\makeatother
\begin{longtable}{ll}
\noalign{\phantomsection\label{\detokenize{schemos/xsd:xsd-type-conversion}}}%
\sphinxtoprule
\sphinxstyletheadfamily 
\sphinxAtStartPar
XSD tipas (type)
&\sphinxstyletheadfamily 
\sphinxAtStartPar
DSA tipas (type)
\\
\sphinxmidrule
\endfirsthead

\multicolumn{2}{c}{\sphinxnorowcolor
    \makebox[0pt]{\sphinxtablecontinued{\tablename\ \thetable{} \textendash{} tęsinys iš praeito puslapio}}%
}\\
\sphinxtoprule
\sphinxstyletheadfamily 
\sphinxAtStartPar
XSD tipas (type)
&\sphinxstyletheadfamily 
\sphinxAtStartPar
DSA tipas (type)
\\
\sphinxmidrule
\endhead

\sphinxbottomrule
\multicolumn{2}{r}{\sphinxnorowcolor
    \makebox[0pt][r]{\sphinxtablecontinued{continues on next page}}%
}\\
\endfoot

\endlastfoot
\sphinxtableatstartofbodyhook

\sphinxAtStartPar
string
&
\sphinxAtStartPar
string
\\
\sphinxhline
\sphinxAtStartPar
boolean
&
\sphinxAtStartPar
boolean
\\
\sphinxhline
\sphinxAtStartPar
decimal
&
\sphinxAtStartPar
number
\\
\sphinxhline
\sphinxAtStartPar
float
&
\sphinxAtStartPar
number
\\
\sphinxhline
\sphinxAtStartPar
double
&
\sphinxAtStartPar
number
\\
\sphinxhline
\sphinxAtStartPar
duration
&
\sphinxAtStartPar
string
\\
\sphinxhline
\sphinxAtStartPar
dateTime
&
\sphinxAtStartPar
datetime
\\
\sphinxhline
\sphinxAtStartPar
time
&
\sphinxAtStartPar
time
\\
\sphinxhline
\sphinxAtStartPar
date
&
\sphinxAtStartPar
date
\\
\sphinxhline
\sphinxAtStartPar
gYearMonth
&
\sphinxAtStartPar
date;enum;M
\\
\sphinxhline
\sphinxAtStartPar
gYear
&
\sphinxAtStartPar
date;enum;Y
\\
\sphinxhline
\sphinxAtStartPar
gMonthDay
&
\sphinxAtStartPar
string
\\
\sphinxhline
\sphinxAtStartPar
gDay
&
\sphinxAtStartPar
string
\\
\sphinxhline
\sphinxAtStartPar
gMonth
&
\sphinxAtStartPar
string
\\
\sphinxhline
\sphinxAtStartPar
hexBinary
&
\sphinxAtStartPar
string
\\
\sphinxhline
\sphinxAtStartPar
base64Binary
&
\sphinxAtStartPar
binary;prepare;base64
\\
\sphinxhline
\sphinxAtStartPar
anyURI
&
\sphinxAtStartPar
uri
\\
\sphinxhline
\sphinxAtStartPar
QName
&
\sphinxAtStartPar
string
\\
\sphinxhline
\sphinxAtStartPar
NOTATION
&
\sphinxAtStartPar
string
\\
\sphinxhline
\sphinxAtStartPar
normalizedString
&
\sphinxAtStartPar
string
\\
\sphinxhline
\sphinxAtStartPar
token
&
\sphinxAtStartPar
string
\\
\sphinxhline
\sphinxAtStartPar
language
&
\sphinxAtStartPar
string
\\
\sphinxhline
\sphinxAtStartPar
NMTOKEN
&
\sphinxAtStartPar
string
\\
\sphinxhline
\sphinxAtStartPar
NMTOKENS
&
\sphinxAtStartPar
string
\\
\sphinxhline
\sphinxAtStartPar
Name
&
\sphinxAtStartPar
string
\\
\sphinxhline
\sphinxAtStartPar
NCName
&
\sphinxAtStartPar
string
\\
\sphinxhline
\sphinxAtStartPar
ID
&
\sphinxAtStartPar
string
\\
\sphinxhline
\sphinxAtStartPar
IDREF
&
\sphinxAtStartPar
string
\\
\sphinxhline
\sphinxAtStartPar
IDREFS
&
\sphinxAtStartPar
string
\\
\sphinxhline
\sphinxAtStartPar
ENTITY
&
\sphinxAtStartPar
string
\\
\sphinxhline
\sphinxAtStartPar
ENTITIES
&
\sphinxAtStartPar
string
\\
\sphinxhline
\sphinxAtStartPar
integer
&
\sphinxAtStartPar
integer
\\
\sphinxhline
\sphinxAtStartPar
nonPositiveInteger
&
\sphinxAtStartPar
integer
\\
\sphinxhline
\sphinxAtStartPar
negativeInteger
&
\sphinxAtStartPar
integer
\\
\sphinxhline
\sphinxAtStartPar
long
&
\sphinxAtStartPar
integer
\\
\sphinxhline
\sphinxAtStartPar
int
&
\sphinxAtStartPar
integer
\\
\sphinxhline
\sphinxAtStartPar
short
&
\sphinxAtStartPar
integer
\\
\sphinxhline
\sphinxAtStartPar
byte
&
\sphinxAtStartPar
integer
\\
\sphinxhline
\sphinxAtStartPar
nonNegativeInteger
&
\sphinxAtStartPar
integer
\\
\sphinxhline
\sphinxAtStartPar
unsignedLong
&
\sphinxAtStartPar
integer
\\
\sphinxhline
\sphinxAtStartPar
unsignedInt
&
\sphinxAtStartPar
integer
\\
\sphinxhline
\sphinxAtStartPar
unsignedShort
&
\sphinxAtStartPar
integer
\\
\sphinxhline
\sphinxAtStartPar
unsignedByte
&
\sphinxAtStartPar
integer
\\
\sphinxhline
\sphinxAtStartPar
positiveInteger
&
\sphinxAtStartPar
integer
\\
\sphinxhline
\sphinxAtStartPar
yearMonthDuration
&
\sphinxAtStartPar
integer
\\
\sphinxhline
\sphinxAtStartPar
dayTimeDuration
&
\sphinxAtStartPar
integer
\\
\sphinxhline
\sphinxAtStartPar
dateTimeStamp
&
\sphinxAtStartPar
datetime
\\
\sphinxhline&
\sphinxAtStartPar
string
\\
\sphinxbottomrule
\end{longtable}
\sphinxtableafterendhook
\sphinxatlongtableend
\end{savenotes}

\sphinxstepscope


\section{Duomenų šaltiniai}
\label{\detokenize{saltiniai:duomenu-saltiniai}}\label{\detokenize{saltiniai:id1}}\label{\detokenize{saltiniai::doc}}

\subsection{SQL}
\label{\detokenize{saltiniai:sql}}\label{\detokenize{saltiniai:resource-type-sql}}\phantomsection\label{\detokenize{saltiniai:sql-resource-source}}

\begin{fulllineitems}

\pysigstartsignatures
\pysigline
{\sphinxbfcode{\sphinxupquote{resource.source}}}
\pysigstopsignatures
\sphinxAtStartPar
Duomenų bazės URI. Duomenų bazės URI formuojamas naudojant tokį \sphinxhref{https://en.wikipedia.org/wiki/Augmented\_Backus–Naur\_form}{ABNF}
šabloną:

\begin{sphinxVerbatim}[commandchars=\\\{\}]
\PYG{n+nc}{uri}\PYG{+w}{ }\PYG{o}{=}\PYG{+w}{ }\PYG{n+nc}{type}\PYG{+w}{ }\PYG{p}{[}\PYG{l}{\PYGZdq{}+\PYGZdq{}}\PYG{+w}{ }\PYG{n+nc}{driver}\PYG{p}{]}\PYG{+w}{ }\PYG{l}{\PYGZdq{}://\PYGZdq{}}
\PYG{+w}{      }\PYG{p}{[}\PYG{n+nc}{user}\PYG{+w}{ }\PYG{p}{[}\PYG{l}{\PYGZdq{}:\PYGZdq{}}\PYG{+w}{ }\PYG{n+nc}{password}\PYG{p}{]}\PYG{+w}{ }\PYG{l}{\PYGZdq{}@\PYGZdq{}}\PYG{p}{]}
\PYG{+w}{      }\PYG{n+nc}{host}\PYG{+w}{ }\PYG{p}{[}\PYG{l}{\PYGZdq{}:\PYGZdq{}}\PYG{+w}{ }\PYG{n+nc}{port}\PYG{p}{]}
\PYG{+w}{      }\PYG{l}{\PYGZdq{}/\PYGZdq{}}\PYG{+w}{ }\PYG{n+nc}{database}\PYG{+w}{ }\PYG{p}{[}\PYG{l}{\PYGZdq{}?\PYGZdq{}}\PYG{+w}{ }\PYG{n+nc}{params}\PYG{p}{]}
\end{sphinxVerbatim}

\sphinxAtStartPar
Šablone naudojamų kintamųjų aprašymas:


\begin{fulllineitems}

\pysigstartsignatures
\pysigline
{\sphinxbfcode{\sphinxupquote{type}}}
\pysigstopsignatures
\sphinxAtStartPar
Duomenų bazių serverio pavadinimas:


\begin{fulllineitems}

\pysigstartsignatures
\pysigline
{\sphinxbfcode{\sphinxupquote{sqlite}}}
\pysigstopsignatures
\end{fulllineitems}



\begin{fulllineitems}

\pysigstartsignatures
\pysigline
{\sphinxbfcode{\sphinxupquote{postgresql}}}
\pysigstopsignatures
\end{fulllineitems}



\begin{fulllineitems}

\pysigstartsignatures
\pysigline
{\sphinxbfcode{\sphinxupquote{mysql}}}
\pysigstopsignatures
\end{fulllineitems}



\begin{fulllineitems}

\pysigstartsignatures
\pysigline
{\sphinxbfcode{\sphinxupquote{oracle}}}
\pysigstopsignatures
\end{fulllineitems}



\begin{fulllineitems}

\pysigstartsignatures
\pysigline
{\sphinxbfcode{\sphinxupquote{mssql}}}
\pysigstopsignatures
\end{fulllineitems}


\end{fulllineitems}



\begin{fulllineitems}

\pysigstartsignatures
\pysigline
{\sphinxbfcode{\sphinxupquote{driver}}}
\pysigstopsignatures
\sphinxAtStartPar
Konkretaus duomenų bazių serverio tvarkyklė naudojama komunikacijai su
duomenų baze.

\end{fulllineitems}



\begin{fulllineitems}

\pysigstartsignatures
\pysigline
{\sphinxbfcode{\sphinxupquote{user}}}
\pysigstopsignatures
\sphinxAtStartPar
Naudotojo vardas jungimuisi prie duomenų bazės.

\end{fulllineitems}



\begin{fulllineitems}

\pysigstartsignatures
\pysigline
{\sphinxbfcode{\sphinxupquote{password}}}
\pysigstopsignatures
\sphinxAtStartPar
Duomenų bazės naudotojo slaptažodis.

\end{fulllineitems}



\begin{fulllineitems}

\pysigstartsignatures
\pysigline
{\sphinxbfcode{\sphinxupquote{host}}}
\pysigstopsignatures
\sphinxAtStartPar
Duomenų bazių serverio adresas.

\end{fulllineitems}



\begin{fulllineitems}

\pysigstartsignatures
\pysigline
{\sphinxbfcode{\sphinxupquote{port}}}
\pysigstopsignatures
\sphinxAtStartPar
Duomenų bazių serverio prievadas.

\end{fulllineitems}



\begin{fulllineitems}

\pysigstartsignatures
\pysigline
{\sphinxbfcode{\sphinxupquote{database}}}
\pysigstopsignatures
\sphinxAtStartPar
Konkrečios duomenų bazės pavadinimas.

\end{fulllineitems}



\begin{fulllineitems}

\pysigstartsignatures
\pysigline
{\sphinxbfcode{\sphinxupquote{params}}}
\pysigstopsignatures
\sphinxAtStartPar
Papildomi parametrai \sphinxcode{\sphinxupquote{Query string}} formatu.

\end{fulllineitems}


\end{fulllineitems}



\begin{fulllineitems}

\pysigstartsignatures
\pysigline
{\sphinxbfcode{\sphinxupquote{resource.prepare}}}
\pysigstopsignatures
\sphinxAtStartPar
Formulė skirta papildomiems veiksmams reikalingiems ryšiui su duomenų baze
užmegzti ir duomenų bazės paruošimui, kad būtų galima skaityt duomenis.

\end{fulllineitems}



\begin{fulllineitems}

\pysigstartsignatures
\pysigline
{\sphinxbfcode{\sphinxupquote{resource.type}}}
\pysigstopsignatures
\sphinxAtStartPar
Galimos reikšmės: \sphinxcode{\sphinxupquote{sql}}.

\end{fulllineitems}



\begin{fulllineitems}

\pysigstartsignatures
\pysigline
{\sphinxbfcode{\sphinxupquote{resource.prepare}}}
\pysigstopsignatures\index{built\sphinxhyphen{}in function@\spxentry{built\sphinxhyphen{}in function}!connect()@\spxentry{connect()}}\index{connect()@\spxentry{connect()}!built\sphinxhyphen{}in function@\spxentry{built\sphinxhyphen{}in function}}

\begin{fulllineitems}
\phantomsection\label{\detokenize{saltiniai:connect}}
\pysigstartsignatures
\pysiglinewithargsret
{\sphinxbfcode{\sphinxupquote{connect}}}
{\sphinxparam{\DUrole{n}{dsn}}\sphinxparamcomma \sphinxparam{\DUrole{n}{schema}\DUrole{p}{:}\DUrole{w}{ }\DUrole{n}{str}\DUrole{w}{ }\DUrole{o}{=}\DUrole{w}{ }\DUrole{default_value}{None}}\sphinxparamcomma \sphinxparam{\DUrole{n}{encoding}\DUrole{p}{:}\DUrole{w}{ }\DUrole{n}{str}\DUrole{w}{ }\DUrole{o}{=}\DUrole{w}{ }\DUrole{default_value}{'utf\sphinxhyphen{}8'}}}
{}
\pysigstopsignatures\begin{quote}\begin{description}
\sphinxlineitem{Parametrai}\begin{itemize}
\item {} 
\sphinxAtStartPar
\sphinxstyleliteralstrong{\sphinxupquote{dsn}} \sphinxhyphen{}\sphinxhyphen{} Duomenų bazės URI, kaip nurodyta {\hyperref[\detokenize{saltiniai:sql-resource-source}]{\sphinxcrossref{\DUrole{std}{\DUrole{std-ref}{resource.source}}}}}.

\item {} 
\sphinxAtStartPar
\sphinxstyleliteralstrong{\sphinxupquote{schema}} \sphinxhyphen{}\sphinxhyphen{} Duomenų bazės schema.

\item {} 
\sphinxAtStartPar
\sphinxstyleliteralstrong{\sphinxupquote{encoding}} \sphinxhyphen{}\sphinxhyphen{} Duomenų bazės koduotė.

\end{itemize}

\end{description}\end{quote}

\sphinxAtStartPar
Naudojama tais atvejais, kai jungiantis prie duomenų bazės reikia
perduoti papildomus parametrus.

\end{fulllineitems}


\end{fulllineitems}



\begin{fulllineitems}

\pysigstartsignatures
\pysigline
{\sphinxbfcode{\sphinxupquote{model.source}}}
\pysigstopsignatures
\sphinxAtStartPar
Duomenų bazėje esančios lentelės pavadinimas.

\end{fulllineitems}



\begin{fulllineitems}

\pysigstartsignatures
\pysigline
{\sphinxbfcode{\sphinxupquote{property.source}}}
\pysigstopsignatures
\sphinxAtStartPar
Lentelės stulpelio pavadinimas.

\end{fulllineitems}



\subsection{CSV}
\label{\detokenize{saltiniai:csv}}

\begin{fulllineitems}

\pysigstartsignatures
\pysigline
{\sphinxbfcode{\sphinxupquote{resource.type}}}
\pysigstopsignatures
\sphinxAtStartPar
Galimos reikšmės: \sphinxcode{\sphinxupquote{csv}}, \sphinxcode{\sphinxupquote{tsv}}.

\end{fulllineitems}



\begin{fulllineitems}

\pysigstartsignatures
\pysigline
{\sphinxbfcode{\sphinxupquote{resource.source}}}
\pysigstopsignatures
\sphinxAtStartPar
Žiūrėti {\hyperref[\detokenize{formules:failai}]{\sphinxcrossref{\DUrole{std}{\DUrole{std-ref}{Failai}}}}}.

\end{fulllineitems}



\begin{fulllineitems}

\pysigstartsignatures
\pysigline
{\sphinxbfcode{\sphinxupquote{resource.prepare}}}
\pysigstopsignatures\index{built\sphinxhyphen{}in function@\spxentry{built\sphinxhyphen{}in function}!tabular()@\spxentry{tabular()}}\index{tabular()@\spxentry{tabular()}!built\sphinxhyphen{}in function@\spxentry{built\sphinxhyphen{}in function}}

\begin{fulllineitems}
\phantomsection\label{\detokenize{saltiniai:tabular}}
\pysigstartsignatures
\pysiglinewithargsret
{\sphinxbfcode{\sphinxupquote{tabular}}}
{\sphinxparam{\DUrole{n}{sep}\DUrole{p}{:}\DUrole{w}{ }\DUrole{n}{\DUrole{s}{','}}}}
{}
\pysigstopsignatures
\sphinxAtStartPar
Nurodoma kaip CSV faile atskirti stulpeliai. Pagal nutylėjimą
\sphinxcode{\sphinxupquote{separator}} reikšmė yra \sphinxcode{\sphinxupquote{,}}.

\end{fulllineitems}


\end{fulllineitems}



\begin{fulllineitems}

\pysigstartsignatures
\pysigline
{\sphinxbfcode{\sphinxupquote{model.source}}}
\pysigstopsignatures
\sphinxAtStartPar
Nenaudojama, kadangi CSV resursas gali turėti tik vieną lentelę.

\end{fulllineitems}



\begin{fulllineitems}

\pysigstartsignatures
\pysigline
{\sphinxbfcode{\sphinxupquote{model.prepare}}}
\pysigstopsignatures
\sphinxAtStartPar
Žiūrėti {\hyperref[\detokenize{formules:stulpeliai-lenteleje}]{\sphinxcrossref{\DUrole{std}{\DUrole{std-ref}{Stulpeliai lentelėje}}}}}.

\end{fulllineitems}



\begin{fulllineitems}

\pysigstartsignatures
\pysigline
{\sphinxbfcode{\sphinxupquote{property.source}}}
\pysigstopsignatures
\sphinxAtStartPar
Žiūrėti {\hyperref[\detokenize{formules:stulpeliai-lenteleje}]{\sphinxcrossref{\DUrole{std}{\DUrole{std-ref}{Stulpeliai lentelėje}}}}}.

\end{fulllineitems}



\subsection{JSON}
\label{\detokenize{saltiniai:json}}

\begin{fulllineitems}

\pysigstartsignatures
\pysigline
{\sphinxbfcode{\sphinxupquote{resource.type}}}
\pysigstopsignatures
\sphinxAtStartPar
Galimos reikšmės: \sphinxcode{\sphinxupquote{json}}, \sphinxcode{\sphinxupquote{jsonl}}.

\end{fulllineitems}



\begin{fulllineitems}

\pysigstartsignatures
\pysigline
{\sphinxbfcode{\sphinxupquote{resource.source}}}
\pysigstopsignatures
\sphinxAtStartPar
Žiūrėti {\hyperref[\detokenize{formules:failai}]{\sphinxcrossref{\DUrole{std}{\DUrole{std-ref}{Failai}}}}}.

\end{fulllineitems}



\begin{fulllineitems}

\pysigstartsignatures
\pysigline
{\sphinxbfcode{\sphinxupquote{model.source}}}
\pysigstopsignatures
\sphinxAtStartPar
JSON objekto savybės pavadinimas, kuri rodo į masyvą reikšmių, kurios bus
naudojamos kaip modelio duomenų eilutės. Kiekvienas masyvo elementas
atskirai aprašomas {\hyperref[\detokenize{formatas:property}]{\sphinxcrossref{\sphinxcode{\sphinxupquote{property}}}}} dimensijoje. Jei JSON objektas yra
kompleksinis žiūrėti {\hyperref[\detokenize{formules:kompleksines-strukturos}]{\sphinxcrossref{\DUrole{std}{\DUrole{std-ref}{Kompleksinės struktūros}}}}}.

\end{fulllineitems}



\begin{fulllineitems}

\pysigstartsignatures
\pysigline
{\sphinxbfcode{\sphinxupquote{property.source}}}
\pysigstopsignatures
\sphinxAtStartPar
JSON objekto savybė, kurioje pateikiami aprašomo stulpelio duomenys.

\end{fulllineitems}



\begin{fulllineitems}

\pysigstartsignatures
\pysigline
{\sphinxbfcode{\sphinxupquote{property.prepare}}}
\pysigstopsignatures
\sphinxAtStartPar
Žiūrėti {\hyperref[\detokenize{formules:kompleksines-strukturos}]{\sphinxcrossref{\DUrole{std}{\DUrole{std-ref}{Kompleksinės struktūros}}}}}.

\end{fulllineitems}



\subsection{XML}
\label{\detokenize{saltiniai:xml}}

\begin{fulllineitems}

\pysigstartsignatures
\pysigline
{\sphinxbfcode{\sphinxupquote{resource.type}}}
\pysigstopsignatures
\sphinxAtStartPar
Galimos reikšmės: \sphinxcode{\sphinxupquote{xml}}, \sphinxcode{\sphinxupquote{html}}.

\end{fulllineitems}



\begin{fulllineitems}

\pysigstartsignatures
\pysigline
{\sphinxbfcode{\sphinxupquote{resource.source}}}
\pysigstopsignatures
\sphinxAtStartPar
Žiūrėti {\hyperref[\detokenize{formules:failai}]{\sphinxcrossref{\DUrole{std}{\DUrole{std-ref}{Failai}}}}}.

\end{fulllineitems}



\begin{fulllineitems}

\pysigstartsignatures
\pysigline
{\sphinxbfcode{\sphinxupquote{model.source}}}
\pysigstopsignatures
\sphinxAtStartPar
\sphinxhref{https://en.wikipedia.org/wiki/XPath}{XPath} iki elementų sąrašo kuriame
yra modelio duomenys.

\end{fulllineitems}



\begin{fulllineitems}

\pysigstartsignatures
\pysigline
{\sphinxbfcode{\sphinxupquote{model.prepare}}}
\pysigstopsignatures
\sphinxAtStartPar
Jei neužpildyta, vykdoma \sphinxcode{\sphinxupquote{xpath(self)}} funkcija.
\index{built\sphinxhyphen{}in function@\spxentry{built\sphinxhyphen{}in function}!xpath()@\spxentry{xpath()}}\index{xpath()@\spxentry{xpath()}!built\sphinxhyphen{}in function@\spxentry{built\sphinxhyphen{}in function}}

\begin{fulllineitems}
\phantomsection\label{\detokenize{saltiniai:id0}}
\pysigstartsignatures
\pysiglinewithargsret
{\sphinxbfcode{\sphinxupquote{xpath}}}
{\sphinxparam{\DUrole{n}{expr}}}
{}
\pysigstopsignatures
\sphinxAtStartPar
Vykdo nurodyta \sphinxcode{\sphinxupquote{expr}}, viso XML dokumento kontekste.

\end{fulllineitems}


\end{fulllineitems}



\begin{fulllineitems}

\pysigstartsignatures
\pysigline
{\sphinxbfcode{\sphinxupquote{property.source}}}
\pysigstopsignatures
\sphinxAtStartPar
\sphinxhref{https://en.wikipedia.org/wiki/XPath}{XPath} iki elemento kuriame yra
duomenys.

\sphinxAtStartPar
XPath nurodomas reliatyvus modeliui, arba kitai daugiareikšmei savybei,
kurios sudėtyje savybė yra. Daugiareikšmės savybės žymimos \sphinxcode{\sphinxupquote{{[}{]}}} simboliais
savybės kodiniame pavadinime, įprastai tai yra \sphinxcode{\sphinxupquote{array}} tipo savybės.

\end{fulllineitems}



\begin{fulllineitems}

\pysigstartsignatures
\pysigline
{\sphinxbfcode{\sphinxupquote{model.prepare}}}
\pysigstopsignatures
\sphinxAtStartPar
Jei neužpildyta, vykdoma \sphinxcode{\sphinxupquote{xpath(self)}} funkcija, iš
{\hyperref[\detokenize{formatas:model}]{\sphinxcrossref{\sphinxcode{\sphinxupquote{model}}}}} gauto elemento kontekste.

\end{fulllineitems}


\begin{sphinxadmonition}{note}{Pavyzdys}

\begin{sphinxVerbatim}[commandchars=\\\{\}]
\PYG{n+nt}{\PYGZlt{}countries}\PYG{n+nt}{\PYGZgt{}}
\PYG{+w}{    }\PYG{n+nt}{\PYGZlt{}country}\PYG{+w}{ }\PYG{n+na}{id=}\PYG{l+s}{\PYGZdq{}1\PYGZdq{}}\PYG{+w}{ }\PYG{n+na}{name=}\PYG{l+s}{\PYGZdq{}Lithuania\PYGZdq{}}\PYG{n+nt}{\PYGZgt{}}
\PYG{+w}{        }\PYG{n+nt}{\PYGZlt{}cities}\PYG{n+nt}{\PYGZgt{}}
\PYG{+w}{            }\PYG{n+nt}{\PYGZlt{}city}\PYG{+w}{ }\PYG{n+na}{id=}\PYG{l+s}{\PYGZdq{}10\PYGZdq{}}\PYG{+w}{ }\PYG{n+na}{name=}\PYG{l+s}{\PYGZdq{}Vilnius\PYGZdq{}}\PYG{n+nt}{\PYGZgt{}}
\PYG{+w}{                }\PYG{n+nt}{\PYGZlt{}streets}\PYG{n+nt}{\PYGZgt{}}
\PYG{+w}{                    }\PYG{n+nt}{\PYGZlt{}street}\PYG{+w}{ }\PYG{n+na}{id=}\PYG{l+s}{\PYGZdq{}100\PYGZdq{}}\PYG{n+nt}{\PYGZgt{}}Gedimino\PYG{+w}{ }st.\PYG{n+nt}{\PYGZlt{}/street\PYGZgt{}}
\PYG{+w}{                    }\PYG{n+nt}{\PYGZlt{}street}\PYG{+w}{ }\PYG{n+na}{id=}\PYG{l+s}{\PYGZdq{}101\PYGZdq{}}\PYG{n+nt}{\PYGZgt{}}Konstitucijos\PYG{+w}{ }st.\PYG{n+nt}{\PYGZlt{}/street\PYGZgt{}}
\PYG{+w}{                }\PYG{n+nt}{\PYGZlt{}/streets\PYGZgt{}}
\PYG{+w}{            }\PYG{n+nt}{\PYGZlt{}/city\PYGZgt{}}
\PYG{+w}{            }\PYG{n+nt}{\PYGZlt{}city}\PYG{+w}{ }\PYG{n+na}{id=}\PYG{l+s}{\PYGZdq{}11\PYGZdq{}}\PYG{+w}{ }\PYG{n+na}{name=}\PYG{l+s}{\PYGZdq{}Kaunas\PYGZdq{}}\PYG{n+nt}{\PYGZgt{}}
\PYG{+w}{                }\PYG{n+nt}{\PYGZlt{}streets}\PYG{n+nt}{\PYGZgt{}}
\PYG{+w}{                    }\PYG{n+nt}{\PYGZlt{}street}\PYG{+w}{ }\PYG{n+na}{id=}\PYG{l+s}{\PYGZdq{}102\PYGZdq{}}\PYG{n+nt}{\PYGZgt{}}Laisves\PYG{+w}{ }st.\PYG{n+nt}{\PYGZlt{}/street\PYGZgt{}}
\PYG{+w}{                    }\PYG{n+nt}{\PYGZlt{}street}\PYG{+w}{ }\PYG{n+na}{id=}\PYG{l+s}{\PYGZdq{}103\PYGZdq{}}\PYG{n+nt}{\PYGZgt{}}Daukanto\PYG{+w}{ }st.\PYG{n+nt}{\PYGZlt{}/street\PYGZgt{}}
\PYG{+w}{                }\PYG{n+nt}{\PYGZlt{}/streets\PYGZgt{}}
\PYG{+w}{            }\PYG{n+nt}{\PYGZlt{}/city\PYGZgt{}}
\PYG{+w}{        }\PYG{n+nt}{\PYGZlt{}/cities\PYGZgt{}}
\PYG{+w}{    }\PYG{n+nt}{\PYGZlt{}/country\PYGZgt{}}
\PYG{n+nt}{\PYGZlt{}/countries\PYGZgt{}}
\end{sphinxVerbatim}

\begin{DUlineblock}{0em}
\item[] 
\end{DUlineblock}

\sphinxAtStartPar
Pagal aukščiau duotus duomenis ir koncepcinį modelį, struktūros aprašas
atrodys taip:


\begin{savenotes}\sphinxattablestart
\sphinxthistablewithglobalstyle
\centering
\begin{tabulary}{\linewidth}[t]{TTTTT}
\sphinxtoprule
\sphinxstyletheadfamily 
\sphinxAtStartPar
model
&\sphinxstyletheadfamily 
\sphinxAtStartPar
property
&\sphinxstyletheadfamily 
\sphinxAtStartPar
type
&\sphinxstyletheadfamily 
\sphinxAtStartPar
ref
&\sphinxstyletheadfamily 
\sphinxAtStartPar
source
\\
\sphinxmidrule
\sphinxtableatstartofbodyhook\sphinxstartmulticolumn{2}%
\begin{varwidth}[t]{\sphinxcolwidth{2}{5}}
\sphinxAtStartPar
\sphinxstylestrong{Country}
\par
\vskip-\baselineskip\vbox{\hbox{\strut}}\end{varwidth}%
\sphinxstopmulticolumn
&&
\sphinxAtStartPar
id
&
\sphinxAtStartPar
\sphinxstylestrong{countries/country}
\\
\sphinxhline
\sphinxAtStartPar

&
\sphinxAtStartPar
id
&
\sphinxAtStartPar
integer
&&
\sphinxAtStartPar
@id
\\
\sphinxhline
\sphinxAtStartPar

&
\sphinxAtStartPar
name@en
&
\sphinxAtStartPar
string
&&
\sphinxAtStartPar
@name
\\
\sphinxhline
\sphinxAtStartPar

&
\sphinxAtStartPar
cities{[}{]}
&
\sphinxAtStartPar
backref
&
\sphinxAtStartPar
\sphinxstylestrong{City}
&
\sphinxAtStartPar
cities/city
\\
\sphinxhline
\sphinxAtStartPar

&
\sphinxAtStartPar
cities{[}{]}.id
&
\sphinxAtStartPar
integer
&&
\sphinxAtStartPar
@id
\\
\sphinxhline
\sphinxAtStartPar

&
\sphinxAtStartPar
cities{[}{]}.name@en
&
\sphinxAtStartPar
string
&&
\sphinxAtStartPar
@name
\\
\sphinxhline
\sphinxAtStartPar

&
\sphinxAtStartPar
cities{[}{]}.country
&
\sphinxAtStartPar
ref
&
\sphinxAtStartPar
\sphinxstylestrong{Country}
&
\sphinxAtStartPar
../../@id
\\
\sphinxhline
\sphinxAtStartPar

&
\sphinxAtStartPar
cities{[}{]}.streets{[}{]}
&
\sphinxAtStartPar
backref
&
\sphinxAtStartPar
\sphinxstylestrong{Street}
&
\sphinxAtStartPar
streets/street
\\
\sphinxhline
\sphinxAtStartPar

&
\sphinxAtStartPar
cities{[}{]}.streets{[}{]}.id
&
\sphinxAtStartPar
integer
&&
\sphinxAtStartPar
@id
\\
\sphinxhline
\sphinxAtStartPar

&
\sphinxAtStartPar
cities{[}{]}.streets{[}{]}.name@en
&
\sphinxAtStartPar
string
&&
\sphinxAtStartPar
@name
\\
\sphinxhline
\sphinxAtStartPar

&
\sphinxAtStartPar
cities{[}{]}.streets{[}{]}.city
&
\sphinxAtStartPar
ref
&
\sphinxAtStartPar
\sphinxstylestrong{City}
&
\sphinxAtStartPar
../../@id
\\
\sphinxhline\sphinxstartmulticolumn{2}%
\begin{varwidth}[t]{\sphinxcolwidth{2}{5}}
\sphinxAtStartPar
\sphinxstylestrong{City}
\par
\vskip-\baselineskip\vbox{\hbox{\strut}}\end{varwidth}%
\sphinxstopmulticolumn
&&
\sphinxAtStartPar
id
&
\sphinxAtStartPar
\sphinxstylestrong{countries/country/cities/city}
\\
\sphinxhline
\sphinxAtStartPar

&
\sphinxAtStartPar
id
&
\sphinxAtStartPar
integer
&&
\sphinxAtStartPar
@id
\\
\sphinxhline
\sphinxAtStartPar

&
\sphinxAtStartPar
name@en
&
\sphinxAtStartPar
string
&&
\sphinxAtStartPar
@name
\\
\sphinxhline
\sphinxAtStartPar

&
\sphinxAtStartPar
country
&
\sphinxAtStartPar
ref
&
\sphinxAtStartPar
\sphinxstylestrong{Country}
&
\sphinxAtStartPar
../../@id
\\
\sphinxhline\sphinxstartmulticolumn{2}%
\begin{varwidth}[t]{\sphinxcolwidth{2}{5}}
\sphinxAtStartPar
\sphinxstylestrong{Street}
\par
\vskip-\baselineskip\vbox{\hbox{\strut}}\end{varwidth}%
\sphinxstopmulticolumn
&&
\sphinxAtStartPar
id
&
\sphinxAtStartPar
\sphinxstylestrong{countries/country/cities/city/streets/street}
\\
\sphinxhline
\sphinxAtStartPar

&
\sphinxAtStartPar
id
&
\sphinxAtStartPar
integer
&&
\sphinxAtStartPar
@id
\\
\sphinxhline
\sphinxAtStartPar

&
\sphinxAtStartPar
name@en
&
\sphinxAtStartPar
string
&&
\sphinxAtStartPar
@name
\\
\sphinxhline
\sphinxAtStartPar

&
\sphinxAtStartPar
country
&
\sphinxAtStartPar
ref
&
\sphinxAtStartPar
\sphinxstylestrong{City}
&
\sphinxAtStartPar
../../@id
\\
\sphinxbottomrule
\end{tabulary}
\sphinxtableafterendhook\par
\sphinxattableend\end{savenotes}

\sphinxAtStartPar
Struktūros apraše matome du variantus, kaip gali būti aprašomi duomenys.
Pirmu atveju \sphinxcode{\sphinxupquote{Country}} modelyje naudojama objektų kompozicija, kur vieno
\sphinxcode{\sphinxupquote{Country}} objekto apimtyje, pateikiami ir kiti objektai.

\sphinxAtStartPar
Reikia atkreipti dėmesį, kad savybės esančios kitos daugiareikšmės savybės
sudėtyje, {\hyperref[\detokenize{dimensijos:property.source}]{\sphinxcrossref{\sphinxcode{\sphinxupquote{property.source}}}}} stulpelyje nurodo XPath išraišką
reliatyvią daugereikšmei savybei. Daugiareikšmės savybės žymymos \sphinxcode{\sphinxupquote{{[}{]}}} žyme.

\sphinxAtStartPar
Pavyzdyje \sphinxcode{\sphinxupquote{cities{[}{]}.id}} {\hyperref[\detokenize{dimensijos:property.source}]{\sphinxcrossref{\sphinxcode{\sphinxupquote{property.source}}}}} stulpelyje nurodo \sphinxcode{\sphinxupquote{@id}},
kuris yra reliatyvus \sphinxcode{\sphinxupquote{cities{[}{]}}} savybės \sphinxcode{\sphinxupquote{streets/street}} atžvilgiu.

\sphinxAtStartPar
Pagal struktūros aprašą pateiktą aukščiau, kreipiantis į \sphinxcode{\sphinxupquote{/Country}},
gausime tokius \sphinxhref{https://ivpk.github.io/uapi}{UDTS} specifikaciją atitinkančius duomenis:

\begin{sphinxVerbatim}[commandchars=\\\{\}]
\PYG{p}{\PYGZob{}}
\PYG{+w}{    }\PYG{n+nt}{\PYGZdq{}\PYGZus{}type\PYGZdq{}}\PYG{p}{:}\PYG{+w}{ }\PYG{l+s+s2}{\PYGZdq{}Country\PYGZdq{}}\PYG{p}{,}
\PYG{+w}{    }\PYG{n+nt}{\PYGZdq{}\PYGZus{}id\PYGZdq{}}\PYG{p}{:}\PYG{+w}{ }\PYG{l+s+s2}{\PYGZdq{}29df0534\PYGZhy{}389d\PYGZhy{}4eac\PYGZhy{}a048\PYGZhy{}799ac64d5103\PYGZdq{}}\PYG{p}{,}
\PYG{+w}{    }\PYG{n+nt}{\PYGZdq{}id\PYGZdq{}}\PYG{p}{:}\PYG{+w}{ }\PYG{l+m+mi}{1}\PYG{p}{,}
\PYG{+w}{    }\PYG{n+nt}{\PYGZdq{}name\PYGZdq{}}\PYG{p}{:}\PYG{+w}{ }\PYG{p}{\PYGZob{}}\PYG{n+nt}{\PYGZdq{}en\PYGZdq{}}\PYG{p}{:}\PYG{+w}{ }\PYG{l+s+s2}{\PYGZdq{}Lithuana\PYGZdq{}}\PYG{p}{\PYGZcb{},}
\PYG{+w}{    }\PYG{n+nt}{\PYGZdq{}cities\PYGZdq{}}\PYG{p}{:}\PYG{+w}{ }\PYG{p}{[}
\PYG{+w}{        }\PYG{p}{\PYGZob{}}
\PYG{+w}{            }\PYG{n+nt}{\PYGZdq{}\PYGZus{}type\PYGZdq{}}\PYG{p}{:}\PYG{+w}{ }\PYG{l+s+s2}{\PYGZdq{}City\PYGZdq{}}\PYG{p}{,}
\PYG{+w}{            }\PYG{n+nt}{\PYGZdq{}\PYGZus{}id\PYGZdq{}}\PYG{p}{:}\PYG{+w}{ }\PYG{l+s+s2}{\PYGZdq{}4a7a3214\PYGZhy{}e6c3\PYGZhy{}4a5b\PYGZhy{}99a8\PYGZhy{}04be88eac3d4\PYGZdq{}}\PYG{p}{,}
\PYG{+w}{            }\PYG{n+nt}{\PYGZdq{}id\PYGZdq{}}\PYG{p}{:}\PYG{+w}{ }\PYG{l+m+mi}{10}\PYG{p}{,}
\PYG{+w}{            }\PYG{n+nt}{\PYGZdq{}name\PYGZdq{}}\PYG{p}{:}\PYG{+w}{ }\PYG{p}{\PYGZob{}}\PYG{n+nt}{\PYGZdq{}en\PYGZdq{}}\PYG{p}{:}\PYG{+w}{ }\PYG{l+s+s2}{\PYGZdq{}Vilnius\PYGZdq{}}\PYG{p}{\PYGZcb{},}
\PYG{+w}{            }\PYG{n+nt}{\PYGZdq{}country\PYGZdq{}}\PYG{p}{:}\PYG{+w}{ }\PYG{p}{\PYGZob{}}
\PYG{+w}{                }\PYG{n+nt}{\PYGZdq{}\PYGZus{}type\PYGZdq{}}\PYG{p}{:}\PYG{+w}{ }\PYG{l+s+s2}{\PYGZdq{}Country\PYGZdq{}}\PYG{p}{,}
\PYG{+w}{                }\PYG{n+nt}{\PYGZdq{}\PYGZus{}id\PYGZdq{}}\PYG{p}{:}\PYG{+w}{ }\PYG{l+s+s2}{\PYGZdq{}29df0534\PYGZhy{}389d\PYGZhy{}4eac\PYGZhy{}a048\PYGZhy{}799ac64d5103\PYGZdq{}}
\PYG{+w}{            }\PYG{p}{\PYGZcb{},}
\PYG{+w}{            }\PYG{n+nt}{\PYGZdq{}streets\PYGZdq{}}\PYG{p}{:}\PYG{+w}{ }\PYG{p}{[}
\PYG{+w}{                }\PYG{p}{\PYGZob{}}
\PYG{+w}{                    }\PYG{n+nt}{\PYGZdq{}\PYGZus{}type\PYGZdq{}}\PYG{p}{:}\PYG{+w}{ }\PYG{l+s+s2}{\PYGZdq{}Street\PYGZdq{}}\PYG{p}{,}
\PYG{+w}{                    }\PYG{n+nt}{\PYGZdq{}\PYGZus{}id\PYGZdq{}}\PYG{p}{:}\PYG{+w}{ }\PYG{l+s+s2}{\PYGZdq{}c1380514\PYGZhy{}549f\PYGZhy{}4cdd\PYGZhy{}b258\PYGZhy{}6fecc3a5bbda\PYGZdq{}}\PYG{p}{,}
\PYG{+w}{                    }\PYG{n+nt}{\PYGZdq{}id\PYGZdq{}}\PYG{p}{:}\PYG{+w}{ }\PYG{l+m+mi}{100}\PYG{p}{,}
\PYG{+w}{                    }\PYG{n+nt}{\PYGZdq{}name\PYGZdq{}}\PYG{p}{:}\PYG{+w}{ }\PYG{p}{\PYGZob{}}\PYG{n+nt}{\PYGZdq{}en\PYGZdq{}}\PYG{p}{:}\PYG{+w}{ }\PYG{l+s+s2}{\PYGZdq{}Gedimino st.\PYGZdq{}}\PYG{p}{\PYGZcb{},}
\PYG{+w}{                    }\PYG{n+nt}{\PYGZdq{}city\PYGZdq{}}\PYG{p}{:}\PYG{+w}{ }\PYG{p}{\PYGZob{}}
\PYG{+w}{                        }\PYG{n+nt}{\PYGZdq{}\PYGZus{}type\PYGZdq{}}\PYG{p}{:}\PYG{+w}{ }\PYG{l+s+s2}{\PYGZdq{}City\PYGZdq{}}\PYG{p}{,}
\PYG{+w}{                        }\PYG{n+nt}{\PYGZdq{}\PYGZus{}id\PYGZdq{}}\PYG{p}{:}\PYG{+w}{ }\PYG{l+s+s2}{\PYGZdq{}4a7a3214\PYGZhy{}e6c3\PYGZhy{}4a5b\PYGZhy{}99a8\PYGZhy{}04be88eac3d4\PYGZdq{}}
\PYG{+w}{                    }\PYG{p}{\PYGZcb{},}
\PYG{+w}{                }\PYG{p}{\PYGZcb{},}
\PYG{+w}{                }\PYG{p}{\PYGZob{}}
\PYG{+w}{                    }\PYG{n+nt}{\PYGZdq{}\PYGZus{}type\PYGZdq{}}\PYG{p}{:}\PYG{+w}{ }\PYG{l+s+s2}{\PYGZdq{}Street\PYGZdq{}}\PYG{p}{,}
\PYG{+w}{                    }\PYG{n+nt}{\PYGZdq{}\PYGZus{}id\PYGZdq{}}\PYG{p}{:}\PYG{+w}{ }\PYG{l+s+s2}{\PYGZdq{}5c02f700\PYGZhy{}6478\PYGZhy{}43a0\PYGZhy{}a147\PYGZhy{}959927cb3c1c\PYGZdq{}}\PYG{p}{,}
\PYG{+w}{                    }\PYG{n+nt}{\PYGZdq{}id\PYGZdq{}}\PYG{p}{:}\PYG{+w}{ }\PYG{l+m+mi}{101}\PYG{p}{,}
\PYG{+w}{                    }\PYG{n+nt}{\PYGZdq{}name\PYGZdq{}}\PYG{p}{:}\PYG{+w}{ }\PYG{p}{\PYGZob{}}\PYG{n+nt}{\PYGZdq{}en\PYGZdq{}}\PYG{p}{:}\PYG{+w}{ }\PYG{l+s+s2}{\PYGZdq{}Konstitucijos st.\PYGZdq{}}\PYG{p}{\PYGZcb{},}
\PYG{+w}{                    }\PYG{n+nt}{\PYGZdq{}city\PYGZdq{}}\PYG{p}{:}\PYG{+w}{ }\PYG{p}{\PYGZob{}}
\PYG{+w}{                        }\PYG{n+nt}{\PYGZdq{}\PYGZus{}type\PYGZdq{}}\PYG{p}{:}\PYG{+w}{ }\PYG{l+s+s2}{\PYGZdq{}City\PYGZdq{}}\PYG{p}{,}
\PYG{+w}{                        }\PYG{n+nt}{\PYGZdq{}\PYGZus{}id\PYGZdq{}}\PYG{p}{:}\PYG{+w}{ }\PYG{l+s+s2}{\PYGZdq{}4a7a3214\PYGZhy{}e6c3\PYGZhy{}4a5b\PYGZhy{}99a8\PYGZhy{}04be88eac3d4\PYGZdq{}}
\PYG{+w}{                    }\PYG{p}{\PYGZcb{},}
\PYG{+w}{                }\PYG{p}{\PYGZcb{}}
\PYG{+w}{            }\PYG{p}{]}
\PYG{+w}{        }\PYG{p}{\PYGZcb{},}
\PYG{+w}{        }\PYG{p}{\PYGZob{}}
\PYG{+w}{            }\PYG{n+nt}{\PYGZdq{}\PYGZus{}type\PYGZdq{}}\PYG{p}{:}\PYG{+w}{ }\PYG{l+s+s2}{\PYGZdq{}City\PYGZdq{}}\PYG{p}{,}
\PYG{+w}{            }\PYG{n+nt}{\PYGZdq{}\PYGZus{}id\PYGZdq{}}\PYG{p}{:}\PYG{+w}{ }\PYG{l+s+s2}{\PYGZdq{}0fee7d9a\PYGZhy{}6827\PYGZhy{}4931\PYGZhy{}bbea\PYGZhy{}d44d197faef2\PYGZdq{}}\PYG{p}{,}
\PYG{+w}{            }\PYG{n+nt}{\PYGZdq{}id\PYGZdq{}}\PYG{p}{:}\PYG{+w}{ }\PYG{l+m+mi}{11}\PYG{p}{,}
\PYG{+w}{            }\PYG{n+nt}{\PYGZdq{}name\PYGZdq{}}\PYG{p}{:}\PYG{+w}{ }\PYG{p}{\PYGZob{}}\PYG{n+nt}{\PYGZdq{}en\PYGZdq{}}\PYG{p}{:}\PYG{+w}{ }\PYG{l+s+s2}{\PYGZdq{}Kaunas\PYGZdq{}}\PYG{p}{\PYGZcb{},}
\PYG{+w}{            }\PYG{n+nt}{\PYGZdq{}country\PYGZdq{}}\PYG{p}{:}\PYG{+w}{ }\PYG{p}{\PYGZob{}}
\PYG{+w}{                }\PYG{n+nt}{\PYGZdq{}\PYGZus{}type\PYGZdq{}}\PYG{p}{:}\PYG{+w}{ }\PYG{l+s+s2}{\PYGZdq{}Country\PYGZdq{}}\PYG{p}{,}
\PYG{+w}{                }\PYG{n+nt}{\PYGZdq{}\PYGZus{}id\PYGZdq{}}\PYG{p}{:}\PYG{+w}{ }\PYG{l+s+s2}{\PYGZdq{}29df0534\PYGZhy{}389d\PYGZhy{}4eac\PYGZhy{}a048\PYGZhy{}799ac64d5103\PYGZdq{}}
\PYG{+w}{            }\PYG{p}{\PYGZcb{},}
\PYG{+w}{            }\PYG{n+nt}{\PYGZdq{}streets\PYGZdq{}}\PYG{p}{:}\PYG{+w}{ }\PYG{p}{[}
\PYG{+w}{                }\PYG{p}{\PYGZob{}}
\PYG{+w}{                    }\PYG{n+nt}{\PYGZdq{}\PYGZus{}type\PYGZdq{}}\PYG{p}{:}\PYG{+w}{ }\PYG{l+s+s2}{\PYGZdq{}Street\PYGZdq{}}\PYG{p}{,}
\PYG{+w}{                    }\PYG{n+nt}{\PYGZdq{}\PYGZus{}id\PYGZdq{}}\PYG{p}{:}\PYG{+w}{ }\PYG{l+s+s2}{\PYGZdq{}399a37d6\PYGZhy{}63a7\PYGZhy{}43a4\PYGZhy{}82de\PYGZhy{}d3d5c75f5d02\PYGZdq{}}\PYG{p}{,}
\PYG{+w}{                    }\PYG{n+nt}{\PYGZdq{}id\PYGZdq{}}\PYG{p}{:}\PYG{+w}{ }\PYG{l+m+mi}{102}\PYG{p}{,}
\PYG{+w}{                    }\PYG{n+nt}{\PYGZdq{}name\PYGZdq{}}\PYG{p}{:}\PYG{+w}{ }\PYG{p}{\PYGZob{}}\PYG{n+nt}{\PYGZdq{}en\PYGZdq{}}\PYG{p}{:}\PYG{+w}{ }\PYG{l+s+s2}{\PYGZdq{}Laisves st.\PYGZdq{}}\PYG{p}{\PYGZcb{},}
\PYG{+w}{                    }\PYG{n+nt}{\PYGZdq{}city\PYGZdq{}}\PYG{p}{:}\PYG{+w}{ }\PYG{p}{\PYGZob{}}
\PYG{+w}{                        }\PYG{n+nt}{\PYGZdq{}\PYGZus{}type\PYGZdq{}}\PYG{p}{:}\PYG{+w}{ }\PYG{l+s+s2}{\PYGZdq{}City\PYGZdq{}}\PYG{p}{,}
\PYG{+w}{                        }\PYG{n+nt}{\PYGZdq{}\PYGZus{}id\PYGZdq{}}\PYG{p}{:}\PYG{+w}{ }\PYG{l+s+s2}{\PYGZdq{}0fee7d9a\PYGZhy{}6827\PYGZhy{}4931\PYGZhy{}bbea\PYGZhy{}d44d197faef2\PYGZdq{}}
\PYG{+w}{                    }\PYG{p}{\PYGZcb{},}
\PYG{+w}{                }\PYG{p}{\PYGZcb{},}
\PYG{+w}{                }\PYG{p}{\PYGZob{}}
\PYG{+w}{                    }\PYG{n+nt}{\PYGZdq{}\PYGZus{}type\PYGZdq{}}\PYG{p}{:}\PYG{+w}{ }\PYG{l+s+s2}{\PYGZdq{}Street\PYGZdq{}}\PYG{p}{,}
\PYG{+w}{                    }\PYG{n+nt}{\PYGZdq{}\PYGZus{}id\PYGZdq{}}\PYG{p}{:}\PYG{+w}{ }\PYG{l+s+s2}{\PYGZdq{}5b04fecd\PYGZhy{}5fff\PYGZhy{}48f6\PYGZhy{}8674\PYGZhy{}7cc6da840281\PYGZdq{}}\PYG{p}{,}
\PYG{+w}{                    }\PYG{n+nt}{\PYGZdq{}id\PYGZdq{}}\PYG{p}{:}\PYG{+w}{ }\PYG{l+m+mi}{103}\PYG{p}{,}
\PYG{+w}{                    }\PYG{n+nt}{\PYGZdq{}name\PYGZdq{}}\PYG{p}{:}\PYG{+w}{ }\PYG{p}{\PYGZob{}}\PYG{n+nt}{\PYGZdq{}en\PYGZdq{}}\PYG{p}{:}\PYG{+w}{ }\PYG{l+s+s2}{\PYGZdq{}Daukanto st.\PYGZdq{}}\PYG{p}{\PYGZcb{},}
\PYG{+w}{                    }\PYG{n+nt}{\PYGZdq{}city\PYGZdq{}}\PYG{p}{:}\PYG{+w}{ }\PYG{p}{\PYGZob{}}
\PYG{+w}{                        }\PYG{n+nt}{\PYGZdq{}\PYGZus{}type\PYGZdq{}}\PYG{p}{:}\PYG{+w}{ }\PYG{l+s+s2}{\PYGZdq{}City\PYGZdq{}}\PYG{p}{,}
\PYG{+w}{                        }\PYG{n+nt}{\PYGZdq{}\PYGZus{}id\PYGZdq{}}\PYG{p}{:}\PYG{+w}{ }\PYG{l+s+s2}{\PYGZdq{}0fee7d9a\PYGZhy{}6827\PYGZhy{}4931\PYGZhy{}bbea\PYGZhy{}d44d197faef2\PYGZdq{}}
\PYG{+w}{                    }\PYG{p}{\PYGZcb{},}
\PYG{+w}{                }\PYG{p}{\PYGZcb{}}
\PYG{+w}{            }\PYG{p}{]}
\PYG{+w}{        }\PYG{p}{\PYGZcb{}}
\PYG{+w}{    }\PYG{p}{]}
\PYG{p}{\PYGZcb{}}
\end{sphinxVerbatim}

\sphinxAtStartPar
Analogiškai, jei kreiptumėmės į \sphinxcode{\sphinxupquote{/Street}}, gautume visas gatves iš visų miestų:

\begin{sphinxVerbatim}[commandchars=\\\{\}]
\PYG{p}{\PYGZob{}}
\PYG{+w}{    }\PYG{n+nt}{\PYGZdq{}\PYGZus{}data\PYGZdq{}}\PYG{p}{:}\PYG{+w}{ }\PYG{p}{[}
\PYG{+w}{        }\PYG{p}{\PYGZob{}}
\PYG{+w}{            }\PYG{n+nt}{\PYGZdq{}\PYGZus{}type\PYGZdq{}}\PYG{p}{:}\PYG{+w}{ }\PYG{l+s+s2}{\PYGZdq{}Street\PYGZdq{}}\PYG{p}{,}
\PYG{+w}{            }\PYG{n+nt}{\PYGZdq{}\PYGZus{}id\PYGZdq{}}\PYG{p}{:}\PYG{+w}{ }\PYG{l+s+s2}{\PYGZdq{}c1380514\PYGZhy{}549f\PYGZhy{}4cdd\PYGZhy{}b258\PYGZhy{}6fecc3a5bbda\PYGZdq{}}\PYG{p}{,}
\PYG{+w}{            }\PYG{n+nt}{\PYGZdq{}id\PYGZdq{}}\PYG{p}{:}\PYG{+w}{ }\PYG{l+m+mi}{100}\PYG{p}{,}
\PYG{+w}{            }\PYG{n+nt}{\PYGZdq{}name\PYGZdq{}}\PYG{p}{:}\PYG{+w}{ }\PYG{p}{\PYGZob{}}\PYG{n+nt}{\PYGZdq{}en\PYGZdq{}}\PYG{p}{:}\PYG{+w}{ }\PYG{l+s+s2}{\PYGZdq{}Gedimino st.\PYGZdq{}}\PYG{p}{\PYGZcb{},}
\PYG{+w}{            }\PYG{n+nt}{\PYGZdq{}city\PYGZdq{}}\PYG{p}{:}\PYG{+w}{ }\PYG{p}{\PYGZob{}}
\PYG{+w}{                }\PYG{n+nt}{\PYGZdq{}\PYGZus{}type\PYGZdq{}}\PYG{p}{:}\PYG{+w}{ }\PYG{l+s+s2}{\PYGZdq{}City\PYGZdq{}}\PYG{p}{,}
\PYG{+w}{                }\PYG{n+nt}{\PYGZdq{}\PYGZus{}id\PYGZdq{}}\PYG{p}{:}\PYG{+w}{ }\PYG{l+s+s2}{\PYGZdq{}4a7a3214\PYGZhy{}e6c3\PYGZhy{}4a5b\PYGZhy{}99a8\PYGZhy{}04be88eac3d4\PYGZdq{}}
\PYG{+w}{            }\PYG{p}{\PYGZcb{},}
\PYG{+w}{        }\PYG{p}{\PYGZcb{},}
\PYG{+w}{        }\PYG{p}{\PYGZob{}}
\PYG{+w}{            }\PYG{n+nt}{\PYGZdq{}\PYGZus{}type\PYGZdq{}}\PYG{p}{:}\PYG{+w}{ }\PYG{l+s+s2}{\PYGZdq{}Street\PYGZdq{}}\PYG{p}{,}
\PYG{+w}{            }\PYG{n+nt}{\PYGZdq{}\PYGZus{}id\PYGZdq{}}\PYG{p}{:}\PYG{+w}{ }\PYG{l+s+s2}{\PYGZdq{}5c02f700\PYGZhy{}6478\PYGZhy{}43a0\PYGZhy{}a147\PYGZhy{}959927cb3c1c\PYGZdq{}}\PYG{p}{,}
\PYG{+w}{            }\PYG{n+nt}{\PYGZdq{}id\PYGZdq{}}\PYG{p}{:}\PYG{+w}{ }\PYG{l+m+mi}{101}\PYG{p}{,}
\PYG{+w}{            }\PYG{n+nt}{\PYGZdq{}name\PYGZdq{}}\PYG{p}{:}\PYG{+w}{ }\PYG{p}{\PYGZob{}}\PYG{n+nt}{\PYGZdq{}en\PYGZdq{}}\PYG{p}{:}\PYG{+w}{ }\PYG{l+s+s2}{\PYGZdq{}Konstitucijos st.\PYGZdq{}}\PYG{p}{\PYGZcb{},}
\PYG{+w}{            }\PYG{n+nt}{\PYGZdq{}city\PYGZdq{}}\PYG{p}{:}\PYG{+w}{ }\PYG{p}{\PYGZob{}}
\PYG{+w}{                }\PYG{n+nt}{\PYGZdq{}\PYGZus{}type\PYGZdq{}}\PYG{p}{:}\PYG{+w}{ }\PYG{l+s+s2}{\PYGZdq{}City\PYGZdq{}}\PYG{p}{,}
\PYG{+w}{                }\PYG{n+nt}{\PYGZdq{}\PYGZus{}id\PYGZdq{}}\PYG{p}{:}\PYG{+w}{ }\PYG{l+s+s2}{\PYGZdq{}4a7a3214\PYGZhy{}e6c3\PYGZhy{}4a5b\PYGZhy{}99a8\PYGZhy{}04be88eac3d4\PYGZdq{}}
\PYG{+w}{            }\PYG{p}{\PYGZcb{},}
\PYG{+w}{        }\PYG{p}{\PYGZcb{},}
\PYG{+w}{        }\PYG{p}{\PYGZob{}}
\PYG{+w}{            }\PYG{n+nt}{\PYGZdq{}\PYGZus{}type\PYGZdq{}}\PYG{p}{:}\PYG{+w}{ }\PYG{l+s+s2}{\PYGZdq{}Street\PYGZdq{}}\PYG{p}{,}
\PYG{+w}{            }\PYG{n+nt}{\PYGZdq{}\PYGZus{}id\PYGZdq{}}\PYG{p}{:}\PYG{+w}{ }\PYG{l+s+s2}{\PYGZdq{}399a37d6\PYGZhy{}63a7\PYGZhy{}43a4\PYGZhy{}82de\PYGZhy{}d3d5c75f5d02\PYGZdq{}}\PYG{p}{,}
\PYG{+w}{            }\PYG{n+nt}{\PYGZdq{}id\PYGZdq{}}\PYG{p}{:}\PYG{+w}{ }\PYG{l+m+mi}{102}\PYG{p}{,}
\PYG{+w}{            }\PYG{n+nt}{\PYGZdq{}name\PYGZdq{}}\PYG{p}{:}\PYG{+w}{ }\PYG{p}{\PYGZob{}}\PYG{n+nt}{\PYGZdq{}en\PYGZdq{}}\PYG{p}{:}\PYG{+w}{ }\PYG{l+s+s2}{\PYGZdq{}Laisves st.\PYGZdq{}}\PYG{p}{\PYGZcb{},}
\PYG{+w}{            }\PYG{n+nt}{\PYGZdq{}city\PYGZdq{}}\PYG{p}{:}\PYG{+w}{ }\PYG{p}{\PYGZob{}}
\PYG{+w}{                }\PYG{n+nt}{\PYGZdq{}\PYGZus{}type\PYGZdq{}}\PYG{p}{:}\PYG{+w}{ }\PYG{l+s+s2}{\PYGZdq{}City\PYGZdq{}}\PYG{p}{,}
\PYG{+w}{                }\PYG{n+nt}{\PYGZdq{}\PYGZus{}id\PYGZdq{}}\PYG{p}{:}\PYG{+w}{ }\PYG{l+s+s2}{\PYGZdq{}0fee7d9a\PYGZhy{}6827\PYGZhy{}4931\PYGZhy{}bbea\PYGZhy{}d44d197faef2\PYGZdq{}}
\PYG{+w}{            }\PYG{p}{\PYGZcb{},}
\PYG{+w}{        }\PYG{p}{\PYGZcb{},}
\PYG{+w}{        }\PYG{p}{\PYGZob{}}
\PYG{+w}{            }\PYG{n+nt}{\PYGZdq{}\PYGZus{}type\PYGZdq{}}\PYG{p}{:}\PYG{+w}{ }\PYG{l+s+s2}{\PYGZdq{}Street\PYGZdq{}}\PYG{p}{,}
\PYG{+w}{            }\PYG{n+nt}{\PYGZdq{}\PYGZus{}id\PYGZdq{}}\PYG{p}{:}\PYG{+w}{ }\PYG{l+s+s2}{\PYGZdq{}5b04fecd\PYGZhy{}5fff\PYGZhy{}48f6\PYGZhy{}8674\PYGZhy{}7cc6da840281\PYGZdq{}}\PYG{p}{,}
\PYG{+w}{            }\PYG{n+nt}{\PYGZdq{}id\PYGZdq{}}\PYG{p}{:}\PYG{+w}{ }\PYG{l+m+mi}{103}\PYG{p}{,}
\PYG{+w}{            }\PYG{n+nt}{\PYGZdq{}name\PYGZdq{}}\PYG{p}{:}\PYG{+w}{ }\PYG{p}{\PYGZob{}}\PYG{n+nt}{\PYGZdq{}en\PYGZdq{}}\PYG{p}{:}\PYG{+w}{ }\PYG{l+s+s2}{\PYGZdq{}Daukanto st.\PYGZdq{}}\PYG{p}{\PYGZcb{},}
\PYG{+w}{            }\PYG{n+nt}{\PYGZdq{}city\PYGZdq{}}\PYG{p}{:}\PYG{+w}{ }\PYG{p}{\PYGZob{}}
\PYG{+w}{                }\PYG{n+nt}{\PYGZdq{}\PYGZus{}type\PYGZdq{}}\PYG{p}{:}\PYG{+w}{ }\PYG{l+s+s2}{\PYGZdq{}City\PYGZdq{}}\PYG{p}{,}
\PYG{+w}{                }\PYG{n+nt}{\PYGZdq{}\PYGZus{}id\PYGZdq{}}\PYG{p}{:}\PYG{+w}{ }\PYG{l+s+s2}{\PYGZdq{}0fee7d9a\PYGZhy{}6827\PYGZhy{}4931\PYGZhy{}bbea\PYGZhy{}d44d197faef2\PYGZdq{}}
\PYG{+w}{            }\PYG{p}{\PYGZcb{},}
\PYG{+w}{        }\PYG{p}{\PYGZcb{}}
\PYG{+w}{    }\PYG{p}{]}
\PYG{p}{\PYGZcb{}}
\end{sphinxVerbatim}
\end{sphinxadmonition}


\subsection{XLSX}
\label{\detokenize{saltiniai:xlsx}}

\begin{fulllineitems}

\pysigstartsignatures
\pysigline
{\sphinxbfcode{\sphinxupquote{resource.type}}}
\pysigstopsignatures
\sphinxAtStartPar
Galimos reikšmės: \sphinxcode{\sphinxupquote{xlsx}}, \sphinxcode{\sphinxupquote{xls}} arba \sphinxcode{\sphinxupquote{odt}}.

\end{fulllineitems}



\begin{fulllineitems}

\pysigstartsignatures
\pysigline
{\sphinxbfcode{\sphinxupquote{resource.source}}}
\pysigstopsignatures
\sphinxAtStartPar
Žiūrėti {\hyperref[\detokenize{formules:failai}]{\sphinxcrossref{\DUrole{std}{\DUrole{std-ref}{Failai}}}}}.

\end{fulllineitems}



\begin{fulllineitems}

\pysigstartsignatures
\pysigline
{\sphinxbfcode{\sphinxupquote{model.source}}}
\pysigstopsignatures
\sphinxAtStartPar
Skaičiuoklės faile esančio lapo pavadinimas.

\end{fulllineitems}



\begin{fulllineitems}

\pysigstartsignatures
\pysigline
{\sphinxbfcode{\sphinxupquote{model.prepare}}}
\pysigstopsignatures
\sphinxAtStartPar
Žiūrėti {\hyperref[\detokenize{formules:stulpeliai-lenteleje}]{\sphinxcrossref{\DUrole{std}{\DUrole{std-ref}{Stulpeliai lentelėje}}}}}.

\end{fulllineitems}



\begin{fulllineitems}

\pysigstartsignatures
\pysigline
{\sphinxbfcode{\sphinxupquote{property.source}}}
\pysigstopsignatures
\sphinxAtStartPar
Žiūrėti {\hyperref[\detokenize{formules:stulpeliai-lenteleje}]{\sphinxcrossref{\DUrole{std}{\DUrole{std-ref}{Stulpeliai lentelėje}}}}}.

\end{fulllineitems}


\sphinxstepscope


\section{Vardų erdvės}
\label{\detokenize{vardu-erdves:vardu-erdves}}\label{\detokenize{vardu-erdves:ns}}\label{\detokenize{vardu-erdves::doc}}
\sphinxAtStartPar
{\hyperref[\detokenize{formatas:dataset}]{\sphinxcrossref{\sphinxcode{\sphinxupquote{dataset}}}}} ir {\hyperref[\detokenize{formatas:model}]{\sphinxcrossref{\sphinxcode{\sphinxupquote{model}}}}} esantys pavadinimai turi būti globaliai
(Lietuvos mastu) unikalūs. Kad užtikrinti pavadinimų unikalumą {\hyperref[\detokenize{formatas:dataset}]{\sphinxcrossref{\sphinxcode{\sphinxupquote{dataset}}}}}
ir {\hyperref[\detokenize{formatas:model}]{\sphinxcrossref{\sphinxcode{\sphinxupquote{model}}}}} pavadinimai formuojami pasitelkiant vardų erdves.


\begin{fulllineitems}

\pysigstartsignatures
\pysigline
{\sphinxbfcode{\sphinxupquote{/\{vocabulary\}/}}}
\pysigstopsignatures
\sphinxAtStartPar
\sphinxstylestrong{Standartinių žodynų vardų erdvė}

\sphinxAtStartPar
Standartinių žodynų vardų erdvė formuojama egzistuojančių standartų ir
išorinių žodynų pagrindu suteikiant vardų erdvei \sphinxcode{\sphinxupquote{\{vocabulary\}}} žodyno
sutrumpintą pavadinimą. Pavyzdžiui duomenų katalogo metaduomenų žodynas
DCAT turėtų keliauti į \sphinxcode{\sphinxupquote{/dcat/}} vardų erdvę. Standartų sutrumpintus
pavadinimus rekomenduojame imti iš \sphinxhref{https://lov.linkeddata.es/dataset/lov/}{Linked Open Vocabularies} katalogo.

\end{fulllineitems}



\begin{fulllineitems}

\pysigstartsignatures
\pysigline
{\sphinxbfcode{\sphinxupquote{/datasets/\{form\}/\{org\}/}}}
\pysigstopsignatures
\sphinxAtStartPar
\sphinxstylestrong{Įstaigų vardų erdvė}

\sphinxAtStartPar
Konkrečios organizacijos vietinė rinkinio vardų erdvė. Rekomenduojama
\sphinxcode{\sphinxupquote{\{org\}}} reikšmei naudoti organizacijos trumpinį, kad bendras modelio
pavadinimas nebūtų per daug ilgas.

\sphinxAtStartPar
Galimos \sphinxcode{\sphinxupquote{\{form\}}} reikšmės:


\begin{savenotes}\sphinxattablestart
\sphinxthistablewithglobalstyle
\centering
\begin{tabulary}{\linewidth}[t]{TT}
\sphinxtoprule
\sphinxtableatstartofbodyhook
\sphinxAtStartPar
\sphinxcode{\sphinxupquote{gov}}
&
\sphinxAtStartPar
Valstybinės įstaigos.
\\
\sphinxhline
\sphinxAtStartPar
\sphinxcode{\sphinxupquote{com}}
&
\sphinxAtStartPar
Verslo įmonės.
\\
\sphinxbottomrule
\end{tabulary}
\sphinxtableafterendhook\par
\sphinxattableend\end{savenotes}

\end{fulllineitems}



\begin{fulllineitems}

\pysigstartsignatures
\pysigline
{\sphinxbfcode{\sphinxupquote{/datasets/\{form\}/\{org\}/\{catalog\}/}}}
\pysigstopsignatures
\sphinxAtStartPar
\sphinxstylestrong{Informacinių sistemų vardų erdvė}

\sphinxAtStartPar
Informacinės sistemos vardų erdvė, kuri teikia duomenų rinkinius.

\end{fulllineitems}



\begin{fulllineitems}

\pysigstartsignatures
\pysigline
{\sphinxbfcode{\sphinxupquote{/datasets/\{form\}/\{org\}/\{catalog\}/\{dataset\}/}}}
\pysigstopsignatures
\sphinxAtStartPar
\sphinxstylestrong{Įstaigų duomenų rinkinių vardų erdvė}

\sphinxAtStartPar
Įstaigos duomenų rinkinio vardų erdvė į kurią patenka visi įstaigos duomenų
modeliai.

\end{fulllineitems}


\sphinxAtStartPar
Naujai atveriami {\hyperref[\detokenize{savokos:term-DSA}]{\sphinxtermref{\DUrole{xref}{\DUrole{std}{\DUrole{std-term}{duomenų struktūros aprašai}}}}}} sudaromi {\hyperref[\detokenize{savokos:term-SDSA}]{\sphinxtermref{\DUrole{xref}{\DUrole{std}{\DUrole{std-term}{ŠDSA}}}}}}
pagrindu. Įprastai duomenų bazių struktūra nėra kuriama vadovaujantis
standartais. Vidinės struktūros dažniausiai kuriamos vadovaujantis sistemai
keliamais reikalavimais. Todėl naujai atveriamų duomenų rinkiniai turi keliauti
į duomenų rinkinio vardų erdvę \sphinxcode{\sphinxupquote{/datasets/\{form\}/\{org\}/\{catalog\}/\{dataset\}/}},
išlaikant pirminę duomenų struktūrą ir neprarandant duomenų.

\sphinxAtStartPar
Tačiau su laiku, dalis įstaigos duomenų iš duomenų rinkinio vardų erdvės turėtu
būti perkeliami į globalią duomenų erdvę. Į globalią duomenų erdvę pirmiausiai
turėtų būti perkeliami tie duomenys, kurie yra plačiai naudojami. Perkėlimas į
globalią duomenų erdvę nepanaikina duomenų rinkinio iš ankstesnės vardų erdvės,
tiesiog duomenų rinkinio duomenų pagrindu kuriama kopija į globalią duomenų
erdvę.


\subsection{Reliatyvūs pavadinimai}
\label{\detokenize{vardu-erdves:reliatyvus-pavadinimai}}\label{\detokenize{vardu-erdves:relative-model-names}}
\sphinxAtStartPar
Modelio pavadinimas gali būti absoliutus arba reliatyvus. Absoliutūs
pavadinimai prasideda \sphinxcode{\sphinxupquote{/}} simboliu, reliatyvūs pavadinimai prasideda be \sphinxcode{\sphinxupquote{/}}
simbolio ir yra jungiami su vardų erdvės pavadinimu, kurios kontekste yra
apibrėžtas modelis.

\sphinxAtStartPar
Pavyzdžiui, turinti tokį duomenų struktūros aprašą:


\begin{savenotes}\sphinxattablestart
\sphinxthistablewithglobalstyle
\centering
\begin{tabulary}{\linewidth}[t]{TTTTTTT}
\sphinxtoprule
\sphinxstyletheadfamily 
\sphinxAtStartPar
id
&\sphinxstyletheadfamily 
\sphinxAtStartPar
d
&\sphinxstyletheadfamily 
\sphinxAtStartPar
r
&\sphinxstyletheadfamily 
\sphinxAtStartPar
b
&\sphinxstyletheadfamily 
\sphinxAtStartPar
m
&\sphinxstyletheadfamily 
\sphinxAtStartPar
property
&\sphinxstyletheadfamily 
\sphinxAtStartPar
type
\\
\sphinxmidrule
\sphinxtableatstartofbodyhook
\sphinxAtStartPar
1
&\sphinxstartmulticolumn{5}%
\begin{varwidth}[t]{\sphinxcolwidth{5}{7}}
\sphinxAtStartPar
\sphinxstylestrong{dcat}
\par
\vskip-\baselineskip\vbox{\hbox{\strut}}\end{varwidth}%
\sphinxstopmulticolumn
&
\sphinxAtStartPar
ns
\\
\sphinxhline
\sphinxAtStartPar
2
&&&&\sphinxstartmulticolumn{2}%
\begin{varwidth}[t]{\sphinxcolwidth{2}{7}}
\sphinxAtStartPar
\sphinxstylestrong{dataset}
\par
\vskip-\baselineskip\vbox{\hbox{\strut}}\end{varwidth}%
\sphinxstopmulticolumn
&\\
\sphinxhline
\sphinxAtStartPar
3
&&&&&
\sphinxAtStartPar
title
&\\
\sphinxhline
\sphinxAtStartPar
4
&\sphinxstartmulticolumn{5}%
\begin{varwidth}[t]{\sphinxcolwidth{5}{7}}
\sphinxAtStartPar
\sphinxstylestrong{datasets/gov/ivpk/adk}
\par
\vskip-\baselineskip\vbox{\hbox{\strut}}\end{varwidth}%
\sphinxstopmulticolumn
&\\
\sphinxhline
\sphinxAtStartPar
5
&&\sphinxstartmulticolumn{4}%
\begin{varwidth}[t]{\sphinxcolwidth{4}{7}}
\sphinxAtStartPar
adk
\par
\vskip-\baselineskip\vbox{\hbox{\strut}}\end{varwidth}%
\sphinxstopmulticolumn
&\\
\sphinxhline
\sphinxAtStartPar
6
&&&\sphinxstartmulticolumn{3}%
\begin{varwidth}[t]{\sphinxcolwidth{3}{7}}
\sphinxAtStartPar
\sphinxstylestrong{/dcat/dataset}
\par
\vskip-\baselineskip\vbox{\hbox{\strut}}\end{varwidth}%
\sphinxstopmulticolumn
&
\sphinxAtStartPar
alias
\\
\sphinxhline
\sphinxAtStartPar
7
&&&&\sphinxstartmulticolumn{2}%
\begin{varwidth}[t]{\sphinxcolwidth{2}{7}}
\sphinxAtStartPar
\sphinxstylestrong{dataset}
\par
\vskip-\baselineskip\vbox{\hbox{\strut}}\end{varwidth}%
\sphinxstopmulticolumn
&\\
\sphinxhline
\sphinxAtStartPar
8
&&&&&
\sphinxAtStartPar
title
&\\
\sphinxbottomrule
\end{tabulary}
\sphinxtableafterendhook\par
\sphinxattableend\end{savenotes}

\sphinxAtStartPar
Matome, kad yra apibrėžti du modeliai:
\begin{itemize}
\item {} 
\sphinxAtStartPar
\sphinxcode{\sphinxupquote{dcat/dataset}}

\item {} 
\sphinxAtStartPar
\sphinxcode{\sphinxupquote{datasets/gov/ivpk/adk/dataset}}

\end{itemize}

\sphinxAtStartPar
Vienas \sphinxcode{\sphinxupquote{dataset}} modelis yra apibrėžtas \sphinxcode{\sphinxupquote{dcat}} vardų erdvės kontekste, kitas
\sphinxcode{\sphinxupquote{datasets/gov/ivpk/adk}} vardų erdvės kontekste.

\sphinxAtStartPar
Kai modelio pavadinimas yra naudojamas vardų erdvės kontekste ir pavadinimas
neprasideda \sphinxcode{\sphinxupquote{/}} simboliu, tada tai yra reliatyvus modelio pavadinimas.
Reliatyvus modelio pavadinimas yra jungiamas su vardų erdvės pavadinimu,
kurios kontekste yra apibrėžtas modelis.

\sphinxAtStartPar
Jei tam tikros vardų erdvės kontekste norime įvardinti modelį, kuris yra už
tos vardų erdvės konteksto ribų, būtina naudoti absoliutų modelio pavadinimą,
kuris prasideda \sphinxcode{\sphinxupquote{/}} simboliu. Taip yra padaryta 6\sphinxhyphen{}oje eilutėje, kur nurodyta,
kad \sphinxcode{\sphinxupquote{datasets/gov/ivpk/adk/dataset}} bazė yra \sphinxcode{\sphinxupquote{dcat/dataset}} modelis iš kitos
vardų erdvės.

\sphinxAtStartPar
Visais atvejais, kai modelio pavadinimas naudojamas nenurodant jokio vardų
erdvės konteksto, \sphinxcode{\sphinxupquote{/}} simbolio pavadinimo pradžioje naudoti nereikia.
Pavyzdžiui šiame tekste įvardinti \sphinxcode{\sphinxupquote{dcat/dataset}} ir
\sphinxcode{\sphinxupquote{datasets/gov/ivpk/adk/dataset}} modelių pavadinimai neprasideda \sphinxcode{\sphinxupquote{/}} simboliu.

\sphinxstepscope


\section{Išoriniai žodynai}
\label{\detokenize{zodynai:isoriniai-zodynai}}\label{\detokenize{zodynai:vocab}}\label{\detokenize{zodynai::doc}}
\sphinxAtStartPar
Išorinių žodynų pagalba, galima susieti aprašomus duomenis su išoriniais
žodynais. Susiejimas atliekamas {\hyperref[\detokenize{dimensijos:model.uri}]{\sphinxcrossref{\sphinxcode{\sphinxupquote{model.uri}}}}} ir {\hyperref[\detokenize{dimensijos:property.uri}]{\sphinxcrossref{\sphinxcode{\sphinxupquote{property.uri}}}}},
naudojant \DUrole{xref}{\DUrole{std}{\DUrole{std-ref}{išorinių žodynų URI prefiksus}}}.

\sphinxAtStartPar
Pavyzdžiui turint tokį duomenų šaltinį:


\begin{savenotes}\sphinxattablestart
\sphinxthistablewithglobalstyle
\centering
\begin{tabulary}{\linewidth}[t]{TT}
\sphinxtoprule
\sphinxstartmulticolumn{2}%
\begin{varwidth}[t]{\sphinxcolwidth{2}{2}}
\sphinxstyletheadfamily \sphinxAtStartPar
country
\par
\vskip-\baselineskip\vbox{\hbox{\strut}}\end{varwidth}%
\sphinxstopmulticolumn
\\
\sphinxhline\sphinxstyletheadfamily 
\sphinxAtStartPar
id
&\sphinxstyletheadfamily 
\sphinxAtStartPar
name
\\
\sphinxmidrule
\sphinxtableatstartofbodyhook
\sphinxAtStartPar
1
&
\sphinxAtStartPar
Lietuva
\\
\sphinxbottomrule
\end{tabulary}
\sphinxtableafterendhook\par
\sphinxattableend\end{savenotes}


\begin{savenotes}\sphinxattablestart
\sphinxthistablewithglobalstyle
\centering
\begin{tabulary}{\linewidth}[t]{TTT}
\sphinxtoprule
\sphinxstartmulticolumn{3}%
\begin{varwidth}[t]{\sphinxcolwidth{3}{3}}
\sphinxstyletheadfamily \sphinxAtStartPar
city
\par
\vskip-\baselineskip\vbox{\hbox{\strut}}\end{varwidth}%
\sphinxstopmulticolumn
\\
\sphinxhline\sphinxstyletheadfamily 
\sphinxAtStartPar
id
&\sphinxstyletheadfamily 
\sphinxAtStartPar
name
&\sphinxstyletheadfamily 
\sphinxAtStartPar
country
\\
\sphinxmidrule
\sphinxtableatstartofbodyhook
\sphinxAtStartPar
1
&
\sphinxAtStartPar
Vilnius
&
\sphinxAtStartPar
1
\\
\sphinxbottomrule
\end{tabulary}
\sphinxtableafterendhook\par
\sphinxattableend\end{savenotes}

\sphinxAtStartPar
Ir šį šaltinį atitinkančią {\hyperref[\detokenize{savokos:term-DSA}]{\sphinxtermref{\DUrole{xref}{\DUrole{std}{\DUrole{std-term}{DSA}}}}}} lentelę:


\begin{savenotes}\sphinxattablestart
\sphinxthistablewithglobalstyle
\centering
\begin{tabulary}{\linewidth}[t]{TTTTTTTTT}
\sphinxtoprule
\sphinxstyletheadfamily 
\sphinxAtStartPar
id
&\sphinxstyletheadfamily 
\sphinxAtStartPar
d
&\sphinxstyletheadfamily 
\sphinxAtStartPar
r
&\sphinxstyletheadfamily 
\sphinxAtStartPar
b
&\sphinxstyletheadfamily 
\sphinxAtStartPar
m
&\sphinxstyletheadfamily 
\sphinxAtStartPar
property
&\sphinxstyletheadfamily 
\sphinxAtStartPar
type
&\sphinxstyletheadfamily 
\sphinxAtStartPar
ref
&\sphinxstyletheadfamily 
\sphinxAtStartPar
uri
\\
\sphinxmidrule
\sphinxtableatstartofbodyhook&&&&&&
\sphinxAtStartPar
prefix
&
\sphinxAtStartPar
locn
&
\sphinxAtStartPar
http://www.w3.org/ns/locn\#
\\
\sphinxhline&&&&&&&
\sphinxAtStartPar
dbpedia\sphinxhyphen{}owl
&
\sphinxAtStartPar
http://dbpedia.org/ontology/
\\
\sphinxhline&\sphinxstartmulticolumn{5}%
\begin{varwidth}[t]{\sphinxcolwidth{5}{9}}
\sphinxAtStartPar
datasets/example/geo
\par
\vskip-\baselineskip\vbox{\hbox{\strut}}\end{varwidth}%
\sphinxstopmulticolumn
&&&\\
\sphinxhline&&\sphinxstartmulticolumn{4}%
\begin{varwidth}[t]{\sphinxcolwidth{4}{9}}
\sphinxAtStartPar
salys
\par
\vskip-\baselineskip\vbox{\hbox{\strut}}\end{varwidth}%
\sphinxstopmulticolumn
&
\sphinxAtStartPar
sql
&&\\
\sphinxhline&&&&\sphinxstartmulticolumn{2}%
\begin{varwidth}[t]{\sphinxcolwidth{2}{9}}
\sphinxAtStartPar
Country
\par
\vskip-\baselineskip\vbox{\hbox{\strut}}\end{varwidth}%
\sphinxstopmulticolumn
&&
\sphinxAtStartPar
id
&
\sphinxAtStartPar
locn:Location
\\
\sphinxhline&&&&&
\sphinxAtStartPar
id
&
\sphinxAtStartPar
integer
&&
\sphinxAtStartPar
dct:identifier
\\
\sphinxhline&&&&&
\sphinxAtStartPar
name@lt
&
\sphinxAtStartPar
text
&&
\sphinxAtStartPar
locn:geographicName
\\
\sphinxhline&&&&&
\sphinxAtStartPar
capital
&
\sphinxAtStartPar
ref
&
\sphinxAtStartPar
City
&
\sphinxAtStartPar
dbpedia\sphinxhyphen{}owl:capital
\\
\sphinxhline&&&&\sphinxstartmulticolumn{2}%
\begin{varwidth}[t]{\sphinxcolwidth{2}{9}}
\sphinxAtStartPar
City
\par
\vskip-\baselineskip\vbox{\hbox{\strut}}\end{varwidth}%
\sphinxstopmulticolumn
&&
\sphinxAtStartPar
id
&
\sphinxAtStartPar
locn:Location
\\
\sphinxhline&&&&&
\sphinxAtStartPar
id
&
\sphinxAtStartPar
integer
&&
\sphinxAtStartPar
dct:identifier
\\
\sphinxhline&&&&&
\sphinxAtStartPar
name@lt
&
\sphinxAtStartPar
text
&&
\sphinxAtStartPar
locn:geographicName
\\
\sphinxhline&&&&&
\sphinxAtStartPar
country
&
\sphinxAtStartPar
ref
&
\sphinxAtStartPar
Country
&
\sphinxAtStartPar
dbpedia\sphinxhyphen{}owl:country
\\
\sphinxbottomrule
\end{tabulary}
\sphinxtableafterendhook\par
\sphinxattableend\end{savenotes}

\sphinxAtStartPar
Jei {\hyperref[\detokenize{formatas:property}]{\sphinxcrossref{\sphinxcode{\sphinxupquote{property}}}}} pavadinimai turi \sphinxcode{\sphinxupquote{@}} žymes, tada generuojant RDF, prie
reikšmės pridedama atitinkama kalbos žymė.

\sphinxAtStartPar
Galima duomenis eksportuoti \sphinxhref{https://en.wikipedia.org/wiki/Turtle\_(syntax)}{RDF Turtle} formatu, kas atrodytų taip:

\begin{sphinxVerbatim}[commandchars=\\\{\}]
@\PYG{n+nv}{base}\PYG{+w}{ }\PYG{o}{\PYGZlt{}}\PYG{n+nv}{https}:\PYG{o}{//}\PYG{n+nv}{get}.\PYG{n+nv}{data}.\PYG{n+nv}{gov}.\PYG{n+nv}{lt}\PYG{o}{/\PYGZgt{}}\PYG{+w}{ }.
@\PYG{n+nv}{prefix}\PYG{+w}{ }\PYG{n+nv}{dct}:\PYG{+w}{ }\PYG{o}{\PYGZlt{}}\PYG{n+nv}{http}:\PYG{o}{//}\PYG{n+nv}{purl}.\PYG{n+nv}{org}\PYG{o}{/}\PYG{n+nv}{dc}\PYG{o}{/}\PYG{n+nv}{terms}\PYG{o}{/\PYGZgt{}}\PYG{+w}{ }.
@\PYG{n+nv}{prefix}\PYG{+w}{ }\PYG{n+nv}{locn}:\PYG{+w}{ }\PYG{o}{\PYGZlt{}}\PYG{n+nv}{http}:\PYG{o}{//}\PYG{n+nv}{www}.\PYG{n+nv}{w3}.\PYG{n+nv}{org}\PYG{o}{/}\PYG{n+nv}{ns}\PYG{o}{/}\PYG{n+nv}{locn}\PYGZsh{}\PYG{o}{\PYGZgt{}}\PYG{+w}{ }.
@\PYG{n+nv}{prefix}\PYG{+w}{ }\PYG{n+nv}{dbpedia}\PYG{o}{\PYGZhy{}}\PYG{n+nv}{owl}:\PYG{+w}{ }\PYG{o}{\PYGZlt{}}\PYG{n+nv}{http}:\PYG{o}{//}\PYG{n+nv}{dbpedia}.\PYG{n+nv}{org}\PYG{o}{/}\PYG{n+nv}{ontology}\PYG{o}{/\PYGZgt{}}\PYG{+w}{ }.

\PYG{o}{\PYGZlt{}}\PYG{n+nv}{datasets}\PYG{o}{/}\PYG{n+nv}{example}\PYG{o}{/}\PYG{n+nv}{geo}\PYG{o}{/}\PYG{n+nv}{Country}\PYG{o}{/}\PYG{n+nv}{eb09946c}\PYG{o}{\PYGZhy{}}\PYG{l+m+mi}{26}\PYG{n+nv}{e1}\PYG{o}{\PYGZhy{}}\PYG{l+m+mi}{4698}\PYG{o}{\PYGZhy{}}\PYG{n+nv}{a298}\PYG{o}{\PYGZhy{}}\PYG{l+m+mi}{7}\PYG{n+nv}{bb1e468b165}\PYG{o}{\PYGZgt{}}
\PYG{+w}{    }\PYG{n+nv}{a}\PYG{+w}{ }\PYG{n+nv}{locn}:\PYG{n+nv}{Location}\PYG{+w}{ }\PYG{c+c1}{;}
\PYG{+w}{    }\PYG{n+nv}{dct}:\PYG{n+nv}{identifier}\PYG{+w}{ }\PYG{l+m+mi}{1}\PYG{+w}{ }\PYG{c+c1}{;}
\PYG{+w}{    }\PYG{n+nv}{locn}:\PYG{n+nv}{geographicName}\PYG{+w}{ }\PYG{l+s+s2}{\PYGZdq{}Lietuva\PYGZdq{}}@\PYG{n+nv}{lt}\PYG{+w}{ }\PYG{c+c1}{;}
\PYG{+w}{    }\PYG{n+nv}{dbpedia}\PYG{o}{\PYGZhy{}}\PYG{n+nv}{owl}:\PYG{n+nv}{capital}\PYG{+w}{ }\PYG{o}{\PYGZlt{}}\PYG{n+nv}{datasets}\PYG{o}{/}\PYG{n+nv}{example}\PYG{o}{/}\PYG{n+nv}{geo}\PYG{o}{/}\PYG{n+nv}{City}\PYG{o}{/}\PYG{n+nv}{b54c21f6}\PYG{o}{\PYGZhy{}}\PYG{l+m+mi}{08}\PYG{n+nv}{b8}\PYG{o}{\PYGZhy{}}\PYG{l+m+mi}{4}\PYG{n+nv}{bdd}\PYG{o}{\PYGZhy{}}\PYG{n+nv}{b785}\PYG{o}{\PYGZhy{}}\PYG{n+nv}{be1cb2e93a98}\PYG{o}{\PYGZgt{}}\PYG{+w}{ }.

\PYG{o}{\PYGZlt{}}\PYG{n+nv}{datasets}\PYG{o}{/}\PYG{n+nv}{example}\PYG{o}{/}\PYG{n+nv}{geo}\PYG{o}{/}\PYG{n+nv}{City}\PYG{o}{/}\PYG{n+nv}{b54c21f6}\PYG{o}{\PYGZhy{}}\PYG{l+m+mi}{08}\PYG{n+nv}{b8}\PYG{o}{\PYGZhy{}}\PYG{l+m+mi}{4}\PYG{n+nv}{bdd}\PYG{o}{\PYGZhy{}}\PYG{n+nv}{b785}\PYG{o}{\PYGZhy{}}\PYG{n+nv}{be1cb2e93a98}\PYG{o}{\PYGZgt{}}
\PYG{+w}{    }\PYG{n+nv}{a}\PYG{+w}{ }\PYG{n+nv}{locn}:\PYG{n+nv}{Location}\PYG{+w}{ }\PYG{c+c1}{;}
\PYG{+w}{    }\PYG{n+nv}{dct}:\PYG{n+nv}{identifier}\PYG{+w}{ }\PYG{l+m+mi}{1}\PYG{+w}{ }\PYG{c+c1}{;}
\PYG{+w}{    }\PYG{n+nv}{locn}:\PYG{n+nv}{geographicName}\PYG{+w}{ }\PYG{l+s+s2}{\PYGZdq{}Vilnius\PYGZdq{}}@\PYG{n+nv}{lt}\PYG{+w}{ }\PYG{c+c1}{;}
\PYG{+w}{    }\PYG{n+nv}{dbpedia}\PYG{o}{\PYGZhy{}}\PYG{n+nv}{owl}:\PYG{n+nv}{country}\PYG{+w}{ }\PYG{o}{\PYGZlt{}}\PYG{n+nv}{datasets}\PYG{o}{/}\PYG{n+nv}{example}\PYG{o}{/}\PYG{n+nv}{geo}\PYG{o}{/}\PYG{n+nv}{Country}\PYG{o}{/}\PYG{n+nv}{eb09946c}\PYG{o}{\PYGZhy{}}\PYG{l+m+mi}{26}\PYG{n+nv}{e1}\PYG{o}{\PYGZhy{}}\PYG{l+m+mi}{4698}\PYG{o}{\PYGZhy{}}\PYG{n+nv}{a298}\PYG{o}{\PYGZhy{}}\PYG{l+m+mi}{7}\PYG{n+nv}{bb1e468b165}\PYG{o}{\PYGZgt{}}\PYG{+w}{ }.
\end{sphinxVerbatim}

\sphinxAtStartPar
Analogiškai, tie patys duomenys gali būti eksportuojami \sphinxhref{https://www.w3.org/TR/rdf-syntax-grammar/}{RDF/XML} formatu:

\begin{sphinxVerbatim}[commandchars=\\\{\}]
\PYG{c+cp}{\PYGZlt{}?xml version=\PYGZdq{}1.0\PYGZdq{} encoding=\PYGZdq{}utf\PYGZhy{}8\PYGZdq{}?\PYGZgt{}}
\PYG{n+nt}{\PYGZlt{}rdf:RDF}
\PYG{+w}{    }\PYG{n+na}{xml:base=}\PYG{l+s}{\PYGZdq{}https://get.data.gov.lt/\PYGZdq{}}
\PYG{+w}{    }\PYG{n+na}{xmlns:dct=}\PYG{l+s}{\PYGZdq{}http://purl.org/dc/terms/\PYGZdq{}}
\PYG{+w}{    }\PYG{n+na}{xmlns:locn=}\PYG{l+s}{\PYGZdq{}http://www.w3.org/ns/locn\PYGZsh{}\PYGZdq{}}
\PYG{+w}{    }\PYG{n+na}{xmlns:dbpedia\PYGZhy{}owl=}\PYG{l+s}{\PYGZdq{}http://dbpedia.org/ontology/\PYGZdq{}}
\PYG{+w}{    }\PYG{n+na}{xmlns:rdf=}\PYG{l+s}{\PYGZdq{}http://www.w3.org/1999/02/22\PYGZhy{}rdf\PYGZhy{}syntax\PYGZhy{}ns\PYGZsh{}\PYGZdq{}}\PYG{n+nt}{\PYGZgt{}}

\PYG{+w}{    }\PYG{n+nt}{\PYGZlt{}locn:Location}
\PYG{+w}{        }\PYG{n+na}{rdf:about=}\PYG{l+s}{\PYGZdq{}datasets/example/geo/Country/eb09946c\PYGZhy{}26e1\PYGZhy{}4698\PYGZhy{}a298\PYGZhy{}7bb1e468b165\PYGZdq{}}
\PYG{+w}{        }\PYG{n+na}{dct:identifier=}\PYG{l+s}{\PYGZdq{}1\PYGZdq{}}\PYG{n+nt}{\PYGZgt{}}
\PYG{+w}{        }\PYG{n+nt}{\PYGZlt{}locn:geographicName}\PYG{+w}{ }\PYG{n+na}{xml:lang=}\PYG{l+s}{\PYGZdq{}lt\PYGZdq{}}\PYG{n+nt}{\PYGZgt{}}Lietuva\PYG{n+nt}{\PYGZlt{}/locn:geographicName\PYGZgt{}}
\PYG{+w}{        }\PYG{n+nt}{\PYGZlt{}dbpedia\PYGZhy{}owl:capital}
\PYG{+w}{            }\PYG{n+na}{rdf:resource=}\PYG{l+s}{\PYGZdq{}datasets/example/geo/City/b54c21f6\PYGZhy{}08b8\PYGZhy{}4bdd\PYGZhy{}b785\PYGZhy{}be1cb2e93a98\PYGZdq{}}\PYG{+w}{ }\PYG{n+nt}{/\PYGZgt{}}
\PYG{+w}{    }\PYG{n+nt}{\PYGZlt{}/locn:Location\PYGZgt{}}

\PYG{+w}{    }\PYG{n+nt}{\PYGZlt{}locn:Location}
\PYG{+w}{        }\PYG{n+na}{rdf:about=}\PYG{l+s}{\PYGZdq{}datasets/example/geo/City/b54c21f6\PYGZhy{}08b8\PYGZhy{}4bdd\PYGZhy{}b785\PYGZhy{}be1cb2e93a98\PYGZdq{}}
\PYG{+w}{        }\PYG{n+na}{dct:identifier=}\PYG{l+s}{\PYGZdq{}1\PYGZdq{}}\PYG{n+nt}{\PYGZgt{}}
\PYG{+w}{        }\PYG{n+nt}{\PYGZlt{}locn:geographicName}\PYG{+w}{ }\PYG{n+na}{xml:lang=}\PYG{l+s}{\PYGZdq{}lt\PYGZdq{}}\PYG{n+nt}{\PYGZgt{}}Vilnius\PYG{n+nt}{\PYGZlt{}/locn:geographicName\PYGZgt{}}
\PYG{+w}{        }\PYG{n+nt}{\PYGZlt{}dbpedia\PYGZhy{}owl:country}
\PYG{+w}{            }\PYG{n+na}{rdf:resource=}\PYG{l+s}{\PYGZdq{}datasets/example/geo/Country/eb09946c\PYGZhy{}26e1\PYGZhy{}4698\PYGZhy{}a298\PYGZhy{}7bb1e468b165\PYGZdq{}}\PYG{+w}{ }\PYG{n+nt}{/\PYGZgt{}}
\PYG{+w}{    }\PYG{n+nt}{\PYGZlt{}/locn:Location\PYGZgt{}}
\PYG{n+nt}{\PYGZlt{}/rdf:RDF\PYGZgt{}}
\end{sphinxVerbatim}

\sphinxAtStartPar
Išoriniai žodynai suteikia galimybę eksportuoti duomenis {\hyperref[\detokenize{savokos:term-RDF}]{\sphinxtermref{\DUrole{xref}{\DUrole{std}{\DUrole{std-term}{RDF}}}}}} formatu.

\sphinxAtStartPar
Jei struktūros apraše nėra užpildytas {\hyperref[\detokenize{formatas:uri}]{\sphinxcrossref{\sphinxcode{\sphinxupquote{uri}}}}} stulpelis, tada, turėtu būti
generuojamas tokie RDF duomenys:

\begin{sphinxVerbatim}[commandchars=\\\{\}]
@\PYG{n+nv}{base}\PYG{+w}{ }\PYG{o}{\PYGZlt{}}\PYG{n+nv}{https}:\PYG{o}{//}\PYG{n+nv}{get}.\PYG{n+nv}{data}.\PYG{n+nv}{gov}.\PYG{n+nv}{lt}\PYG{o}{/\PYGZgt{}}\PYG{+w}{ }.
@\PYG{n+nv}{prefix}\PYG{+w}{ }\PYG{n+nv}{dct}:\PYG{+w}{ }\PYG{o}{\PYGZlt{}}\PYG{n+nv}{http}:\PYG{o}{//}\PYG{n+nv}{purl}.\PYG{n+nv}{org}\PYG{o}{/}\PYG{n+nv}{dc}\PYG{o}{/}\PYG{n+nv}{terms}\PYG{o}{/\PYGZgt{}}\PYG{+w}{ }.
@\PYG{n+nv}{prefix}\PYG{+w}{ }\PYG{n+nv}{locn}:\PYG{+w}{ }\PYG{o}{\PYGZlt{}}\PYG{n+nv}{http}:\PYG{o}{//}\PYG{n+nv}{www}.\PYG{n+nv}{w3}.\PYG{n+nv}{org}\PYG{o}{/}\PYG{n+nv}{ns}\PYG{o}{/}\PYG{n+nv}{locn}\PYGZsh{}\PYG{o}{\PYGZgt{}}\PYG{+w}{ }.
@\PYG{n+nv}{prefix}\PYG{+w}{ }\PYG{n+nv}{dbpedia}\PYG{o}{\PYGZhy{}}\PYG{n+nv}{owl}:\PYG{+w}{ }\PYG{o}{\PYGZlt{}}\PYG{n+nv}{http}:\PYG{o}{//}\PYG{n+nv}{dbpedia}.\PYG{n+nv}{org}\PYG{o}{/}\PYG{n+nv}{ontology}\PYG{o}{/\PYGZgt{}}\PYG{+w}{ }.

\PYG{o}{\PYGZlt{}}\PYG{n+nv}{datasets}\PYG{o}{/}\PYG{n+nv}{example}\PYG{o}{/}\PYG{n+nv}{geo}\PYG{o}{/}\PYG{n+nv}{Country}\PYG{o}{/}\PYG{n+nv}{eb09946c}\PYG{o}{\PYGZhy{}}\PYG{l+m+mi}{26}\PYG{n+nv}{e1}\PYG{o}{\PYGZhy{}}\PYG{l+m+mi}{4698}\PYG{o}{\PYGZhy{}}\PYG{n+nv}{a298}\PYG{o}{\PYGZhy{}}\PYG{l+m+mi}{7}\PYG{n+nv}{bb1e468b165}\PYG{o}{\PYGZgt{}}
\PYG{+w}{    }\PYG{n+nv}{a}\PYG{+w}{ }\PYG{o}{\PYGZlt{}}\PYG{n+nv}{datasets}\PYG{o}{/}\PYG{n+nv}{example}\PYG{o}{/}\PYG{n+nv}{geo}\PYG{o}{/}\PYG{n+nv}{Country}\PYG{o}{\PYGZgt{}}\PYG{+w}{ }\PYG{c+c1}{;}
\PYG{+w}{    }\PYG{o}{\PYGZlt{}}\PYG{n+nv}{datasets}\PYG{o}{/}\PYG{n+nv}{example}\PYG{o}{/}\PYG{n+nv}{geo}\PYG{o}{/}\PYG{n+nv}{Country}\PYG{o}{/}\PYG{n+nv}{id}\PYG{o}{\PYGZgt{}}\PYG{+w}{ }\PYG{l+m+mi}{1}\PYG{+w}{ }\PYG{c+c1}{;}
\PYG{+w}{    }\PYG{o}{\PYGZlt{}}\PYG{n+nv}{datasets}\PYG{o}{/}\PYG{n+nv}{example}\PYG{o}{/}\PYG{n+nv}{geo}\PYG{o}{/}\PYG{n+nv}{Country}\PYG{o}{/}\PYG{n+nv}{name}\PYG{o}{\PYGZgt{}}\PYG{+w}{ }\PYG{l+s+s2}{\PYGZdq{}Lietuva\PYGZdq{}}@\PYG{n+nv}{lt}\PYG{+w}{ }\PYG{c+c1}{;}
\PYG{+w}{    }\PYG{o}{\PYGZlt{}}\PYG{n+nv}{datasets}\PYG{o}{/}\PYG{n+nv}{example}\PYG{o}{/}\PYG{n+nv}{geo}\PYG{o}{/}\PYG{n+nv}{Country}\PYG{o}{/}\PYG{n+nv}{capital}\PYG{o}{\PYGZgt{}}\PYG{+w}{ }\PYG{o}{\PYGZlt{}}\PYG{n+nv}{datasets}\PYG{o}{/}\PYG{n+nv}{example}\PYG{o}{/}\PYG{n+nv}{geo}\PYG{o}{/}\PYG{n+nv}{City}\PYG{o}{/}\PYG{n+nv}{b54c21f6}\PYG{o}{\PYGZhy{}}\PYG{l+m+mi}{08}\PYG{n+nv}{b8}\PYG{o}{\PYGZhy{}}\PYG{l+m+mi}{4}\PYG{n+nv}{bdd}\PYG{o}{\PYGZhy{}}\PYG{n+nv}{b785}\PYG{o}{\PYGZhy{}}\PYG{n+nv}{be1cb2e93a98}\PYG{o}{\PYGZgt{}}\PYG{+w}{ }.

\PYG{o}{\PYGZlt{}}\PYG{n+nv}{datasets}\PYG{o}{/}\PYG{n+nv}{example}\PYG{o}{/}\PYG{n+nv}{geo}\PYG{o}{/}\PYG{n+nv}{City}\PYG{o}{/}\PYG{n+nv}{b54c21f6}\PYG{o}{\PYGZhy{}}\PYG{l+m+mi}{08}\PYG{n+nv}{b8}\PYG{o}{\PYGZhy{}}\PYG{l+m+mi}{4}\PYG{n+nv}{bdd}\PYG{o}{\PYGZhy{}}\PYG{n+nv}{b785}\PYG{o}{\PYGZhy{}}\PYG{n+nv}{be1cb2e93a98}\PYG{o}{\PYGZgt{}}
\PYG{+w}{    }\PYG{n+nv}{a}\PYG{+w}{ }\PYG{o}{\PYGZlt{}}\PYG{n+nv}{datasets}\PYG{o}{/}\PYG{n+nv}{example}\PYG{o}{/}\PYG{n+nv}{geo}\PYG{o}{/}\PYG{n+nv}{City}\PYG{o}{\PYGZgt{}}\PYG{+w}{ }\PYG{c+c1}{;}
\PYG{+w}{    }\PYG{o}{\PYGZlt{}}\PYG{n+nv}{datasets}\PYG{o}{/}\PYG{n+nv}{example}\PYG{o}{/}\PYG{n+nv}{geo}\PYG{o}{/}\PYG{n+nv}{City}\PYG{o}{/}\PYG{n+nv}{id}\PYG{o}{\PYGZgt{}}\PYG{+w}{ }\PYG{l+m+mi}{1}\PYG{+w}{ }\PYG{c+c1}{;}
\PYG{+w}{    }\PYG{o}{\PYGZlt{}}\PYG{n+nv}{datasets}\PYG{o}{/}\PYG{n+nv}{example}\PYG{o}{/}\PYG{n+nv}{geo}\PYG{o}{/}\PYG{n+nv}{City}\PYG{o}{/}\PYG{n+nv}{name}\PYG{o}{\PYGZgt{}}\PYG{+w}{ }\PYG{l+s+s2}{\PYGZdq{}Vilnius\PYGZdq{}}@\PYG{n+nv}{lt}\PYG{+w}{ }\PYG{c+c1}{;}
\PYG{+w}{    }\PYG{o}{\PYGZlt{}}\PYG{n+nv}{datasets}\PYG{o}{/}\PYG{n+nv}{example}\PYG{o}{/}\PYG{n+nv}{geo}\PYG{o}{/}\PYG{n+nv}{City}\PYG{o}{/}\PYG{n+nv}{country}\PYG{o}{\PYGZgt{}}\PYG{+w}{ }\PYG{o}{\PYGZlt{}}\PYG{n+nv}{datasets}\PYG{o}{/}\PYG{n+nv}{example}\PYG{o}{/}\PYG{n+nv}{geo}\PYG{o}{/}\PYG{n+nv}{Country}\PYG{o}{/}\PYG{n+nv}{eb09946c}\PYG{o}{\PYGZhy{}}\PYG{l+m+mi}{26}\PYG{n+nv}{e1}\PYG{o}{\PYGZhy{}}\PYG{l+m+mi}{4698}\PYG{o}{\PYGZhy{}}\PYG{n+nv}{a298}\PYG{o}{\PYGZhy{}}\PYG{l+m+mi}{7}\PYG{n+nv}{bb1e468b165}\PYG{o}{\PYGZgt{}}\PYG{+w}{ }.
\end{sphinxVerbatim}


\subsection{Subjekto URI}
\label{\detokenize{zodynai:subjekto-uri}}
\sphinxAtStartPar
Pagal nutylėjimą {\hyperref[\detokenize{savokos:term-subjektas}]{\sphinxtermref{\DUrole{xref}{\DUrole{std}{\DUrole{std-term}{subjekto}}}}}} URI yra automatiškai generuojamas
ir atrodo taip:

\begin{sphinxVerbatim}[commandchars=\\\{\}]
\PYG{n}{https}\PYG{p}{:}\PYG{o}{/}\PYG{o}{/}\PYG{n}{get}\PYG{o}{.}\PYG{n}{data}\PYG{o}{.}\PYG{n}{gov}\PYG{o}{.}\PYG{n}{lt}\PYG{o}{/}\PYG{n}{datasets}\PYG{o}{/}\PYG{n}{example}\PYG{o}{/}\PYG{n}{geo}\PYG{o}{/}\PYG{n}{Country}\PYG{o}{/}\PYG{n}{eb09946c}\PYG{o}{\PYGZhy{}}\PYG{l+m+mf}{26e1}\PYG{o}{\PYGZhy{}}\PYG{l+m+mi}{4698}\PYG{o}{\PYGZhy{}}\PYG{n}{a298}\PYG{o}{\PYGZhy{}}\PYG{l+m+mi}{7}\PYG{n}{bb1e468b165}
\end{sphinxVerbatim}

\sphinxAtStartPar
Tačiau naudojant kontroliuojamus žodynus, galima nurodyti kitą identifikatorių
tokiu būdu:


\begin{savenotes}\sphinxattablestart
\sphinxthistablewithglobalstyle
\centering
\begin{tabulary}{\linewidth}[t]{TTTTT}
\sphinxtoprule
\sphinxstyletheadfamily 
\sphinxAtStartPar
m
&\sphinxstyletheadfamily 
\sphinxAtStartPar
property
&\sphinxstyletheadfamily 
\sphinxAtStartPar
type
&\sphinxstyletheadfamily 
\sphinxAtStartPar
ref
&\sphinxstyletheadfamily 
\sphinxAtStartPar
uri
\\
\sphinxmidrule
\sphinxtableatstartofbodyhook\sphinxstartmulticolumn{2}%
\begin{varwidth}[t]{\sphinxcolwidth{2}{5}}
\sphinxAtStartPar
Country
\par
\vskip-\baselineskip\vbox{\hbox{\strut}}\end{varwidth}%
\sphinxstopmulticolumn
&&
\sphinxAtStartPar
id
&
\sphinxAtStartPar
locn:Location
\\
\sphinxhline
\sphinxAtStartPar

&
\sphinxAtStartPar
id
&
\sphinxAtStartPar
integer
&&\\
\sphinxhline
\sphinxAtStartPar

&
\sphinxAtStartPar
uri
&
\sphinxAtStartPar
uri
&&
\sphinxAtStartPar
locn:Location
\\
\sphinxhline
\sphinxAtStartPar

&
\sphinxAtStartPar
name@lt
&
\sphinxAtStartPar
text
&&\\
\sphinxbottomrule
\end{tabulary}
\sphinxtableafterendhook\par
\sphinxattableend\end{savenotes}

\sphinxAtStartPar
Jei {\hyperref[\detokenize{dimensijos:property.uri}]{\sphinxcrossref{\sphinxcode{\sphinxupquote{property.uri}}}}} sutampa su {\hyperref[\detokenize{dimensijos:model.uri}]{\sphinxcrossref{\sphinxcode{\sphinxupquote{model.uri}}}}} ir {\hyperref[\detokenize{dimensijos:property.type}]{\sphinxcrossref{\sphinxcode{\sphinxupquote{property.type}}}}}
yra {\hyperref[\detokenize{formatas:uri}]{\sphinxcrossref{\sphinxcode{\sphinxupquote{uri}}}}}, tada formuojant duomenis RDF formatu naudojame ne generuotą
subjekto URI, o naudojame lauko reikšmę, kurio {\hyperref[\detokenize{dimensijos:property.uri}]{\sphinxcrossref{\sphinxcode{\sphinxupquote{property.uri}}}}} sutampa su
{\hyperref[\detokenize{dimensijos:model.uri}]{\sphinxcrossref{\sphinxcode{\sphinxupquote{model.uri}}}}}.

\sphinxAtStartPar
Gali būti ne daugiau kaip vienas {\hyperref[\detokenize{dimensijos:property.uri}]{\sphinxcrossref{\sphinxcode{\sphinxupquote{property.uri}}}}} su {\hyperref[\detokenize{dimensijos:property.type}]{\sphinxcrossref{\sphinxcode{\sphinxupquote{property.type}}}}}
{\hyperref[\detokenize{formatas:uri}]{\sphinxcrossref{\sphinxcode{\sphinxupquote{uri}}}}}, kuris sutampa su {\hyperref[\detokenize{dimensijos:model.uri}]{\sphinxcrossref{\sphinxcode{\sphinxupquote{model.uri}}}}}.

\sphinxAtStartPar
Jei yra keli {\hyperref[\detokenize{formatas:uri}]{\sphinxcrossref{\sphinxcode{\sphinxupquote{uri}}}}} tipo laukai, kurie identifikuoja tą patį subjektą,
tada kitiems atvejams reikia naudoti ne {\hyperref[\detokenize{dimensijos:model.uri}]{\sphinxcrossref{\sphinxcode{\sphinxupquote{model.uri}}}}}, o \sphinxhref{https://www.w3.org/TR/owl-ref/\#sameAs-def}{owl:sameAs}.

\sphinxAtStartPar
Jei \sphinxcode{\sphinxupquote{uri}} reikšmė bus \sphinxcode{\sphinxupquote{https://sws.geonames.org/597427/}}, tada gautume tokius
RDF duomenis:

\begin{sphinxVerbatim}[commandchars=\\\{\}]
@\PYG{n+nv}{base}\PYG{+w}{ }\PYG{o}{\PYGZlt{}}\PYG{n+nv}{https}:\PYG{o}{//}\PYG{n+nv}{get}.\PYG{n+nv}{data}.\PYG{n+nv}{gov}.\PYG{n+nv}{lt}\PYG{o}{/}\PYG{n+nv}{example}\PYG{o}{/\PYGZgt{}}\PYG{+w}{ }.
@\PYG{n+nv}{prefix}\PYG{+w}{ }\PYG{n+nv}{locn}:\PYG{+w}{ }\PYG{o}{\PYGZlt{}}\PYG{n+nv}{http}:\PYG{o}{//}\PYG{n+nv}{www}.\PYG{n+nv}{w3}.\PYG{n+nv}{org}\PYG{o}{/}\PYG{n+nv}{ns}\PYG{o}{/}\PYG{n+nv}{locn}\PYGZsh{}\PYG{o}{\PYGZgt{}}\PYG{+w}{ }.

\PYG{o}{\PYGZlt{}}\PYG{n+nv}{https}:\PYG{o}{//}\PYG{n+nv}{sws}.\PYG{n+nv}{geonames}.\PYG{n+nv}{org}\PYG{o}{/}\PYG{l+m+mi}{597427}\PYG{o}{/\PYGZgt{}}
\PYG{+w}{    }\PYG{n+nv}{a}\PYG{+w}{ }\PYG{n+nv}{locn}:\PYG{n+nv}{Location}\PYG{+w}{ }\PYG{c+c1}{;}
\PYG{+w}{    }\PYG{o}{\PYGZlt{}}\PYG{n+nv}{Country}\PYG{o}{/}\PYG{n+nv}{id}\PYG{o}{\PYGZgt{}}\PYG{+w}{ }\PYG{l+m+mi}{1}\PYG{+w}{ }\PYG{c+c1}{;}
\PYG{+w}{    }\PYG{o}{\PYGZlt{}}\PYG{n+nv}{Country}\PYG{o}{/}\PYG{n+nv}{name}\PYG{o}{\PYGZgt{}}\PYG{+w}{ }\PYG{l+s+s2}{\PYGZdq{}Lietuva\PYGZdq{}}@\PYG{n+nv}{lt}\PYG{+w}{ }.
\end{sphinxVerbatim}

\sphinxAtStartPar
Atkreipkite dėmesį, kad pats \sphinxcode{\sphinxupquote{uri}} laukas nėra įtrautkas į RDF duomenis.

\sphinxAtStartPar
Analogiškai, jei {\hyperref[\detokenize{formatas:ref}]{\sphinxcrossref{\sphinxcode{\sphinxupquote{ref}}}}} tipo laukas rodo į modelį, kurio {\hyperref[\detokenize{dimensijos:model.uri}]{\sphinxcrossref{\sphinxcode{\sphinxupquote{model.uri}}}}}
sutampa su {\hyperref[\detokenize{dimensijos:property.uri}]{\sphinxcrossref{\sphinxcode{\sphinxupquote{property.uri}}}}} kuris yra {\hyperref[\detokenize{formatas:ref}]{\sphinxcrossref{\sphinxcode{\sphinxupquote{ref}}}}} tipo, tada {\hyperref[\detokenize{formatas:ref}]{\sphinxcrossref{\sphinxcode{\sphinxupquote{ref}}}}}
lauko reikšmė taip pat įgyja ne generuotą URI, o URI iš duomenų.

\sphinxAtStartPar
Pratęsiant tą patį pavyzdį:


\begin{savenotes}\sphinxattablestart
\sphinxthistablewithglobalstyle
\centering
\begin{tabulary}{\linewidth}[t]{TTTTT}
\sphinxtoprule
\sphinxstyletheadfamily 
\sphinxAtStartPar
m
&\sphinxstyletheadfamily 
\sphinxAtStartPar
property
&\sphinxstyletheadfamily 
\sphinxAtStartPar
type
&\sphinxstyletheadfamily 
\sphinxAtStartPar
ref
&\sphinxstyletheadfamily 
\sphinxAtStartPar
uri
\\
\sphinxmidrule
\sphinxtableatstartofbodyhook\sphinxstartmulticolumn{2}%
\begin{varwidth}[t]{\sphinxcolwidth{2}{5}}
\sphinxAtStartPar
Country
\par
\vskip-\baselineskip\vbox{\hbox{\strut}}\end{varwidth}%
\sphinxstopmulticolumn
&&
\sphinxAtStartPar
id
&
\sphinxAtStartPar
locn:Location
\\
\sphinxhline
\sphinxAtStartPar

&
\sphinxAtStartPar
id
&
\sphinxAtStartPar
integer
&&\\
\sphinxhline
\sphinxAtStartPar

&
\sphinxAtStartPar
uri
&
\sphinxAtStartPar
uri
&&
\sphinxAtStartPar
locn:Location
\\
\sphinxhline
\sphinxAtStartPar

&
\sphinxAtStartPar
name@lt
&
\sphinxAtStartPar
text
&&\\
\sphinxhline\sphinxstartmulticolumn{2}%
\begin{varwidth}[t]{\sphinxcolwidth{2}{5}}
\sphinxAtStartPar
City
\par
\vskip-\baselineskip\vbox{\hbox{\strut}}\end{varwidth}%
\sphinxstopmulticolumn
&&
\sphinxAtStartPar
id
&
\sphinxAtStartPar
locn:Location
\\
\sphinxhline
\sphinxAtStartPar

&
\sphinxAtStartPar
id
&
\sphinxAtStartPar
integer
&&\\
\sphinxhline
\sphinxAtStartPar

&
\sphinxAtStartPar
name@lt
&
\sphinxAtStartPar
text
&&\\
\sphinxhline
\sphinxAtStartPar

&
\sphinxAtStartPar
country
&
\sphinxAtStartPar
ref
&
\sphinxAtStartPar
Country
&
\sphinxAtStartPar
locn:Location
\\
\sphinxbottomrule
\end{tabulary}
\sphinxtableafterendhook\par
\sphinxattableend\end{savenotes}

\sphinxAtStartPar
Gautume tokius duomenis:

\begin{sphinxVerbatim}[commandchars=\\\{\}]
@\PYG{n+nv}{base}\PYG{+w}{ }\PYG{o}{\PYGZlt{}}\PYG{n+nv}{https}:\PYG{o}{//}\PYG{n+nv}{get}.\PYG{n+nv}{data}.\PYG{n+nv}{gov}.\PYG{n+nv}{lt}\PYG{o}{/}\PYG{n+nv}{example}\PYG{o}{/\PYGZgt{}}\PYG{+w}{ }.
@\PYG{n+nv}{prefix}\PYG{+w}{ }\PYG{n+nv}{locn}:\PYG{+w}{ }\PYG{o}{\PYGZlt{}}\PYG{n+nv}{http}:\PYG{o}{//}\PYG{n+nv}{www}.\PYG{n+nv}{w3}.\PYG{n+nv}{org}\PYG{o}{/}\PYG{n+nv}{ns}\PYG{o}{/}\PYG{n+nv}{locn}\PYGZsh{}\PYG{o}{\PYGZgt{}}\PYG{+w}{ }.

\PYG{o}{\PYGZlt{}}\PYG{n+nv}{https}:\PYG{o}{//}\PYG{n+nv}{sws}.\PYG{n+nv}{geonames}.\PYG{n+nv}{org}\PYG{o}{/}\PYG{l+m+mi}{597427}\PYG{o}{/\PYGZgt{}}
\PYG{+w}{    }\PYG{n+nv}{a}\PYG{+w}{ }\PYG{n+nv}{locn}:\PYG{n+nv}{Location}\PYG{+w}{ }\PYG{c+c1}{;}
\PYG{+w}{    }\PYG{o}{\PYGZlt{}}\PYG{n+nv}{Country}\PYG{o}{/}\PYG{n+nv}{id}\PYG{o}{\PYGZgt{}}\PYG{+w}{ }\PYG{l+m+mi}{1}\PYG{+w}{ }\PYG{c+c1}{;}
\PYG{+w}{    }\PYG{o}{\PYGZlt{}}\PYG{n+nv}{Country}\PYG{o}{/}\PYG{n+nv}{name}\PYG{o}{\PYGZgt{}}\PYG{+w}{ }\PYG{l+s+s2}{\PYGZdq{}Lietuva\PYGZdq{}}@\PYG{n+nv}{lt}\PYG{+w}{ }.

\PYG{o}{\PYGZlt{}}\PYG{n+nv}{City}\PYG{o}{/}\PYG{n+nv}{b54c21f6}\PYG{o}{\PYGZhy{}}\PYG{l+m+mi}{08}\PYG{n+nv}{b8}\PYG{o}{\PYGZhy{}}\PYG{l+m+mi}{4}\PYG{n+nv}{bdd}\PYG{o}{\PYGZhy{}}\PYG{n+nv}{b785}\PYG{o}{\PYGZhy{}}\PYG{n+nv}{be1cb2e93a98}\PYG{o}{\PYGZgt{}}
\PYG{+w}{    }\PYG{n+nv}{a}\PYG{+w}{ }\PYG{n+nv}{locn}:\PYG{n+nv}{Location}\PYG{+w}{ }\PYG{c+c1}{;}
\PYG{+w}{    }\PYG{o}{\PYGZlt{}}\PYG{n+nv}{City}\PYG{o}{/}\PYG{n+nv}{id}\PYG{o}{\PYGZgt{}}\PYG{+w}{ }\PYG{l+m+mi}{1}\PYG{+w}{ }\PYG{c+c1}{;}
\PYG{+w}{    }\PYG{o}{\PYGZlt{}}\PYG{n+nv}{City}\PYG{o}{/}\PYG{n+nv}{name}\PYG{o}{\PYGZgt{}}\PYG{+w}{ }\PYG{l+s+s2}{\PYGZdq{}Vilnius\PYGZdq{}}@\PYG{n+nv}{lt}\PYG{+w}{ }\PYG{c+c1}{;}
\PYG{+w}{    }\PYG{o}{\PYGZlt{}}\PYG{n+nv}{City}\PYG{o}{/}\PYG{n+nv}{country}\PYG{o}{\PYGZgt{}}\PYG{+w}{ }\PYG{o}{\PYGZlt{}}\PYG{n+nv}{https}:\PYG{o}{//}\PYG{n+nv}{sws}.\PYG{n+nv}{geonames}.\PYG{n+nv}{org}\PYG{o}{/}\PYG{l+m+mi}{597427}\PYG{o}{/\PYGZgt{}}\PYG{+w}{ }.
\end{sphinxVerbatim}

\sphinxstepscope


\section{Formulės}
\label{\detokenize{formules:formules}}\label{\detokenize{formules:id1}}\label{\detokenize{formules::doc}}
\sphinxAtStartPar
Formulės rašomos {\hyperref[\detokenize{formatas:prepare}]{\sphinxcrossref{\sphinxcode{\sphinxupquote{prepare}}}}} stulpelyje. Formulių pagalba galima atlikti
įvairius duomenų transformavimo, nuasmeninimo, filtravimo ir kokybės tikrinimo
veiksmus.

\sphinxAtStartPar
Kadangi yra labai didelė įvairovė duomenų formatų ir duomenų valdymo mechanizmų,
siekiant suvaldyti visą šią įvairovę {\hyperref[\detokenize{savokos:term-DSA}]{\sphinxtermref{\DUrole{xref}{\DUrole{std}{\DUrole{std-term}{DSA}}}}}} formulės leidžia vieningai
aprašyti veiksmus su duomenimis. Vėliau formulės verčiamos į vieningą \sphinxhref{https://en.wikipedia.org/wiki/Abstract\_syntax\_tree}{AST}, kurį
gali interpretuoti automatizuotos priemonės, priklausomai nuo duomenų šaltinio
ir konteksto ir DSA sluoksnio.


\subsection{Gramatika}
\label{\detokenize{formules:gramatika}}
\sphinxAtStartPar
Formulių sintaksė atitinką šią \sphinxhref{https://en.wikipedia.org/wiki/Augmented\_Backus–Naur\_form}{ABNF} gramatiką:

\begin{sphinxVerbatim}[commandchars=\\\{\}]
\PYG{n+nc}{formula}\PYG{+w}{     }\PYG{o}{=}\PYG{+w}{ }\PYG{n+nc}{testlist}
\PYG{n+nc}{testlist}\PYG{+w}{    }\PYG{o}{=}\PYG{+w}{ }\PYG{n+nc}{test}\PYG{+w}{ }\PYG{o}{*}\PYG{p}{(}\PYG{l}{\PYGZdq{},\PYGZdq{}}\PYG{+w}{ }\PYG{n+nc}{test}\PYG{p}{)}\PYG{+w}{ }\PYG{o}{*}\PYG{o}{1}\PYG{l}{\PYGZdq{},\PYGZdq{}}
\PYG{n+nc}{test}\PYG{+w}{        }\PYG{o}{=}\PYG{+w}{ }\PYG{n+nc}{or}
\PYG{n+nc}{or}\PYG{+w}{          }\PYG{o}{=}\PYG{+w}{ }\PYG{n+nc}{and}\PYG{+w}{ }\PYG{o}{*}\PYG{p}{(}\PYG{l}{\PYGZdq{}|\PYGZdq{}}\PYG{+w}{ }\PYG{n+nc}{and}\PYG{p}{)}
\PYG{n+nc}{and}\PYG{+w}{         }\PYG{o}{=}\PYG{+w}{ }\PYG{n+nc}{not}\PYG{+w}{ }\PYG{o}{*}\PYG{p}{(}\PYG{l}{\PYGZdq{}\PYGZam{}\PYGZdq{}}\PYG{+w}{ }\PYG{n+nc}{not}\PYG{p}{)}
\PYG{n+nc}{not}\PYG{+w}{         }\PYG{o}{=}\PYG{+w}{ }\PYG{l}{\PYGZdq{}!\PYGZdq{}}\PYG{+w}{ }\PYG{n+nc}{not}\PYG{+w}{ }\PYG{o}{/}\PYG{+w}{ }\PYG{n+nc}{comp}
\PYG{n+nc}{comp}\PYG{+w}{        }\PYG{o}{=}\PYG{+w}{ }\PYG{n+nc}{expr}\PYG{+w}{ }\PYG{o}{*}\PYG{p}{(}\PYG{n+nc}{compop}\PYG{+w}{ }\PYG{n+nc}{expr}\PYG{p}{)}
\PYG{n+nc}{expr}\PYG{+w}{        }\PYG{o}{=}\PYG{+w}{ }\PYG{n+nc}{term}\PYG{+w}{ }\PYG{o}{*}\PYG{p}{(}\PYG{n+nc}{termop}\PYG{+w}{ }\PYG{n+nc}{term}\PYG{p}{)}
\PYG{n+nc}{term}\PYG{+w}{        }\PYG{o}{=}\PYG{+w}{ }\PYG{n+nc}{factor}\PYG{+w}{ }\PYG{o}{*}\PYG{p}{(}\PYG{n+nc}{factorop}\PYG{+w}{ }\PYG{n+nc}{factor}\PYG{p}{)}
\PYG{n+nc}{factor}\PYG{+w}{      }\PYG{o}{=}\PYG{+w}{ }\PYG{n+nc}{sign}\PYG{+w}{ }\PYG{n+nc}{factor}\PYG{+w}{ }\PYG{o}{/}\PYG{+w}{ }\PYG{n+nc}{composition}
\PYG{n+nc}{composition}\PYG{+w}{ }\PYG{o}{=}\PYG{+w}{ }\PYG{n+nc}{atom}\PYG{+w}{ }\PYG{o}{*}\PYG{n+nc}{trailer}
\PYG{n+nc}{atom}\PYG{+w}{        }\PYG{o}{=}\PYG{+w}{ }\PYG{l}{\PYGZdq{}(\PYGZdq{}}\PYG{+w}{ }\PYG{o}{*}\PYG{o}{1}\PYG{n+nc}{group}\PYG{+w}{ }\PYG{l}{\PYGZdq{})\PYGZdq{}}
\PYG{+w}{            }\PYG{o}{/}\PYG{+w}{ }\PYG{l}{\PYGZdq{}[\PYGZdq{}}\PYG{+w}{ }\PYG{o}{*}\PYG{o}{1}\PYG{n+nc}{list}\PYG{+w}{ }\PYG{l}{\PYGZdq{}]\PYGZdq{}}
\PYG{+w}{            }\PYG{o}{/}\PYG{+w}{ }\PYG{n+nc}{func}\PYG{+w}{ }\PYG{o}{/}\PYG{+w}{ }\PYG{n+nc}{value}\PYG{+w}{ }\PYG{o}{/}\PYG{+w}{ }\PYG{n+nc}{name}
\PYG{n+nc}{group}\PYG{+w}{       }\PYG{o}{=}\PYG{+w}{ }\PYG{n+nc}{test}\PYG{+w}{ }\PYG{o}{*}\PYG{p}{(}\PYG{l}{\PYGZdq{},\PYGZdq{}}\PYG{+w}{ }\PYG{n+nc}{test}\PYG{p}{)}\PYG{+w}{ }\PYG{o}{*}\PYG{o}{1}\PYG{l}{\PYGZdq{},\PYGZdq{}}
\PYG{n+nc}{list}\PYG{+w}{        }\PYG{o}{=}\PYG{+w}{ }\PYG{n+nc}{test}\PYG{+w}{ }\PYG{o}{*}\PYG{p}{(}\PYG{l}{\PYGZdq{},\PYGZdq{}}\PYG{+w}{ }\PYG{n+nc}{test}\PYG{p}{)}\PYG{+w}{ }\PYG{o}{*}\PYG{o}{1}\PYG{l}{\PYGZdq{},\PYGZdq{}}
\PYG{n+nc}{trailer}\PYG{+w}{     }\PYG{o}{=}\PYG{+w}{ }\PYG{l}{\PYGZdq{}[\PYGZdq{}}\PYG{+w}{ }\PYG{o}{*}\PYG{o}{1}\PYG{n+nc}{filter}\PYG{+w}{ }\PYG{l}{\PYGZdq{}]\PYGZdq{}}\PYG{+w}{ }\PYG{o}{/}\PYG{+w}{ }\PYG{n+nc}{method}\PYG{+w}{ }\PYG{o}{/}\PYG{+w}{ }\PYG{n+nc}{attr}
\PYG{n+nc}{func}\PYG{+w}{        }\PYG{o}{=}\PYG{+w}{ }\PYG{n+nc}{name}\PYG{+w}{ }\PYG{n+nc}{call}
\PYG{n+nc}{method}\PYG{+w}{      }\PYG{o}{=}\PYG{+w}{ }\PYG{l}{\PYGZdq{}.\PYGZdq{}}\PYG{+w}{ }\PYG{n+nc}{name}\PYG{+w}{ }\PYG{n+nc}{call}
\PYG{n+nc}{call}\PYG{+w}{        }\PYG{o}{=}\PYG{+w}{ }\PYG{l}{\PYGZdq{}(\PYGZdq{}}\PYG{+w}{ }\PYG{o}{*}\PYG{o}{1}\PYG{n+nc}{arglist}\PYG{+w}{ }\PYG{l}{\PYGZdq{})\PYGZdq{}}
\PYG{n+nc}{arglist}\PYG{+w}{     }\PYG{o}{=}\PYG{+w}{ }\PYG{n+nc}{argument}\PYG{+w}{ }\PYG{o}{*}\PYG{p}{(}\PYG{l}{\PYGZdq{},\PYGZdq{}}\PYG{+w}{ }\PYG{n+nc}{argument}\PYG{p}{)}\PYG{+w}{ }\PYG{o}{*}\PYG{o}{1}\PYG{l}{\PYGZdq{},\PYGZdq{}}
\PYG{n+nc}{argument}\PYG{+w}{    }\PYG{o}{=}\PYG{+w}{ }\PYG{n+nc}{test}\PYG{+w}{ }\PYG{o}{/}\PYG{+w}{ }\PYG{n+nc}{kwarg}
\PYG{n+nc}{kwarg}\PYG{+w}{       }\PYG{o}{=}\PYG{+w}{ }\PYG{n+nc}{name}\PYG{+w}{ }\PYG{l}{\PYGZdq{}:\PYGZdq{}}\PYG{+w}{ }\PYG{n+nc}{test}
\PYG{n+nc}{filter}\PYG{+w}{      }\PYG{o}{=}\PYG{+w}{ }\PYG{n+nc}{test}\PYG{+w}{ }\PYG{o}{*}\PYG{p}{(}\PYG{l}{\PYGZdq{},\PYGZdq{}}\PYG{+w}{ }\PYG{n+nc}{test}\PYG{p}{)}\PYG{+w}{ }\PYG{o}{*}\PYG{o}{1}\PYG{l}{\PYGZdq{},\PYGZdq{}}
\PYG{n+nc}{attr}\PYG{+w}{        }\PYG{o}{=}\PYG{+w}{ }\PYG{l}{\PYGZdq{}.\PYGZdq{}}\PYG{+w}{ }\PYG{n+nc}{name}
\PYG{n+nc}{value}\PYG{+w}{       }\PYG{o}{=}\PYG{+w}{ }\PYG{n+nc}{null}\PYG{+w}{ }\PYG{o}{/}\PYG{+w}{ }\PYG{n+nc}{bool}\PYG{+w}{ }\PYG{o}{/}\PYG{+w}{ }\PYG{n+nc}{integer}\PYG{+w}{ }\PYG{o}{/}\PYG{+w}{ }\PYG{n+nc}{number}\PYG{+w}{ }\PYG{o}{/}\PYG{+w}{ }\PYG{n+nc}{string}\PYG{+w}{ }\PYG{o}{/}\PYG{+w}{ }\PYG{n+nc}{star}
\PYG{n+nc}{compop}\PYG{+w}{      }\PYG{o}{=}\PYG{+w}{ }\PYG{l}{\PYGZdq{}\PYGZgt{}=\PYGZdq{}}\PYG{+w}{ }\PYG{o}{/}\PYG{+w}{ }\PYG{l}{\PYGZdq{}\PYGZlt{}=\PYGZdq{}}\PYG{+w}{ }\PYG{o}{/}\PYG{+w}{ }\PYG{l}{\PYGZdq{}!=\PYGZdq{}}\PYG{+w}{ }\PYG{o}{/}\PYG{+w}{ }\PYG{l}{\PYGZdq{}=\PYGZdq{}}\PYG{+w}{ }\PYG{o}{/}\PYG{+w}{ }\PYG{l}{\PYGZdq{}\PYGZlt{}\PYGZdq{}}\PYG{+w}{ }\PYG{o}{/}\PYG{+w}{ }\PYG{l}{\PYGZdq{}\PYGZgt{}\PYGZdq{}}
\PYG{n+nc}{termpop}\PYG{+w}{     }\PYG{o}{=}\PYG{+w}{ }\PYG{l}{\PYGZdq{}+\PYGZdq{}}\PYG{+w}{ }\PYG{o}{/}\PYG{+w}{ }\PYG{l}{\PYGZdq{}\PYGZhy{}\PYGZdq{}}
\PYG{n+nc}{factorop}\PYG{+w}{    }\PYG{o}{=}\PYG{+w}{ }\PYG{l}{\PYGZdq{}*\PYGZdq{}}\PYG{+w}{ }\PYG{o}{/}\PYG{+w}{ }\PYG{l}{\PYGZdq{}/\PYGZdq{}}\PYG{+w}{ }\PYG{o}{/}\PYG{+w}{ }\PYG{l}{\PYGZdq{}\PYGZpc{}\PYGZdq{}}
\PYG{n+nc}{sign}\PYG{+w}{        }\PYG{o}{=}\PYG{+w}{ }\PYG{l}{\PYGZdq{}+\PYGZdq{}}\PYG{+w}{ }\PYG{o}{/}\PYG{+w}{ }\PYG{l}{\PYGZdq{}\PYGZhy{}\PYGZdq{}}
\PYG{n+nc}{star}\PYG{+w}{        }\PYG{o}{=}\PYG{+w}{ }\PYG{l}{\PYGZdq{}*\PYGZdq{}}
\PYG{n+nc}{name}\PYG{+w}{        }\PYG{o}{=}\PYG{+w}{ }\PYGZti{}\PYG{o}{/}\PYG{p}{[}\PYG{n+nc}{a\PYGZhy{}}z\PYGZus{}\PYG{p}{]}\PYG{p}{[}\PYG{n+nc}{a\PYGZhy{}z0\PYGZhy{}}\PYG{o}{9}\PYGZus{}\PYG{p}{]}\PYG{o}{*}\PYG{o}{/}\PYG{n+nc}{i}
\PYG{n+nc}{number}\PYG{+w}{      }\PYG{o}{=}\PYG{+w}{ }\PYGZti{}\PYG{o}{/}\PYGZbs{}\PYG{n+nc}{d}+\PYG{p}{(}\PYGZbs{}.\PYGZbs{}\PYG{n+nc}{d}+\PYG{p}{)}?\PYG{o}{/}
\PYG{n+nc}{integer}\PYG{+w}{     }\PYG{o}{=}\PYG{+w}{ }\PYGZti{}\PYG{o}{/}\PYG{o}{0}|\PYG{p}{[}\PYG{o}{1}\PYGZhy{}\PYG{o}{9}\PYG{p}{]}\PYGZbs{}\PYG{n+nc}{d}\PYG{o}{*}\PYG{o}{/}
\PYG{n+nc}{bool}\PYG{+w}{        }\PYG{o}{=}\PYG{+w}{ }\PYG{l}{\PYGZdq{}false\PYGZdq{}}\PYG{+w}{ }\PYG{o}{/}\PYG{+w}{ }\PYG{l}{\PYGZdq{}true\PYGZdq{}}
\PYG{n+nc}{null}\PYG{+w}{        }\PYG{o}{=}\PYG{+w}{ }\PYG{l}{\PYGZdq{}null\PYGZdq{}}
\PYG{n+nc}{string}\PYG{+w}{      }\PYG{o}{=}\PYG{+w}{ }\PYGZti{}\PYG{o}{/}\PYG{p}{(}?!\PYG{l}{\PYGZdq{}\PYGZdq{}}\PYG{p}{)}.\PYG{o}{*}?\PYG{p}{(}?\PYGZlt{}!\PYGZbs{}\PYGZbs{}\PYG{p}{)}\PYG{p}{(}\PYGZbs{}\PYGZbs{}\PYGZbs{}\PYGZbs{}\PYG{p}{)}\PYG{o}{*}?\PYGZdq{}|\PYGZsq{}\PYG{p}{(}?!\PYGZsq{}\PYGZsq{}\PYG{p}{)}.\PYG{o}{*}?
\PYG{+w}{                }\PYG{p}{(}?\PYGZlt{}!\PYGZbs{}\PYGZbs{}\PYG{p}{)}\PYG{p}{(}\PYGZbs{}\PYGZbs{}\PYGZbs{}\PYGZbs{}\PYG{p}{)}\PYG{o}{*}?\PYGZsq{}\PYG{o}{/}\PYG{n+nc}{i}
\end{sphinxVerbatim}


\subsection{Sintaksės medis}
\label{\detokenize{formules:sintakses-medis}}
\sphinxAtStartPar
Formulės verčiamos į vieningą abstraktų sintaksės medį. Vieningas abstraktus
sintaksės medis leidžia atskirti formulės skaitymo ir interpretavimo veiklas.

\sphinxAtStartPar
Abstraktus sintaksės medis sudarytas iš vienodų elementų turinčių tokias
savybes:


\begin{fulllineitems}

\pysigstartsignatures
\pysigline
{\sphinxbfcode{\sphinxupquote{name}}}
\pysigstopsignatures
\sphinxAtStartPar
Funkcijos pavadinimas.

\end{fulllineitems}



\begin{fulllineitems}

\pysigstartsignatures
\pysigline
{\sphinxbfcode{\sphinxupquote{args}}}
\pysigstopsignatures
\sphinxAtStartPar
Funkcijos argumentų sąrašas, kurį gali sudaryti konkrečios reikšmės ar kiti
medžio elementai, veiksmai.

\end{fulllineitems}


\sphinxAtStartPar
Visos formulėje naudojamos išraiškos sintaksės medyje verčiamos į funkcijų ir
argumentų medį. Pavyzdžiui \sphinxcode{\sphinxupquote{test("a", "b")}} bus verčiamas į:

\begin{sphinxVerbatim}[commandchars=\\\{\}]
\PYG{p}{\PYGZob{}}
    \PYG{l+s+s2}{\PYGZdq{}}\PYG{l+s+s2}{name}\PYG{l+s+s2}{\PYGZdq{}}\PYG{p}{:} \PYG{l+s+s2}{\PYGZdq{}}\PYG{l+s+s2}{test}\PYG{l+s+s2}{\PYGZdq{}}\PYG{p}{,}
    \PYG{l+s+s2}{\PYGZdq{}}\PYG{l+s+s2}{args}\PYG{l+s+s2}{\PYGZdq{}}\PYG{p}{:} \PYG{p}{[}\PYG{l+s+s2}{\PYGZdq{}}\PYG{l+s+s2}{a}\PYG{l+s+s2}{\PYGZdq{}}\PYG{p}{,} \PYG{l+s+s2}{\PYGZdq{}}\PYG{l+s+s2}{b}\PYG{l+s+s2}{\PYGZdq{}}\PYG{p}{]}\PYG{p}{,}
\PYG{p}{\PYGZcb{}}
\end{sphinxVerbatim}


\subsection{Funkcijų iškvietimas}
\label{\detokenize{formules:funkciju-iskvietimas}}
\sphinxAtStartPar
Formulės susideda iš vykdomų funkcijų sekos. Pavyzdžiui funkcijos pavadinimu
\sphinxcode{\sphinxupquote{test}} vykdymas formulėje atrodys taip:

\begin{sphinxVerbatim}[commandchars=\\\{\}]
\PYG{n}{test}\PYG{p}{(}\PYG{p}{)}
\end{sphinxVerbatim}

\sphinxAtStartPar
Aukščiau pavyzdyje pateikta formulė vykdo funkciją \sphinxcode{\sphinxupquote{test}}, be argumentų. Tačiau
funkcijos gali turėti pozicinius ir vardinius argumentus.


\subsection{Poziciniai argumentai}
\label{\detokenize{formules:poziciniai-argumentai}}
\sphinxAtStartPar
Poziciniai argumentai perduodami taip:

\begin{sphinxVerbatim}[commandchars=\\\{\}]
\PYG{n}{test}\PYG{p}{(}\PYG{n}{a}\PYG{p}{,} \PYG{n}{b}\PYG{p}{,} \PYG{n}{c}\PYG{p}{)}
\end{sphinxVerbatim}

\sphinxAtStartPar
Pavyzdyje, funkcijai \sphinxcode{\sphinxupquote{test}} perduodami trys argumentai \sphinxcode{\sphinxupquote{a}}, \sphinxcode{\sphinxupquote{b}} ir \sphinxcode{\sphinxupquote{c}}. Šioje
dokumentacijoje, tais atvejais, kai funkcijos pozicinių argumentų skaičius nėra
fiksuotas, naudojama \sphinxcode{\sphinxupquote{*args}} išraiška, kur \sphinxcode{\sphinxupquote{*}} nurodo, kad pozicinių argumentų
gali būti 0 ar daugiau.


\subsection{Vardiniai argumentai}
\label{\detokenize{formules:vardiniai-argumentai}}
\sphinxAtStartPar
Vardiniai argumentai funkcijai perduodami taip:

\begin{sphinxVerbatim}[commandchars=\\\{\}]
\PYG{n}{test}\PYG{p}{(}\PYG{n}{a}\PYG{p}{:} \PYG{l+m+mi}{1}\PYG{p}{,} \PYG{n}{b}\PYG{p}{:} \PYG{l+m+mi}{2}\PYG{p}{,} \PYG{n}{c}\PYG{p}{:} \PYG{l+m+mi}{3}\PYG{p}{)}
\end{sphinxVerbatim}

\sphinxAtStartPar
Pozicinius argumentus būtina perduoti tiksliai tokia tvarka, kokios tikisi
funkcija. Tačiau vardinius argumentus, galima perduoti, bet kuria tvarka.

\sphinxAtStartPar
Jei vardinių argumentų sąrašas nėra fiksuotas, dokumentacijoje toks argumentų
sąrašas užrašomas \sphinxcode{\sphinxupquote{**kwargs}} forma, kur \sphinxcode{\sphinxupquote{**}} nurodo, kad vardinių argumentų
gali būti 0 ar daugiau.


\subsection{Alternatyvus funkcijos iškvietimas}
\label{\detokenize{formules:alternatyvus-funkcijos-iskvietimas}}
\sphinxAtStartPar
Funkcijų iškvietimas gali būti užrašomas įprastiniu būdu, pavyzdžiui:

\begin{sphinxVerbatim}[commandchars=\\\{\}]
\PYG{n}{test}\PYG{p}{(}\PYG{n}{test}\PYG{p}{(}\PYG{n}{test}\PYG{p}{(}\PYG{n}{a}\PYG{p}{)}\PYG{p}{,} \PYG{n}{b}\PYG{p}{)}\PYG{p}{,} \PYG{n}{c}\PYG{p}{)}
\end{sphinxVerbatim}

\sphinxAtStartPar
Arba funkcijų grandinės (angl. \sphinxhref{https://en.wikipedia.org/wiki/Method\_chaining}{Method chain}) būdu:

\begin{sphinxVerbatim}[commandchars=\\\{\}]
\PYG{n}{a}\PYG{o}{.}\PYG{n}{test}\PYG{p}{(}\PYG{p}{)}\PYG{o}{.}\PYG{n}{test}\PYG{p}{(}\PYG{n}{b}\PYG{p}{)}\PYG{o}{.}\PYG{n}{test}\PYG{p}{(}\PYG{n}{c}\PYG{p}{)}
\end{sphinxVerbatim}

\sphinxAtStartPar
Kadangi formulės dažnai naudojamos tam tikros reikšmės transformavimui, todėl
dažnai formulė yra lengviau skaitoma, naudojant funkcijų grandinę.

\sphinxAtStartPar
\sphinxcode{\sphinxupquote{test(a)}} yra \sphinxcode{\sphinxupquote{a.test()}} arba \sphinxcode{\sphinxupquote{test(a, b)}} ir \sphinxcode{\sphinxupquote{a.test(b)}} yra ekvivalentūs
(\sphinxhref{https://en.wikipedia.org/wiki/Uniform\_Function\_Call\_Syntax}{UFCS}).


\subsection{Standartinės funkcijos}
\label{\detokenize{formules:standartines-funkcijos}}
\sphinxAtStartPar
Priklausomai nuo duomenų šaltinio ar konteksto gali būti naudojami skirtingi
veiksmai, tačiau žemiau yra pateikti bendrosios paskirties veiksmai:
\index{built\sphinxhyphen{}in function@\spxentry{built\sphinxhyphen{}in function}!bind()@\spxentry{bind()}}\index{bind()@\spxentry{bind()}!built\sphinxhyphen{}in function@\spxentry{built\sphinxhyphen{}in function}}

\begin{fulllineitems}
\phantomsection\label{\detokenize{formules:bind}}
\pysigstartsignatures
\pysiglinewithargsret
{\sphinxbfcode{\sphinxupquote{bind}}}
{\sphinxparam{\DUrole{n}{name}}}
{}
\pysigstopsignatures
\sphinxAtStartPar
Rodo į reikšmę pavadinimu \sphinxcode{\sphinxupquote{name}} iš konteksto. Reikšmės ieškoma tokia
tvarka:
\begin{itemize}
\item {} 
\sphinxAtStartPar
{\hyperref[\detokenize{formules:var}]{\sphinxcrossref{\sphinxcode{\sphinxupquote{var()}}}}}

\item {} 
\sphinxAtStartPar
{\hyperref[\detokenize{formules:param}]{\sphinxcrossref{\sphinxcode{\sphinxupquote{param()}}}}}

\item {} 
\sphinxAtStartPar
{\hyperref[\detokenize{formules:item}]{\sphinxcrossref{\sphinxcode{\sphinxupquote{item()}}}}}

\item {} 
\sphinxAtStartPar
{\hyperref[\detokenize{formules:prop}]{\sphinxcrossref{\sphinxcode{\sphinxupquote{prop()}}}}}

\end{itemize}

\end{fulllineitems}

\index{built\sphinxhyphen{}in function@\spxentry{built\sphinxhyphen{}in function}!prop()@\spxentry{prop()}}\index{prop()@\spxentry{prop()}!built\sphinxhyphen{}in function@\spxentry{built\sphinxhyphen{}in function}}

\begin{fulllineitems}
\phantomsection\label{\detokenize{formules:prop}}
\pysigstartsignatures
\pysiglinewithargsret
{\sphinxbfcode{\sphinxupquote{prop}}}
{\sphinxparam{\DUrole{n}{name}}}
{}
\pysigstopsignatures
\sphinxAtStartPar
Modelio savybė pavadinimu \sphinxcode{\sphinxupquote{name}} iš {\hyperref[\detokenize{formatas:property}]{\sphinxcrossref{\sphinxcode{\sphinxupquote{property}}}}} stulpelio.

\end{fulllineitems}

\index{built\sphinxhyphen{}in function@\spxentry{built\sphinxhyphen{}in function}!item()@\spxentry{item()}}\index{item()@\spxentry{item()}!built\sphinxhyphen{}in function@\spxentry{built\sphinxhyphen{}in function}}

\begin{fulllineitems}
\phantomsection\label{\detokenize{formules:item}}
\pysigstartsignatures
\pysiglinewithargsret
{\sphinxbfcode{\sphinxupquote{item}}}
{\sphinxparam{\DUrole{n}{name}}}
{}
\pysigstopsignatures
\sphinxAtStartPar
Sąrašo elemento savybė pavadinimu \sphinxcode{\sphinxupquote{name}}.

\end{fulllineitems}

\index{built\sphinxhyphen{}in function@\spxentry{built\sphinxhyphen{}in function}!param()@\spxentry{param()}}\index{param()@\spxentry{param()}!built\sphinxhyphen{}in function@\spxentry{built\sphinxhyphen{}in function}}

\begin{fulllineitems}
\phantomsection\label{\detokenize{formules:param}}
\pysigstartsignatures
\pysiglinewithargsret
{\sphinxbfcode{\sphinxupquote{param}}}
{\sphinxparam{\DUrole{n}{name}}}
{}
\pysigstopsignatures
\sphinxAtStartPar
Parametras pavadinimu \sphinxcode{\sphinxupquote{name}}. Žiūrėti {\hyperref[\detokenize{dimensijos:param}]{\sphinxcrossref{\DUrole{std}{\DUrole{std-ref}{param}}}}}.

\end{fulllineitems}

\index{built\sphinxhyphen{}in function@\spxentry{built\sphinxhyphen{}in function}!var()@\spxentry{var()}}\index{var()@\spxentry{var()}!built\sphinxhyphen{}in function@\spxentry{built\sphinxhyphen{}in function}}

\begin{fulllineitems}
\phantomsection\label{\detokenize{formules:var}}
\pysigstartsignatures
\pysiglinewithargsret
{\sphinxbfcode{\sphinxupquote{var}}}
{\sphinxparam{\DUrole{n}{name}}}
{}
\pysigstopsignatures
\sphinxAtStartPar
Kintamasis apibrėžtas {\hyperref[\detokenize{formules:id16}]{\sphinxcrossref{\sphinxcode{\sphinxupquote{set()}}}}} funkcijos pagalba.

\end{fulllineitems}

\index{built\sphinxhyphen{}in function@\spxentry{built\sphinxhyphen{}in function}!self()@\spxentry{self()}}\index{self()@\spxentry{self()}!built\sphinxhyphen{}in function@\spxentry{built\sphinxhyphen{}in function}}

\begin{fulllineitems}
\phantomsection\label{\detokenize{formules:self}}
\pysigstartsignatures
\pysiglinewithargsret
{\sphinxbfcode{\sphinxupquote{self}}}
{}
{}
\pysigstopsignatures
\sphinxAtStartPar
Rodo į aktyvią reikšmę, naudojamas {\hyperref[\detokenize{dimensijos:property.prepare}]{\sphinxcrossref{\sphinxcode{\sphinxupquote{property.prepare}}}}} formulėse.

\end{fulllineitems}

\index{built\sphinxhyphen{}in function@\spxentry{built\sphinxhyphen{}in function}!or()@\spxentry{or()}}\index{or()@\spxentry{or()}!built\sphinxhyphen{}in function@\spxentry{built\sphinxhyphen{}in function}}

\begin{fulllineitems}
\phantomsection\label{\detokenize{formules:or}}
\pysigstartsignatures
\pysiglinewithargsret
{\sphinxbfcode{\sphinxupquote{or}}}
{\sphinxparam{\DUrole{o}{*}\DUrole{n}{args}}}
{}
\pysigstopsignatures
\sphinxAtStartPar
Taip pat galima naudoti tokia išraiška:

\begin{sphinxVerbatim}[commandchars=\\\{\}]
\PYG{n}{a} \PYG{o}{|} \PYG{n}{b} \PYG{o}{|} \PYG{n}{c}
\end{sphinxVerbatim}

\sphinxAtStartPar
Grąžiną pirmą netuščią reikšmę. Pirmoji netuščia reikšmė nutraukia sekančių
\sphinxcode{\sphinxupquote{args}} argumentų interpretavimą.

\end{fulllineitems}

\index{built\sphinxhyphen{}in function@\spxentry{built\sphinxhyphen{}in function}!and()@\spxentry{and()}}\index{and()@\spxentry{and()}!built\sphinxhyphen{}in function@\spxentry{built\sphinxhyphen{}in function}}

\begin{fulllineitems}
\phantomsection\label{\detokenize{formules:and}}
\pysigstartsignatures
\pysiglinewithargsret
{\sphinxbfcode{\sphinxupquote{and}}}
{\sphinxparam{\DUrole{o}{*}\DUrole{n}{args}}}
{}
\pysigstopsignatures
\sphinxAtStartPar
Taip pat galima naudoti tokia išraiška:

\begin{sphinxVerbatim}[commandchars=\\\{\}]
\PYG{n}{a} \PYG{o}{\PYGZam{}} \PYG{n}{b} \PYG{o}{\PYGZam{}} \PYG{n}{c}
\end{sphinxVerbatim}

\sphinxAtStartPar
Grąžina pirmą tuščią reikšmę arba paskutinę reikšmę, jei prieš tai esančios
reikšmės netuščios.

\end{fulllineitems}

\index{built\sphinxhyphen{}in function@\spxentry{built\sphinxhyphen{}in function}!not()@\spxentry{not()}}\index{not()@\spxentry{not()}!built\sphinxhyphen{}in function@\spxentry{built\sphinxhyphen{}in function}}

\begin{fulllineitems}
\phantomsection\label{\detokenize{formules:not}}
\pysigstartsignatures
\pysiglinewithargsret
{\sphinxbfcode{\sphinxupquote{not}}}
{\sphinxparam{\DUrole{n}{arg}}}
{}
\pysigstopsignatures
\sphinxAtStartPar
Taip pat galima naudoti tokia išraiška:

\begin{sphinxVerbatim}[commandchars=\\\{\}]
!arg
\end{sphinxVerbatim}

\sphinxAtStartPar
Jei \sphinxcode{\sphinxupquote{arg}} tuščia grąžina \sphinxcode{\sphinxupquote{true}}, priešingu atveju \sphinxcode{\sphinxupquote{false}}.

\end{fulllineitems}

\index{built\sphinxhyphen{}in function@\spxentry{built\sphinxhyphen{}in function}!eq()@\spxentry{eq()}}\index{eq()@\spxentry{eq()}!built\sphinxhyphen{}in function@\spxentry{built\sphinxhyphen{}in function}}

\begin{fulllineitems}
\phantomsection\label{\detokenize{formules:eq}}
\pysigstartsignatures
\pysiglinewithargsret
{\sphinxbfcode{\sphinxupquote{eq}}}
{\sphinxparam{\DUrole{n}{a}}\sphinxparamcomma \sphinxparam{\DUrole{n}{b}}}
{}
\pysigstopsignatures
\sphinxAtStartPar
Taip pat galima naudoti tokia išraiška:

\begin{sphinxVerbatim}[commandchars=\\\{\}]
\PYG{n}{a} \PYG{o}{=} \PYG{n}{b}
\end{sphinxVerbatim}

\sphinxAtStartPar
\sphinxcode{\sphinxupquote{a}} lygus \sphinxcode{\sphinxupquote{b}}.

\end{fulllineitems}

\index{built\sphinxhyphen{}in function@\spxentry{built\sphinxhyphen{}in function}!ne()@\spxentry{ne()}}\index{ne()@\spxentry{ne()}!built\sphinxhyphen{}in function@\spxentry{built\sphinxhyphen{}in function}}

\begin{fulllineitems}
\phantomsection\label{\detokenize{formules:ne}}
\pysigstartsignatures
\pysiglinewithargsret
{\sphinxbfcode{\sphinxupquote{ne}}}
{\sphinxparam{\DUrole{n}{a}}\sphinxparamcomma \sphinxparam{\DUrole{n}{b}}}
{}
\pysigstopsignatures
\sphinxAtStartPar
Taip pat galima naudoti tokia išraiška:

\begin{sphinxVerbatim}[commandchars=\\\{\}]
\PYG{n}{a} \PYG{o}{!=} \PYG{n}{b}
\end{sphinxVerbatim}

\sphinxAtStartPar
\sphinxcode{\sphinxupquote{a}} nelygus \sphinxcode{\sphinxupquote{b}}.

\end{fulllineitems}

\index{built\sphinxhyphen{}in function@\spxentry{built\sphinxhyphen{}in function}!lt()@\spxentry{lt()}}\index{lt()@\spxentry{lt()}!built\sphinxhyphen{}in function@\spxentry{built\sphinxhyphen{}in function}}

\begin{fulllineitems}
\phantomsection\label{\detokenize{formules:lt}}
\pysigstartsignatures
\pysiglinewithargsret
{\sphinxbfcode{\sphinxupquote{lt}}}
{\sphinxparam{\DUrole{n}{a}}\sphinxparamcomma \sphinxparam{\DUrole{n}{b}}}
{}
\pysigstopsignatures
\sphinxAtStartPar
Taip pat galima naudoti tokia išraiška:

\begin{sphinxVerbatim}[commandchars=\\\{\}]
\PYG{n}{a} \PYG{o}{\PYGZlt{}} \PYG{n}{b}
\end{sphinxVerbatim}

\sphinxAtStartPar
\sphinxcode{\sphinxupquote{a}} mažiau už \sphinxcode{\sphinxupquote{b}}.

\end{fulllineitems}

\index{built\sphinxhyphen{}in function@\spxentry{built\sphinxhyphen{}in function}!le()@\spxentry{le()}}\index{le()@\spxentry{le()}!built\sphinxhyphen{}in function@\spxentry{built\sphinxhyphen{}in function}}

\begin{fulllineitems}
\phantomsection\label{\detokenize{formules:le}}
\pysigstartsignatures
\pysiglinewithargsret
{\sphinxbfcode{\sphinxupquote{le}}}
{\sphinxparam{\DUrole{n}{a}}\sphinxparamcomma \sphinxparam{\DUrole{n}{b}}}
{}
\pysigstopsignatures
\sphinxAtStartPar
Taip pat galima naudoti tokia išraiška:

\begin{sphinxVerbatim}[commandchars=\\\{\}]
\PYG{n}{a} \PYG{o}{\PYGZlt{}}\PYG{o}{=} \PYG{n}{b}
\end{sphinxVerbatim}

\sphinxAtStartPar
\sphinxcode{\sphinxupquote{a}} mažiau arba lygu už \sphinxcode{\sphinxupquote{b}}.

\end{fulllineitems}

\index{built\sphinxhyphen{}in function@\spxentry{built\sphinxhyphen{}in function}!gt()@\spxentry{gt()}}\index{gt()@\spxentry{gt()}!built\sphinxhyphen{}in function@\spxentry{built\sphinxhyphen{}in function}}

\begin{fulllineitems}
\phantomsection\label{\detokenize{formules:gt}}
\pysigstartsignatures
\pysiglinewithargsret
{\sphinxbfcode{\sphinxupquote{gt}}}
{\sphinxparam{\DUrole{n}{a}}\sphinxparamcomma \sphinxparam{\DUrole{n}{b}}}
{}
\pysigstopsignatures
\sphinxAtStartPar
Taip pat galima naudoti tokia išraiška:

\begin{sphinxVerbatim}[commandchars=\\\{\}]
\PYG{n}{a} \PYG{o}{\PYGZgt{}} \PYG{n}{b}
\end{sphinxVerbatim}

\sphinxAtStartPar
\sphinxcode{\sphinxupquote{a}} daugiau už \sphinxcode{\sphinxupquote{b}}.

\end{fulllineitems}

\index{built\sphinxhyphen{}in function@\spxentry{built\sphinxhyphen{}in function}!ge()@\spxentry{ge()}}\index{ge()@\spxentry{ge()}!built\sphinxhyphen{}in function@\spxentry{built\sphinxhyphen{}in function}}

\begin{fulllineitems}
\phantomsection\label{\detokenize{formules:ge}}
\pysigstartsignatures
\pysiglinewithargsret
{\sphinxbfcode{\sphinxupquote{ge}}}
{\sphinxparam{\DUrole{n}{a}}\sphinxparamcomma \sphinxparam{\DUrole{n}{b}}}
{}
\pysigstopsignatures
\sphinxAtStartPar
Taip pat galima naudoti tokia išraiška:

\begin{sphinxVerbatim}[commandchars=\\\{\}]
\PYG{n}{a} \PYG{o}{\PYGZgt{}}\PYG{o}{=} \PYG{n}{b}
\end{sphinxVerbatim}

\sphinxAtStartPar
\sphinxcode{\sphinxupquote{a}} daugiau arba lygu už \sphinxcode{\sphinxupquote{b}}.

\end{fulllineitems}

\index{built\sphinxhyphen{}in function@\spxentry{built\sphinxhyphen{}in function}!add()@\spxentry{add()}}\index{add()@\spxentry{add()}!built\sphinxhyphen{}in function@\spxentry{built\sphinxhyphen{}in function}}

\begin{fulllineitems}
\phantomsection\label{\detokenize{formules:add}}
\pysigstartsignatures
\pysiglinewithargsret
{\sphinxbfcode{\sphinxupquote{add}}}
{\sphinxparam{\DUrole{n}{a}}\sphinxparamcomma \sphinxparam{\DUrole{n}{b}}}
{}
\pysigstopsignatures
\sphinxAtStartPar
Taip pat galima naudoti tokia išraiška:

\begin{sphinxVerbatim}[commandchars=\\\{\}]
\PYG{n}{a} \PYG{o}{+} \PYG{n}{b}
\end{sphinxVerbatim}

\sphinxAtStartPar
\sphinxcode{\sphinxupquote{a}} ir \sphinxcode{\sphinxupquote{b}} suma.

\end{fulllineitems}

\index{built\sphinxhyphen{}in function@\spxentry{built\sphinxhyphen{}in function}!sub()@\spxentry{sub()}}\index{sub()@\spxentry{sub()}!built\sphinxhyphen{}in function@\spxentry{built\sphinxhyphen{}in function}}

\begin{fulllineitems}
\phantomsection\label{\detokenize{formules:sub}}
\pysigstartsignatures
\pysiglinewithargsret
{\sphinxbfcode{\sphinxupquote{sub}}}
{\sphinxparam{\DUrole{n}{a}}\sphinxparamcomma \sphinxparam{\DUrole{n}{b}}}
{}
\pysigstopsignatures
\sphinxAtStartPar
Taip pat galima naudoti tokia išraiška:

\begin{sphinxVerbatim}[commandchars=\\\{\}]
\PYG{n}{a} \PYG{o}{\PYGZhy{}} \PYG{n}{b}
\end{sphinxVerbatim}

\sphinxAtStartPar
\sphinxcode{\sphinxupquote{a}} ir \sphinxcode{\sphinxupquote{b}} skirtumas.

\end{fulllineitems}

\index{built\sphinxhyphen{}in function@\spxentry{built\sphinxhyphen{}in function}!mul()@\spxentry{mul()}}\index{mul()@\spxentry{mul()}!built\sphinxhyphen{}in function@\spxentry{built\sphinxhyphen{}in function}}

\begin{fulllineitems}
\phantomsection\label{\detokenize{formules:mul}}
\pysigstartsignatures
\pysiglinewithargsret
{\sphinxbfcode{\sphinxupquote{mul}}}
{\sphinxparam{\DUrole{n}{a}}\sphinxparamcomma \sphinxparam{\DUrole{n}{b}}}
{}
\pysigstopsignatures
\sphinxAtStartPar
Taip pat galima naudoti tokia išraiška:

\begin{sphinxVerbatim}[commandchars=\\\{\}]
\PYG{n}{a} \PYG{o}{*} \PYG{n}{b}
\end{sphinxVerbatim}

\sphinxAtStartPar
\sphinxcode{\sphinxupquote{a}} ir \sphinxcode{\sphinxupquote{b}} sandauga.

\end{fulllineitems}

\index{built\sphinxhyphen{}in function@\spxentry{built\sphinxhyphen{}in function}!div()@\spxentry{div()}}\index{div()@\spxentry{div()}!built\sphinxhyphen{}in function@\spxentry{built\sphinxhyphen{}in function}}

\begin{fulllineitems}
\phantomsection\label{\detokenize{formules:div}}
\pysigstartsignatures
\pysiglinewithargsret
{\sphinxbfcode{\sphinxupquote{div}}}
{\sphinxparam{\DUrole{n}{a}}\sphinxparamcomma \sphinxparam{\DUrole{n}{b}}}
{}
\pysigstopsignatures
\sphinxAtStartPar
Taip pat galima naudoti tokia išraiška:

\begin{sphinxVerbatim}[commandchars=\\\{\}]
\PYG{n}{a} \PYG{o}{/} \PYG{n}{b}
\end{sphinxVerbatim}

\sphinxAtStartPar
\sphinxcode{\sphinxupquote{a}} ir \sphinxcode{\sphinxupquote{b}} dalyba.

\end{fulllineitems}

\index{built\sphinxhyphen{}in function@\spxentry{built\sphinxhyphen{}in function}!mod()@\spxentry{mod()}}\index{mod()@\spxentry{mod()}!built\sphinxhyphen{}in function@\spxentry{built\sphinxhyphen{}in function}}

\begin{fulllineitems}
\phantomsection\label{\detokenize{formules:mod}}
\pysigstartsignatures
\pysiglinewithargsret
{\sphinxbfcode{\sphinxupquote{mod}}}
{\sphinxparam{\DUrole{n}{a}}\sphinxparamcomma \sphinxparam{\DUrole{n}{b}}}
{}
\pysigstopsignatures
\sphinxAtStartPar
Taip pat galima naudoti tokia išraiška:

\begin{sphinxVerbatim}[commandchars=\\\{\}]
\PYG{n}{a} \PYG{o}{\PYGZpc{}} \PYG{n}{b}
\end{sphinxVerbatim}

\sphinxAtStartPar
\sphinxcode{\sphinxupquote{a}} ir \sphinxcode{\sphinxupquote{b}} modulis.

\end{fulllineitems}

\index{built\sphinxhyphen{}in function@\spxentry{built\sphinxhyphen{}in function}!positive()@\spxentry{positive()}}\index{positive()@\spxentry{positive()}!built\sphinxhyphen{}in function@\spxentry{built\sphinxhyphen{}in function}}

\begin{fulllineitems}
\phantomsection\label{\detokenize{formules:positive}}
\pysigstartsignatures
\pysiglinewithargsret
{\sphinxbfcode{\sphinxupquote{positive}}}
{\sphinxparam{\DUrole{n}{a}}}
{}
\pysigstopsignatures
\sphinxAtStartPar
Taip pat galima naudoti tokia išraiška:

\begin{sphinxVerbatim}[commandchars=\\\{\}]
\PYG{o}{+}\PYG{n}{a}
\end{sphinxVerbatim}

\sphinxAtStartPar
Gali būti interpretuojamas skirtingai, priklausomai nuo konteksto.
Įprastiniu atveju keičia skaičiaus ženklą.

\end{fulllineitems}

\index{built\sphinxhyphen{}in function@\spxentry{built\sphinxhyphen{}in function}!negative()@\spxentry{negative()}}\index{negative()@\spxentry{negative()}!built\sphinxhyphen{}in function@\spxentry{built\sphinxhyphen{}in function}}

\begin{fulllineitems}
\phantomsection\label{\detokenize{formules:negative}}
\pysigstartsignatures
\pysiglinewithargsret
{\sphinxbfcode{\sphinxupquote{negative}}}
{\sphinxparam{\DUrole{n}{a}}}
{}
\pysigstopsignatures
\sphinxAtStartPar
Taip pat galima naudoti tokia išraiška:

\begin{sphinxVerbatim}[commandchars=\\\{\}]
\PYG{o}{\PYGZhy{}}\PYG{n}{a}
\end{sphinxVerbatim}

\sphinxAtStartPar
Gali būti interpretuojamas skirtingai, priklausomai nuo konteksto.
Įprastiniu atveju keičia skaičiaus ženklą.

\end{fulllineitems}

\index{built\sphinxhyphen{}in function@\spxentry{built\sphinxhyphen{}in function}!tuple()@\spxentry{tuple()}}\index{tuple()@\spxentry{tuple()}!built\sphinxhyphen{}in function@\spxentry{built\sphinxhyphen{}in function}}

\begin{fulllineitems}
\phantomsection\label{\detokenize{formules:tuple}}
\pysigstartsignatures
\pysiglinewithargsret
{\sphinxbfcode{\sphinxupquote{tuple}}}
{\sphinxparam{\DUrole{o}{*}\DUrole{n}{args}}}
{}
\pysigstopsignatures
\sphinxAtStartPar
Taip pat galima naudoti tokia išraiška:

\begin{sphinxVerbatim}[commandchars=\\\{\}]
\PYG{p}{(}\PYG{o}{*}\PYG{n}{args}\PYG{p}{)}
\end{sphinxVerbatim}

\sphinxAtStartPar
Grupė argumentų.
\begin{description}
\sphinxlineitem{()}
\sphinxAtStartPar
Tuščia grupė.

\sphinxlineitem{a, b}
\sphinxAtStartPar
Tas pats, kas \sphinxcode{\sphinxupquote{tuple(a, b)}}.

\end{description}

\end{fulllineitems}

\index{built\sphinxhyphen{}in function@\spxentry{built\sphinxhyphen{}in function}!list()@\spxentry{list()}}\index{list()@\spxentry{list()}!built\sphinxhyphen{}in function@\spxentry{built\sphinxhyphen{}in function}}

\begin{fulllineitems}
\phantomsection\label{\detokenize{formules:list}}
\pysigstartsignatures
\pysiglinewithargsret
{\sphinxbfcode{\sphinxupquote{list}}}
{\sphinxparam{\DUrole{o}{*}\DUrole{n}{args}}}
{}
\pysigstopsignatures
\sphinxAtStartPar
Taip pat galima naudoti tokia išraiška:

\begin{sphinxVerbatim}[commandchars=\\\{\}]
\PYG{p}{[}\PYG{o}{*}\PYG{n}{args}\PYG{p}{]}
\end{sphinxVerbatim}

\sphinxAtStartPar
Sąrašas reikšmių.

\end{fulllineitems}

\index{built\sphinxhyphen{}in function@\spxentry{built\sphinxhyphen{}in function}!getattr()@\spxentry{getattr()}}\index{getattr()@\spxentry{getattr()}!built\sphinxhyphen{}in function@\spxentry{built\sphinxhyphen{}in function}}

\begin{fulllineitems}
\phantomsection\label{\detokenize{formules:getattr}}
\pysigstartsignatures
\pysiglinewithargsret
{\sphinxbfcode{\sphinxupquote{getattr}}}
{\sphinxparam{\DUrole{n}{object}}\sphinxparamcomma \sphinxparam{\DUrole{n}{attr}}}
{}
\pysigstopsignatures
\sphinxAtStartPar
Taip pat galima naudoti tokia išraiška:

\begin{sphinxVerbatim}[commandchars=\\\{\}]
\PYG{n+nb}{object}\PYG{o}{.}\PYG{n}{attr}
\end{sphinxVerbatim}

\sphinxAtStartPar
Gaunamos reikšmės pagal atributą arba raktą.

\end{fulllineitems}

\index{built\sphinxhyphen{}in function@\spxentry{built\sphinxhyphen{}in function}!getitem()@\spxentry{getitem()}}\index{getitem()@\spxentry{getitem()}!built\sphinxhyphen{}in function@\spxentry{built\sphinxhyphen{}in function}}

\begin{fulllineitems}
\phantomsection\label{\detokenize{formules:getitem}}
\pysigstartsignatures
\pysiglinewithargsret
{\sphinxbfcode{\sphinxupquote{getitem}}}
{\sphinxparam{\DUrole{n}{object}}\sphinxparamcomma \sphinxparam{\DUrole{n}{item}}}
{}
\pysigstopsignatures
\sphinxAtStartPar
Taip pat galima naudoti tokia išraiška:

\begin{sphinxVerbatim}[commandchars=\\\{\}]
\PYG{n}{a}\PYG{p}{[}\PYG{n}{item}\PYG{p}{]}
\end{sphinxVerbatim}

\sphinxAtStartPar
Gaunamos reikšmės pagal atributą arba raktą.

\sphinxAtStartPar
{\hyperref[\detokenize{formules:id22}]{\sphinxcrossref{\sphinxcode{\sphinxupquote{getitem()}}}}} gali būti interpretuojamas kaip sąrašo reikšmių filtras:

\begin{sphinxVerbatim}[commandchars=\\\{\}]
\PYG{n}{a}\PYG{p}{[}\PYG{n}{b} \PYG{o}{\PYGZgt{}} \PYG{n}{c}\PYG{p}{]}
\end{sphinxVerbatim}

\end{fulllineitems}

\index{built\sphinxhyphen{}in function@\spxentry{built\sphinxhyphen{}in function}!dict()@\spxentry{dict()}}\index{dict()@\spxentry{dict()}!built\sphinxhyphen{}in function@\spxentry{built\sphinxhyphen{}in function}}

\begin{fulllineitems}
\phantomsection\label{\detokenize{formules:dict}}
\pysigstartsignatures
\pysiglinewithargsret
{\sphinxbfcode{\sphinxupquote{dict}}}
{\sphinxparam{\DUrole{o}{**}\DUrole{n}{kwargs}}}
{}
\pysigstopsignatures
\sphinxAtStartPar
Taip pat galima naudoti tokia išraiška:

\begin{sphinxVerbatim}[commandchars=\\\{\}]
\PYG{p}{\PYGZob{}}\PYG{n}{a}\PYG{p}{:} \PYG{n}{b}\PYG{p}{\PYGZcb{}}
\end{sphinxVerbatim}

\sphinxAtStartPar
Sudėtinė duomenų struktūra.

\end{fulllineitems}

\index{built\sphinxhyphen{}in function@\spxentry{built\sphinxhyphen{}in function}!set()@\spxentry{set()}}\index{set()@\spxentry{set()}!built\sphinxhyphen{}in function@\spxentry{built\sphinxhyphen{}in function}}

\begin{fulllineitems}
\phantomsection\label{\detokenize{formules:set}}
\pysigstartsignatures
\pysiglinewithargsret
{\sphinxbfcode{\sphinxupquote{set}}}
{\sphinxparam{\DUrole{o}{**}\DUrole{n}{kwargs}}}
{}
\pysigstopsignatures
\sphinxAtStartPar
Taip pat galima naudoti tokia išraiška:

\begin{sphinxVerbatim}[commandchars=\\\{\}]
\PYG{p}{\PYGZob{}}\PYG{n}{a}\PYG{p}{,} \PYG{n}{b}\PYG{p}{\PYGZcb{}}
\end{sphinxVerbatim}

\sphinxAtStartPar
Reikšmių aibė.

\end{fulllineitems}

\index{built\sphinxhyphen{}in function@\spxentry{built\sphinxhyphen{}in function}!op()@\spxentry{op()}}\index{op()@\spxentry{op()}!built\sphinxhyphen{}in function@\spxentry{built\sphinxhyphen{}in function}}

\begin{fulllineitems}
\phantomsection\label{\detokenize{formules:op}}
\pysigstartsignatures
\pysiglinewithargsret
{\sphinxbfcode{\sphinxupquote{op}}}
{\sphinxparam{\DUrole{n}{operator}}}
{}
\pysigstopsignatures
\sphinxAtStartPar
Taip pat galima naudoti tokia išraiška:

\begin{sphinxVerbatim}[commandchars=\\\{\}]
\PYG{n}{a}\PYG{p}{(}\PYG{o}{*}\PYG{p}{)}
\end{sphinxVerbatim}

\sphinxAtStartPar
Operatoriai gali būti naudojami kaip argumentai.

\end{fulllineitems}

\index{built\sphinxhyphen{}in function@\spxentry{built\sphinxhyphen{}in function}!stack()@\spxentry{stack()}}\index{stack()@\spxentry{stack()}!built\sphinxhyphen{}in function@\spxentry{built\sphinxhyphen{}in function}}

\begin{fulllineitems}
\phantomsection\label{\detokenize{formules:stack}}
\pysigstartsignatures
\pysiglinewithargsret
{\sphinxbfcode{\sphinxupquote{stack}}}
{\sphinxparam{\DUrole{n}{columns}}\sphinxparamcomma \sphinxparam{\DUrole{n}{values}}\sphinxparamcomma \sphinxparam{\DUrole{n}{exclude}}}
{}
\pysigstopsignatures
\sphinxAtStartPar
Visus stulpelius išskyrus \sphinxcode{\sphinxupquote{exclude}} verčia į vieną stulpelių eilutei
suteikiant \sphinxcode{\sphinxupquote{columns}} pavadinimą, o reikšmių stulpeliui \sphinxcode{\sphinxupquote{values}} pavadinimą.
Pavyzdžiui:


\begin{savenotes}\sphinxattablestart
\sphinxthistablewithglobalstyle
\centering
\begin{tabulary}{\linewidth}[t]{TTT}
\sphinxtoprule
\sphinxstyletheadfamily 
\sphinxAtStartPar
vertinimas
&\sphinxstyletheadfamily 
\sphinxAtStartPar
2015P2
&\sphinxstyletheadfamily 
\sphinxAtStartPar
2016P2
\\
\sphinxmidrule
\sphinxtableatstartofbodyhook
\sphinxAtStartPar
Neigiamai
&
\sphinxAtStartPar
0
&
\sphinxAtStartPar
1
\\
\sphinxhline
\sphinxAtStartPar
Teigiamai
&
\sphinxAtStartPar
39
&
\sphinxAtStartPar
28
\\
\sphinxbottomrule
\end{tabulary}
\sphinxtableafterendhook\par
\sphinxattableend\end{savenotes}

\sphinxAtStartPar
Tokiai lentelei pritaikius \sphinxcode{\sphinxupquote{stack("data", "rodiklis", {[}"vertinimas"{]})}}
transformaciją, gausime tokį rezultatą:


\begin{savenotes}\sphinxattablestart
\sphinxthistablewithglobalstyle
\centering
\begin{tabulary}{\linewidth}[t]{TTT}
\sphinxtoprule
\sphinxstyletheadfamily 
\sphinxAtStartPar
vertinimas
&\sphinxstyletheadfamily 
\sphinxAtStartPar
data
&\sphinxstyletheadfamily 
\sphinxAtStartPar
rodiklis
\\
\sphinxmidrule
\sphinxtableatstartofbodyhook
\sphinxAtStartPar
Neigiamai
&
\sphinxAtStartPar
2015P2
&
\sphinxAtStartPar
0
\\
\sphinxhline
\sphinxAtStartPar
Neigiamai
&
\sphinxAtStartPar
2016P2
&
\sphinxAtStartPar
1
\\
\sphinxhline
\sphinxAtStartPar
Teigiamai
&
\sphinxAtStartPar
2015P2
&
\sphinxAtStartPar
39
\\
\sphinxhline
\sphinxAtStartPar
Teigiamai
&
\sphinxAtStartPar
2016P2
&
\sphinxAtStartPar
28
\\
\sphinxbottomrule
\end{tabulary}
\sphinxtableafterendhook\par
\sphinxattableend\end{savenotes}

\end{fulllineitems}

\index{built\sphinxhyphen{}in function@\spxentry{built\sphinxhyphen{}in function}!datetime()@\spxentry{datetime()}}\index{datetime()@\spxentry{datetime()}!built\sphinxhyphen{}in function@\spxentry{built\sphinxhyphen{}in function}}

\begin{fulllineitems}
\phantomsection\label{\detokenize{formules:datetime}}
\pysigstartsignatures
\pysiglinewithargsret
{\sphinxbfcode{\sphinxupquote{datetime}}}
{\sphinxparam{\DUrole{n}{str}}\sphinxparamcomma \sphinxparam{\DUrole{n}{format}}}
{}
\pysigstopsignatures
\sphinxAtStartPar
Išgaunama data ir laikas iš str, naudojant \sphinxhref{https://strftime.org/}{strftime} formatą.

\end{fulllineitems}

\index{built\sphinxhyphen{}in function@\spxentry{built\sphinxhyphen{}in function}!date()@\spxentry{date()}}\index{date()@\spxentry{date()}!built\sphinxhyphen{}in function@\spxentry{built\sphinxhyphen{}in function}}

\begin{fulllineitems}
\phantomsection\label{\detokenize{formules:date}}
\pysigstartsignatures
\pysiglinewithargsret
{\sphinxbfcode{\sphinxupquote{date}}}
{\sphinxparam{\DUrole{n}{str}}\sphinxparamcomma \sphinxparam{\DUrole{n}{format}}}
{}
\pysigstopsignatures
\sphinxAtStartPar
išgaunama data iš str, naudojant \sphinxhref{https://strftime.org/}{strftime} formatą.

\end{fulllineitems}

\index{built\sphinxhyphen{}in function@\spxentry{built\sphinxhyphen{}in function}!date()@\spxentry{date()}}\index{date()@\spxentry{date()}!built\sphinxhyphen{}in function@\spxentry{built\sphinxhyphen{}in function}}

\begin{fulllineitems}
\phantomsection\label{\detokenize{formules:id0}}
\pysigstartsignatures
\pysiglinewithargsret
{\sphinxbfcode{\sphinxupquote{date}}}
{\sphinxparam{\DUrole{n}{datetime}}}
{}
\pysigstopsignatures
\sphinxAtStartPar
Gražinama data iš datos ir laiko.

\end{fulllineitems}



\subsection{Failai}
\label{\detokenize{formules:failai}}\label{\detokenize{formules:id3}}
\sphinxAtStartPar
Dažnai duomenys teikiami failų pavidalu, kurie gali būti saugomi tiek lokaliai
failų sistemoje, tiek nuotoliniame serveryje. Failai gali būti suspausti ir
patalpinti į archyvo konteinerius. {\hyperref[\detokenize{savokos:term-DSA}]{\sphinxtermref{\DUrole{xref}{\DUrole{std}{\DUrole{std-term}{DSA}}}}}} leidžia aprašyti įvairius
prieigos prie duomenų, saugomų failuose, atvejus.


\begin{fulllineitems}

\pysigstartsignatures
\pysigline
{\sphinxbfcode{\sphinxupquote{resource.source}}}
\pysigstopsignatures
\sphinxAtStartPar
Nutolusiame serveryje saugomo failo {\hyperref[\detokenize{savokos:term-URI}]{\sphinxtermref{\DUrole{xref}{\DUrole{std}{\DUrole{std-term}{URI}}}}}} arba kelias iki lokalaus
katalogo. Lokalaus katalogo kelias gali būti pateikiamas tiek {\hyperref[\detokenize{savokos:term-POSIX}]{\sphinxtermref{\DUrole{xref}{\DUrole{std}{\DUrole{std-term}{POSIX}}}}}},
tiek {\hyperref[\detokenize{savokos:term-DOS}]{\sphinxtermref{\DUrole{xref}{\DUrole{std}{\DUrole{std-term}{DOS}}}}}} formatais, priklausomai nuo to, kokioje operacinėje
sistemoje failai saugomi.

\end{fulllineitems}



\begin{fulllineitems}

\pysigstartsignatures
\pysigline
{\sphinxbfcode{\sphinxupquote{resource.prepare}}}
\pysigstopsignatures\index{built\sphinxhyphen{}in function@\spxentry{built\sphinxhyphen{}in function}!file()@\spxentry{file()}}\index{file()@\spxentry{file()}!built\sphinxhyphen{}in function@\spxentry{built\sphinxhyphen{}in function}}

\begin{fulllineitems}
\phantomsection\label{\detokenize{formules:file}}
\pysigstartsignatures
\pysiglinewithargsret
{\sphinxbfcode{\sphinxupquote{file}}}
{\sphinxparam{\DUrole{n}{resource}}\sphinxparamcomma \sphinxparam{\DUrole{n}{encoding}\DUrole{p}{:}\DUrole{w}{ }\DUrole{n}{\DUrole{s}{'utf\sphinxhyphen{}8'}}}}
{}
\pysigstopsignatures\begin{quote}\begin{description}
\sphinxlineitem{Parametrai}\begin{itemize}
\item {} 
\sphinxAtStartPar
\sphinxstyleliteralstrong{\sphinxupquote{resource}} \sphinxhyphen{}\sphinxhyphen{} Kelias arba URI iki failo.

\item {} 
\sphinxAtStartPar
\sphinxstyleliteralstrong{\sphinxupquote{encoding}} \sphinxhyphen{}\sphinxhyphen{} Failo koduotę.

\end{itemize}

\end{description}\end{quote}

\sphinxAtStartPar
Ši funkcija leidžia nurodyti failo koduotę, jei failas yra užkoduotas
kita, nei UTF\sphinxhyphen{}8 koduote. Pilną palaikomų koduočių sąrašą galite rasti
\sphinxhref{https://docs.python.org/3/library/codecs.html\#standard-encodings}{šiame sąraše}.

\end{fulllineitems}

\index{built\sphinxhyphen{}in function@\spxentry{built\sphinxhyphen{}in function}!extract()@\spxentry{extract()}}\index{extract()@\spxentry{extract()}!built\sphinxhyphen{}in function@\spxentry{built\sphinxhyphen{}in function}}

\begin{fulllineitems}
\phantomsection\label{\detokenize{formules:extract}}
\pysigstartsignatures
\pysiglinewithargsret
{\sphinxbfcode{\sphinxupquote{extract}}}
{\sphinxparam{\DUrole{n}{resource}}\sphinxparamcomma \sphinxparam{\DUrole{n}{type}}}
{}
\pysigstopsignatures\begin{quote}\begin{description}
\sphinxlineitem{Parametrai}\begin{itemize}
\item {} 
\sphinxAtStartPar
\sphinxstyleliteralstrong{\sphinxupquote{resource}} \sphinxhyphen{}\sphinxhyphen{} Kelias arba URI iki archyvo failo arba failo objektas.

\item {} 
\sphinxAtStartPar
\sphinxstyleliteralstrong{\sphinxupquote{type}} \sphinxhyphen{}\sphinxhyphen{} Archyvo tipas.

\end{itemize}

\end{description}\end{quote}

\sphinxAtStartPar
Išpakuoja archyvą, kuriame saugomi failai. Galimos \sphinxcode{\sphinxupquote{type}} reikšmės:


\begin{fulllineitems}

\pysigstartsignatures
\pysigline
{\sphinxbfcode{\sphinxupquote{zip}}}
\pysigstopsignatures
\end{fulllineitems}



\begin{fulllineitems}

\pysigstartsignatures
\pysigline
{\sphinxbfcode{\sphinxupquote{tar}}}
\pysigstopsignatures
\end{fulllineitems}



\begin{fulllineitems}

\pysigstartsignatures
\pysigline
{\sphinxbfcode{\sphinxupquote{rar}}}
\pysigstopsignatures
\end{fulllineitems}


\sphinxAtStartPar
Funkcijos rezultatas yra archyvo objektas, kuris leidžia pasiekti
esančius archyvo failus {\hyperref[\detokenize{formules:id22}]{\sphinxcrossref{\sphinxcode{\sphinxupquote{getitem()}}}}} funkcijos pagalba.

\end{fulllineitems}

\index{built\sphinxhyphen{}in function@\spxentry{built\sphinxhyphen{}in function}!decompress()@\spxentry{decompress()}}\index{decompress()@\spxentry{decompress()}!built\sphinxhyphen{}in function@\spxentry{built\sphinxhyphen{}in function}}

\begin{fulllineitems}
\phantomsection\label{\detokenize{formules:decompress}}
\pysigstartsignatures
\pysiglinewithargsret
{\sphinxbfcode{\sphinxupquote{decompress}}}
{\sphinxparam{\DUrole{n}{resource}}\sphinxparamcomma \sphinxparam{\DUrole{n}{type}}}
{}
\pysigstopsignatures\begin{quote}\begin{description}
\sphinxlineitem{Parametrai}\begin{itemize}
\item {} 
\sphinxAtStartPar
\sphinxstyleliteralstrong{\sphinxupquote{resource}} \sphinxhyphen{}\sphinxhyphen{} Kelias arba URI iki archyvo failo arba failo objektas.

\item {} 
\sphinxAtStartPar
\sphinxstyleliteralstrong{\sphinxupquote{type}} \sphinxhyphen{}\sphinxhyphen{} Archyvo tipas.

\end{itemize}

\end{description}\end{quote}

\sphinxAtStartPar
Taikomas srautinis failo glaudinimo filtras. Galimos \sphinxcode{\sphinxupquote{type}} reikšmės:


\begin{fulllineitems}

\pysigstartsignatures
\pysigline
{\sphinxbfcode{\sphinxupquote{gz}}}
\pysigstopsignatures
\end{fulllineitems}



\begin{fulllineitems}

\pysigstartsignatures
\pysigline
{\sphinxbfcode{\sphinxupquote{bz2}}}
\pysigstopsignatures
\end{fulllineitems}



\begin{fulllineitems}

\pysigstartsignatures
\pysigline
{\sphinxbfcode{\sphinxupquote{xz}}}
\pysigstopsignatures
\end{fulllineitems}


\end{fulllineitems}


\end{fulllineitems}



\subsection{Stulpeliai lentelėje}
\label{\detokenize{formules:stulpeliai-lenteleje}}\label{\detokenize{formules:id5}}
\sphinxAtStartPar
CSV ar skaičiuoklių lentelėse stulpelių pavadinimai pateikiami pačioje
lentelėje. Eilutė, kurioje surašyti pavadinimai nebūtinai gali būti pirma.
Stulpelių pavadinimai gali būti pateikti keliose eilutėse iš kurių formuojamos
kompleksinės struktūros (žiūrėti {\hyperref[\detokenize{formules:kompleksines-strukturos}]{\sphinxcrossref{\DUrole{std}{\DUrole{std-ref}{Kompleksinės struktūros}}}}}). Įvairias
situacijas galima aprašyti formulių pagalba.


\begin{fulllineitems}

\pysigstartsignatures
\pysigline
{\sphinxbfcode{\sphinxupquote{model.prepare}}}
\pysigstopsignatures\index{built\sphinxhyphen{}in function@\spxentry{built\sphinxhyphen{}in function}!header()@\spxentry{header()}}\index{header()@\spxentry{header()}!built\sphinxhyphen{}in function@\spxentry{built\sphinxhyphen{}in function}}

\begin{fulllineitems}
\phantomsection\label{\detokenize{formules:header}}
\pysigstartsignatures
\pysiglinewithargsret
{\sphinxbfcode{\sphinxupquote{header}}}
{\sphinxparam{\DUrole{o}{*}\DUrole{n}{line}}}
{}
\pysigstopsignatures

\begin{fulllineitems}

\pysigstartsignatures
\pysigline
{\sphinxbfcode{\sphinxupquote{null}}}
\pysigstopsignatures
\sphinxAtStartPar
Lentelėje eilučių pavadinimų nėra. Tokiu atveju,
{\hyperref[\detokenize{dimensijos:property.source}]{\sphinxcrossref{\sphinxcode{\sphinxupquote{property.source}}}}} stulpelyje reikia pateikti stulpelio numerį,
pradedant skaičiuoti nuo 0.

\end{fulllineitems}



\begin{fulllineitems}

\pysigstartsignatures
\pysigline
{\sphinxbfcode{\sphinxupquote{line}}}
\pysigstopsignatures
\sphinxAtStartPar
Nurodomas eilutės numeris, pradedant eilutes skaičiuoti nuo 0, kur
yra pateikti lentelės stulpelių pavadinimai. Pagal nutylėjimą
stulpelių pavadinimų ieškoma pirmoje eilutėje.

\end{fulllineitems}



\begin{fulllineitems}

\pysigstartsignatures
\pysigline
{\sphinxbfcode{\sphinxupquote{*line}}}
\pysigstopsignatures
\sphinxAtStartPar
Jei lentelė turi kompleksinę stulpelių struktūrą, tada galima
pateikti daugiau nei vieną eilutės numerį iš kurių bus nustatomi
stulpelių pavadinimai.

\end{fulllineitems}


\end{fulllineitems}

\index{built\sphinxhyphen{}in function@\spxentry{built\sphinxhyphen{}in function}!head()@\spxentry{head()}}\index{head()@\spxentry{head()}!built\sphinxhyphen{}in function@\spxentry{built\sphinxhyphen{}in function}}

\begin{fulllineitems}
\phantomsection\label{\detokenize{formules:head}}
\pysigstartsignatures
\pysiglinewithargsret
{\sphinxbfcode{\sphinxupquote{head}}}
{\sphinxparam{\DUrole{n}{n}}}
{}
\pysigstopsignatures
\sphinxAtStartPar
Praleisti \sphinxcode{\sphinxupquote{n}} einančių po stulpelių pavadinimų eilutės.

\end{fulllineitems}

\index{built\sphinxhyphen{}in function@\spxentry{built\sphinxhyphen{}in function}!tail()@\spxentry{tail()}}\index{tail()@\spxentry{tail()}!built\sphinxhyphen{}in function@\spxentry{built\sphinxhyphen{}in function}}

\begin{fulllineitems}
\phantomsection\label{\detokenize{formules:tail}}
\pysigstartsignatures
\pysiglinewithargsret
{\sphinxbfcode{\sphinxupquote{tail}}}
{\sphinxparam{\DUrole{n}{n}}}
{}
\pysigstopsignatures
\sphinxAtStartPar
Ignoruoti \sphinxcode{\sphinxupquote{n}} eilučių failo pabaigoje.

\end{fulllineitems}


\end{fulllineitems}



\begin{fulllineitems}

\pysigstartsignatures
\pysigline
{\sphinxbfcode{\sphinxupquote{property.source}}}
\pysigstopsignatures
\sphinxAtStartPar
Jei naudojamas {\hyperref[\detokenize{formules:header}]{\sphinxcrossref{\sphinxcode{\sphinxupquote{header(null)}}}}}, tada nurodomas stulpelio
numeris, pradedant nuo 0.

\sphinxAtStartPar
Jei naudojamas {\hyperref[\detokenize{formules:header}]{\sphinxcrossref{\sphinxcode{\sphinxupquote{header(line)}}}}}, tada nurodomas stulpelio
pavadinimas, toks koks įrašytas lentelės line eilutėje.

\sphinxAtStartPar
Jei naudojamas {\hyperref[\detokenize{formules:header}]{\sphinxcrossref{\sphinxcode{\sphinxupquote{header(*line)}}}}}, tada nurodomas stulpelio
pavadinimas, toks koks įrašymas lentelės pirmajame line argumente.

\end{fulllineitems}



\begin{fulllineitems}

\pysigstartsignatures
\pysigline
{\sphinxbfcode{\sphinxupquote{property.prepare}}}
\pysigstopsignatures
\sphinxAtStartPar
Jei naudojamas \sphinxcode{\sphinxupquote{header(*line)}}, žiūrėti {\hyperref[\detokenize{formules:kompleksines-strukturos}]{\sphinxcrossref{\DUrole{std}{\DUrole{std-ref}{Kompleksinės struktūros}}}}}.

\end{fulllineitems}



\subsection{Duomenų atranka}
\label{\detokenize{formules:duomenu-atranka}}\label{\detokenize{formules:id6}}
\sphinxAtStartPar
Duomenų filtravimui naudojamas {\hyperref[\detokenize{dimensijos:model.prepare}]{\sphinxcrossref{\sphinxcode{\sphinxupquote{model.prepare}}}}} stulpelis, kuriame galima
apriboti iš šaltinio skaitomų duomenų imtį.

\sphinxAtStartPar
Tarkime, jei turime tokias dvi duomenų lenteles:


\begin{savenotes}\sphinxattablestart
\sphinxthistablewithglobalstyle
\centering
\begin{tabulary}{\linewidth}[t]{TT}
\sphinxtoprule
\sphinxstartmulticolumn{2}%
\begin{varwidth}[t]{\sphinxcolwidth{2}{2}}
\sphinxstyletheadfamily \sphinxAtStartPar
COUNTRIES
\par
\vskip-\baselineskip\vbox{\hbox{\strut}}\end{varwidth}%
\sphinxstopmulticolumn
\\
\sphinxhline\sphinxstyletheadfamily 
\sphinxAtStartPar
COUNTRY
&\sphinxstyletheadfamily 
\sphinxAtStartPar
CODE
\\
\sphinxmidrule
\sphinxtableatstartofbodyhook
\sphinxAtStartPar
Lietuva
&
\sphinxAtStartPar
lt
\\
\sphinxhline
\sphinxAtStartPar
Latvija
&
\sphinxAtStartPar
lv
\\
\sphinxbottomrule
\end{tabulary}
\sphinxtableafterendhook\par
\sphinxattableend\end{savenotes}


\begin{savenotes}\sphinxattablestart
\sphinxthistablewithglobalstyle
\centering
\begin{tabulary}{\linewidth}[t]{TTT}
\sphinxtoprule
\sphinxstartmulticolumn{3}%
\begin{varwidth}[t]{\sphinxcolwidth{3}{3}}
\sphinxstyletheadfamily \sphinxAtStartPar
CITIES
\par
\vskip-\baselineskip\vbox{\hbox{\strut}}\end{varwidth}%
\sphinxstopmulticolumn
\\
\sphinxhline\sphinxstyletheadfamily 
\sphinxAtStartPar
ID
&\sphinxstyletheadfamily 
\sphinxAtStartPar
CITY
&\sphinxstyletheadfamily 
\sphinxAtStartPar
COUNTRY
\\
\sphinxmidrule
\sphinxtableatstartofbodyhook
\sphinxAtStartPar
1
&
\sphinxAtStartPar
Vilnius
&
\sphinxAtStartPar
lt
\\
\sphinxhline
\sphinxAtStartPar
2
&
\sphinxAtStartPar
Kaunas
&
\sphinxAtStartPar
lt
\\
\sphinxhline
\sphinxAtStartPar
3
&
\sphinxAtStartPar
Ryga
&
\sphinxAtStartPar
lv
\\
\sphinxbottomrule
\end{tabulary}
\sphinxtableafterendhook\par
\sphinxattableend\end{savenotes}

\sphinxAtStartPar
Jei norėtume atveri ne visų šalių duomenis, o tik Lietuvos, tada duomenų
struktūros aprašas turėtu atrodyti taip:


\begin{savenotes}\sphinxattablestart
\sphinxthistablewithglobalstyle
\centering
\begin{tabulary}{\linewidth}[t]{TTTTTTTTT}
\sphinxtoprule
\sphinxstyletheadfamily 
\sphinxAtStartPar
d
&\sphinxstyletheadfamily 
\sphinxAtStartPar
r
&\sphinxstyletheadfamily 
\sphinxAtStartPar
b
&\sphinxstyletheadfamily 
\sphinxAtStartPar
m
&\sphinxstyletheadfamily 
\sphinxAtStartPar
property
&\sphinxstyletheadfamily 
\sphinxAtStartPar
type
&\sphinxstyletheadfamily 
\sphinxAtStartPar
ref
&\sphinxstyletheadfamily 
\sphinxAtStartPar
source
&\sphinxstyletheadfamily 
\sphinxAtStartPar
prepare
\\
\sphinxmidrule
\sphinxtableatstartofbodyhook\sphinxstartmulticolumn{5}%
\begin{varwidth}[t]{\sphinxcolwidth{5}{9}}
\sphinxAtStartPar
datasets/example/countries
\par
\vskip-\baselineskip\vbox{\hbox{\strut}}\end{varwidth}%
\sphinxstopmulticolumn
&&&&\\
\sphinxhline&\sphinxstartmulticolumn{4}%
\begin{varwidth}[t]{\sphinxcolwidth{4}{9}}
\sphinxAtStartPar
salys
\par
\vskip-\baselineskip\vbox{\hbox{\strut}}\end{varwidth}%
\sphinxstopmulticolumn
&
\sphinxAtStartPar
sql
&&
\sphinxAtStartPar
sqlite://
&\\
\sphinxhline&&&\sphinxstartmulticolumn{2}%
\begin{varwidth}[t]{\sphinxcolwidth{2}{9}}
\sphinxAtStartPar
Country
\par
\vskip-\baselineskip\vbox{\hbox{\strut}}\end{varwidth}%
\sphinxstopmulticolumn
&&
\sphinxAtStartPar
code
&
\sphinxAtStartPar
COUNTRIES
&
\sphinxAtStartPar
\sphinxstylestrong{code = "lt"}
\\
\sphinxhline&&&&
\sphinxAtStartPar
name
&
\sphinxAtStartPar
string
&&
\sphinxAtStartPar
COUNTRY
&\\
\sphinxhline&&&&
\sphinxAtStartPar
code
&
\sphinxAtStartPar
string
&&
\sphinxAtStartPar
CODE
&\\
\sphinxhline&&&\sphinxstartmulticolumn{2}%
\begin{varwidth}[t]{\sphinxcolwidth{2}{9}}
\sphinxAtStartPar
City
\par
\vskip-\baselineskip\vbox{\hbox{\strut}}\end{varwidth}%
\sphinxstopmulticolumn
&&
\sphinxAtStartPar
id
&
\sphinxAtStartPar
CITIES
&\\
\sphinxhline&&&&
\sphinxAtStartPar
id
&
\sphinxAtStartPar
integer
&&
\sphinxAtStartPar
ID
&\\
\sphinxhline&&&&
\sphinxAtStartPar
name
&
\sphinxAtStartPar
string
&&
\sphinxAtStartPar
CITY
&\\
\sphinxhline&&&&
\sphinxAtStartPar
country
&
\sphinxAtStartPar
ref
&
\sphinxAtStartPar
Country
&
\sphinxAtStartPar
COUNTRY
&\\
\sphinxbottomrule
\end{tabulary}
\sphinxtableafterendhook\par
\sphinxattableend\end{savenotes}

\sphinxAtStartPar
Kaip ir visur, formulės reikia naudoti pavadinimus ne iš {\hyperref[\detokenize{formatas:source}]{\sphinxcrossref{\sphinxcode{\sphinxupquote{source}}}}}
stulpelio, o iš {\hyperref[\detokenize{formatas:property}]{\sphinxcrossref{\sphinxcode{\sphinxupquote{property}}}}}, {\hyperref[\detokenize{formatas:model}]{\sphinxcrossref{\sphinxcode{\sphinxupquote{model}}}}} arba {\hyperref[\detokenize{formatas:dataset}]{\sphinxcrossref{\sphinxcode{\sphinxupquote{dataset}}}}}.

\sphinxAtStartPar
Jei lentelės yra susijusios ryšiais tarpusavyje, užtenka filtrą nurodyti tik
vienoje lentelėje, visose kitose susijusios lentelėse filtrai bus taikomi
automatiškai, kad užtikrinti duomenų vientisumą.

\sphinxAtStartPar
Nurodant filtrus yra galimybė naudoti ne tik vienos lentelės laukus, bet ir
susijusių lentelių laukus, pavyzdžiui yra galimybė nurodyti tokį filtrą:


\begin{savenotes}\sphinxattablestart
\sphinxthistablewithglobalstyle
\centering
\begin{tabulary}{\linewidth}[t]{TTTTTTTTT}
\sphinxtoprule
\sphinxstyletheadfamily 
\sphinxAtStartPar
d
&\sphinxstyletheadfamily 
\sphinxAtStartPar
r
&\sphinxstyletheadfamily 
\sphinxAtStartPar
b
&\sphinxstyletheadfamily 
\sphinxAtStartPar
m
&\sphinxstyletheadfamily 
\sphinxAtStartPar
property
&\sphinxstyletheadfamily 
\sphinxAtStartPar
type
&\sphinxstyletheadfamily 
\sphinxAtStartPar
ref
&\sphinxstyletheadfamily 
\sphinxAtStartPar
source
&\sphinxstyletheadfamily 
\sphinxAtStartPar
prepare
\\
\sphinxmidrule
\sphinxtableatstartofbodyhook&&&\sphinxstartmulticolumn{2}%
\begin{varwidth}[t]{\sphinxcolwidth{2}{9}}
\sphinxAtStartPar
City
\par
\vskip-\baselineskip\vbox{\hbox{\strut}}\end{varwidth}%
\sphinxstopmulticolumn
&&
\sphinxAtStartPar
id
&
\sphinxAtStartPar
CITIES
&
\sphinxAtStartPar
\sphinxstylestrong{country.code = "lt"}
\\
\sphinxbottomrule
\end{tabulary}
\sphinxtableafterendhook\par
\sphinxattableend\end{savenotes}

\sphinxAtStartPar
Tačiau šiuo atveju, toks filtras būtų perteklinis, nes toks filtras
generuojamas automatiškai ir susijusio \sphinxcode{\sphinxupquote{Country}} modelio, kadangi negalime
publikuoti Latvijos miestų, jei publikuojama tik Lietuvos šalis.

\sphinxAtStartPar
Pilnas galimų filtrų sąrašas:


\begin{fulllineitems}

\pysigstartsignatures
\pysigline
{\sphinxbfcode{\sphinxupquote{model.prepare}}}
\pysigstopsignatures\index{built\sphinxhyphen{}in function@\spxentry{built\sphinxhyphen{}in function}!eq()@\spxentry{eq()}}\index{eq()@\spxentry{eq()}!built\sphinxhyphen{}in function@\spxentry{built\sphinxhyphen{}in function}}

\begin{fulllineitems}
\phantomsection\label{\detokenize{formules:id7}}
\pysigstartsignatures
\pysiglinewithargsret
{\sphinxbfcode{\sphinxupquote{eq}}}
{\sphinxparam{\DUrole{n}{a}}\sphinxparamcomma \sphinxparam{\DUrole{n}{b}}}
{}
\pysigstopsignatures
\sphinxAtStartPar
Sutrumpinta forma:

\begin{sphinxVerbatim}[commandchars=\\\{\}]
\PYG{n}{a} \PYG{o}{=} \PYG{n}{b}
\end{sphinxVerbatim}

\sphinxAtStartPar
Sąlyga tenkinama, jei \sphinxcode{\sphinxupquote{a}} ir \sphinxcode{\sphinxupquote{b}} reikšmės yra lygios.

\end{fulllineitems}

\index{built\sphinxhyphen{}in function@\spxentry{built\sphinxhyphen{}in function}!ne()@\spxentry{ne()}}\index{ne()@\spxentry{ne()}!built\sphinxhyphen{}in function@\spxentry{built\sphinxhyphen{}in function}}

\begin{fulllineitems}
\phantomsection\label{\detokenize{formules:id8}}
\pysigstartsignatures
\pysiglinewithargsret
{\sphinxbfcode{\sphinxupquote{ne}}}
{\sphinxparam{\DUrole{n}{a}}\sphinxparamcomma \sphinxparam{\DUrole{n}{b}}}
{}
\pysigstopsignatures
\sphinxAtStartPar
Sutrumpinta forma:

\begin{sphinxVerbatim}[commandchars=\\\{\}]
\PYG{n}{a} \PYG{o}{!=} \PYG{n}{b}
\end{sphinxVerbatim}

\sphinxAtStartPar
Sąlyga tenkinama, jei \sphinxcode{\sphinxupquote{a}} ir \sphinxcode{\sphinxupquote{b}} reikšmės nėra lygios.

\end{fulllineitems}

\index{built\sphinxhyphen{}in function@\spxentry{built\sphinxhyphen{}in function}!gt()@\spxentry{gt()}}\index{gt()@\spxentry{gt()}!built\sphinxhyphen{}in function@\spxentry{built\sphinxhyphen{}in function}}

\begin{fulllineitems}
\phantomsection\label{\detokenize{formules:id9}}
\pysigstartsignatures
\pysiglinewithargsret
{\sphinxbfcode{\sphinxupquote{gt}}}
{\sphinxparam{\DUrole{n}{a}}\sphinxparamcomma \sphinxparam{\DUrole{n}{b}}}
{}
\pysigstopsignatures
\sphinxAtStartPar
Sutrumpinta forma:

\begin{sphinxVerbatim}[commandchars=\\\{\}]
\PYG{n}{a} \PYG{o}{\PYGZgt{}} \PYG{n}{b}
\end{sphinxVerbatim}

\sphinxAtStartPar
Sąlyga tenkinama, jei \sphinxcode{\sphinxupquote{a}} reikšmė yra didesnė už \sphinxcode{\sphinxupquote{b}}.

\sphinxAtStartPar
Ši funkcija veikia tik su kiekybiniais duomenimis, kuriuos galima palyginti.

\end{fulllineitems}

\index{built\sphinxhyphen{}in function@\spxentry{built\sphinxhyphen{}in function}!lt()@\spxentry{lt()}}\index{lt()@\spxentry{lt()}!built\sphinxhyphen{}in function@\spxentry{built\sphinxhyphen{}in function}}

\begin{fulllineitems}
\phantomsection\label{\detokenize{formules:id10}}
\pysigstartsignatures
\pysiglinewithargsret
{\sphinxbfcode{\sphinxupquote{lt}}}
{\sphinxparam{\DUrole{n}{a}}\sphinxparamcomma \sphinxparam{\DUrole{n}{b}}}
{}
\pysigstopsignatures
\sphinxAtStartPar
Sutrumpinta forma:

\begin{sphinxVerbatim}[commandchars=\\\{\}]
\PYG{n}{a} \PYG{o}{\PYGZlt{}} \PYG{n}{b}
\end{sphinxVerbatim}

\sphinxAtStartPar
Sąlyga tenkinama, jei \sphinxcode{\sphinxupquote{a}} reikšmė yra mažesnė už \sphinxcode{\sphinxupquote{b}}.

\sphinxAtStartPar
Ši funkcija veikia tik su kiekybiniais duomenimis, kuriuos galima palyginti.

\end{fulllineitems}

\index{built\sphinxhyphen{}in function@\spxentry{built\sphinxhyphen{}in function}!ge()@\spxentry{ge()}}\index{ge()@\spxentry{ge()}!built\sphinxhyphen{}in function@\spxentry{built\sphinxhyphen{}in function}}

\begin{fulllineitems}
\phantomsection\label{\detokenize{formules:id11}}
\pysigstartsignatures
\pysiglinewithargsret
{\sphinxbfcode{\sphinxupquote{ge}}}
{\sphinxparam{\DUrole{n}{a}}\sphinxparamcomma \sphinxparam{\DUrole{n}{b}}}
{}
\pysigstopsignatures
\sphinxAtStartPar
Sutrumpinta forma:

\begin{sphinxVerbatim}[commandchars=\\\{\}]
\PYG{n}{a} \PYG{o}{\PYGZgt{}}\PYG{o}{=} \PYG{n}{b}
\end{sphinxVerbatim}

\sphinxAtStartPar
Sąlyga tenkinama, jei \sphinxcode{\sphinxupquote{a}} reikšmė yra didesnė arba lygi \sphinxcode{\sphinxupquote{b}} reikšmei.

\sphinxAtStartPar
Ši funkcija veikia tik su kiekybiniais duomenimis, kuriuos galima palyginti.

\end{fulllineitems}

\index{built\sphinxhyphen{}in function@\spxentry{built\sphinxhyphen{}in function}!le()@\spxentry{le()}}\index{le()@\spxentry{le()}!built\sphinxhyphen{}in function@\spxentry{built\sphinxhyphen{}in function}}

\begin{fulllineitems}
\phantomsection\label{\detokenize{formules:id12}}
\pysigstartsignatures
\pysiglinewithargsret
{\sphinxbfcode{\sphinxupquote{le}}}
{\sphinxparam{\DUrole{n}{a}}\sphinxparamcomma \sphinxparam{\DUrole{n}{b}}}
{}
\pysigstopsignatures
\sphinxAtStartPar
Sutrumpinta forma:

\begin{sphinxVerbatim}[commandchars=\\\{\}]
\PYG{n}{a} \PYG{o}{\PYGZlt{}}\PYG{o}{=} \PYG{n}{b}
\end{sphinxVerbatim}

\sphinxAtStartPar
Sąlyga tenkinama, jei \sphinxcode{\sphinxupquote{a}} reikšmė yra mažesnė arba lygi \sphinxcode{\sphinxupquote{b}} reikšmei.

\sphinxAtStartPar
Ši funkcija veikia tik su kiekybiniais duomenimis, kuriuos galima palyginti.

\end{fulllineitems}

\index{built\sphinxhyphen{}in function@\spxentry{built\sphinxhyphen{}in function}!in()@\spxentry{in()}}\index{in()@\spxentry{in()}!built\sphinxhyphen{}in function@\spxentry{built\sphinxhyphen{}in function}}

\begin{fulllineitems}
\phantomsection\label{\detokenize{formules:in}}
\pysigstartsignatures
\pysiglinewithargsret
{\sphinxbfcode{\sphinxupquote{in}}}
{\sphinxparam{\DUrole{n}{a}}\sphinxparamcomma \sphinxparam{\DUrole{n}{b}}}
{}
\pysigstopsignatures
\sphinxAtStartPar
Sutrumpinta forma:

\begin{sphinxVerbatim}[commandchars=\\\{\}]
\PYG{n}{a}\PYG{o}{.}\PYG{o+ow}{in}\PYG{p}{(}\PYG{n}{b}\PYG{p}{)}
\end{sphinxVerbatim}

\sphinxAtStartPar
Sąlyga tenkinama, jei \sphinxcode{\sphinxupquote{a}} lygi bent vienai iš \sphinxcode{\sphinxupquote{b}} sekos reikšmių.

\end{fulllineitems}

\index{built\sphinxhyphen{}in function@\spxentry{built\sphinxhyphen{}in function}!notin()@\spxentry{notin()}}\index{notin()@\spxentry{notin()}!built\sphinxhyphen{}in function@\spxentry{built\sphinxhyphen{}in function}}

\begin{fulllineitems}
\phantomsection\label{\detokenize{formules:notin}}
\pysigstartsignatures
\pysiglinewithargsret
{\sphinxbfcode{\sphinxupquote{notin}}}
{\sphinxparam{\DUrole{n}{a}}\sphinxparamcomma \sphinxparam{\DUrole{n}{b}}}
{}
\pysigstopsignatures
\sphinxAtStartPar
Sutrumpinta forma:

\begin{sphinxVerbatim}[commandchars=\\\{\}]
\PYG{n}{a}\PYG{o}{.}\PYG{n}{notin}\PYG{p}{(}\PYG{n}{b}\PYG{p}{)}
\end{sphinxVerbatim}

\sphinxAtStartPar
Sąlyga tenkinama, jei \sphinxcode{\sphinxupquote{a}} nelygi nei vienai iš \sphinxcode{\sphinxupquote{b}} sekos reikšmių.

\end{fulllineitems}

\index{built\sphinxhyphen{}in function@\spxentry{built\sphinxhyphen{}in function}!contains()@\spxentry{contains()}}\index{contains()@\spxentry{contains()}!built\sphinxhyphen{}in function@\spxentry{built\sphinxhyphen{}in function}}

\begin{fulllineitems}
\phantomsection\label{\detokenize{formules:contains}}
\pysigstartsignatures
\pysiglinewithargsret
{\sphinxbfcode{\sphinxupquote{contains}}}
{\sphinxparam{\DUrole{n}{a}}\sphinxparamcomma \sphinxparam{\DUrole{n}{b}}}
{}
\pysigstopsignatures
\sphinxAtStartPar
Sutrumpinta forma:

\begin{sphinxVerbatim}[commandchars=\\\{\}]
\PYG{n}{a}\PYG{o}{.}\PYG{n}{contains}\PYG{p}{(}\PYG{n}{b}\PYG{p}{)}
\end{sphinxVerbatim}

\sphinxAtStartPar
Sąlyga tenkinama, jei bent viena \sphinxcode{\sphinxupquote{a}} sekos reikšmių sutampa su \sphinxcode{\sphinxupquote{b}} reikšme.

\end{fulllineitems}

\index{built\sphinxhyphen{}in function@\spxentry{built\sphinxhyphen{}in function}!startswith()@\spxentry{startswith()}}\index{startswith()@\spxentry{startswith()}!built\sphinxhyphen{}in function@\spxentry{built\sphinxhyphen{}in function}}

\begin{fulllineitems}
\phantomsection\label{\detokenize{formules:startswith}}
\pysigstartsignatures
\pysiglinewithargsret
{\sphinxbfcode{\sphinxupquote{startswith}}}
{\sphinxparam{\DUrole{n}{a}}\sphinxparamcomma \sphinxparam{\DUrole{n}{b}}}
{}
\pysigstopsignatures
\sphinxAtStartPar
Sutrumpinta forma:

\begin{sphinxVerbatim}[commandchars=\\\{\}]
\PYG{n}{a}\PYG{o}{.}\PYG{n}{startswith}\PYG{p}{(}\PYG{n}{b}\PYG{p}{)}
\end{sphinxVerbatim}

\sphinxAtStartPar
Sąlyga tenkinama, jei \sphinxcode{\sphinxupquote{a}} simbolių eilutė prasideda \sphinxcode{\sphinxupquote{b}} simbolių eilute.

\end{fulllineitems}

\index{built\sphinxhyphen{}in function@\spxentry{built\sphinxhyphen{}in function}!endswith()@\spxentry{endswith()}}\index{endswith()@\spxentry{endswith()}!built\sphinxhyphen{}in function@\spxentry{built\sphinxhyphen{}in function}}

\begin{fulllineitems}
\phantomsection\label{\detokenize{formules:endswith}}
\pysigstartsignatures
\pysiglinewithargsret
{\sphinxbfcode{\sphinxupquote{endswith}}}
{\sphinxparam{\DUrole{n}{a}}\sphinxparamcomma \sphinxparam{\DUrole{n}{b}}}
{}
\pysigstopsignatures
\sphinxAtStartPar
Sutrumpinta forma:

\begin{sphinxVerbatim}[commandchars=\\\{\}]
\PYG{n}{a}\PYG{o}{.}\PYG{n}{endswith}\PYG{p}{(}\PYG{n}{b}\PYG{p}{)}
\end{sphinxVerbatim}

\sphinxAtStartPar
Sąlyga tenkinama, jei \sphinxcode{\sphinxupquote{a}} simbolių eilutė baigiasi \sphinxcode{\sphinxupquote{b}} simbolių eilute.

\end{fulllineitems}

\index{built\sphinxhyphen{}in function@\spxentry{built\sphinxhyphen{}in function}!and()@\spxentry{and()}}\index{and()@\spxentry{and()}!built\sphinxhyphen{}in function@\spxentry{built\sphinxhyphen{}in function}}

\begin{fulllineitems}
\phantomsection\label{\detokenize{formules:id13}}
\pysigstartsignatures
\pysiglinewithargsret
{\sphinxbfcode{\sphinxupquote{and}}}
{\sphinxparam{\DUrole{n}{a}}\sphinxparamcomma \sphinxparam{\DUrole{n}{b}}}
{}
\pysigstopsignatures
\sphinxAtStartPar
Sutrumpinta forma:

\begin{sphinxVerbatim}[commandchars=\\\{\}]
\PYG{n}{a} \PYG{o}{\PYGZam{}} \PYG{n}{b}
\end{sphinxVerbatim}

\sphinxAtStartPar
Sąlyga tenkinama, jei abi \sphinxcode{\sphinxupquote{a}} ir \sphinxcode{\sphinxupquote{b}} sąlygos tenkinamos.

\end{fulllineitems}

\index{built\sphinxhyphen{}in function@\spxentry{built\sphinxhyphen{}in function}!or()@\spxentry{or()}}\index{or()@\spxentry{or()}!built\sphinxhyphen{}in function@\spxentry{built\sphinxhyphen{}in function}}

\begin{fulllineitems}
\phantomsection\label{\detokenize{formules:id14}}
\pysigstartsignatures
\pysiglinewithargsret
{\sphinxbfcode{\sphinxupquote{or}}}
{\sphinxparam{\DUrole{n}{a}}\sphinxparamcomma \sphinxparam{\DUrole{n}{b}}}
{}
\pysigstopsignatures
\sphinxAtStartPar
Sutrumpinta forma:

\begin{sphinxVerbatim}[commandchars=\\\{\}]
\PYG{n}{a} \PYG{o}{|} \PYG{n}{b}
\end{sphinxVerbatim}

\sphinxAtStartPar
Sąlyga tenkinama, jei bent viena \sphinxcode{\sphinxupquote{a}} arba \sphinxcode{\sphinxupquote{b}} sąlyga tenkinama.

\end{fulllineitems}

\index{built\sphinxhyphen{}in function@\spxentry{built\sphinxhyphen{}in function}!sort()@\spxentry{sort()}}\index{sort()@\spxentry{sort()}!built\sphinxhyphen{}in function@\spxentry{built\sphinxhyphen{}in function}}

\begin{fulllineitems}
\phantomsection\label{\detokenize{formules:sort}}
\pysigstartsignatures
\pysiglinewithargsret
{\sphinxbfcode{\sphinxupquote{sort}}}
{\sphinxparam{\DUrole{n}{+a}}\sphinxparamcomma \sphinxparam{\DUrole{n}{\sphinxhyphen{}b}}}
{}
\pysigstopsignatures
\sphinxAtStartPar
Rūšiuoti didėjimo tvarka  pagal \sphinxcode{\sphinxupquote{a}} ir mažėjimo tvarka pagal \sphinxcode{\sphinxupquote{b}}.

\end{fulllineitems}

\index{built\sphinxhyphen{}in function@\spxentry{built\sphinxhyphen{}in function}!select()@\spxentry{select()}}\index{select()@\spxentry{select()}!built\sphinxhyphen{}in function@\spxentry{built\sphinxhyphen{}in function}}

\begin{fulllineitems}
\phantomsection\label{\detokenize{formules:select}}
\pysigstartsignatures
\pysiglinewithargsret
{\sphinxbfcode{\sphinxupquote{select}}}
{\sphinxparam{\DUrole{o}{*}\DUrole{n}{props}}}
{}
\pysigstopsignatures
\sphinxAtStartPar
Pateikiant duomenis, grąžinamos tik nurodytos savybės.

\sphinxAtStartPar
Jei nenurodyta jokia savybė, įtraukia visas, išskyrus daugiareikšmes
savybes.

\end{fulllineitems}

\index{built\sphinxhyphen{}in function@\spxentry{built\sphinxhyphen{}in function}!include()@\spxentry{include()}}\index{include()@\spxentry{include()}!built\sphinxhyphen{}in function@\spxentry{built\sphinxhyphen{}in function}}

\begin{fulllineitems}
\phantomsection\label{\detokenize{formules:include}}
\pysigstartsignatures
\pysiglinewithargsret
{\sphinxbfcode{\sphinxupquote{include}}}
{\sphinxparam{\DUrole{o}{*}\DUrole{n}{props}}}
{}
\pysigstopsignatures
\sphinxAtStartPar
Prie grąžinamų savybių, papildomai įtraukiamos nurodytos savybės. Gali
būti naudojama kartu su \sphinxcode{\sphinxupquote{select()}}, papildomai įtraukiant
daugiareikšmes ar sudėtinių tipų savybės.

\end{fulllineitems}

\index{built\sphinxhyphen{}in function@\spxentry{built\sphinxhyphen{}in function}!exclude()@\spxentry{exclude()}}\index{exclude()@\spxentry{exclude()}!built\sphinxhyphen{}in function@\spxentry{built\sphinxhyphen{}in function}}

\begin{fulllineitems}
\phantomsection\label{\detokenize{formules:exclude}}
\pysigstartsignatures
\pysiglinewithargsret
{\sphinxbfcode{\sphinxupquote{exclude}}}
{\sphinxparam{\DUrole{o}{*}\DUrole{n}{props}}}
{}
\pysigstopsignatures
\sphinxAtStartPar
Pašalina savybę iš \sphinxcode{\sphinxupquote{select()}} ar \sphinxcode{\sphinxupquote{expand()}} funkcijos pagalba atrinktų
savybių sąrašo.

\end{fulllineitems}

\index{built\sphinxhyphen{}in function@\spxentry{built\sphinxhyphen{}in function}!expand()@\spxentry{expand()}}\index{expand()@\spxentry{expand()}!built\sphinxhyphen{}in function@\spxentry{built\sphinxhyphen{}in function}}

\begin{fulllineitems}
\phantomsection\label{\detokenize{formules:expand}}
\pysigstartsignatures
\pysiglinewithargsret
{\sphinxbfcode{\sphinxupquote{expand}}}
{\sphinxparam{\DUrole{o}{*}\DUrole{n}{props}}}
{}
\pysigstopsignatures
\sphinxAtStartPar
Veikia panašiai, kaip \sphinxcode{\sphinxupquote{include()}}, tačiau įtraukia visas nurodyto
sudėtinio tipo savybes.

\sphinxAtStartPar
Jei nenurodyta konkreti savybė, įtraukia visų sudėtinių tipų savybes.

\sphinxAtStartPar
\sphinxcode{\sphinxupquote{expand()}} veikia tik su tiesiogiai modeliui priklausančiomis savybėmis.


\begin{sphinxseealso}{Taip pat žiūrėkite:}
\begin{itemize}
\item {} 
\sphinxAtStartPar
{\hyperref[\detokenize{identifikatoriai:prop-expand}]{\sphinxcrossref{\DUrole{std}{\DUrole{std-ref}{Savybių įtraukimas}}}}}

\end{itemize}


\end{sphinxseealso}


\end{fulllineitems}

\index{built\sphinxhyphen{}in function@\spxentry{built\sphinxhyphen{}in function}!extends()@\spxentry{extends()}}\index{extends()@\spxentry{extends()}!built\sphinxhyphen{}in function@\spxentry{built\sphinxhyphen{}in function}}

\begin{fulllineitems}
\phantomsection\label{\detokenize{formules:extends}}
\pysigstartsignatures
\pysiglinewithargsret
{\sphinxbfcode{\sphinxupquote{extends}}}
{\sphinxparam{\DUrole{n}{model}}\sphinxparamcomma \sphinxparam{\DUrole{o}{*}\DUrole{n}{props}}}
{}
\pysigstopsignatures
\sphinxAtStartPar
Įtraukia nurodyto \sphinxcode{\sphinxupquote{model}} modelio \sphinxcode{\sphinxupquote{props}} savybes.

\sphinxAtStartPar
Jei nenurodytas \sphinxcode{\sphinxupquote{model}} ir \sphinxcode{\sphinxupquote{props}}, įtraukia visas {\hyperref[\detokenize{formatas:base}]{\sphinxcrossref{\sphinxcode{\sphinxupquote{base}}}}} modelio savybes.

\sphinxAtStartPar
Jei nenurodyta \sphinxcode{\sphinxupquote{props}}, įtraukia, visas \sphinxcode{\sphinxupquote{model}} modelio savybes.

\end{fulllineitems}

\index{built\sphinxhyphen{}in function@\spxentry{built\sphinxhyphen{}in function}!limit()@\spxentry{limit()}}\index{limit()@\spxentry{limit()}!built\sphinxhyphen{}in function@\spxentry{built\sphinxhyphen{}in function}}

\begin{fulllineitems}
\phantomsection\label{\detokenize{formules:limit}}
\pysigstartsignatures
\pysiglinewithargsret
{\sphinxbfcode{\sphinxupquote{limit}}}
{\sphinxparam{\DUrole{n}{num}}}
{}
\pysigstopsignatures
\sphinxAtStartPar
Riboja gražinamų objektų skaičių iki nurodyti \sphinxcode{\sphinxupquote{num}} skaičiaus.

\sphinxAtStartPar
Gali būti naudojamas su {\hyperref[\detokenize{formules:page}]{\sphinxcrossref{\sphinxcode{\sphinxupquote{page()}}}}}.

\end{fulllineitems}

\index{built\sphinxhyphen{}in function@\spxentry{built\sphinxhyphen{}in function}!page()@\spxentry{page()}}\index{page()@\spxentry{page()}!built\sphinxhyphen{}in function@\spxentry{built\sphinxhyphen{}in function}}

\begin{fulllineitems}
\phantomsection\label{\detokenize{formules:page}}
\pysigstartsignatures
\pysiglinewithargsret
{\sphinxbfcode{\sphinxupquote{page}}}
{\sphinxparam{\DUrole{o}{*}\DUrole{n}{props}}}
{}
\pysigstopsignatures
\sphinxAtStartPar
Nurodo sekančio \sphinxcode{\sphinxupquote{next}} puslapio reikšmę.

\end{fulllineitems}


\end{fulllineitems}



\subsection{Periodiškumas}
\label{\detokenize{formules:periodiskumas}}
\sphinxAtStartPar
Periodiškumui nurodyti naudojamas model.prepare stulpelis, kuriame galima
naudoti tokias formules:


\begin{fulllineitems}

\pysigstartsignatures
\pysigline
{\sphinxbfcode{\sphinxupquote{model.prepare}}}
\pysigstopsignatures\index{built\sphinxhyphen{}in function@\spxentry{built\sphinxhyphen{}in function}!cron()@\spxentry{cron()}}\index{cron()@\spxentry{cron()}!built\sphinxhyphen{}in function@\spxentry{built\sphinxhyphen{}in function}}

\begin{fulllineitems}
\phantomsection\label{\detokenize{formules:cron}}
\pysigstartsignatures
\pysiglinewithargsret
{\sphinxbfcode{\sphinxupquote{cron}}}
{\sphinxparam{\DUrole{n}{line}}}
{}
\pysigstopsignatures
\sphinxAtStartPar
Duomenų atnaujinimo laikas, analogiškas \sphinxhref{https://en.wikipedia.org/wiki/Cron}{cron} formatui.

\sphinxAtStartPar
\sphinxcode{\sphinxupquote{line}} argumentas aprašomas taip:
\begin{description}
\sphinxlineitem{\sphinxcode{\sphinxupquote{n}}m}
\sphinxAtStartPar
\sphinxcode{\sphinxupquote{n}}\sphinxhyphen{}toji minutė, \sphinxcode{\sphinxupquote{n}} ∊ 0\sphinxhyphen{}59.

\sphinxlineitem{\sphinxcode{\sphinxupquote{n}}h}
\sphinxAtStartPar
\sphinxcode{\sphinxupquote{n}}\sphinxhyphen{}toji valanda, \sphinxcode{\sphinxupquote{n}} ∊ 0\sphinxhyphen{}23.

\sphinxlineitem{\sphinxcode{\sphinxupquote{n}}d}
\sphinxAtStartPar
\sphinxcode{\sphinxupquote{n}}\sphinxhyphen{}toji mėnesio diena, \sphinxcode{\sphinxupquote{n}} ∊ 1\sphinxhyphen{}31.

\sphinxlineitem{\$d}
\sphinxAtStartPar
Paskutinė mėnesio diena.

\sphinxlineitem{\sphinxcode{\sphinxupquote{n}}M}
\sphinxAtStartPar
\sphinxcode{\sphinxupquote{n}}\sphinxhyphen{}tasis mėnuo, \sphinxcode{\sphinxupquote{n}} ∊ 1\sphinxhyphen{}12.

\sphinxlineitem{\sphinxcode{\sphinxupquote{n}}w}
\sphinxAtStartPar
\sphinxcode{\sphinxupquote{n}}\sphinxhyphen{}toji savaitės diena, \sphinxcode{\sphinxupquote{n}} ∊ 0\sphinxhyphen{}6 (sekmadienis\sphinxhyphen{}šeštadienis).

\sphinxlineitem{\sphinxcode{\sphinxupquote{n}}\#\sphinxcode{\sphinxupquote{i}}w}
\sphinxAtStartPar
\sphinxcode{\sphinxupquote{n}}\sphinxhyphen{}toji savaitės diena, \sphinxcode{\sphinxupquote{i}}\sphinxhyphen{}toji mėnesio savaitė, \sphinxcode{\sphinxupquote{i}} ∊ 1\sphinxhyphen{}6.

\sphinxlineitem{\sphinxcode{\sphinxupquote{n}}\$\sphinxcode{\sphinxupquote{i}}w}
\sphinxAtStartPar
\sphinxcode{\sphinxupquote{n}}\sphinxhyphen{}toji savaitės diena, \sphinxcode{\sphinxupquote{i}}\sphinxhyphen{}toji savaitė nuo mėnesio galo, \sphinxcode{\sphinxupquote{i}}
∊ 1\sphinxhyphen{}6.

\end{description}
\begin{description}
\sphinxlineitem{,}
\sphinxAtStartPar
Kableliu galim atskirt kelias laiko vertes.

\sphinxlineitem{\sphinxhyphen{}}
\sphinxAtStartPar
Brūkšneliu galima atskirti laiko verčių intervalą.

\end{description}
\begin{description}
\sphinxlineitem{/}
\sphinxAtStartPar
Pasvyruoju brūkšniu galima atskirti laiko verčių kartojimo
žingsnį.

\end{description}

\sphinxAtStartPar
Laiko vertės atskiriamos tarpo simbolių. Jei laiko vertė nenurodyta,
reiškia įeina visos įmanomos laiko vertės reikšmės.

\end{fulllineitems}

\index{built\sphinxhyphen{}in function@\spxentry{built\sphinxhyphen{}in function}!hourly()@\spxentry{hourly()}}\index{hourly()@\spxentry{hourly()}!built\sphinxhyphen{}in function@\spxentry{built\sphinxhyphen{}in function}}

\begin{fulllineitems}
\phantomsection\label{\detokenize{formules:hourly}}
\pysigstartsignatures
\pysiglinewithargsret
{\sphinxbfcode{\sphinxupquote{hourly}}}
{}
{}
\pysigstopsignatures
\sphinxAtStartPar
{\hyperref[\detokenize{formules:cron}]{\sphinxcrossref{\sphinxcode{\sphinxupquote{cron('0m')}}}}}

\end{fulllineitems}

\index{built\sphinxhyphen{}in function@\spxentry{built\sphinxhyphen{}in function}!daily()@\spxentry{daily()}}\index{daily()@\spxentry{daily()}!built\sphinxhyphen{}in function@\spxentry{built\sphinxhyphen{}in function}}

\begin{fulllineitems}
\phantomsection\label{\detokenize{formules:daily}}
\pysigstartsignatures
\pysiglinewithargsret
{\sphinxbfcode{\sphinxupquote{daily}}}
{}
{}
\pysigstopsignatures
\sphinxAtStartPar
{\hyperref[\detokenize{formules:cron}]{\sphinxcrossref{\sphinxcode{\sphinxupquote{cron('0m 0d')}}}}}

\end{fulllineitems}

\index{built\sphinxhyphen{}in function@\spxentry{built\sphinxhyphen{}in function}!weekly()@\spxentry{weekly()}}\index{weekly()@\spxentry{weekly()}!built\sphinxhyphen{}in function@\spxentry{built\sphinxhyphen{}in function}}

\begin{fulllineitems}
\phantomsection\label{\detokenize{formules:weekly}}
\pysigstartsignatures
\pysiglinewithargsret
{\sphinxbfcode{\sphinxupquote{weekly}}}
{}
{}
\pysigstopsignatures
\sphinxAtStartPar
{\hyperref[\detokenize{formules:cron}]{\sphinxcrossref{\sphinxcode{\sphinxupquote{cron('0m 0h 0w')}}}}}

\end{fulllineitems}

\index{built\sphinxhyphen{}in function@\spxentry{built\sphinxhyphen{}in function}!monthly()@\spxentry{monthly()}}\index{monthly()@\spxentry{monthly()}!built\sphinxhyphen{}in function@\spxentry{built\sphinxhyphen{}in function}}

\begin{fulllineitems}
\phantomsection\label{\detokenize{formules:monthly}}
\pysigstartsignatures
\pysiglinewithargsret
{\sphinxbfcode{\sphinxupquote{monthly}}}
{}
{}
\pysigstopsignatures
\sphinxAtStartPar
{\hyperref[\detokenize{formules:cron}]{\sphinxcrossref{\sphinxcode{\sphinxupquote{cron('0m 0h 1d')}}}}}

\end{fulllineitems}

\index{built\sphinxhyphen{}in function@\spxentry{built\sphinxhyphen{}in function}!yearly()@\spxentry{yearly()}}\index{yearly()@\spxentry{yearly()}!built\sphinxhyphen{}in function@\spxentry{built\sphinxhyphen{}in function}}

\begin{fulllineitems}
\phantomsection\label{\detokenize{formules:yearly}}
\pysigstartsignatures
\pysiglinewithargsret
{\sphinxbfcode{\sphinxupquote{yearly}}}
{}
{}
\pysigstopsignatures
\sphinxAtStartPar
{\hyperref[\detokenize{formules:cron}]{\sphinxcrossref{\sphinxcode{\sphinxupquote{cron('0m 0h 1d 1M')}}}}}

\end{fulllineitems}


\end{fulllineitems}



\subsection{Statinės reikšmės}
\label{\detokenize{formules:statines-reiksmes}}
\sphinxAtStartPar
Statinės reikšmės arba konstantos duomenų laukams gali būti nurodomos
{\hyperref[\detokenize{dimensijos:property.prepare}]{\sphinxcrossref{\sphinxcode{\sphinxupquote{property.prepare}}}}} stulpelyje naudojant formulės sintaksę. Plačiau apie
formules žiūrėti {\hyperref[\detokenize{formules:formules}]{\sphinxcrossref{\DUrole{std}{\DUrole{std-ref}{Formulės}}}}} skyrelyje.


\subsection{Transformavimas}
\label{\detokenize{formules:transformavimas}}
\sphinxAtStartPar
{\hyperref[\detokenize{dimensijos:property.prepare}]{\sphinxcrossref{\sphinxcode{\sphinxupquote{property.prepare}}}}} stulpelyje gauta šaltinio reikšmė gali būti pasiekiama
per \sphinxcode{\sphinxupquote{self}} kintamąjį.

\sphinxAtStartPar
{\hyperref[\detokenize{dimensijos:property.prepare}]{\sphinxcrossref{\sphinxcode{\sphinxupquote{property.prepare}}}}} formulėje gali būti aprašomos kelios reikšmės atskirtos
kableliu, tai naudojama ryšio laukams, kai ryšiui aprašyti reikia daugiau nei
vieno duomenų lauko.

\sphinxAtStartPar
Formulėje galima naudoti kitus to pačio modelio property pavadinimus, kai
aprašomo {\hyperref[\detokenize{formatas:property}]{\sphinxcrossref{\sphinxcode{\sphinxupquote{property}}}}} reikšmės formuojamos dinamiškai naudojant vieną ar
kelis jau aprašytus laukus.

\sphinxAtStartPar
{\hyperref[\detokenize{dimensijos:property.prepare}]{\sphinxcrossref{\sphinxcode{\sphinxupquote{property.prepare}}}}} stulpelyje galima naudoti tokias formules:


\begin{fulllineitems}

\pysigstartsignatures
\pysigline
{\sphinxbfcode{\sphinxupquote{property.prepare}}}
\pysigstopsignatures\index{built\sphinxhyphen{}in function@\spxentry{built\sphinxhyphen{}in function}!null()@\spxentry{null()}}\index{null()@\spxentry{null()}!built\sphinxhyphen{}in function@\spxentry{built\sphinxhyphen{}in function}}

\begin{fulllineitems}
\phantomsection\label{\detokenize{formules:null}}
\pysigstartsignatures
\pysiglinewithargsret
{\sphinxbfcode{\sphinxupquote{null}}}
{}
{}
\pysigstopsignatures
\sphinxAtStartPar
Grąžina \sphinxcode{\sphinxupquote{null}} reikšmę, jei toliau einančios transformacijos grąžina
\sphinxcode{\sphinxupquote{null}}.

\end{fulllineitems}

\index{built\sphinxhyphen{}in function@\spxentry{built\sphinxhyphen{}in function}!replace()@\spxentry{replace()}}\index{replace()@\spxentry{replace()}!built\sphinxhyphen{}in function@\spxentry{built\sphinxhyphen{}in function}}

\begin{fulllineitems}
\phantomsection\label{\detokenize{formules:replace}}
\pysigstartsignatures
\pysiglinewithargsret
{\sphinxbfcode{\sphinxupquote{replace}}}
{\sphinxparam{\DUrole{n}{old}}\sphinxparamcomma \sphinxparam{\DUrole{n}{new}}}
{}
\pysigstopsignatures
\sphinxAtStartPar
Pakeičia visus \sphinxcode{\sphinxupquote{old}} į \sphinxcode{\sphinxupquote{new}} simbolius eilutėje.

\end{fulllineitems}

\index{built\sphinxhyphen{}in function@\spxentry{built\sphinxhyphen{}in function}!re()@\spxentry{re()}}\index{re()@\spxentry{re()}!built\sphinxhyphen{}in function@\spxentry{built\sphinxhyphen{}in function}}

\begin{fulllineitems}
\phantomsection\label{\detokenize{formules:re}}
\pysigstartsignatures
\pysiglinewithargsret
{\sphinxbfcode{\sphinxupquote{re}}}
{\sphinxparam{\DUrole{n}{pattern}}}
{}
\pysigstopsignatures
\sphinxAtStartPar
Grąžina atitinkančią reguliariosios išraiškos \sphinxcode{\sphinxupquote{pattern}} reikšmę arba
pirmos grupės reikšmę jei naudojama tik viena grupė arba reikšmių grupę
jei \sphinxcode{\sphinxupquote{pattern}} yra daugiau nei viena grupė.

\end{fulllineitems}

\index{built\sphinxhyphen{}in function@\spxentry{built\sphinxhyphen{}in function}!cast()@\spxentry{cast()}}\index{cast()@\spxentry{cast()}!built\sphinxhyphen{}in function@\spxentry{built\sphinxhyphen{}in function}}

\begin{fulllineitems}
\phantomsection\label{\detokenize{formules:cast}}
\pysigstartsignatures
\pysiglinewithargsret
{\sphinxbfcode{\sphinxupquote{cast}}}
{\sphinxparam{\DUrole{n}{type}}}
{}
\pysigstopsignatures
\sphinxAtStartPar
Konvertuoja šaltinio tipą į nurodytą \sphinxcode{\sphinxupquote{type}} tipą. Tipų konvertavimas yra
įmanomas tik tam tikrais atvejais. Jei tipų konvertuoti neįmanoma, tada
metodas turėtų grąžinti klaidą.

\end{fulllineitems}

\index{built\sphinxhyphen{}in function@\spxentry{built\sphinxhyphen{}in function}!split()@\spxentry{split()}}\index{split()@\spxentry{split()}!built\sphinxhyphen{}in function@\spxentry{built\sphinxhyphen{}in function}}

\begin{fulllineitems}
\phantomsection\label{\detokenize{formules:split}}
\pysigstartsignatures
\pysiglinewithargsret
{\sphinxbfcode{\sphinxupquote{split}}}
{}
{}
\pysigstopsignatures
\sphinxAtStartPar
Dalina simbolių eilutę naudojant \sphinxcode{\sphinxupquote{s+}} {\hyperref[\detokenize{savokos:term-reguliarioji-israiska}]{\sphinxtermref{\DUrole{xref}{\DUrole{std}{\DUrole{std-term}{reguliariąją išraišką}}}}}}. Grąžina masyvą.

\end{fulllineitems}

\index{built\sphinxhyphen{}in function@\spxentry{built\sphinxhyphen{}in function}!strip()@\spxentry{strip()}}\index{strip()@\spxentry{strip()}!built\sphinxhyphen{}in function@\spxentry{built\sphinxhyphen{}in function}}

\begin{fulllineitems}
\phantomsection\label{\detokenize{formules:strip}}
\pysigstartsignatures
\pysiglinewithargsret
{\sphinxbfcode{\sphinxupquote{strip}}}
{}
{}
\pysigstopsignatures
\sphinxAtStartPar
Pašalina tarpo simbolius iš pradžios ir pabaigos.

\end{fulllineitems}

\index{built\sphinxhyphen{}in function@\spxentry{built\sphinxhyphen{}in function}!lower()@\spxentry{lower()}}\index{lower()@\spxentry{lower()}!built\sphinxhyphen{}in function@\spxentry{built\sphinxhyphen{}in function}}

\begin{fulllineitems}
\phantomsection\label{\detokenize{formules:lower}}
\pysigstartsignatures
\pysiglinewithargsret
{\sphinxbfcode{\sphinxupquote{lower}}}
{}
{}
\pysigstopsignatures
\sphinxAtStartPar
Verčia visas raides mažosiomis.

\end{fulllineitems}

\index{built\sphinxhyphen{}in function@\spxentry{built\sphinxhyphen{}in function}!upper()@\spxentry{upper()}}\index{upper()@\spxentry{upper()}!built\sphinxhyphen{}in function@\spxentry{built\sphinxhyphen{}in function}}

\begin{fulllineitems}
\phantomsection\label{\detokenize{formules:upper}}
\pysigstartsignatures
\pysiglinewithargsret
{\sphinxbfcode{\sphinxupquote{upper}}}
{}
{}
\pysigstopsignatures
\sphinxAtStartPar
Verčia visas raides didžiosiomis.

\end{fulllineitems}

\index{built\sphinxhyphen{}in function@\spxentry{built\sphinxhyphen{}in function}!len()@\spxentry{len()}}\index{len()@\spxentry{len()}!built\sphinxhyphen{}in function@\spxentry{built\sphinxhyphen{}in function}}

\begin{fulllineitems}
\phantomsection\label{\detokenize{formules:len}}
\pysigstartsignatures
\pysiglinewithargsret
{\sphinxbfcode{\sphinxupquote{len}}}
{}
{}
\pysigstopsignatures
\sphinxAtStartPar
Grąžina elementų skaičių sekoje.

\end{fulllineitems}

\index{built\sphinxhyphen{}in function@\spxentry{built\sphinxhyphen{}in function}!choose()@\spxentry{choose()}}\index{choose()@\spxentry{choose()}!built\sphinxhyphen{}in function@\spxentry{built\sphinxhyphen{}in function}}

\begin{fulllineitems}
\phantomsection\label{\detokenize{formules:choose}}
\pysigstartsignatures
\pysiglinewithargsret
{\sphinxbfcode{\sphinxupquote{choose}}}
{\sphinxparam{\DUrole{n}{default}}}
{}
\pysigstopsignatures
\sphinxAtStartPar
Jei šaltinio reikšmė nėra viena iš {\hyperref[\detokenize{dimensijos:enum}]{\sphinxcrossref{\DUrole{std}{\DUrole{std-ref}{enum}}}}}, tada grąžinama default
reikšmė.

\sphinxAtStartPar
Jei \sphinxcode{\sphinxupquote{default}} nepateiktas, grąžina vieną iš {\hyperref[\detokenize{dimensijos:property.enum}]{\sphinxcrossref{\sphinxcode{\sphinxupquote{property.enum}}}}}
reikšmių, jei duomenų šaltinio reikšmė nėra viena iš
{\hyperref[\detokenize{dimensijos:property.enum}]{\sphinxcrossref{\sphinxcode{\sphinxupquote{property.enum}}}}}, tada grąžinama klaida.

\end{fulllineitems}

\index{built\sphinxhyphen{}in function@\spxentry{built\sphinxhyphen{}in function}!switch()@\spxentry{switch()}}\index{switch()@\spxentry{switch()}!built\sphinxhyphen{}in function@\spxentry{built\sphinxhyphen{}in function}}

\begin{fulllineitems}
\phantomsection\label{\detokenize{formules:switch}}
\pysigstartsignatures
\pysiglinewithargsret
{\sphinxbfcode{\sphinxupquote{switch}}}
{\sphinxparam{\DUrole{o}{*}\DUrole{n}{cases}}}
{}
\pysigstopsignatures
\end{fulllineitems}

\index{built\sphinxhyphen{}in function@\spxentry{built\sphinxhyphen{}in function}!case()@\spxentry{case()}}\index{case()@\spxentry{case()}!built\sphinxhyphen{}in function@\spxentry{built\sphinxhyphen{}in function}}

\begin{fulllineitems}
\phantomsection\label{\detokenize{formules:case}}
\pysigstartsignatures
\pysiglinewithargsret
{\sphinxbfcode{\sphinxupquote{case}}}
{\sphinxparam{\DUrole{n}{cond}}\sphinxparamcomma \sphinxparam{\DUrole{n}{value}}}
{}
\pysigstopsignatures
\end{fulllineitems}

\index{built\sphinxhyphen{}in function@\spxentry{built\sphinxhyphen{}in function}!case()@\spxentry{case()}}\index{case()@\spxentry{case()}!built\sphinxhyphen{}in function@\spxentry{built\sphinxhyphen{}in function}}

\begin{fulllineitems}
\phantomsection\label{\detokenize{formules:id15}}
\pysigstartsignatures
\pysiglinewithargsret
{\sphinxbfcode{\sphinxupquote{case}}}
{\sphinxparam{\DUrole{n}{default}}}
{}
\pysigstopsignatures
\sphinxAtStartPar
Grąžina \sphinxcode{\sphinxupquote{value}}, jei tenkina \sphinxcode{\sphinxupquote{cond}} arba \sphinxcode{\sphinxupquote{default}}. Jei \sphinxcode{\sphinxupquote{case(default)}}
nepateiktas, tada grąžina pradinę reikšmę.

\sphinxAtStartPar
Jei, \sphinxcode{\sphinxupquote{cases}} nepateikti, grąžina vieną iš \sphinxcode{\sphinxupquote{switch.source}}
reikšmių, tenkinančių switch prepare sąlygą.

\end{fulllineitems}

\index{built\sphinxhyphen{}in function@\spxentry{built\sphinxhyphen{}in function}!swap()@\spxentry{swap()}}\index{swap()@\spxentry{swap()}!built\sphinxhyphen{}in function@\spxentry{built\sphinxhyphen{}in function}}

\begin{fulllineitems}
\phantomsection\label{\detokenize{formules:swap}}
\pysigstartsignatures
\pysiglinewithargsret
{\sphinxbfcode{\sphinxupquote{swap}}}
{\sphinxparam{\DUrole{n}{old}}\sphinxparamcomma \sphinxparam{\DUrole{n}{new}}}
{}
\pysigstopsignatures
\sphinxAtStartPar
Pakeičia \sphinxcode{\sphinxupquote{old}} reikšmę \sphinxcode{\sphinxupquote{new}}, jeigu \sphinxcode{\sphinxupquote{self}} atitinka \sphinxcode{\sphinxupquote{old}}.

\end{fulllineitems}

\index{built\sphinxhyphen{}in function@\spxentry{built\sphinxhyphen{}in function}!return()@\spxentry{return()}}\index{return()@\spxentry{return()}!built\sphinxhyphen{}in function@\spxentry{built\sphinxhyphen{}in function}}

\begin{fulllineitems}
\phantomsection\label{\detokenize{formules:return}}
\pysigstartsignatures
\pysiglinewithargsret
{\sphinxbfcode{\sphinxupquote{return}}}
{}
{}
\pysigstopsignatures
\sphinxAtStartPar
Nutraukia transformacijų grandinę ir grąžina reikšmę.

\end{fulllineitems}

\index{built\sphinxhyphen{}in function@\spxentry{built\sphinxhyphen{}in function}!set()@\spxentry{set()}}\index{set()@\spxentry{set()}!built\sphinxhyphen{}in function@\spxentry{built\sphinxhyphen{}in function}}

\begin{fulllineitems}
\phantomsection\label{\detokenize{formules:id16}}
\pysigstartsignatures
\pysiglinewithargsret
{\sphinxbfcode{\sphinxupquote{set}}}
{\sphinxparam{\DUrole{n}{name}}}
{}
\pysigstopsignatures
\sphinxAtStartPar
Išsaugo reikšmę į kintamąjį \sphinxcode{\sphinxupquote{name}}.

\end{fulllineitems}

\index{built\sphinxhyphen{}in function@\spxentry{built\sphinxhyphen{}in function}!url()@\spxentry{url()}}\index{url()@\spxentry{url()}!built\sphinxhyphen{}in function@\spxentry{built\sphinxhyphen{}in function}}

\begin{fulllineitems}
\phantomsection\label{\detokenize{formules:url}}
\pysigstartsignatures
\pysiglinewithargsret
{\sphinxbfcode{\sphinxupquote{url}}}
{}
{}
\pysigstopsignatures
\sphinxAtStartPar
Skaido URI į objektą turintį tokias savybes:
\begin{description}
\sphinxlineitem{scheme}
\sphinxAtStartPar
URI schema.

\sphinxlineitem{netloc}
\sphinxAtStartPar
Visada URI dalis tarp scheme ir path.

\sphinxlineitem{username}
\sphinxAtStartPar
Naudotojo vardas.

\sphinxlineitem{password}
\sphinxAtStartPar
Slaptažodis.

\sphinxlineitem{host}
\sphinxAtStartPar
Domeno vardas arba IP adresas.

\sphinxlineitem{port}
\sphinxAtStartPar
Prievado numeris.

\sphinxlineitem{path}
\sphinxAtStartPar
Kelias.

\sphinxlineitem{query}
\sphinxAtStartPar
URL dalis einanti tarp \sphinxcode{\sphinxupquote{?}} ir \sphinxcode{\sphinxupquote{\#}}.

\sphinxlineitem{fragment}
\sphinxAtStartPar
URL dalis einanti po \#.

\end{description}

\end{fulllineitems}

\index{built\sphinxhyphen{}in function@\spxentry{built\sphinxhyphen{}in function}!query()@\spxentry{query()}}\index{query()@\spxentry{query()}!built\sphinxhyphen{}in function@\spxentry{built\sphinxhyphen{}in function}}

\begin{fulllineitems}
\phantomsection\label{\detokenize{formules:query}}
\pysigstartsignatures
\pysiglinewithargsret
{\sphinxbfcode{\sphinxupquote{query}}}
{}
{}
\pysigstopsignatures
\sphinxAtStartPar
Funkcija skirta darbui su URI query, skaido URI query dalį į parametrus
arba leidžia pakeisti duomenų šaltinio URI query naujomis reikšmimis.


\begin{sphinxseealso}{Taip pat žiūrėkite:}

\begin{DUlineblock}{0em}
\item[] {\hyperref[\detokenize{dimensijos:param.query}]{\sphinxcrossref{\sphinxcode{\sphinxupquote{param.query()}}}}}
\end{DUlineblock}


\end{sphinxseealso}


\end{fulllineitems}

\index{built\sphinxhyphen{}in function@\spxentry{built\sphinxhyphen{}in function}!path()@\spxentry{path()}}\index{path()@\spxentry{path()}!built\sphinxhyphen{}in function@\spxentry{built\sphinxhyphen{}in function}}

\begin{fulllineitems}
\phantomsection\label{\detokenize{formules:path}}
\pysigstartsignatures
\pysiglinewithargsret
{\sphinxbfcode{\sphinxupquote{path}}}
{}
{}
\pysigstopsignatures
\sphinxAtStartPar
Skaido failų sistemos arba URI kelią į tokias savybes:
\begin{description}
\sphinxlineitem{parts}
\sphinxAtStartPar
Skaido kelią į dalis (\sphinxhref{https://docs.python.org/3/library/pathlib.html\#pathlib.PurePath.parts}{plačiau}).

\sphinxlineitem{drive}
\sphinxAtStartPar
Diskas (\sphinxhref{https://docs.python.org/3/library/pathlib.html\#pathlib.PurePath.drive}{plačiau}).

\sphinxlineitem{root}
\sphinxAtStartPar
Šaknis (\sphinxhref{https://docs.python.org/3/library/pathlib.html\#pathlib.PurePath.root}{plačiau}).

\end{description}


\begin{sphinxseealso}{Taip pat žiūrėkite:}

\begin{DUlineblock}{0em}
\item[] {\hyperref[\detokenize{dimensijos:param.path}]{\sphinxcrossref{\sphinxcode{\sphinxupquote{param.path()}}}}}
\end{DUlineblock}


\end{sphinxseealso}


\end{fulllineitems}


\end{fulllineitems}



\subsection{Kompleksinės struktūros}
\label{\detokenize{formules:kompleksines-strukturos}}\label{\detokenize{formules:id20}}
\sphinxAtStartPar
Daugelis duomenų šaltiniu turi galimybę saugoti kompleksines struktūras. Jei
duomenys yra kompleksiniai, tada {\hyperref[\detokenize{dimensijos:property.source}]{\sphinxcrossref{\sphinxcode{\sphinxupquote{property.source}}}}} stulpelyje galima
nurodyti tik duomens pavadinimą iš pirmojo lygmens, gilesniuose lygmenyse
esančius duomenis galima aprašyti naudojant formules {\hyperref[\detokenize{dimensijos:property.prepare}]{\sphinxcrossref{\sphinxcode{\sphinxupquote{property.prepare}}}}}
stulpelyje.

\sphinxAtStartPar
Analogiškai duomenų atranką galima daryti ir \sphinxcode{\sphinxupquote{model}} eilutėse, jei tai leidžia
duomenų šaltinis.

\sphinxAtStartPar
Kaip pavyzdį naudosime tokią {\hyperref[\detokenize{savokos:term-JSON}]{\sphinxtermref{\DUrole{xref}{\DUrole{std}{\DUrole{std-term}{JSON}}}}}} duomenų struktūrą:

\begin{sphinxVerbatim}[commandchars=\\\{\}]
\PYG{p}{\PYGZob{}}
\PYG{+w}{    }\PYG{n+nt}{\PYGZdq{}result\PYGZdq{}}\PYG{p}{:}\PYG{+w}{ }\PYG{p}{\PYGZob{}}
\PYG{+w}{        }\PYG{n+nt}{\PYGZdq{}count\PYGZdq{}}\PYG{p}{:}\PYG{+w}{ }\PYG{l+m+mi}{1}\PYG{p}{,}
\PYG{+w}{        }\PYG{n+nt}{\PYGZdq{}results\PYGZdq{}}\PYG{p}{:}\PYG{+w}{ }\PYG{p}{[}
\PYG{+w}{            }\PYG{p}{\PYGZob{}}
\PYG{+w}{                }\PYG{n+nt}{\PYGZdq{}type\PYGZdq{}}\PYG{p}{:}\PYG{+w}{ }\PYG{l+s+s2}{\PYGZdq{}dataset\PYGZdq{}}\PYG{p}{,}
\PYG{+w}{                }\PYG{n+nt}{\PYGZdq{}tags\PYGZdq{}}\PYG{p}{:}\PYG{+w}{ }\PYG{p}{[}\PYG{l+s+s2}{\PYGZdq{}CSV\PYGZdq{}}\PYG{p}{]}
\PYG{+w}{            }\PYG{p}{\PYGZcb{}}
\PYG{+w}{        }\PYG{p}{]}
\PYG{+w}{    }\PYG{p}{\PYGZcb{}}
\PYG{p}{\PYGZcb{}}
\end{sphinxVerbatim}


\begin{fulllineitems}

\pysigstartsignatures
\pysigline
{\sphinxbfcode{\sphinxupquote{property.prepare}}}
\pysigstopsignatures\index{built\sphinxhyphen{}in function@\spxentry{built\sphinxhyphen{}in function}!getattr()@\spxentry{getattr()}}\index{getattr()@\spxentry{getattr()}!built\sphinxhyphen{}in function@\spxentry{built\sphinxhyphen{}in function}}

\begin{fulllineitems}
\phantomsection\label{\detokenize{formules:id21}}
\pysigstartsignatures
\pysiglinewithargsret
{\sphinxbfcode{\sphinxupquote{getattr}}}
{\sphinxparam{\DUrole{n}{object}}\sphinxparamcomma \sphinxparam{\DUrole{n}{name}}}
{}
\pysigstopsignatures
\sphinxAtStartPar
Grąžina \sphinxcode{\sphinxupquote{object}} savybę \sphinxcode{\sphinxupquote{name}}.

\begin{sphinxVerbatim}[commandchars=\\\{\}]
\PYG{g+gp}{\PYGZgt{}\PYGZgt{}\PYGZgt{} }\PYG{n+nb+bp}{self}\PYG{o}{.}\PYG{n}{result}\PYG{o}{.}\PYG{n}{count}
\PYG{g+go}{1}
\end{sphinxVerbatim}

\end{fulllineitems}

\index{built\sphinxhyphen{}in function@\spxentry{built\sphinxhyphen{}in function}!getitem()@\spxentry{getitem()}}\index{getitem()@\spxentry{getitem()}!built\sphinxhyphen{}in function@\spxentry{built\sphinxhyphen{}in function}}

\begin{fulllineitems}
\phantomsection\label{\detokenize{formules:id22}}
\pysigstartsignatures
\pysiglinewithargsret
{\sphinxbfcode{\sphinxupquote{getitem}}}
{\sphinxparam{\DUrole{n}{object}}\sphinxparamcomma \sphinxparam{\DUrole{n}{item}}}
{}
\pysigstopsignatures
\sphinxAtStartPar
Grąžina \sphinxcode{\sphinxupquote{object}} objekto \sphinxcode{\sphinxupquote{item}} savybę arba \sphinxcode{\sphinxupquote{object}} masyvo \sphinxcode{\sphinxupquote{item}}
elementą.

\begin{sphinxVerbatim}[commandchars=\\\{\}]
\PYG{g+gp}{\PYGZgt{}\PYGZgt{}\PYGZgt{} }\PYG{n+nb+bp}{self}\PYG{p}{[}\PYG{l+s+s2}{\PYGZdq{}}\PYG{l+s+s2}{result}\PYG{l+s+s2}{\PYGZdq{}}\PYG{p}{]}\PYG{p}{[}\PYG{l+s+s2}{\PYGZdq{}}\PYG{l+s+s2}{count}\PYG{l+s+s2}{\PYGZdq{}}\PYG{p}{]}
\PYG{g+go}{1}
\end{sphinxVerbatim}

\sphinxAtStartPar
{\hyperref[\detokenize{formules:id22}]{\sphinxcrossref{\sphinxcode{\sphinxupquote{getitem()}}}}} ir {\hyperref[\detokenize{formules:id21}]{\sphinxcrossref{\sphinxcode{\sphinxupquote{getattr()}}}}} gali būti naudojami kartu.

\begin{sphinxVerbatim}[commandchars=\\\{\}]
\PYG{g+gp}{\PYGZgt{}\PYGZgt{}\PYGZgt{} }\PYG{n+nb+bp}{self}\PYG{o}{.}\PYG{n}{result}\PYG{o}{.}\PYG{n}{results}\PYG{p}{[}\PYG{l+m+mi}{0}\PYG{p}{]}\PYG{o}{.}\PYG{n}{type}
\PYG{g+go}{\PYGZdq{}dataset\PYGZdq{}}
\end{sphinxVerbatim}

\sphinxAtStartPar
{\hyperref[\detokenize{formules:id22}]{\sphinxcrossref{\sphinxcode{\sphinxupquote{getitem()}}}}} gali būti naudojamas, kaip masyvo elementų filtras
pateikiant filtro sąlygą.

\begin{sphinxVerbatim}[commandchars=\\\{\}]
\PYG{g+gp}{\PYGZgt{}\PYGZgt{}\PYGZgt{} }\PYG{n+nb+bp}{self}\PYG{o}{.}\PYG{n}{result}\PYG{o}{.}\PYG{n}{results}\PYG{p}{[}\PYG{n}{tags} \PYG{o}{=} \PYG{l+s+s2}{\PYGZdq{}}\PYG{l+s+s2}{CSV}\PYG{l+s+s2}{\PYGZdq{}}\PYG{p}{]}\PYG{o}{.}\PYG{n}{type}
\PYG{g+go}{[\PYGZdq{}dataset\PYGZdq{}]}

\PYG{g+gp}{\PYGZgt{}\PYGZgt{}\PYGZgt{} }\PYG{n+nb+bp}{self}\PYG{o}{.}\PYG{n}{result}\PYG{o}{.}\PYG{n}{results}\PYG{p}{[}\PYG{n}{item}\PYG{p}{(}\PYG{n}{tags}\PYG{p}{)} \PYG{o}{=} \PYG{l+s+s2}{\PYGZdq{}}\PYG{l+s+s2}{CSV}\PYG{l+s+s2}{\PYGZdq{}}\PYG{p}{]}\PYG{o}{.}\PYG{n}{type}
\PYG{g+go}{[\PYGZdq{}dataset\PYGZdq{}]}
\end{sphinxVerbatim}

\sphinxAtStartPar
Norint gauti visus masyvo elementus, galima naudoti tokią išraišką:

\begin{sphinxVerbatim}[commandchars=\\\{\}]
\PYG{g+gp}{\PYGZgt{}\PYGZgt{}\PYGZgt{} }\PYG{n+nb+bp}{self}\PYG{o}{.}\PYG{n}{result}\PYG{o}{.}\PYG{n}{results}\PYG{p}{[}\PYG{p}{]}\PYG{o}{.}\PYG{n}{tags}\PYG{p}{[}\PYG{p}{]}
\PYG{g+go}{[\PYGZdq{}CSV\PYGZdq{}]}
\end{sphinxVerbatim}

\end{fulllineitems}

\index{built\sphinxhyphen{}in function@\spxentry{built\sphinxhyphen{}in function}!first()@\spxentry{first()}}\index{first()@\spxentry{first()}!built\sphinxhyphen{}in function@\spxentry{built\sphinxhyphen{}in function}}

\begin{fulllineitems}
\phantomsection\label{\detokenize{formules:first}}
\pysigstartsignatures
\pysiglinewithargsret
{\sphinxbfcode{\sphinxupquote{first}}}
{\sphinxparam{\DUrole{n}{object}}\sphinxparamcomma \sphinxparam{\DUrole{n}{default}}}
{}
\pysigstopsignatures
\sphinxAtStartPar
Grąžina pirmą \sphinxcode{\sphinxupquote{object}} sąrašo reikšmę, jei sąrašas tuščias, tada
grąžina \sphinxcode{\sphinxupquote{default}} reikšmę, jei \sphinxcode{\sphinxupquote{default}} nenurodytas, tada nutraukia
vykdymą su klaidą.

\begin{sphinxVerbatim}[commandchars=\\\{\}]
\PYG{g+gp}{\PYGZgt{}\PYGZgt{}\PYGZgt{} }\PYG{n+nb+bp}{self}\PYG{o}{.}\PYG{n}{result}\PYG{o}{.}\PYG{n}{results}\PYG{p}{[}\PYG{p}{]}\PYG{o}{.}\PYG{n}{tags}\PYG{o}{.}\PYG{n}{first}\PYG{p}{(}\PYG{p}{)}
\PYG{g+go}{\PYGZdq{}CSV\PYGZdq{}}
\end{sphinxVerbatim}

\sphinxAtStartPar
Jei {\color{red}\bfseries{}`}self.result.results{[}{]}.tags būtų tuščias, tada:

\begin{sphinxVerbatim}[commandchars=\\\{\}]
\PYG{g+gp}{\PYGZgt{}\PYGZgt{}\PYGZgt{} }\PYG{n+nb+bp}{self}\PYG{o}{.}\PYG{n}{result}\PYG{o}{.}\PYG{n}{results}\PYG{p}{[}\PYG{p}{]}\PYG{o}{.}\PYG{n}{tags}\PYG{o}{.}\PYG{n}{first}\PYG{p}{(}\PYG{n}{null}\PYG{p}{)}
\PYG{g+go}{null}
\end{sphinxVerbatim}

\end{fulllineitems}


\end{fulllineitems}


\sphinxAtStartPar
Analogiška struktūra gali būti gaunama ir lentelėse, kai stulpelių pavadinimai
nurodyti keliose eilutėse, pavyzdyje pateiktą struktūrą atitiktų tokia lentelė:


\begin{savenotes}\sphinxattablestart
\sphinxthistablewithglobalstyle
\centering
\begin{tabulary}{\linewidth}[t]{TTT}
\sphinxtoprule
\sphinxstyletheadfamily 
\sphinxAtStartPar
result
&\sphinxstyletheadfamily &\sphinxstyletheadfamily \\
\sphinxhline\sphinxstyletheadfamily 
\sphinxAtStartPar
count
&\sphinxstyletheadfamily 
\sphinxAtStartPar
results
&\sphinxstyletheadfamily \\
\sphinxhline
\sphinxAtStartPar

&\sphinxstyletheadfamily 
\sphinxAtStartPar
type
&\sphinxstyletheadfamily 
\sphinxAtStartPar
tags
\\
\sphinxmidrule
\sphinxtableatstartofbodyhook
\sphinxAtStartPar
1
&
\sphinxAtStartPar
dataset
&
\sphinxAtStartPar
CSV
\\
\sphinxbottomrule
\end{tabulary}
\sphinxtableafterendhook\par
\sphinxattableend\end{savenotes}

\sphinxAtStartPar
Šioje lentelėje stulpelių pavadinimai pateikti trijose eilutėse, todėl
\sphinxcode{\sphinxupquote{model.prepare}} reikėtų naudoti {\hyperref[\detokenize{formules:header}]{\sphinxcrossref{\sphinxcode{\sphinxupquote{header(0, 1, 2)}}}}}.

\sphinxstepscope


\section{Sąvokos}
\label{\detokenize{savokos:savokos}}\label{\detokenize{savokos::doc}}\begin{description}
\sphinxlineitem{ADK\index{ADK@\spxentry{ADK}|spxpagem}\phantomsection\label{\detokenize{savokos:term-ADK}}}
\sphinxAtStartPar
Lietuvos atvirų duomenų katalogas, prieinamas adresu \sphinxhref{https://data.gov.lt/}{data.gov.lt}.

\sphinxlineitem{ADP\index{ADP@\spxentry{ADP}|spxpagem}\phantomsection\label{\detokenize{savokos:term-ADP}}}
\sphinxAtStartPar
Atvirų duomenų portalas, sudarytas iš {\hyperref[\detokenize{savokos:term-ADK}]{\sphinxtermref{\DUrole{xref}{\DUrole{std}{\DUrole{std-term}{atvirų duomenų katalogo}}}}}} ir {\hyperref[\detokenize{savokos:term-ADS}]{\sphinxtermref{\DUrole{xref}{\DUrole{std}{\DUrole{std-term}{duomenų saugyklos}}}}}}.

\sphinxlineitem{ADS\index{ADS@\spxentry{ADS}|spxpagem}\phantomsection\label{\detokenize{savokos:term-ADS}}}
\sphinxAtStartPar
\DUrole{xref}{\DUrole{std}{\DUrole{std-ref}{Atvirų duomenų saugykla}}}, skirta pakartotinio
panaudojimo duomenų publikavimui, valstybinė atvirų duomenų saugykla
pasiekiama \sphinxhref{https://get.data.gov.lt/}{get.data.gov.lt} adresu.

\sphinxlineitem{ADSA\index{ADSA@\spxentry{ADSA}|spxpagem}\phantomsection\label{\detokenize{savokos:term-ADSA}}}
\sphinxAtStartPar
{\hyperref[\detokenize{savokos:term-DSA}]{\sphinxtermref{\DUrole{xref}{\DUrole{std}{\DUrole{std-term}{DSA}}}}}} lentelė, kurioje aprašomi jau atverti ir viešai prieinami
duomenys.

\sphinxlineitem{aplinkos kintamasis\index{aplinkos kintamasis@\spxentry{aplinkos kintamasis}|spxpagem}\phantomsection\label{\detokenize{savokos:term-aplinkos-kintamasis}}}
\sphinxAtStartPar
Angliškai tai vadinama \sphinxstyleemphasis{environment variables}, tai yra operacinės
sistemos aplinkos kintamieji.

\sphinxAtStartPar
Plačiau apie tai skaitykite \sphinxhref{https://en.wikipedia.org/wiki/Environment\_variable}{Vikipedijoje}.

\sphinxlineitem{atvirų duomenų direktyva\index{atvirų duomenų direktyva@\spxentry{atvirų duomenų direktyva}|spxpagem}\phantomsection\label{\detokenize{savokos:term-atviru-duomenu-direktyva}}}
\sphinxAtStartPar
2019 m. birželio 20 d. Europos Parlamento ir Tarybos direktyva (ES)
\sphinxhref{https://eur-lex.europa.eu/legal-content/LT/TXT/?uri=CELEX:32019L1024}{2019/1024} dėl atvirųjų duomenų ir viešojo sektoriaus informacijos
pakartotinio naudojimo.

\sphinxlineitem{BDAR\index{BDAR@\spxentry{BDAR}|spxpagem}\phantomsection\label{\detokenize{savokos:term-BDAR}}}
\sphinxAtStartPar
2016 m. balandžio 27 d. Europos Parlamento ir Tarybos reglamentas (ES)
\sphinxhref{https://eur-lex.europa.eu/legal-content/LT/TXT/?uri=CELEX:32016R0679}{2016/679} dėl fizinių asmenų apsaugos tvarkant asmens duomenis ir dėl
laisvo tokių duomenų judėjimo ir kuriuo panaikinama Direktyva
\sphinxhref{https://eur-lex.europa.eu/legal-content/LT/TXT/?uri=CELEX:31995L0046}{95/46/EB} (Bendrasis duomenų apsaugos reglamentas).

\sphinxlineitem{brandos lygis\index{brandos lygis@\spxentry{brandos lygis}|spxpagem}\phantomsection\label{\detokenize{savokos:term-brandos-lygis}}}
\sphinxAtStartPar
Duomenų brandos lygiai yra apibrėžti \sphinxhref{https://5stardata.info/}{5 ★ Open Data} svetainėje.
Viso yra penki brandos lygiai, tačiau papildomai verta įtraukti ir
nulinį brandos lygį, kai duomenų poreikis yra, tačiau duomenys
nekaupiami arba negali būti publikuojami dėl teisinių ar kitų
apribojimų.

\sphinxAtStartPar
\sphinxhref{https://5stardata.info/}{5 ★ Open Data} svetainėje brandos lygia apibrėžti, kaip pavyzdį
nurodant formatus. Nors formatus galima naudoti kaip pavyzdį labai
abstrakčiai apibūdinant ką reiškia brandos lygiai, tačiau tikslus
brandos lygis gali būti suteiktas tik atskiriems duomenų  laukams, o
ne formatui.

\sphinxAtStartPar
Duomenų brandos lygiai yra tokie:
\begin{description}
\sphinxlineitem{0}
\sphinxAtStartPar
Duomenys nekaupiami, tačiau poreikis tokiems duomenims yra. Gali
būti ir tokių atvejų, kai duomenys yra kaupiami, tačiau dėl teisinių
ar kitų priežasčių negali būti publikuojami.

\sphinxlineitem{1}
\sphinxAtStartPar
Duomenys kaupiami ir publikuojami viešai, bet kokia forma ir bet
kokiu formatu. Pavyzdžiui datos tipo laukas gali būti pateikiamas
įvairiais formatais „Pirmadienis“, „2021 gegužės 10 d.“,
„5/10/21“ ir pan. Kadangi šiuo atveju data gali būti užrašyta bet
kokia forma ir bet kokiu tikslumu, nėra galimybės automatinėmis
priemonėmis patikimai nuskaityti tokių duomenų.

\sphinxlineitem{2}
\sphinxAtStartPar
Publikuojami duomenys turi aiškią, mašininiu būdu nuskaitomą
struktūrą, tačiau pateikiami nestandartiniu arba nuosavybiniu
formatu. Pavyzdžiui datos tipo lauko duomenys pateikiami
nestandartiniu formatu, tačiau visos reikšmės pateiktos naudojant tą
patį formatą, „5/10/21“, „6/10/21“ ir pan. Šiuo atveju, automatiškai
nuskaityti tokius duomenis įmanoma tik papildomai įgyvendinant
duomenų nuskaitymo priemones, kuriose yra įgyvendintas būtent tokio
nestandartinio formato duomenų skaitymas.

\sphinxlineitem{3}
\sphinxAtStartPar
Duomenys pateikiami naudojant standartinį formatą. Lietuvos atvirų
duomenų kontekste, {\hyperref[\detokenize{tipai:duomenu-tipai}]{\sphinxcrossref{\DUrole{std}{\DUrole{std-ref}{standartiniai formatai yra apibrėžti
duomenų struktūros aprašo specifikacijoje}}}}}.
Pavyzdžiui datos tipo lauko duomenys pateikiami standartiniu \sphinxhref{https://en.wikipedia.org/wiki/ISO\_8601}{ISO
8601} formatu. Kadangi duomenys yra pateikti standartiniu formatu,
pačio formato specifikacija yra atvira ir viešai publikuojama, o
duomenų nuskaitymo priemonės tokį atvirą formatą palaiko, todėl
tokių duomenų nuskaitymui nereikia įdėti jokio papildomo darbo.

\sphinxlineitem{4}
\sphinxAtStartPar
Kiekvienas publikuojamų duomenų {\hyperref[\detokenize{savokos:term-objektas}]{\sphinxtermref{\DUrole{xref}{\DUrole{std}{\DUrole{std-term}{objektas}}}}}} turi unikalų
identifikatorių ir naudojant tokius unikalius objektų
identifikatorius, skirtingų tipų objektai siejami tarpusavyje.
Kartu su duomenimis pateikiami ir metaduomenys apie tai, kaip
skirtingų tipų objektai siejasi tarpusavyje.

\sphinxAtStartPar
Pavyzdžiui miesto tipo objektui „Vilnius“ yra suteiktas unikalus
identifikatorius \sphinxtitleref{6868eca7\sphinxhyphen{}0ae1\sphinxhyphen{}4390\sphinxhyphen{}83d0\sphinxhyphen{}7af642a62863}, o šalies
tipo objekto „Lietuva“ duomenų lauko „sostinė“ reikšmė yra objekto
„Vilnius“ unikalus identifikatorius
\sphinxtitleref{6868eca7\sphinxhyphen{}0ae1\sphinxhyphen{}4390\sphinxhyphen{}83d0\sphinxhyphen{}7af642a62863}.

\sphinxAtStartPar
Turint tokį brandos lygį, duomenis galima ne tik nuskaityti, bet ir
jungti tarpusavyje, o jungiant skirtingus duomenis tarpusavyje
atsiveria daugiau galimybių juos naudoti įvairiuose taikymuose.

\sphinxlineitem{5}
\sphinxAtStartPar
Kartu su publikuojamais duomenimis, pateikiami ir metaduomenys
apie tai, kaip publikuojami duomenys siejasi su kitais viešaisiais
duomenų žodynais (ontologijomis). Pavyzdžiui datos duomenų laukas
yra susiejamas su „Dublin Core Metadata Initiative“ publikuojama
ontologija, nurodant, kad datos lauko semantinė prasmė yra tokia
pati, kaip apibrėžta \sphinxhref{http://purl.org/dc/terms/created}{dcterms:created} ontologijoje. Šiuo atveju,
nurodoma, kad datos laukas būtent yra tam tikro resurso sukūrimo
data.

\sphinxAtStartPar
Kai duomenys yra susieti su išoriniais žodynais, atsiranda galimybė
įgyvendinti tokias priemones, kurios veiktų universaliai,
nepriklausomai nuo duomenų šaltinio ar duomenų kilmės.

\end{description}

\sphinxlineitem{daugiareikšmis tipas\index{daugiareikšmis tipas@\spxentry{daugiareikšmis tipas}|spxpagem}\phantomsection\label{\detokenize{savokos:term-daugiareiksmis-tipas}}}
\sphinxAtStartPar
Duomenų tipas nurodantis, kad viena savybė gali turėti daugiau nei
vieną, to pačio tipo reikšmę.

\sphinxAtStartPar
Struktūros aprašuose daugiareimšmės savybės žymimos \sphinxtitleref{{[}{]}} simboliais,
užrašomais savybės pavadinimo pabaigoje.


\begin{sphinxseealso}{Taip pat žiūrėkite:}
\begin{itemize}
\item {} 
\sphinxAtStartPar
{\hyperref[\detokenize{tipai:type.array}]{\sphinxcrossref{\sphinxcode{\sphinxupquote{type.array}}}}}

\end{itemize}


\end{sphinxseealso}


\sphinxlineitem{duomenų valdymo aktas\index{duomenų valdymo aktas@\spxentry{duomenų valdymo aktas}|spxpagem}\phantomsection\label{\detokenize{savokos:term-duomenu-valdymo-aktas}}}
\sphinxAtStartPar
2020 m. lapkričio 25 d. Europos Parlamento ir Tarybos reglamento (ES)
pasiūlymas \sphinxhref{https://eur-lex.europa.eu/legal-content/LT/TXT/?uri=CELEX:52020PC0767}{2020/0340} dėl Europos duomenų valdymo (Duomenų valdymo
aktas).

\sphinxlineitem{duomenų katalogas\index{duomenų katalogas@\spxentry{duomenų katalogas}|spxpagem}\phantomsection\label{\detokenize{savokos:term-duomenu-katalogas}}}
\sphinxAtStartPar
Lietuvos duomenų portalo sudedamoji dalis, skirta metaduomenims apie
duomenų šaltinius registruoti.

\sphinxAtStartPar
Duomenų katalogas prieinamas adresu \sphinxhref{https://data.gov.lt/}{data.gov.lt}.

\sphinxlineitem{didelės vertės duomenys\index{didelės vertės duomenys@\spxentry{didelės vertės duomenys}|spxpagem}\phantomsection\label{\detokenize{savokos:term-dideles-vertes-duomenys}}}\sphinxlineitem{aukštos vertės duomenys\index{aukštos vertės duomenys@\spxentry{aukštos vertės duomenys}|spxpagem}\phantomsection\label{\detokenize{savokos:term-aukstos-vertes-duomenys}}}
\sphinxAtStartPar
Duomenys apibrėžti {\hyperref[\detokenize{savokos:term-atviru-duomenu-direktyva}]{\sphinxtermref{\DUrole{xref}{\DUrole{std}{\DUrole{std-term}{atvirų duomenų direktyvos}}}}}} 5 skyriuje.

\sphinxAtStartPar
\sphinxhref{https://eur-lex.europa.eu/legal-content/LT/TXT/?qid=1561563110433\&uri=CELEX:32019L1024\#d1e32-79-1}{Aukštos vertės duomenų sritys} yra šios:
\begin{itemize}
\item {} 
\sphinxAtStartPar
Geoerdviniai duomenys

\item {} 
\sphinxAtStartPar
Aplinka ir žemės stebėjimai

\item {} 
\sphinxAtStartPar
Meteorologiniai duomenys

\item {} 
\sphinxAtStartPar
Statistika (demografiniai ir ekonominiai rodikliai)

\item {} 
\sphinxAtStartPar
Įmonės ir įmonių savininkai

\item {} 
\sphinxAtStartPar
Judumas

\end{itemize}

\sphinxlineitem{duomenų serializavimo formatas\index{duomenų serializavimo formatas@\spxentry{duomenų serializavimo formatas}|spxpagem}\phantomsection\label{\detokenize{savokos:term-duomenu-serializavimo-formatas}}}
\sphinxAtStartPar
Duomenys gali būti serializuojami įvairiais formatais, pavyzdžiui YAML
formatu:

\begin{sphinxVerbatim}[commandchars=\\\{\}]
\PYG{n+nt}{type}\PYG{p}{:}\PYG{+w}{ }\PYG{l+lScalar+lScalarPlain}{project}
\PYG{n+nt}{title}\PYG{p}{:}\PYG{+w}{ }\PYG{l+lScalar+lScalarPlain}{Manifestas}
\end{sphinxVerbatim}

\sphinxAtStartPar
JSON formatu:

\begin{sphinxVerbatim}[commandchars=\\\{\}]
\PYG{p}{\PYGZob{}}\PYG{n+nt}{\PYGZdq{}type\PYGZdq{}}\PYG{p}{:}\PYG{+w}{ }\PYG{l+s+s2}{\PYGZdq{}project\PYGZdq{}}\PYG{p}{,}\PYG{+w}{ }\PYG{n+nt}{\PYGZdq{}title\PYGZdq{}}\PYG{p}{:}\PYG{+w}{ }\PYG{l+s+s2}{\PYGZdq{}Manifestas\PYGZdq{}}\PYG{p}{\PYGZcb{}}
\end{sphinxVerbatim}

\sphinxAtStartPar
Turtle formatu:

\begin{sphinxVerbatim}[commandchars=\\\{\}]
@\PYG{n+nv}{prefix}\PYG{+w}{ }\PYG{n+nv}{foaf}:\PYG{+w}{ }\PYG{o}{\PYGZlt{}}\PYG{n+nv}{http}:\PYG{o}{//}\PYG{n+nv}{xmlns}.\PYG{n+nv}{com}\PYG{o}{/}\PYG{n+nv}{foaf}\PYG{o}{/}\PYG{l+m+mi}{0}.\PYG{l+m+mi}{1}\PYG{o}{/\PYGZgt{}}\PYG{+w}{ }.
@\PYG{n+nv}{prefix}\PYG{+w}{ }\PYG{n+nv}{rdfs}:\PYG{+w}{ }\PYG{o}{\PYGZlt{}}\PYG{n+nv}{http}:\PYG{o}{//}\PYG{n+nv}{www}.\PYG{n+nv}{w3}.\PYG{n+nv}{org}\PYG{o}{/}\PYG{l+m+mi}{2000}\PYG{o}{/}\PYG{l+m+mi}{01}\PYG{o}{/}\PYG{n+nv}{rdf}\PYG{o}{\PYGZhy{}}\PYG{n+nv}{schema}\PYGZsh{}\PYG{o}{\PYGZgt{}}\PYG{+w}{ }.
\PYG{o}{\PYGZlt{}}\PYG{n+nv}{http}:\PYG{o}{//}\PYG{n+nv}{atviriduomenys}.\PYG{n+nv}{lt}\PYG{o}{\PYGZgt{}}\PYG{+w}{ }\PYG{n+nv}{a}\PYG{+w}{ }\PYG{n+nv}{foaf}:\PYG{n+nv}{Project}\PYG{c+c1}{;}
\PYG{+w}{    }\PYG{n+nv}{rdfs}:\PYG{n+nv}{label}\PYG{+w}{ }\PYG{l+s+s2}{\PYGZdq{}Manifestas\PYGZdq{}}\PYG{+w}{ }.
\end{sphinxVerbatim}

\sphinxAtStartPar
MessagePack dvejetainiu formatu, kurio turinys pateiktas naudojant BASE64
koduotę:

\begin{sphinxVerbatim}[commandchars=\\\{\}]
\PYG{n}{gqR0eXBlp3Byb2plY3SkbmFtZapNYW5pZmVzdGFz}
\end{sphinxVerbatim}

\sphinxAtStartPar
Visuose šiuose pavyzdžiuose yra pateikti tie patys duomenys, tačiau
naudojami skirtingi duomenų serializavimo formatai, koduotės ir skirtingi
žodynai.

\sphinxlineitem{DSA\index{DSA@\spxentry{DSA}|spxpagem}\phantomsection\label{\detokenize{savokos:term-DSA}}}
\sphinxAtStartPar
\DUrole{xref}{\DUrole{std}{\DUrole{std-ref}{Duomenų struktūros aprašas}}} yra
lentelė, kurioje išsamiai aprašyta tam tikro duomenų šaltinio duomenų
struktūra. DSA lentelę sudaro penkios dimensijos (duomenų rinkinys,
resursas, bazė, modelis, savybė) ir dešimt metaduomenų stulpelių.

\sphinxlineitem{EIF\index{EIF@\spxentry{EIF}|spxpagem}\phantomsection\label{\detokenize{savokos:term-EIF}}}\sphinxlineitem{Europos sąveikumo karkasas\index{Europos sąveikumo karkasas@\spxentry{Europos sąveikumo karkasas}|spxpagem}\phantomsection\label{\detokenize{savokos:term-Europos-saveikumo-karkasas}}}
\sphinxAtStartPar
\sphinxhref{https://eur-lex.europa.eu/resource.html?uri=cellar:2c2f2554-0faf-11e7-8a35-01aa75ed71a1.0017.02/DOC\_3\&format=PDF}{Rekomendacijų rinkinys} apie tai, kaip užtikrinti didesnį skaitmeninį
sąveikumą tarp Europos šalių.

\sphinxAtStartPar
Rekomendacijų sąrašas:
\begin{quote}

\sphinxAtStartPar
2. Publish the data you own as open data unless certain restrictions
apply.

\sphinxAtStartPar
3. Ensure a level playing field for open source software and
demonstrate active and fair consideration of using open source
software, taking into account the total cost of ownership of the
solution.

\sphinxAtStartPar
41. Establish procedures and processes to integrate the opening of
data in your common business processes, working routines, and in the
development of new information systems.

\sphinxAtStartPar
42. Publish open data in machine\sphinxhyphen{}readable, non\sphinxhyphen{}proprietary formats.
Ensure that open data is accompanied by high quality,
machine\sphinxhyphen{}readable metadata in non\sphinxhyphen{}proprietary formats, including a
description of their content, the way data is collected and its
level of quality and the licence terms under which it is made
available. The use of common vocabularies for expressing metadata is
recommended.

\sphinxAtStartPar
43. Communicate clearly the right to access and reuse open data. The
legal regimes for facilitating access and reuse, such as licences,
should be standardised as much as possible.

\sphinxAtStartPar
44. Put in place catalogues of public services, public data, and
interoperability solutions and use common models for describing
them.

\sphinxAtStartPar
45. Where useful and feasible to do so, use external information
sources and services while developing European public services.
\end{quote}

\sphinxlineitem{IVPK\index{IVPK@\spxentry{IVPK}|spxpagem}\phantomsection\label{\detokenize{savokos:term-IVPK}}}
\sphinxAtStartPar
Informacinės visuomenės plėtros komitetas.

\begin{sphinxadmonition}{note}{Pastaba:}
\sphinxAtStartPar
IVPK pavadininimas 2024 metais pakeistas į {\hyperref[\detokenize{savokos:term-VSSA}]{\sphinxtermref{\DUrole{xref}{\DUrole{std}{\DUrole{std-term}{VSSA}}}}}}.
\end{sphinxadmonition}

\sphinxlineitem{jungtinis modelis\index{jungtinis modelis@\spxentry{jungtinis modelis}|spxpagem}\phantomsection\label{\detokenize{savokos:term-jungtinis-modelis}}}
\sphinxAtStartPar
Jungtinis modelis yra duomenų modelis, kuriame yra apjungtos kelių
skirtingų klasių savybės į vieną duomenų modelį.

\sphinxAtStartPar
Toks apjungiams yra vadinamas duomenų denormalizavimu.


\begin{sphinxseealso}{Taip pat žiūrėkite:}
\begin{itemize}
\item {} 
\sphinxAtStartPar
{\hyperref[\detokenize{identifikatoriai:ref-denorm}]{\sphinxcrossref{\DUrole{std}{\DUrole{std-ref}{Jungtinis modelis}}}}}

\end{itemize}


\end{sphinxseealso}


\sphinxlineitem{kanoniniai duomenys\index{kanoniniai duomenys@\spxentry{kanoniniai duomenys}|spxpagem}\phantomsection\label{\detokenize{savokos:term-kanoniniai-duomenys}}}
\sphinxAtStartPar
Kanoniniai duomenys yra tarsi duomenų etalonas, kuris nusako kokios
duomenų reikšmės yra teisingos. Pavyzdžiui įmonės pavadinimas gali būti
užrašomas įvairiausiomis formomis, pavyzdžiui:


\begin{savenotes}\sphinxattablestart
\sphinxthistablewithglobalstyle
\centering
\begin{tabulary}{\linewidth}[t]{TT}
\sphinxtoprule
\sphinxstyletheadfamily 
\sphinxAtStartPar
Įmonės kodas
&\sphinxstyletheadfamily 
\sphinxAtStartPar
Įmonės pavadinimas
\\
\sphinxmidrule
\sphinxtableatstartofbodyhook
\sphinxAtStartPar
\sphinxhyphen{}
&
\sphinxAtStartPar
UAB "Duomesta"
\\
\sphinxhline
\sphinxAtStartPar
\sphinxhyphen{}
&
\sphinxAtStartPar
UAB „Duomesta“
\\
\sphinxhline
\sphinxAtStartPar
\sphinxhyphen{}
&
\sphinxAtStartPar
Duomesta
\\
\sphinxhline
\sphinxAtStartPar
\sphinxhyphen{}
&
\sphinxAtStartPar
DUOMESTA
\\
\sphinxhline
\sphinxAtStartPar
\sphinxhyphen{}
&
\sphinxAtStartPar
Uždaroji akcinė bendrovė Duomesta
\\
\sphinxhline
\sphinxAtStartPar
\sphinxhyphen{}
&
\sphinxAtStartPar
Duomesta, UAB
\\
\sphinxhline
\sphinxAtStartPar
\sphinxhyphen{}
&
\sphinxAtStartPar
DSTA UAB
\\
\sphinxbottomrule
\end{tabulary}
\sphinxtableafterendhook\par
\sphinxattableend\end{savenotes}

\sphinxAtStartPar
Jei duomenų rinkinyje nėra pateiktas įmonės registracijos kodas, tada
unikaliai identifikuoti įmonę yra gan sudėtinga.

\sphinxAtStartPar
Tačiau turint autoritetingus kanoninius duomenis:


\begin{savenotes}\sphinxattablestart
\sphinxthistablewithglobalstyle
\centering
\begin{tabulary}{\linewidth}[t]{TT}
\sphinxtoprule
\sphinxstyletheadfamily 
\sphinxAtStartPar
Įmonės kodas
&\sphinxstyletheadfamily 
\sphinxAtStartPar
Įmones pavadinimas
\\
\sphinxmidrule
\sphinxtableatstartofbodyhook
\sphinxAtStartPar
111111111
&
\sphinxAtStartPar
UAB "Duomesta"
\\
\sphinxbottomrule
\end{tabulary}
\sphinxtableafterendhook\par
\sphinxattableend\end{savenotes}

\sphinxAtStartPar
Užduotis unikaliai identifikuoti įmonę pasidaro paprastesnė. Todėl
kanoniniai duomenys yra labai svarbūs.

\sphinxlineitem{kodinis pavadinimas\index{kodinis pavadinimas@\spxentry{kodinis pavadinimas}|spxpagem}\phantomsection\label{\detokenize{savokos:term-kodinis-pavadinimas}}}
\sphinxAtStartPar
Pavadinimas, kuriam keliami tam tikri apribojimai.

\sphinxlineitem{kompozicinis raktas\index{kompozicinis raktas@\spxentry{kompozicinis raktas}|spxpagem}\phantomsection\label{\detokenize{savokos:term-kompozicinis-raktas}}}
\sphinxAtStartPar
Lokalus objekto identifikatorius susidedantis iš daugiau nei vienos
reikšmės.

\sphinxlineitem{manifestas\index{manifestas@\spxentry{manifestas}|spxpagem}\phantomsection\label{\detokenize{savokos:term-manifestas}}}
\sphinxAtStartPar
Atvirų duomenų manifestas yra {\hyperref[\detokenize{savokos:term-DSA}]{\sphinxtermref{\DUrole{xref}{\DUrole{std}{\DUrole{std-term}{DSA}}}}}} lentelių rinkinys, kuriuose
aprašyti duomenų šaltiniai ir juose esančių duomenų struktūra.

\sphinxAtStartPar
Žodis manifestas yra kilęs iš programavimo srityje naudojamo termino
\sphinxhref{https://en.wikipedia.org/wiki/Manifest\_file}{Manifesto failas}, kuriame pateikiami metaduomenys apie programinio
paketo sandarą.

\sphinxAtStartPar
Duomenų kontekste, žodis manifestas turėtu būti suprantamas, kaip
metaduomenų lentelė apie įvairiuose duomenų šaltiniuose publikuojamus
duomenis.

\sphinxlineitem{metaduomenys\index{metaduomenys@\spxentry{metaduomenys}|spxpagem}\phantomsection\label{\detokenize{savokos:term-metaduomenys}}}
\sphinxAtStartPar
Duomenys apie duomenis yra vadinami metaduomenimis. Pavyzdžiui duomenų
struktūros aprašas konkrečiam CSV duomenų failui gali būti vadinamas CSV
failo metaduomenimis.

\sphinxlineitem{normalizavimas\index{normalizavimas@\spxentry{normalizavimas}|spxpagem}\phantomsection\label{\detokenize{savokos:term-normalizavimas}}}
\sphinxAtStartPar
Duomenų normalizavimas yra duomenų struktūros transformavimo procesas
taikant taip vadinamas normalines formas, tam kad sumažinti duomenų
pasikartojimą.

\sphinxAtStartPar
Plačiau apie tai skaitykite \sphinxhref{https://en.wikipedia.org/wiki/Database\_normalization}{Vikipedijoje}.

\sphinxlineitem{prieigos taškas\index{prieigos taškas@\spxentry{prieigos taškas}|spxpagem}\phantomsection\label{\detokenize{savokos:term-prieigos-taskas}}}
\sphinxAtStartPar
Prieigos taškas yra {\hyperref[\detokenize{savokos:term-REST-API}]{\sphinxtermref{\DUrole{xref}{\DUrole{std}{\DUrole{std-term}{REST API}}}}}} terminas, nurodantis URL kelio dalį iki tam
tikro resurso.

\sphinxAtStartPar
Plačiau skaitykite \sphinxhref{https://en.wikipedia.org/wiki/Web\_API\#Endpoints}{Vikipedijoje}.

\sphinxlineitem{REST API\index{REST API@\spxentry{REST API}|spxpagem}\phantomsection\label{\detokenize{savokos:term-REST-API}}}
\sphinxAtStartPar
Representational State Transfer (REST) yra taisyklių ir rekomendacijų
rinkinys sirtas {\hyperref[\detokenize{savokos:term-web-servisas}]{\sphinxtermref{\DUrole{xref}{\DUrole{std}{\DUrole{std-term}{web servisams}}}}}} kurti.

\sphinxAtStartPar
Plačiau skaitykite \sphinxhref{https://en.wikipedia.org/wiki/Representational\_state\_transfer}{Vikipedijoje}.

\sphinxlineitem{web servisas\index{web servisas@\spxentry{web servisas}|spxpagem}\phantomsection\label{\detokenize{savokos:term-web-servisas}}}
\sphinxAtStartPar
Web servisas yra interneto paslauga skirta automatizuotiems robotams.
Interneto svetainės dažniausiai yra skirtos žmonėms, tačiau web servisai
yra skirti mašinoms, kurios gali komunikuoti viena su kita.

\sphinxAtStartPar
Plačiau skaitykite \sphinxhref{https://en.wikipedia.org/wiki/Web\_service}{Vikipedijoje}.

\sphinxlineitem{YAML\index{YAML@\spxentry{YAML}|spxpagem}\phantomsection\label{\detokenize{savokos:term-YAML}}}
\sphinxAtStartPar
YAML yra {\hyperref[\detokenize{savokos:term-duomenu-serializavimo-formatas}]{\sphinxtermref{\DUrole{xref}{\DUrole{std}{\DUrole{std-term}{duomenų serializavimo formatas}}}}}}, kuris skirtas ne tik
mašininiam skaitymui, bet su šio formato turiniu tiesiogiai gali dirbti
ir žmogus. YAML formato pavyzdys:

\begin{sphinxVerbatim}[commandchars=\\\{\}]
\PYG{n+nt}{container}\PYG{p}{:}
\PYG{+w}{  }\PYG{n+nt}{name}\PYG{p}{:}\PYG{+w}{ }\PYG{l+lScalar+lScalarPlain}{value}
\end{sphinxVerbatim}

\sphinxAtStartPar
YAML yra sukurtas JSON formato pagrindu, siekant palengvinti darbą su
JSON serializuotais duomenimis žmonėms. Analogiškas pavyzdys JSON formatu
atrodo taip:

\begin{sphinxVerbatim}[commandchars=\\\{\}]
\PYG{p}{\PYGZob{}}\PYG{n+nt}{\PYGZdq{}container\PYGZdq{}}\PYG{p}{:}\PYG{+w}{ }\PYG{p}{\PYGZob{}}\PYG{n+nt}{\PYGZdq{}name\PYGZdq{}}\PYG{p}{:}\PYG{+w}{ }\PYG{l+s+s2}{\PYGZdq{}value\PYGZdq{}}\PYG{p}{\PYGZcb{}\PYGZcb{}}
\end{sphinxVerbatim}

\sphinxlineitem{viešasis žodynas\index{viešasis žodynas@\spxentry{viešasis žodynas}|spxpagem}\phantomsection\label{\detokenize{savokos:term-viesasis-zodynas}}}
\sphinxAtStartPar
Viešieji žodynai, dar vadinami ontologijomis, šie žodynai dažnai yra
gerai dokumentuoti ir skelbiami viešai, jie yra skirti globaliam
susietųjų duomenų tinkui kurti (angl. \sphinxstyleemphasis{linked data}).

\sphinxlineitem{sudėtinis tipas\index{sudėtinis tipas@\spxentry{sudėtinis tipas}|spxpagem}\phantomsection\label{\detokenize{savokos:term-sudetinis-tipas}}}
\sphinxAtStartPar
Duomenų tipas apjungiantis kelias savybes į vieną grupę. Sudėtiniam
tipui priskirtos savybės gali būti pasiekiamos \sphinxtitleref{.} (taško) pagalba.


\begin{sphinxseealso}{Taip pat žiūrėkite:}
\begin{itemize}
\item {} 
\sphinxAtStartPar
{\hyperref[\detokenize{tipai:type.ref}]{\sphinxcrossref{\sphinxcode{\sphinxupquote{type.ref}}}}}

\item {} 
\sphinxAtStartPar
{\hyperref[\detokenize{tipai:type.backref}]{\sphinxcrossref{\sphinxcode{\sphinxupquote{type.backref}}}}}

\item {} 
\sphinxAtStartPar
{\hyperref[\detokenize{tipai:type.object}]{\sphinxcrossref{\sphinxcode{\sphinxupquote{type.object}}}}}

\end{itemize}


\end{sphinxseealso}


\sphinxlineitem{sisteminis pavadinimas\index{sisteminis pavadinimas@\spxentry{sisteminis pavadinimas}|spxpagem}\phantomsection\label{\detokenize{savokos:term-sisteminis-pavadinimas}}}
\sphinxAtStartPar
Sisteminis pavadinimas yra naudojamas objektų identifikavimui ir yra
naudojamas URL nuorodose ir visur kitur, kur reikia nurodyti ryšį su
objektu, naudojamas to objekto sisteminis pavadinimas.

\sphinxAtStartPar
Sisteminis pavadinimas sudaromas tik iš lotyniškų raidžių ir \sphinxtitleref{\sphinxhyphen{}\_/}
simbolių.

\sphinxlineitem{pirminis šaltinis\index{pirminis šaltinis@\spxentry{pirminis šaltinis}|spxpagem}\phantomsection\label{\detokenize{savokos:term-pirminis-saltinis}}}\sphinxlineitem{pirminis duomenų šaltinis\index{pirminis duomenų šaltinis@\spxentry{pirminis duomenų šaltinis}|spxpagem}\phantomsection\label{\detokenize{savokos:term-pirminis-duomenu-saltinis}}}
\sphinxAtStartPar
Duomenš šaltinis, kuriame duomenys pateikiami arba įrašomi pirmą kartą.

\sphinxAtStartPar
Įprastai, kas būtų užtikrinamas duomenų vientisumas, duomenų objektai
yra registruojami vienoje vietoje, tai yra numatyta ir {\hyperref[\detokenize{savokos:term-VIIVI}]{\sphinxtermref{\DUrole{xref}{\DUrole{std}{\DUrole{std-term}{VIIVĮ}}}}}}
įstatyme, kuriame nurodyta, kad Informacinės sistemos objektai, gali
būti registruojami tik vienoje vietoje.

\sphinxlineitem{DCAT\index{DCAT@\spxentry{DCAT}|spxpagem}\phantomsection\label{\detokenize{savokos:term-DCAT}}}
\sphinxAtStartPar
Duomenų katalogo žodynas (angl. \sphinxhref{https://www.w3.org/TR/vocab-dcat-2/}{Data Catalog Vocabulary})  yra
standartas skirtas duomenų rinkiniams aprašyti. Aprašant duomenis DCAT
standartu reikėtų vadovautis {\hyperref[\detokenize{savokos:term-DCAT-AP}]{\sphinxtermref{\DUrole{xref}{\DUrole{std}{\DUrole{std-term}{DCAT\sphinxhyphen{}AP}}}}}} specifikacijomis.

\sphinxlineitem{DCAT\sphinxhyphen{}AP\index{DCAT\sphinxhyphen{}AP@\spxentry{DCAT\sphinxhyphen{}AP}|spxpagem}\phantomsection\label{\detokenize{savokos:term-DCAT-AP}}}
\sphinxAtStartPar
\sphinxhref{https://joinup.ec.europa.eu/collection/semantic-interoperability-community-semic/solution/dcat-application-profile-data-portals-europe}{DCAT\sphinxhyphen{}AP} (DCAT Application Profile) yra \sphinxhref{https://github.com/SEMICeu/DCAT-AP}{specifikacija}, detalizuojanti
DCAT naudojima, nurodant kurios DCAT klasės ir savybės yra privalomos,
kurios rekomenduojamos ir kaip jas naudoti.

\sphinxlineitem{dimensija\index{dimensija@\spxentry{dimensija}|spxpagem}\phantomsection\label{\detokenize{savokos:term-dimensija}}}
\sphinxAtStartPar
Dimensija yra metaduomenų, aprašomų DSA lentelėje, grupė. DSA lentelėje
metaduomenys skirstomi į tokias dimensijas:
\begin{itemize}
\item {} 
\sphinxAtStartPar
duomenų rinkinys

\item {} 
\sphinxAtStartPar
resursas

\item {} 
\sphinxAtStartPar
bazė

\item {} 
\sphinxAtStartPar
modelis

\item {} 
\sphinxAtStartPar
savybė

\end{itemize}

\sphinxAtStartPar
Kiekviena dimensija turi skirtingą metaduomenų detalumo lygį.

\sphinxAtStartPar
Plačiau apie dimensijas: {\hyperref[\detokenize{dimensijos:dimensijos}]{\sphinxcrossref{\DUrole{std}{\DUrole{std-ref}{Dimensijos}}}}}.

\sphinxlineitem{duomenų rinkinys\index{duomenų rinkinys@\spxentry{duomenų rinkinys}|spxpagem}\phantomsection\label{\detokenize{savokos:term-duomenu-rinkinys}}}
\sphinxAtStartPar
Duomenų rinkinys apibrėžia turimus arba pageidaujamus duomenis,
reikalingus konkrečios organizacijos, konkrečiai veiklai vykdyti.

\sphinxAtStartPar
Duomenų rinkinys gali būti registras, informacinės sistemos duomenų
bazė, interneto svetainės duomenų bazė, skaičiuoklės lentelė, dokumentų
katalogas arba duomenys, kurie dar nėra kaupiami, tačiau yra reikalingi
tam tikrai veiklai vykdyti.

\sphinxAtStartPar
Duomenų rinkinio fizinė reprezentacija, tai yra patys duomenys yra
vadinami {\hyperref[\detokenize{savokos:term-distribucija}]{\sphinxtermref{\DUrole{xref}{\DUrole{std}{\DUrole{std-term}{distribucija}}}}}}. Duomenų rinkinyje gali būti daugiau
nei viena distribucija, jei fiziškai duomenys yra suskaidyti
pagal vietos, laiko, detalumo, struktūros elementus, natūralios kalbos
ar kitus kriterijus.

\sphinxAtStartPar
Dažnai duomenų rinkinys painiojamas su distribucija. Duomenų rinkinys
apibrėžia tam tikrą grupę duomenų, kurie nebūtinai fiziškai egzistuoja,
tuo tarpu distribucija yra fiziniai duomenys įeinantys į duomenų
rinkinio sudėtį.

\sphinxAtStartPar
Duomenų rinkiniai neskaidomi pagal vietos, laiko, detalumo, struktūros
ar kitus kriterijus.

\sphinxAtStartPar
Plačiau apie tai, kaip duomenų rinkiniai aprašomi duomenų struktūros
apraše skaitykite skyriuje {\hyperref[\detokenize{dimensijos:dataset}]{\sphinxcrossref{\DUrole{std}{\DUrole{std-ref}{dataset}}}}}.

\sphinxAtStartPar
Duomenų rinkinys atitinka \sphinxhref{https://www.w3.org/TR/vocab-dcat-2/\#Class:Dataset}{dcat:Dataset} apibrėžimą.

\sphinxlineitem{distribucija\index{distribucija@\spxentry{distribucija}|spxpagem}\phantomsection\label{\detokenize{savokos:term-distribucija}}}
\sphinxAtStartPar
Distribucija yra duomenų rinkinio fizinė reprezentacija. Vienas duomenų
rinkinys gali būti sudarytas iš kelių distribucijų, tuos pačius duomenis
pateikiant skirtingais formatais, suskaidant duomenis pagal laiko,
vietos ar kitus kriterijus, tuos pačius duomenis pateikiant skirtingu
detalumu arba pateikiant agreguotus duomenis įvairiais pjūviais.

\sphinxAtStartPar
Duomenų struktūros aprašo kontekste, distribucija yra tas pats, kas
\DUrole{xref}{\DUrole{std}{\DUrole{std-ref}{resource}}}.

\sphinxAtStartPar
Distribucija atitinka \sphinxhref{https://www.w3.org/TR/vocab-dcat-2/\#Class:Distribution}{dcat:Distribution} apibrėžimą.

\sphinxlineitem{bazė\index{bazė@\spxentry{bazė}|spxpagem}\phantomsection\label{\detokenize{savokos:term-baze}}}
\sphinxAtStartPar
Bazė arba loginė klasė yra modelių grupė turinčių bendras savybes ir
vienodą semantinę prasmę.

\sphinxAtStartPar
Dažnai skirtingų organizacijų veikloje naudojami duomenų rinkiniai turi
vienodą semantinę prasmę. Pavyzdžiui, daugelis organizacijų turi
naujienų duomenis. Norint visų organizacijų naujienų duomenis
aprašyti vieningai, galima pasitelkti vieną bazę, arba vieną duomenų
rinkinį, kurio struktūrą naudoja visi kiti rinkiniai. Tai bazė būtent
ir būtų struktūros šablonas pagal kurį būtų sudaromos visų kitų
analogiškų rinkinių struktūros.

\sphinxAtStartPar
Bazė yra tas pats, kas {\hyperref[\detokenize{savokos:term-modelis}]{\sphinxtermref{\DUrole{xref}{\DUrole{std}{\DUrole{std-term}{modelis}}}}}} arba tiksliau modelio šablonas.

\sphinxAtStartPar
Duomenų struktūros aprašo kontekste apie bazę plačiau skaitykite
skyriuje {\hyperref[\detokenize{dimensijos:base}]{\sphinxcrossref{\DUrole{std}{\DUrole{std-ref}{base}}}}}.

\sphinxlineitem{modelis\index{modelis@\spxentry{modelis}|spxpagem}\phantomsection\label{\detokenize{savokos:term-modelis}}}
\sphinxAtStartPar
Modelis yra gan plati sąvoka turinti daug prasmių, priklausomai nuo
konteksto. Šioje dokumentacijoje, modelis yra duomenų struktūros
aprašo dalis leidžianti aprašyti duomenis pateiktus įvairiais
formatais.

\sphinxAtStartPar
Tiksli modelio prasmė priklauso nuo duomenų šaltinio, kurio duomenys
yra aprašomi:
\begin{itemize}
\item {} 
\sphinxAtStartPar
CSV failo atveju, modelis yra CSV faile esanti lentelė,

\item {} 
\sphinxAtStartPar
Excel failo atveju, modelis yra kiekviena lentelė (arba lapas) esanti
Excel faile,

\item {} 
\sphinxAtStartPar
SQL duomenų bazių atveju, modelis yra viena duomenų bazės lentelė,

\item {} 
\sphinxAtStartPar
JSON dokumento atveju, modelis yra kiekvienas masyvas esantis JSON
dokumente,

\item {} 
\sphinxAtStartPar
XML atveju, modelis yra kiekvienas elementų masyvas esantis XML faile.

\end{itemize}

\sphinxAtStartPar
Duomenų rinkiniai aprašo konkretaus autoriaus duomenis, skirtingi
autoriai gali naudoti tuos pačius duomenis, todėl duomenys skirtinguose
rinkiniuose gali dubliuotis. Tuo tarpu modeliai aprašo duomenis pagal
jų semantinę prasmę, nepriklausomai nuo autoriaus, tai leidžia apjungti
skirtingų autorių naudojamus duomenis, pagal jų semantinę prasmę,
modelių pagalba.

\sphinxAtStartPar
{\hyperref[\detokenize{savokos:term-DSA}]{\sphinxtermref{\DUrole{xref}{\DUrole{std}{\DUrole{std-term}{DSA}}}}}} lentelėje atitinka {\hyperref[\detokenize{formatas:model}]{\sphinxcrossref{\sphinxcode{\sphinxupquote{model}}}}}. Duomenų modelį atitinkanti
fizinė reprezentacija nurodoma {\hyperref[\detokenize{formatas:source}]{\sphinxcrossref{\sphinxcode{\sphinxupquote{source}}}}} stulpelyje. {\hyperref[\detokenize{formatas:source}]{\sphinxcrossref{\sphinxcode{\sphinxupquote{source}}}}}
gali būti duomenų bazės lentelė, CSV failas ar kita, priklauso nuo
duomenų šaltinio tipo. Sąsaja su išoriniais žodynais pateikiama
{\hyperref[\detokenize{formatas:uri}]{\sphinxcrossref{\sphinxcode{\sphinxupquote{uri}}}}} stulpelyje. Siejant su išoriniais žodynais, pateikiama
nuoroda į \sphinxhref{https://www.w3.org/TR/rdf-schema/\#ch\_class}{rdfs:Class}.

\sphinxlineitem{savybė\index{savybė@\spxentry{savybė}|spxpagem}\phantomsection\label{\detokenize{savokos:term-savybe}}}
\sphinxAtStartPar
{\hyperref[\detokenize{savokos:term-modelis}]{\sphinxtermref{\DUrole{xref}{\DUrole{std}{\DUrole{std-term}{Duomenų modeliui}}}}}} priklausančių informacinių
{\hyperref[\detokenize{savokos:term-objektas}]{\sphinxtermref{\DUrole{xref}{\DUrole{std}{\DUrole{std-term}{objektų}}}}}} savybė, pavyzdžiui miesto pavadinimas, šalis
kuriai priklauso miestas. {\hyperref[\detokenize{savokos:term-DSA}]{\sphinxtermref{\DUrole{xref}{\DUrole{std}{\DUrole{std-term}{DSA}}}}}} lentelėje atitinka
{\hyperref[\detokenize{formatas:property}]{\sphinxcrossref{\sphinxcode{\sphinxupquote{property}}}}}. Atitinka \sphinxhref{https://www.w3.org/TR/rdf-schema/\#ch\_property}{rdfs:Property} arba lentelės stulpelį.

\sphinxlineitem{subjektas\index{subjektas@\spxentry{subjektas}|spxpagem}\phantomsection\label{\detokenize{savokos:term-subjektas}}}
\sphinxAtStartPar
\sphinxhref{https://en.wikipedia.org/wiki/Semantic\_triple\#Subject,\_predicate\_and\_object}{Subjektas} lietuvių kalboje vadinamas veiksniu, duomenų kontekste
įvardija objektą apie kurį eina kalba.

\sphinxAtStartPar
Tarkime saknyje „Namas turi stogą“ subjektas yra Namas, todėl, kad
kalba eina apie namą.

\sphinxlineitem{objektas\index{objektas@\spxentry{objektas}|spxpagem}\phantomsection\label{\detokenize{savokos:term-objektas}}}
\sphinxAtStartPar
Vienas duomenų įrašas sudarytas iš savybių ir savybėms priskirtų
reikšmių. Informacinis objektas turi turėti unikalų identifikatorių.
Atitinka \sphinxhref{https://www.w3.org/TR/rdf-schema/\#ch\_resource}{rdfs:Resource} arba lentelės vieną eilutę.

\sphinxAtStartPar
Plačiau apie objektą: {\hyperref[\detokenize{modelis:objektas}]{\sphinxcrossref{\DUrole{std}{\DUrole{std-ref}{Objektas}}}}}.

\sphinxlineitem{žodynas\index{žodynas@\spxentry{žodynas}|spxpagem}\phantomsection\label{\detokenize{savokos:term-zodynas}}}
\sphinxAtStartPar
Duomenų kontekste, žodynas yra susitarimas, kokiais pavadinimais
vadinami objektai ir jų savybės. Dažniausiai kiekvienas duomenų rinkinys
turi savo vidinį naudojamą žodyną, visas Lietuvos atvirų duomenų modelis
turi savo vidinį žodyną, kuris suvienodina skirtingus duomenų rinkinių
naudojamus žodynus. Yra {\hyperref[\detokenize{savokos:term-viesasis-zodynas}]{\sphinxtermref{\DUrole{xref}{\DUrole{std}{\DUrole{std-term}{viešieji žodynai}}}}}}, dar
vadinami ontologijomis, kurie yra skelbiami viešai ir skirti globaliam
susietųjų duomenų tinklui kurti.

\sphinxAtStartPar
Duomenų kontekste, žodynas yra tiesiog {\hyperref[\detokenize{savokos:term-modelis}]{\sphinxtermref{\DUrole{xref}{\DUrole{std}{\DUrole{std-term}{modelių}}}}}} ir
{\hyperref[\detokenize{savokos:term-savybe}]{\sphinxtermref{\DUrole{xref}{\DUrole{std}{\DUrole{std-term}{savybių}}}}}} pavadinimų rinkinys. Skirtingi duomenų
šaltiniai dažniausiai naudoja skirtingus žodynus, t.y. naudoja
skirtingus {\hyperref[\detokenize{savokos:term-modelis}]{\sphinxtermref{\DUrole{xref}{\DUrole{std}{\DUrole{std-term}{modelių}}}}}} ir {\hyperref[\detokenize{savokos:term-savybe}]{\sphinxtermref{\DUrole{xref}{\DUrole{std}{\DUrole{std-term}{savybių}}}}}}
pavadinimus.

\sphinxAtStartPar
{\hyperref[\detokenize{savokos:term-DSA}]{\sphinxtermref{\DUrole{xref}{\DUrole{std}{\DUrole{std-term}{Duomenų struktūros aprašas}}}}}} leidžia skirtinguose duomenų
šaltiniuose naudojamus pavadinimus suvienodinti, taip, kad visi
šaltiniai naudotų vieningą žodyną.

\sphinxAtStartPar
Vieningo žodyno sudarymas yra gan sudėtinga užduotis, todėl, {\hyperref[\detokenize{savokos:term-DSA}]{\sphinxtermref{\DUrole{xref}{\DUrole{std}{\DUrole{std-term}{DSA}}}}}}
leidžia prie vieningo žodyno pereiti palaipsniui:
\begin{itemize}
\item {} 
\sphinxAtStartPar
pirmiausia sudaromas vieno duomenų rinkinio žodynas,

\item {} 
\sphinxAtStartPar
kuris palaipsniui transformuojamas į Lietuvos vieningą žodyną,

\item {} 
\sphinxAtStartPar
o Lietuvos vieningas žodynas palaipsniui transformuojamas į globalų
žodyną, nurodant sąsajas su išoriniais žodynais ir standartais.

\end{itemize}

\sphinxAtStartPar
Žodynai sudaromi pasitelkiant \DUrole{xref}{\DUrole{std}{\DUrole{std-ref}{vardų erdves}}}.

\sphinxlineitem{API\index{API@\spxentry{API}|spxpagem}\phantomsection\label{\detokenize{savokos:term-API}}}
\sphinxAtStartPar
Programavimo sąsaja (\sphinxstyleemphasis{angl. Application Programming Interface}).

\sphinxlineitem{duomenų šaltinis\index{duomenų šaltinis@\spxentry{duomenų šaltinis}|spxpagem}\phantomsection\label{\detokenize{savokos:term-duomenu-saltinis}}}
\sphinxAtStartPar
Resursas, kuriame saugomi duomenys. Toks resursas tampa duomenų
šaltiniu, kai tokius duomenis norima pakartotinai panaudoti, tokiu
atveju, iš pakartotinio panaudojimo perspektyvos toks resursas tampa
duomenų šaltiniu.

\sphinxlineitem{ETL\index{ETL@\spxentry{ETL}|spxpagem}\phantomsection\label{\detokenize{savokos:term-ETL}}}
\sphinxAtStartPar
Duomenų ištraukimas, transformavimas ir užkrovimas (\sphinxstyleemphasis{angl. Extract
Transform Load}).

\sphinxlineitem{iteratorius\index{iteratorius@\spxentry{iteratorius}|spxpagem}\phantomsection\label{\detokenize{savokos:term-iteratorius}}}
\sphinxAtStartPar
Tam tikra funkcija, kuri grąžina keletą elementų, tačiau ne visus iš
karto, o po vieną.

\sphinxlineitem{URI\index{URI@\spxentry{URI}|spxpagem}\phantomsection\label{\detokenize{savokos:term-URI}}}
\sphinxAtStartPar
Universalus resurso identifikatorius (\sphinxstyleemphasis{angl. Universal Resource
Identifier}).

\sphinxlineitem{POSIX\index{POSIX@\spxentry{POSIX}|spxpagem}\phantomsection\label{\detokenize{savokos:term-POSIX}}}
\sphinxAtStartPar
Universali operacinių sistemų sąsaja (\sphinxstyleemphasis{angl. Portable Operating System
Interface}) \sphinxhyphen{} standartas apibrėžiantis operacinių sistemų sąsają, kad
skirtingos operacinės sistemos būtų suderinamos tarpusavyje.

\sphinxAtStartPar
\sphinxurl{https://en.wikipedia.org/wiki/POSIX}

\sphinxlineitem{DOS\index{DOS@\spxentry{DOS}|spxpagem}\phantomsection\label{\detokenize{savokos:term-DOS}}}
\sphinxAtStartPar
\sphinxhref{https://en.wikipedia.org/wiki/MS-DOS}{MS\sphinxhyphen{}DOS}.

\sphinxlineitem{reguliarioji išraiška\index{reguliarioji išraiška@\spxentry{reguliarioji išraiška}|spxpagem}\phantomsection\label{\detokenize{savokos:term-reguliarioji-israiska}}}
\sphinxAtStartPar
Simbolių seka apibrėžianti tam tikrą šabloną tekste (angl.
\sphinxhref{https://en.wikipedia.org/wiki/Regular\_expression}{Regular Expression}).

\sphinxlineitem{JSON\index{JSON@\spxentry{JSON}|spxpagem}\phantomsection\label{\detokenize{savokos:term-JSON}}}
\sphinxAtStartPar
Atviras duomenų formatas (angl. \sphinxhref{https://en.wikipedia.org/wiki/JSON}{JavaScript Object Notation}).

\sphinxlineitem{RDF\index{RDF@\spxentry{RDF}|spxpagem}\phantomsection\label{\detokenize{savokos:term-RDF}}}
\sphinxAtStartPar
Duomenų modelis sudarytas iš subjekto, predikato ir objekto tripletų
(angl. \sphinxhref{https://en.wikipedia.org/wiki/Resource\_Description\_Framework}{Resource Description Framework}).

\sphinxlineitem{TGIĮ\index{TGIĮ@\spxentry{TGIĮ}|spxpagem}\phantomsection\label{\detokenize{savokos:term-TGII}}}
\sphinxAtStartPar
\sphinxhref{https://e-seimas.lrs.lt/portal/legalAct/lt/TAD/TAIS.94745/asr}{Teisės gauti informaciją ir duomenų pakartotinio naudojimo
įstatymas}.

\sphinxAtStartPar
Šis įstatymas įpareigoja valstybės ir savivaldybių institucijas ir
joms pavaldžius subjektus atverti duomenis.

\sphinxAtStartPar
Kelios citatos iš įstatymo:
\begin{quote}

\sphinxAtStartPar
\sphinxstylestrong{4 straipsnis}

\sphinxAtStartPar
1. Institucijos ir valstybės valdomi subjektai privalo teikti
pareiškėjams ar jų atstovams duomenis, įskaitant pakartotiniam
naudojimui skirtus duomenis, išskyrus šio įstatymo ir kitų įstatymų
nustatytus atvejus.

\sphinxAtStartPar
\sphinxstylestrong{15 straipsnis}

\sphinxAtStartPar
1. Visi institucijos ar valstybės valdomo subjekto duomenys turi
būti inventorizuoti laikantis principo, kad duomenys gali būti
skelbiami pakartotinai naudoti, jeigu tai neprieštarauja šiam ir
kitiems įstatymams. Inventorizuotų duomenų sąrašas turi būti
skelbiamas Lietuvos atvirų duomenų portale.

\sphinxAtStartPar
2. Institucijos ir valstybės valdomi subjektai turi sudaryti
duomenų, dėl kurių yra pateiktos užklausos Lietuvos atvirų duomenų
portale arba kurių pakartotinis naudojimas, institucijos ir
valstybės valdomo subjekto vertinimu, gali kurti pridėtinę vertę,
rinkinius ir juos skelbti šiame portale, jeigu tai neprieštarauja
šiam ir kitiems įstatymams.

\sphinxAtStartPar
\sphinxstylestrong{17 straipsnis}

\sphinxAtStartPar
1. Lietuvos atvirų duomenų portalas yra valstybės informacinė
sistema, skirta duomenų rinkiniams ir jų metaduomenims sisteminti ir
skelbti naudojant vienodą metaduomenų aprašymo formatą, taip pat
vieno langelio principu institucijų ir valstybės valdomų subjektų
sudarytiems duomenų rinkiniams ir jų metaduomenims ieškoti,
peržiūrėti, parsisiųsti, pareiškėjų užklausoms registruoti ir kitoms
paslaugoms, susijusioms su šios informacinės sistemos paskirtimi,
teikti.

\sphinxAtStartPar
5. Institucijos ir valstybės valdomi subjektai privalo užtikrinti,
kad inventorizuotų duomenų sąrašai ir sudaryti duomenų rinkiniai
Lietuvos atvirų duomenų portale bus surasti ir pasiekiami šio
portalo tvarkytojo nustatyta tvarka ir priemonėmis.

\sphinxAtStartPar
\sphinxstylestrong{18 straipsnis.}

\sphinxAtStartPar
Pareiškėjo teisės gali būti ginamos šiais būdais:

\sphinxAtStartPar
1) pareiškėjas turi teisę apskųsti institucijos veiksmus, neveikimą
ar administracinį sprendimą, taip pat institucijos vilkinimą atlikti
jos kompetencijai šiuo įstatymu priskirtus veiksmus Viešojo
administravimo įstatymo nustatyta tvarka;

\sphinxAtStartPar
2) pareiškėjas turi teisę apskųsti valstybės valdomo subjekto
veiksmus ar neveikimą, taip pat valstybės valdomo subjekto vilkinimą
atlikti jo kompetencijai šiuo įstatymu priskirtus veiksmus tam
pačiam valstybės valdomam subjektui arba bendrosios kompetencijos
teismui.
\end{quote}

\sphinxlineitem{ŠDSA\index{ŠDSA@\spxentry{ŠDSA}|spxpagem}\phantomsection\label{\detokenize{savokos:term-SDSA}}}
\sphinxAtStartPar
{\hyperref[\detokenize{savokos:term-DSA}]{\sphinxtermref{\DUrole{xref}{\DUrole{std}{\DUrole{std-term}{DSA}}}}}} lentelė, kurioje aprašoma neatvertų, {\hyperref[\detokenize{savokos:term-pirminis-duomenu-saltinis}]{\sphinxtermref{\DUrole{xref}{\DUrole{std}{\DUrole{std-term}{pirminio
duomenų šaltinio}}}}}} duomenų struktūra.

\sphinxlineitem{VSSA\index{VSSA@\spxentry{VSSA}|spxpagem}\phantomsection\label{\detokenize{savokos:term-VSSA}}}
\sphinxAtStartPar
\sphinxhref{https://vssa.lrv.lt/}{Valstybės skaitmeninių sprendimų agentūra}.

\sphinxlineitem{VIIVĮ\index{VIIVĮ@\spxentry{VIIVĮ}|spxpagem}\phantomsection\label{\detokenize{savokos:term-VIIVI}}}
\sphinxAtStartPar
\sphinxhref{https://e-seimas.lrs.lt/portal/legalAct/lt/TAD/TAIS.415499/asr}{Valstybės informacinių išteklių valdymo įstatymas}.

\end{description}

\sphinxstepscope


\section{Keitimų istorija}
\label{\detokenize{keitimai:keitimu-istorija}}\label{\detokenize{keitimai:keitimai}}\label{\detokenize{keitimai::doc}}

\subsection{1.1.0 (neišleista)}
\label{\detokenize{keitimai:neisleista}}\begin{itemize}
\item {} 
\sphinxAtStartPar
Pridėtos \sphinxcode{\sphinxupquote{select()}}, \sphinxcode{\sphinxupquote{expand()}}, \sphinxcode{\sphinxupquote{include()}}, \sphinxcode{\sphinxupquote{exclude()}} ir \sphinxcode{\sphinxupquote{extends()}}
funkcijos.

\item {} 
\sphinxAtStartPar
Naujas skyrius {\hyperref[\detokenize{modeliai/funkciniai:functional-models}]{\sphinxcrossref{\DUrole{std}{\DUrole{std-ref}{Funkciniai modeliai}}}}}.

\end{itemize}


\subsection{1.0.0 (2024\sphinxhyphen{}10\sphinxhyphen{}22)}
\label{\detokenize{keitimai:id1}}

\subsection{0.1.0 (2022\sphinxhyphen{}03\sphinxhyphen{}03)}
\label{\detokenize{keitimai:id2}}
\sphinxAtStartPar
Pirmoji duomenų struktūros aprašo versija.


\renewcommand{\indexname}{Python Module Index}
\begin{sphinxtheindex}
\let\bigletter\sphinxstyleindexlettergroup
\bigletter{b}
\item\relax\sphinxstyleindexentry{base}\sphinxstyleindexpageref{dimensijos:\detokenize{module-base}}
\indexspace
\bigletter{c}
\item\relax\sphinxstyleindexentry{comment}\sphinxstyleindexpageref{dimensijos:\detokenize{module-comment}}
\indexspace
\bigletter{d}
\item\relax\sphinxstyleindexentry{dataset}\sphinxstyleindexpageref{dimensijos:\detokenize{module-dataset}}
\indexspace
\bigletter{e}
\item\relax\sphinxstyleindexentry{enum}\sphinxstyleindexpageref{dimensijos:\detokenize{module-enum}}
\indexspace
\bigletter{l}
\item\relax\sphinxstyleindexentry{lang}\sphinxstyleindexpageref{dimensijos:\detokenize{module-lang}}
\indexspace
\bigletter{m}
\item\relax\sphinxstyleindexentry{migrate}\sphinxstyleindexpageref{dimensijos:\detokenize{module-migrate}}
\item\relax\sphinxstyleindexentry{model}\sphinxstyleindexpageref{dimensijos:\detokenize{module-1}}
\indexspace
\bigletter{p}
\item\relax\sphinxstyleindexentry{param}\sphinxstyleindexpageref{dimensijos:\detokenize{module-2}}
\item\relax\sphinxstyleindexentry{prefix}\sphinxstyleindexpageref{dimensijos:\detokenize{module-prefix}}
\item\relax\sphinxstyleindexentry{property}\sphinxstyleindexpageref{dimensijos:\detokenize{module-property}}
\indexspace
\bigletter{r}
\item\relax\sphinxstyleindexentry{resource}\sphinxstyleindexpageref{dimensijos:\detokenize{module-0}}
\indexspace
\bigletter{s}
\item\relax\sphinxstyleindexentry{switch}\sphinxstyleindexpageref{dimensijos:\detokenize{module-switch}}
\indexspace
\bigletter{t}
\item\relax\sphinxstyleindexentry{type}\sphinxstyleindexpageref{tipai:\detokenize{module-type}}
\end{sphinxtheindex}

\renewcommand{\indexname}{Indeksas}
\printindex
\end{document}